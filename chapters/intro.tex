\section{Introduction}
\label{sec:intro}

The central concern of this thesis is the syntax of mathematics, programming
languages, and their respective mutual influence, as conceived and practiced by
mathematicians and computer scientists.  From one vantage point, the role of
syntax in mathematics may be regarded as a 2nd order concern, a topic for
discussion during a Fika, an artifact of ad hoc development by the working
mathematician whose real goals are producing genuine mathematical knowledge.
For the programmers and computer scientists, syntax may be regarding as a
matter of taste, with friendly debates recurring regarding the use of
semicolons, brackets, and white space.  Yet, when viewed through the lens of
the propositions-as-types paradigm, these discussions intersect in new and
interesting ways.  When one introduces a third paradigm through which to
analyze the use of syntax in mathematics and programming, namely Linguistics, I
propose what some may regard as superficial detail, indeed becomes a central
paradigm, with many interesting and important questions. 

To get a feel for this syntactic paradigm, let us look at a basic mathematical
example: that of a group homomorphism, as expressed in a variety of sources.  

% Wikipedia Defn:

\begin{definition}
In mathematics, given two groups, $(G, \ast)$ and $(H, \cdot)$, a group homomorphism from $(G, \ast)$ to $(H, \cdot)$ is a function $h : G \to H$ such that for all $u$ and $v$ in $G$ it holds that

\begin{center}
  $h(u \ast v) = h ( u ) \cdot h ( v )$ 
\end{center}
\end{definition}

% http://math.mit.edu/~jwellens/Group%20Theory%20Forum.pdf

\begin{definition}
Let $G = (G,\cdot)$ and $G' = (G',\ast)$ be groups, and let $\phi : G \to G'$ be a map between them. We call $\phi$ a \textbf{homomorphism} if for every pair of elements $g, h \in G$, we have 
\begin{center}
  $\phi(g \ast h) = \phi ( g ) \cdot \phi ( h )$ 
\end{center}
\end{definition}

% http://www.maths.gla.ac.uk/~mwemyss/teaching/3alg1-7.pdf

\begin{definition}
Let $G$, $H$, be groups.  A map $\phi : G \to H$ is called a \emph{group homomorphism} if
\begin{center}
  $\phi(xy) = \phi ( x ) \phi ( y )$ for all $x, y \in G$ 
\end{center}
(Note that $xy$ on the left is formed using the group operation in $G$, whilst the product $\phi ( x ) \phi ( y )$ is formed using the group operation $H$.)
\end{definition}

% NLab:

\begin{definition}
Classically, a group is a monoid in which every element has an inverse (necessarily unique).
\end{definition}

We inquire the reader to pay attention to nuance and difference in presentation
that is normally ignored or taken for granted by the fluent mathematician.

If one want to distill the meaning of each of these presentations, there is a
significant amount of subliminal interpretation happening very much analagous
to our innate lingusitic ussage.  The inverse and identity are discarded, even
though they are necessary data when defning a group. The order of presentation
of information is incostent, as well as the choice to use symbolic or natural
language information. In (3), the group operation is used implicitly, and its
clarification a side remark.

Details aside, these all mean the same thing--don't they?  This thesis seeks to provide an
abstract framework to determine whether two lingusitically nuanced presenations
mean the same thing via their syntactic transformations. 

These syntactic transformations come in two flavors : parsing and
linearization, and are natively handled by a Logical Framework (LF) for
specifying grammars : Grammatical Framework (GF).

% previous part of definition was here

We now show yet another definition of a group homomorphism formalized in the
Agda programming language:

[TODO: replace monoidhom with grouphom]

theres supposeed to be agda here

\begin{code}[hide]%
\>[0]\AgdaComment{--\{-\# OPTIONS --cubical \#-\}}\<%
\\
\>[0]\AgdaSymbol{\{-\#}\AgdaSpace{}%
\AgdaKeyword{OPTIONS}\AgdaSpace{}%
\AgdaPragma{--cubical}\AgdaSpace{}%
\AgdaPragma{--no-import-sorts}\AgdaSpace{}%
\AgdaPragma{--safe}\AgdaSpace{}%
\AgdaSymbol{\#-\}}\<%
\\
%
\\[\AgdaEmptyExtraSkip]%
\>[0]\AgdaKeyword{module}\AgdaSpace{}%
\AgdaModule{monoid}\AgdaSpace{}%
\AgdaKeyword{where}\<%
\\
%
\\[\AgdaEmptyExtraSkip]%
\>[0]\AgdaKeyword{module}\AgdaSpace{}%
\AgdaModule{Namespace1}\AgdaSpace{}%
\AgdaKeyword{where}\<%
\\
%
\\[\AgdaEmptyExtraSkip]%
\>[0][@{}l@{\AgdaIndent{0}}]%
\>[2]\AgdaKeyword{open}\AgdaSpace{}%
\AgdaKeyword{import}\AgdaSpace{}%
\AgdaModule{Cubical.Foundations.Prelude}\<%
\\
%
\>[2]\AgdaKeyword{open}\AgdaSpace{}%
\AgdaKeyword{import}\AgdaSpace{}%
\AgdaModule{Cubical.Foundations.Equiv}\<%
\\
%
\>[2]\AgdaKeyword{open}\AgdaSpace{}%
\AgdaKeyword{import}\AgdaSpace{}%
\AgdaModule{Cubical.Foundations.Structure}\<%
\\
%
\>[2]\AgdaKeyword{open}\AgdaSpace{}%
\AgdaKeyword{import}\AgdaSpace{}%
\AgdaModule{Cubical.Algebra.Group.Base}\<%
\\
%
\>[2]\AgdaKeyword{open}\AgdaSpace{}%
\AgdaKeyword{import}\AgdaSpace{}%
\AgdaModule{Cubical.Data.Sigma}\<%
\\
%
\\[\AgdaEmptyExtraSkip]%
%
\>[2]\AgdaKeyword{private}\<%
\\
\>[2][@{}l@{\AgdaIndent{0}}]%
\>[4]\AgdaKeyword{variable}\<%
\\
\>[4][@{}l@{\AgdaIndent{0}}]%
\>[6]\AgdaGeneralizable{ℓ}\AgdaSpace{}%
\AgdaGeneralizable{ℓ'}\AgdaSpace{}%
\AgdaGeneralizable{ℓ''}\AgdaSpace{}%
\AgdaGeneralizable{ℓ'''}\AgdaSpace{}%
\AgdaSymbol{:}\AgdaSpace{}%
\AgdaPostulate{Level}\<%
\end{code}
\begin{code}%
%
\>[2]\AgdaFunction{isGroupHom}\AgdaSpace{}%
\AgdaSymbol{:}\AgdaSpace{}%
\AgdaSymbol{(}\AgdaBound{G}\AgdaSpace{}%
\AgdaSymbol{:}\AgdaSpace{}%
\AgdaFunction{Group}\AgdaSpace{}%
\AgdaSymbol{\{}\AgdaGeneralizable{ℓ}\AgdaSymbol{\})}\AgdaSpace{}%
\AgdaSymbol{(}\AgdaBound{H}\AgdaSpace{}%
\AgdaSymbol{:}\AgdaSpace{}%
\AgdaFunction{Group}\AgdaSpace{}%
\AgdaSymbol{\{}\AgdaGeneralizable{ℓ'}\AgdaSymbol{\})}\AgdaSpace{}%
\AgdaSymbol{(}\AgdaBound{f}\AgdaSpace{}%
\AgdaSymbol{:}\AgdaSpace{}%
\AgdaOperator{\AgdaFunction{⟨}}\AgdaSpace{}%
\AgdaBound{G}\AgdaSpace{}%
\AgdaOperator{\AgdaFunction{⟩}}\AgdaSpace{}%
\AgdaSymbol{→}\AgdaSpace{}%
\AgdaOperator{\AgdaFunction{⟨}}\AgdaSpace{}%
\AgdaBound{H}\AgdaSpace{}%
\AgdaOperator{\AgdaFunction{⟩}}\AgdaSymbol{)}\AgdaSpace{}%
\AgdaSymbol{→}\AgdaSpace{}%
\AgdaPrimitive{Type}\AgdaSpace{}%
\AgdaSymbol{\AgdaUnderscore{}}\<%
\\
%
\>[2]\AgdaFunction{isGroupHom}\AgdaSpace{}%
\AgdaBound{G}\AgdaSpace{}%
\AgdaBound{H}\AgdaSpace{}%
\AgdaBound{f}\AgdaSpace{}%
\AgdaSymbol{=}\AgdaSpace{}%
\AgdaSymbol{(}\AgdaBound{x}\AgdaSpace{}%
\AgdaBound{y}\AgdaSpace{}%
\AgdaSymbol{:}\AgdaSpace{}%
\AgdaOperator{\AgdaFunction{⟨}}\AgdaSpace{}%
\AgdaBound{G}\AgdaSpace{}%
\AgdaOperator{\AgdaFunction{⟩}}\AgdaSymbol{)}\AgdaSpace{}%
\AgdaSymbol{→}\AgdaSpace{}%
\AgdaBound{f}\AgdaSpace{}%
\AgdaSymbol{(}\AgdaBound{x}\AgdaSpace{}%
\AgdaOperator{\AgdaFunction{G.+}}\AgdaSpace{}%
\AgdaBound{y}\AgdaSymbol{)}\AgdaSpace{}%
\AgdaOperator{\AgdaFunction{≡}}\AgdaSpace{}%
\AgdaSymbol{(}\AgdaBound{f}\AgdaSpace{}%
\AgdaBound{x}\AgdaSpace{}%
\AgdaOperator{\AgdaFunction{H.+}}\AgdaSpace{}%
\AgdaBound{f}\AgdaSpace{}%
\AgdaBound{y}\AgdaSymbol{)}\AgdaSpace{}%
\AgdaKeyword{where}\<%
\\
\>[2][@{}l@{\AgdaIndent{0}}]%
\>[4]\AgdaKeyword{module}\AgdaSpace{}%
\AgdaModule{G}\AgdaSpace{}%
\AgdaSymbol{=}\AgdaSpace{}%
\AgdaModule{GroupStr}\AgdaSpace{}%
\AgdaSymbol{(}\AgdaField{snd}\AgdaSpace{}%
\AgdaBound{G}\AgdaSymbol{)}\<%
\\
%
\>[4]\AgdaKeyword{module}\AgdaSpace{}%
\AgdaModule{H}\AgdaSpace{}%
\AgdaSymbol{=}\AgdaSpace{}%
\AgdaModule{GroupStr}\AgdaSpace{}%
\AgdaSymbol{(}\AgdaField{snd}\AgdaSpace{}%
\AgdaBound{H}\AgdaSymbol{)}\<%
\\
%
\\[\AgdaEmptyExtraSkip]%
%
\>[2]\AgdaKeyword{record}\AgdaSpace{}%
\AgdaRecord{GroupHom}\AgdaSpace{}%
\AgdaSymbol{(}\AgdaBound{G}\AgdaSpace{}%
\AgdaSymbol{:}\AgdaSpace{}%
\AgdaFunction{Group}\AgdaSpace{}%
\AgdaSymbol{\{}\AgdaGeneralizable{ℓ}\AgdaSymbol{\})}\AgdaSpace{}%
\AgdaSymbol{(}\AgdaBound{H}\AgdaSpace{}%
\AgdaSymbol{:}\AgdaSpace{}%
\AgdaFunction{Group}\AgdaSpace{}%
\AgdaSymbol{\{}\AgdaGeneralizable{ℓ'}\AgdaSymbol{\})}\AgdaSpace{}%
\AgdaSymbol{:}\AgdaSpace{}%
\AgdaPrimitive{Type}\AgdaSpace{}%
\AgdaSymbol{(}\AgdaPrimitive{ℓ-max}\AgdaSpace{}%
\AgdaBound{ℓ}\AgdaSpace{}%
\AgdaBound{ℓ'}\AgdaSymbol{)}\AgdaSpace{}%
\AgdaKeyword{where}\<%
\\
\>[2][@{}l@{\AgdaIndent{0}}]%
\>[4]\AgdaKeyword{constructor}\AgdaSpace{}%
\AgdaInductiveConstructor{grouphom}\<%
\\
%
\\[\AgdaEmptyExtraSkip]%
%
\>[4]\AgdaKeyword{field}\<%
\\
\>[4][@{}l@{\AgdaIndent{0}}]%
\>[6]\AgdaField{fun}\AgdaSpace{}%
\AgdaSymbol{:}\AgdaSpace{}%
\AgdaOperator{\AgdaFunction{⟨}}\AgdaSpace{}%
\AgdaBound{G}\AgdaSpace{}%
\AgdaOperator{\AgdaFunction{⟩}}\AgdaSpace{}%
\AgdaSymbol{→}\AgdaSpace{}%
\AgdaOperator{\AgdaFunction{⟨}}\AgdaSpace{}%
\AgdaBound{H}\AgdaSpace{}%
\AgdaOperator{\AgdaFunction{⟩}}\<%
\\
%
\>[6]\AgdaField{isHom}\AgdaSpace{}%
\AgdaSymbol{:}\AgdaSpace{}%
\AgdaFunction{isGroupHom}\AgdaSpace{}%
\AgdaBound{G}\AgdaSpace{}%
\AgdaBound{H}\AgdaSpace{}%
\AgdaField{fun}\<%
\end{code}
This actually \emph{was} the Cubical Agda implementation of a group homomorphism
sometime around the end of 2020. We see that, while a mathematician might be
able to infer the meaning of some of the syntax, the use of levels,
distinguising between isGroupHom and GroupHom itself, and many other details
might obscure what's going on.

We finally provide the current (May 2021) definition via Cubical Agda. One may
witness a significant number of differences from the previous version : concrete
syntax differences via changes in camel case, new uses of Group vs GroupStr, as
well as, most significantly, the identity and inverse preservation data not
appearing as corollaries, but part of the definition. Additionally, we had to
refactor the commented lines to those shown below to be compatible with our
outdated version of cubical. These changes reflect interesting syntactic
changes.

\begin{code}%
%
\>[2]\AgdaKeyword{record}\AgdaSpace{}%
\AgdaRecord{IsGroupHom}\AgdaSpace{}%
\AgdaSymbol{\{}\AgdaBound{A}\AgdaSpace{}%
\AgdaSymbol{:}\AgdaSpace{}%
\AgdaPrimitive{Type}\AgdaSpace{}%
\AgdaGeneralizable{ℓ}\AgdaSymbol{\}}\AgdaSpace{}%
\AgdaSymbol{\{}\AgdaBound{B}\AgdaSpace{}%
\AgdaSymbol{:}\AgdaSpace{}%
\AgdaPrimitive{Type}\AgdaSpace{}%
\AgdaGeneralizable{ℓ'}\AgdaSymbol{\}}\<%
\\
\>[2][@{}l@{\AgdaIndent{0}}]%
\>[4]\AgdaSymbol{(}\AgdaBound{M}\AgdaSpace{}%
\AgdaSymbol{:}\AgdaSpace{}%
\AgdaRecord{GroupStr}\AgdaSpace{}%
\AgdaBound{A}\AgdaSymbol{)}\AgdaSpace{}%
\AgdaSymbol{(}\AgdaBound{f}\AgdaSpace{}%
\AgdaSymbol{:}\AgdaSpace{}%
\AgdaBound{A}\AgdaSpace{}%
\AgdaSymbol{→}\AgdaSpace{}%
\AgdaBound{B}\AgdaSymbol{)}\AgdaSpace{}%
\AgdaSymbol{(}\AgdaBound{N}\AgdaSpace{}%
\AgdaSymbol{:}\AgdaSpace{}%
\AgdaRecord{GroupStr}\AgdaSpace{}%
\AgdaBound{B}\AgdaSymbol{)}\<%
\\
%
\>[4]\AgdaSymbol{:}\AgdaSpace{}%
\AgdaPrimitive{Type}\AgdaSpace{}%
\AgdaSymbol{(}\AgdaPrimitive{ℓ-max}\AgdaSpace{}%
\AgdaBound{ℓ}\AgdaSpace{}%
\AgdaBound{ℓ'}\AgdaSymbol{)}\<%
\\
%
\>[4]\AgdaKeyword{where}\<%
\\
%
\\[\AgdaEmptyExtraSkip]%
%
\>[4]\AgdaComment{-- Shorter qualified names}\<%
\\
%
\>[4]\AgdaKeyword{private}\<%
\\
\>[4][@{}l@{\AgdaIndent{0}}]%
\>[6]\AgdaKeyword{module}\AgdaSpace{}%
\AgdaModule{M}\AgdaSpace{}%
\AgdaSymbol{=}\AgdaSpace{}%
\AgdaModule{GroupStr}\AgdaSpace{}%
\AgdaBound{M}\<%
\\
%
\>[6]\AgdaKeyword{module}\AgdaSpace{}%
\AgdaModule{N}\AgdaSpace{}%
\AgdaSymbol{=}\AgdaSpace{}%
\AgdaModule{GroupStr}\AgdaSpace{}%
\AgdaBound{N}\<%
\\
%
\\[\AgdaEmptyExtraSkip]%
%
\>[4]\AgdaKeyword{field}\<%
\\
\>[4][@{}l@{\AgdaIndent{0}}]%
\>[6]\AgdaField{pres·}\AgdaSpace{}%
\AgdaSymbol{:}\AgdaSpace{}%
\AgdaSymbol{(}\AgdaBound{x}\AgdaSpace{}%
\AgdaBound{y}\AgdaSpace{}%
\AgdaSymbol{:}\AgdaSpace{}%
\AgdaBound{A}\AgdaSymbol{)}\AgdaSpace{}%
\AgdaSymbol{→}\AgdaSpace{}%
\AgdaBound{f}\AgdaSpace{}%
\AgdaSymbol{(}\AgdaOperator{\AgdaFunction{M.\AgdaUnderscore{}+\AgdaUnderscore{}}}\AgdaSpace{}%
\AgdaBound{x}\AgdaSpace{}%
\AgdaBound{y}\AgdaSymbol{)}\AgdaSpace{}%
\AgdaOperator{\AgdaFunction{≡}}\AgdaSpace{}%
\AgdaSymbol{(}\AgdaOperator{\AgdaFunction{N.\AgdaUnderscore{}+\AgdaUnderscore{}}}\AgdaSpace{}%
\AgdaSymbol{(}\AgdaBound{f}\AgdaSpace{}%
\AgdaBound{x}\AgdaSymbol{)}\AgdaSpace{}%
\AgdaSymbol{(}\AgdaBound{f}\AgdaSpace{}%
\AgdaBound{y}\AgdaSymbol{))}\<%
\\
%
\>[6]\AgdaField{pres1}\AgdaSpace{}%
\AgdaSymbol{:}\AgdaSpace{}%
\AgdaBound{f}\AgdaSpace{}%
\AgdaFunction{M.0g}\AgdaSpace{}%
\AgdaOperator{\AgdaFunction{≡}}\AgdaSpace{}%
\AgdaFunction{N.0g}\<%
\\
%
\>[6]\AgdaField{presinv}\AgdaSpace{}%
\AgdaSymbol{:}\AgdaSpace{}%
\AgdaSymbol{(}\AgdaBound{x}\AgdaSpace{}%
\AgdaSymbol{:}\AgdaSpace{}%
\AgdaBound{A}\AgdaSymbol{)}\AgdaSpace{}%
\AgdaSymbol{→}\AgdaSpace{}%
\AgdaBound{f}\AgdaSpace{}%
\AgdaSymbol{(}\AgdaOperator{\AgdaFunction{M.-\AgdaUnderscore{}}}\AgdaSpace{}%
\AgdaBound{x}\AgdaSymbol{)}\AgdaSpace{}%
\AgdaOperator{\AgdaFunction{≡}}\AgdaSpace{}%
\AgdaOperator{\AgdaFunction{N.-\AgdaUnderscore{}}}\AgdaSpace{}%
\AgdaSymbol{(}\AgdaBound{f}\AgdaSpace{}%
\AgdaBound{x}\AgdaSymbol{)}\<%
\\
%
\>[6]\AgdaComment{-- pres· : (x y : A) → f (x M.· y) ≡ f x N.· f y}\<%
\\
%
\>[6]\AgdaComment{-- pres1 : f M.1g ≡ N.1g}\<%
\\
%
\>[6]\AgdaComment{-- presinv : (x : A) → f (M.inv x) ≡ N.inv (f x)}\<%
\\
%
\\[\AgdaEmptyExtraSkip]%
%
\>[2]\AgdaFunction{GroupHom'}\AgdaSpace{}%
\AgdaSymbol{:}\AgdaSpace{}%
\AgdaSymbol{(}\AgdaBound{G}\AgdaSpace{}%
\AgdaSymbol{:}\AgdaSpace{}%
\AgdaFunction{Group}\AgdaSpace{}%
\AgdaSymbol{\{}\AgdaGeneralizable{ℓ}\AgdaSymbol{\})}\AgdaSpace{}%
\AgdaSymbol{(}\AgdaBound{H}\AgdaSpace{}%
\AgdaSymbol{:}\AgdaSpace{}%
\AgdaFunction{Group}\AgdaSpace{}%
\AgdaSymbol{\{}\AgdaGeneralizable{ℓ'}\AgdaSymbol{\})}\AgdaSpace{}%
\AgdaSymbol{→}\AgdaSpace{}%
\AgdaPrimitive{Type}\AgdaSpace{}%
\AgdaSymbol{(}\AgdaPrimitive{ℓ-max}\AgdaSpace{}%
\AgdaGeneralizable{ℓ}\AgdaSpace{}%
\AgdaGeneralizable{ℓ'}\AgdaSymbol{)}\<%
\\
%
\>[2]\AgdaComment{-- GroupHom' : (G : Group ℓ) (H : Group ℓ') → Type (ℓ-max ℓ ℓ')}\<%
\\
%
\>[2]\AgdaFunction{GroupHom'}\AgdaSpace{}%
\AgdaBound{G}\AgdaSpace{}%
\AgdaBound{H}\AgdaSpace{}%
\AgdaSymbol{=}\AgdaSpace{}%
\AgdaFunction{Σ[}\AgdaSpace{}%
\AgdaBound{f}\AgdaSpace{}%
\AgdaFunction{∈}\AgdaSpace{}%
\AgdaSymbol{(}\AgdaBound{G}\AgdaSpace{}%
\AgdaSymbol{.}\AgdaField{fst}\AgdaSpace{}%
\AgdaSymbol{→}\AgdaSpace{}%
\AgdaBound{H}\AgdaSpace{}%
\AgdaSymbol{.}\AgdaField{fst}\AgdaSymbol{)}\AgdaSpace{}%
\AgdaFunction{]}\AgdaSpace{}%
\AgdaRecord{IsGroupHom}\AgdaSpace{}%
\AgdaSymbol{(}\AgdaBound{G}\AgdaSpace{}%
\AgdaSymbol{.}\AgdaField{snd}\AgdaSymbol{)}\AgdaSpace{}%
\AgdaBound{f}\AgdaSpace{}%
\AgdaSymbol{(}\AgdaBound{H}\AgdaSpace{}%
\AgdaSymbol{.}\AgdaField{snd}\AgdaSymbol{)}\<%
\end{code}


% \begin{code}%
\>[0]\<%
\\
\>[0]\AgdaKeyword{module}\AgdaSpace{}%
\AgdaModule{Nat}\AgdaSpace{}%
\AgdaKeyword{where}\<%
\\
%
\\[\AgdaEmptyExtraSkip]%
\>[0]\AgdaKeyword{data}\AgdaSpace{}%
\AgdaDatatype{ℕ}\AgdaSpace{}%
\AgdaSymbol{:}\AgdaSpace{}%
\AgdaPrimitive{Set}\AgdaSpace{}%
\AgdaKeyword{where}\<%
\\
\>[0][@{}l@{\AgdaIndent{0}}]%
\>[2]\AgdaInductiveConstructor{z}\AgdaSpace{}%
\AgdaSymbol{:}\AgdaSpace{}%
\AgdaDatatype{ℕ}\<%
\\
%
\>[2]\AgdaInductiveConstructor{s}\AgdaSpace{}%
\AgdaSymbol{:}\AgdaSpace{}%
\AgdaDatatype{ℕ}\AgdaSpace{}%
\AgdaSymbol{→}\AgdaSpace{}%
\AgdaDatatype{ℕ}\<%
\\
\>[0]\<%
\end{code}

% \begin{code}%
\>[0]\<%
\\
\>[0]\AgdaKeyword{module}\AgdaSpace{}%
\AgdaModule{Plus}\AgdaSpace{}%
\AgdaKeyword{where}\<%
\\
%
\\[\AgdaEmptyExtraSkip]%
\>[0]\AgdaKeyword{open}\AgdaSpace{}%
\AgdaKeyword{import}\AgdaSpace{}%
\AgdaModule{Nat}\<%
\\
%
\\[\AgdaEmptyExtraSkip]%
\>[0]\AgdaFunction{plus}\AgdaSpace{}%
\AgdaSymbol{:}\AgdaSpace{}%
\AgdaDatatype{ℕ}\AgdaSpace{}%
\AgdaSymbol{→}\AgdaSpace{}%
\AgdaDatatype{ℕ}\AgdaSpace{}%
\AgdaSymbol{→}\AgdaSpace{}%
\AgdaDatatype{ℕ}\<%
\\
\>[0]\AgdaFunction{plus}\AgdaSpace{}%
\AgdaInductiveConstructor{z}\AgdaSpace{}%
\AgdaBound{x₁}\AgdaSpace{}%
\AgdaSymbol{=}\AgdaSpace{}%
\AgdaBound{x₁}\<%
\\
\>[0]\AgdaFunction{plus}\AgdaSpace{}%
\AgdaSymbol{(}\AgdaInductiveConstructor{s}\AgdaSpace{}%
\AgdaBound{x}\AgdaSymbol{)}\AgdaSpace{}%
\AgdaBound{x₁}\AgdaSpace{}%
\AgdaSymbol{=}\AgdaSpace{}%
\AgdaInductiveConstructor{s}\AgdaSpace{}%
\AgdaSymbol{(}\AgdaFunction{plus}\AgdaSpace{}%
\AgdaBound{x}\AgdaSpace{}%
\AgdaBound{x₁}\AgdaSymbol{)}\<%
\\
\>[0]\<%
\end{code}

% \begin{code}%
\>[0]\<%
\\
\>[0]\AgdaKeyword{module}\AgdaSpace{}%
\AgdaModule{Times}\AgdaSpace{}%
\AgdaKeyword{where}\<%
\\
%
\\[\AgdaEmptyExtraSkip]%
\>[0]\AgdaKeyword{open}\AgdaSpace{}%
\AgdaKeyword{import}\AgdaSpace{}%
\AgdaModule{Nat}\<%
\\
\>[0]\AgdaKeyword{open}\AgdaSpace{}%
\AgdaKeyword{import}\AgdaSpace{}%
\AgdaModule{Plus}\<%
\\
%
\\[\AgdaEmptyExtraSkip]%
\>[0]\AgdaFunction{times}\AgdaSpace{}%
\AgdaSymbol{:}\AgdaSpace{}%
\AgdaDatatype{ℕ}\AgdaSpace{}%
\AgdaSymbol{→}\AgdaSpace{}%
\AgdaDatatype{ℕ}\AgdaSpace{}%
\AgdaSymbol{→}\AgdaSpace{}%
\AgdaDatatype{ℕ}\<%
\\
\>[0]\AgdaFunction{times}\AgdaSpace{}%
\AgdaInductiveConstructor{z}\AgdaSpace{}%
\AgdaBound{y}\AgdaSpace{}%
\AgdaSymbol{=}\AgdaSpace{}%
\AgdaInductiveConstructor{z}\<%
\\
\>[0]\AgdaFunction{times}\AgdaSpace{}%
\AgdaSymbol{(}\AgdaInductiveConstructor{s}\AgdaSpace{}%
\AgdaBound{x}\AgdaSymbol{)}\AgdaSpace{}%
\AgdaBound{y}\AgdaSpace{}%
\AgdaSymbol{=}\AgdaSpace{}%
\AgdaFunction{plus}\AgdaSpace{}%
\AgdaBound{y}\AgdaSpace{}%
\AgdaSymbol{(}\AgdaFunction{times}\AgdaSpace{}%
\AgdaBound{x}\AgdaSpace{}%
\AgdaBound{y}\AgdaSymbol{)}\<%
\\
\>[0]\<%
\end{code}


While the first three definitions above are should be linguistically
comprehensible to a non-mathematician, this last definition is most certainly
not.  While may carry degree of comprehension to a programmer or mathematician
not exposed to Agda, it is certainly comprehensible to a computer : that is, it
typechecks on a computer where Cubical Agda is installed. While GF is designed
for multilingual syntactic transformations and is targeted for natural language
translation, it's underlying theory is largely based on ideas from the compiler
communities. A cousin of the BNF Converter (BNFC), GF is fully capable of
parsing progamming languages like Agda! And while the above definition is just
another concrete syntactic presentation of a group homomorphism, it is distinct
from the natural language presentations above in that the colors indicate it
has indeed type checked. 

While this example may not exemplify the power of Agda's type checker, it is of
considerable interest to many. The typechecker has merely assured us that
monoidHom, is a well-formed type.  The type-checker is much more useful than is
immediately evident: it delegates the work of verifying that a proof is
correct, that is, the work of judging whether a term has a type, to the
computer. While it's of practical concern is immediate to any exploited grad
student grading papers late on a Sunday night, its theoretical concern has led
to many recent developments in modern mathematics. Thomas Hales solution to the
Kepler Conjecture was seen as unverifiable by those reviewing it. This led to
Hales outsourcing the verification to Interactive Theorem provers HOL Light and
Isabelle, during which led to many minor corrections in the original proof
which were never spotted due to human oversight.

Fields Medalist Vladimir Voevodsky, had the experience of being told one day
his proof of the Milnor conjecture was fatally flawed. Although the leak in the
proof was patched, this experience of temporarily believing much of his life's
work invalid led him to investigate proof assintants as a tool for future
thought. Indeed, this proof verification error was a key event that led to the
Univalent Foundations
Project~\cite{theunivalentfoundationsprogram-homotopytypetheory-2013}.

While Agda and other programming languages are capable of encoding definitions,
theorems, and proofs, they have so far seen little adoption, and in some cases
treated with suspicion and scorn by many mathematicians.  This isn't entirely
unfounded : it's a lot of work to learn how to use Agda or Coq, software
updates may cause proofs to break, and the inevitable errors we humans are
instilled in these Theorem Provers. And that's not to mention that Martin-Löf
Type Theory, the constructive foundational project which underlies these proof
assistants, is rejected by those who dogmatically accept the law of the
excluded middle and ZFC as the word of God.

It should be noted, the constructivist rejects neither the law of the excluded
middle nor ZFC. She merely observes them, and admits their handiness in certain
cituations. Excluded middle is indeed a helpful tool, as many mathematicians
may attest. The contention is that it should be avoided whenever possible -
proofs which don't rely on it, or it's corallary of proof by contradction, are
much more ameanable to formalization in systems with decideable type checking.
And ZFC, while serving the mathematicians of the early 20th century, is 
lacking when it comes to the higher dimensional structure of n-categories and
infinity groupoids.

What these theorem provers give the mathematician is confidence that her work
is correct, and even more importantly, that the work which she takes for
granted and references in her work is also correct. The task before us is then
one of religious conversion. And one doesn't undertake a conversion by simply
by preaching. Foundational details aside, this thesis is meant to provide a
blueprint for the syntactic reformation that must take place.  

It doesn't ask the mathematician to relinquish the beautiful language she has
come to love in expressing her ideas.  Rather, it asks her to make a compromise
for the time being, and use a Controlled Natural Language (CNL) to develop her
work. In exchange she'll get the confidence that Agda provides. Not only that,
she'll be able to search through a library, to see who else has possibly
already postulated and proved her conjecture. A version of this grandiose vision is 
explored in The Formal Abstracts Project.
