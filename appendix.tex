\section{Appendix} \label{appendix}


\subsection{cubicalTT} \label{cubicaltt}
\begin{verbatim}
abstract Exp = {

flags startcat = Decl ;
      -- note, cubical tt doesn't support inductive families, and therefore the datatype (& labels) need to be modified

cat
  Comment ;
  Module  ;
  AIdent ;
  Imp ; --imports, add later
  Decl ;
  Exp ;
  ExpWhere ;
  Tele ;
  Branch ;
  PTele ;
  Label ;
  [AIdent]{0} ; -- "x y z"
  [Decl]{1} ;
  [Tele]{0} ;
  [Branch]{1} ;
  [Label]{1} ;
  [PTele]{1} ;
  -- [Exp]{1};

fun

  DeclDef : AIdent -> [Tele] -> Exp -> ExpWhere -> Decl ;
  -- foo ( b : bool ) : bool = b
  DeclData : AIdent -> [Tele] -> [Label] -> Decl ;
  -- data nat : Set where zero | suc ( n : nat )
  DeclSplit : AIdent -> [Tele] -> Exp -> [Branch] -> Decl ;
  -- caseBool ( x : Set ) ( y z : x ) : bool -> Set = split false -> y || true -> z
  DeclUndef : AIdent -> [Tele] -> Exp -> Decl ;
  -- funExt ( a : Set ) ( b : a -> Set ) ( f g : ( x : a ) -> b x ) ( p : ( x : a ) -> ( b x ) ( f x ) == ( g x ) ) : ( ( y : a ) -> b y ) f == g = undefined

  Where : Exp -> [Decl] -> ExpWhere ;
  -- foo ( b : bool ) : bool =
  -- f b where f : bool -> bool = negb
  NoWhere : Exp -> ExpWhere ;
  -- foo ( b : bool ) : bool =
  -- b

  Split : Exp -> [Branch] -> Exp ;
  --split@ ( nat -> bool ) with zero  -> true || suc n -> false
  Let : [Decl] -> Exp -> Exp ;
  -- foo ( b : bool ) : bool =
  -- let f : bool -> bool = negb in f b
  Lam : [PTele] -> Exp -> Exp ;
  -- \\ ( x : bool ) -> negb x
  -- todo, allow implicit typing
  Fun : Exp -> Exp -> Exp ;
  -- Set -> Set
  -- Set -> ( b : bool ) -> ( x : Set ) -> ( f x )
  Pi    : [PTele] -> Exp -> Exp ;
  --( f : bool -> Set ) -> ( b : bool ) -> ( x : Set ) -> ( f x )
  -- ( f : bool -> Set ) ( b : bool ) ( x : Set ) -> ( f x )
  Sigma : [PTele] -> Exp -> Exp ;
  -- ( f : bool -> Set ) ( b : bool ) ( x : Set ) * ( f x )
  App : Exp -> Exp -> Exp ;
  -- proj1 ( x , y )
  Id : Exp -> Exp -> Exp -> Exp ;
  -- Set bool == nat
  IdJ : Exp -> Exp -> Exp -> Exp -> Exp -> Exp ;
  -- J e d x y p
  Fst : Exp -> Exp ; -- "proj1 x"
  Snd : Exp -> Exp ; -- "proj2 x"
  -- Pair : Exp -> [Exp] -> Exp ;
  Pair : Exp -> Exp -> Exp ;
  -- ( x , y )
  Var : AIdent -> Exp ;
  -- x
  Univ : Exp ;
  -- Set
  Refl : AIdent ; -- Exp ;
  -- refl
  --Hole : HoleIdent -> Exp ; -- need to add holes

  OBranch :  AIdent -> [AIdent] -> ExpWhere -> Branch ;
  -- suc m -> split@ ( nat -> bool ) with zero -> false || suc n -> equalNat m n
  -- for splits

  OLabel : AIdent -> [Tele] -> Label ;
  -- suc ( n : nat )
  -- fora data types

  -- construct telescope
  TeleC : AIdent -> [AIdent] -> Exp -> Tele ;
  -- "( f g : ( x : a ) -> b x )"
  -- ( a : Set ) ( b : ( a ) -> ( Set ) ) ( f g : ( x : a ) -> ( ( b ) ( x ) ) ) ( p : ( x : a ) -> ( ( ( b ) ( x ) ) ( ( f ) ( x ) ) == ( ( g ) ( x ) ) ) )

  -- why does gr with this fail so epically?
  PTeleC : Exp -> Exp -> PTele ; 
  -- ( x : Set ) -- ( y : x -> Set )" -- ( x : f y z )"

  --everything below this is just strings

  Foo : AIdent ;
  A , B , C , D , E , F , G , H , I , J , K , L , M , N , O , P , Q , R , S , T , U , V , W , X , Y , Z : AIdent ;
  True , False , Bool : AIdent ;
  NegB : AIdent ;
  CaseBool : AIdent ;
  IndBool : AIdent ;
  FunExt : AIdent ;
  Nat : AIdent ;
  Zero : AIdent ;
  Suc : AIdent ;
  EqualNat : AIdent ;
  Unit : AIdent ;
  Top : AIdent ;
  Contr : AIdent ;
  Fiber : AIdent ;
  IsEquiv : AIdent ;
  IdIsEquiv : AIdent ;
  IdFun : AIdent ;
  ContrSingl : AIdent ;
  Equiv : AIdent ;
  EqToIso : AIdent ;
  Ybar : AIdent ;
  IdFib : AIdent ;
  Identity : AIdent ;
  Lemma0 : AIdent ;
}
\end{verbatim}
\begin{verbatim}
concrete ExpCubicalTT of Exp = open Prelude, FormalTwo in {

lincat 
  Comment,
  Module ,
  AIdent,
  Imp,
  Decl ,
  ExpWhere,
  Tele,
  Branch ,
  PTele,
  Label,
    -- = Str ;
  [AIdent],
  [Decl] ,
  -- [Exp],
  [Tele],
  [Branch] ,
  [PTele],
  [Label]
    -- = {hd,tl : Str} ;
    = Str ;
  Exp = TermPrec ;

lin

  DeclDef a lt e ew = a ++ lt ++ ":" ++ usePrec 0 e ++ "=" ++ ew ;
  DeclData a t d = "data" ++ a ++ t ++ ": Set where" ++ d ;
  DeclSplit ai lt e lb = ai ++ lt ++ ":" ++ usePrec 0 e ++ "= split" ++ lb ;
  DeclUndef a lt e = a ++ lt ++ ":" ++ usePrec 0 e ++ "= undefined" ; -- postulate in agda

  Where e ld = usePrec 0 e ++ "where" ++ ld ;
  NoWhere e = usePrec 0 e ;

  Let ld e = mkPrec 0 ("let" ++ ld ++ "in" ++ (usePrec 0 e)) ;
  Split e lb = mkPrec 0 ("split@" ++ usePrec 0 e ++ "with" ++ lb) ;
  Lam pt e = mkPrec 0 ("\\" ++ pt ++ "->" ++ usePrec 0 e) ;
  Fun = infixr 1 "->" ; -- A -> Set
  Pi pt e = mkPrec 1 (pt ++ "->" ++ usePrec 1 e) ;
  Sigma pt e = mkPrec 1 (pt ++ "*" ++ usePrec 1 e) ;
  App = infixl 2 "" ;
  Id e1 e2 e3 = mkPrec 3 (usePrec 4 e1 ++ usePrec 4 e2 ++ "==" ++ usePrec 3 e3) ;
-- for an explicit vs implicit use of parameters, may have to use expressions as records, with a parameter is_implicit
  IdJ e1 e2 e3 e4 e5 = mkPrec 3 ("J" ++ usePrec 4 e1 ++ usePrec 4 e2 ++ usePrec 4 e3 ++ usePrec 4 e4 ++ usePrec 4 e5) ;
  Fst e = mkPrec 4 ("fst" ++ usePrec 4 e) ;
  Snd e = mkPrec 4 ("snd" ++ usePrec 4 e) ;
  Pair e1 e2 = mkPrec 5 ("(" ++ usePrec 0 e1 ++ "," ++ usePrec 0 e2 ++ ")") ;
  Var a = constant a ;
  Univ = constant "Set" ;
  Refl = "refl" ; -- constant "refl" ;

  BaseAIdent = "" ;
  ConsAIdent x xs = x ++ xs ;

  -- [Decl] only used in ExpWhere
  BaseDecl x = x ;
  ConsDecl x xs = x ++ "^" ++ xs ;

  -- maybe accomodate so split on empty type just gives () 
  -- BaseBranch = "" ;
  BaseBranch x = x ;
  -- ConsBranch x xs = x ++ "\n" ++ xs ;
  ConsBranch x xs = x ++ "||" ++ xs ;

  -- for data constructors
  BaseLabel x = x ;
  ConsLabel x xs = x ++ "|" ++ xs ; 

  BasePTele x = x ;
  ConsPTele x xs = x ++ xs ;

  BaseTele = "" ;
  ConsTele x xs = x ++ xs ;

  OBranch a la ew = a ++ la ++ "->" ++ ew ;
  TeleC a la e = "(" ++ a ++ la ++ ":" ++ usePrec 0 e ++ ")" ;
  PTeleC e1 e2 = "(" ++ top e1 ++ ":" ++ top e2 ++ ")" ;

  OLabel a lt = a ++ lt ;

  --object language syntax, all variables for now
  Bool = "bool" ;
  True = "true" ;
  False = "false" ;
  CaseBool = "caseBool" ;
  IndBool = "indBool" ;
  FunExt = "funExt" ;
  Nat = "nat" ;
  Zero = "zero" ;
  Suc = "suc" ;
  EqualNat = "equalNat" ;
  Unit = "unit" ;
  Top = "top" ;
  Foo = "foo" ; 
  A = "a" ;
  B = "b" ;
  C = "c" ;
  D = "d" ;
  E = "e" ;
  F = "f" ;
  G = "g" ;
  H = "h" ;
  I = "i" ;
  J = "j" ;
  K = "k" ;
  L = "l" ;
  M = "m" ;
  N = "n" ;
  O = "o" ;
  P = "p" ;
  Q = "q" ;
  R = "r" ;
  S = "s" ;
  T = "t" ;
  U = "u" ;
  V = "v" ;
  W = "w" ;
  X = "x" ;
  Y = "y" ;
  Z = "z" ;
  NegB = "negb" ;
  -- everything below is for contractible proofs
  Contr = "isContr" ;
  Fiber = "fiber" ;
  IsEquiv = "isEquiv" ;
  IdIsEquiv = "idIsEquiv" ;
  IdFun = "idfun" ;
  ContrSingl = "contrSingl" ;
  Equiv = "equiv" ;
  EqToIso = "eqToIso" ;
  Identity = "id" ;
  Ybar = "ybar"  ;
  IdFib = "idFib"  ;
  Lemma0 = "lemma0" ;
}
\end{verbatim}
The resource FormalTwo.gf merely substitutes more precedences than Formal.gf
from the RGL, in the ideal case that we could scale the grammar to include
larger and more complicated fixity information.
\begin{verbatim}
resource FormalTwo = open Prelude in {

----Everything the same up until the definition of Prec in Formal.gf


    Prec : PType = Predef.Ints 9 ;

    highest = 9 ;

    lessPrec : Prec -> Prec -> Bool = \p,q ->
      case <<p,q> : Prec * Prec> of {
        <3,9> | <2,9> | <4,9> | <5,9> | <6,9> | <7,9> | <8,9> => True ;
        <3,8> | <2,8> | <4,8> | <5,8> | <6,8> | <7,8> => True ;
        <3,7> | <2,7> | <4,7> | <5,7> | <6,7> => True ;
        <3,6> | <2,6> | <4,6> | <5,6> => True ;
        <3,5> | <2,5> | <4,5> => True ;
        <3,4> | <2,3> | <2,4> => True ;
        <1,1> | <1,0> | <0,0> => False ;
        <1,_> | <0,_>         => True ;
        _ => False
      } ;

    nextPrec : Prec -> Prec = \p -> case <p : Prec> of {
      9 => 9 ;
      n => Predef.plus n 1
      } ;
\end{verbatim}


\subsection{Hott and cubicalTT Grammars} \label{comparison}

\begin{code}[hide]%
\>[0]\AgdaSymbol{\{-\#}\AgdaSpace{}%
\AgdaKeyword{OPTIONS}\AgdaSpace{}%
\AgdaPragma{--omega-in-omega}\AgdaSpace{}%
\AgdaPragma{--type-in-type}\AgdaSpace{}%
\AgdaSymbol{\#-\}}\<%
\\
%
\\[\AgdaEmptyExtraSkip]%
\>[0]\AgdaKeyword{module}\AgdaSpace{}%
\AgdaModule{compare}\AgdaSpace{}%
\AgdaKeyword{where}\<%
\\
%
\\[\AgdaEmptyExtraSkip]%
\>[0]\AgdaKeyword{open}\AgdaSpace{}%
\AgdaKeyword{import}\AgdaSpace{}%
\AgdaModule{Agda.Builtin.Sigma}\AgdaSpace{}%
\AgdaKeyword{public}\<%
\\
%
\\[\AgdaEmptyExtraSkip]%
\>[0]\AgdaKeyword{variable}\<%
\\
\>[0][@{}l@{\AgdaIndent{0}}]%
\>[2]\AgdaGeneralizable{A}\AgdaSpace{}%
\AgdaGeneralizable{B}\AgdaSpace{}%
\AgdaSymbol{:}\AgdaSpace{}%
\AgdaPrimitive{Set}\<%
\\
%
\\[\AgdaEmptyExtraSkip]%
\>[0]\AgdaKeyword{data}\AgdaSpace{}%
\AgdaOperator{\AgdaDatatype{\AgdaUnderscore{}≡\AgdaUnderscore{}}}\AgdaSpace{}%
\AgdaSymbol{\{}\AgdaBound{A}\AgdaSpace{}%
\AgdaSymbol{:}\AgdaSpace{}%
\AgdaPrimitive{Set}\AgdaSymbol{\}}\AgdaSpace{}%
\AgdaSymbol{(}\AgdaBound{a}\AgdaSpace{}%
\AgdaSymbol{:}\AgdaSpace{}%
\AgdaBound{A}\AgdaSymbol{)}\AgdaSpace{}%
\AgdaSymbol{:}\AgdaSpace{}%
\AgdaBound{A}\AgdaSpace{}%
\AgdaSymbol{→}\AgdaSpace{}%
\AgdaPrimitive{Set}\AgdaSpace{}%
\AgdaKeyword{where}\<%
\\
\>[0][@{}l@{\AgdaIndent{0}}]%
\>[2]\AgdaInductiveConstructor{r}\AgdaSpace{}%
\AgdaSymbol{:}\AgdaSpace{}%
\AgdaBound{a}\AgdaSpace{}%
\AgdaOperator{\AgdaDatatype{≡}}\AgdaSpace{}%
\AgdaBound{a}\<%
\\
%
\\[\AgdaEmptyExtraSkip]%
\>[0]\AgdaKeyword{infix}\AgdaSpace{}%
\AgdaNumber{20}\AgdaSpace{}%
\AgdaOperator{\AgdaDatatype{\AgdaUnderscore{}≡\AgdaUnderscore{}}}\<%
\end{code}
\begin{code}%
\>[0]\AgdaFunction{id}\AgdaSpace{}%
\AgdaSymbol{:}\AgdaSpace{}%
\AgdaGeneralizable{A}\AgdaSpace{}%
\AgdaSymbol{→}\AgdaSpace{}%
\AgdaGeneralizable{A}\<%
\\
\>[0]\AgdaFunction{id}\AgdaSpace{}%
\AgdaSymbol{=}\AgdaSpace{}%
\AgdaSymbol{λ}\AgdaSpace{}%
\AgdaBound{z}\AgdaSpace{}%
\AgdaSymbol{→}\AgdaSpace{}%
\AgdaBound{z}\<%
\\
%
\\[\AgdaEmptyExtraSkip]%
\>[0]\AgdaFunction{iscontr}\AgdaSpace{}%
\AgdaSymbol{:}\AgdaSpace{}%
\AgdaSymbol{(}\AgdaBound{A}\AgdaSpace{}%
\AgdaSymbol{:}\AgdaSpace{}%
\AgdaPrimitive{Set}\AgdaSymbol{)}\AgdaSpace{}%
\AgdaSymbol{→}\AgdaSpace{}%
\AgdaPrimitive{Set}\<%
\\
\>[0]\AgdaFunction{iscontr}\AgdaSpace{}%
\AgdaBound{A}\AgdaSpace{}%
\AgdaSymbol{=}%
\>[13]\AgdaRecord{Σ}\AgdaSpace{}%
\AgdaBound{A}\AgdaSpace{}%
\AgdaSymbol{λ}\AgdaSpace{}%
\AgdaBound{a}\AgdaSpace{}%
\AgdaSymbol{→}\AgdaSpace{}%
\AgdaSymbol{(}\AgdaBound{x}\AgdaSpace{}%
\AgdaSymbol{:}\AgdaSpace{}%
\AgdaBound{A}\AgdaSymbol{)}\AgdaSpace{}%
\AgdaSymbol{→}\AgdaSpace{}%
\AgdaSymbol{(}\AgdaBound{a}\AgdaSpace{}%
\AgdaOperator{\AgdaDatatype{≡}}\AgdaSpace{}%
\AgdaBound{x}\AgdaSymbol{)}\<%
\\
%
\\[\AgdaEmptyExtraSkip]%
\>[0]\AgdaFunction{fiber}\AgdaSpace{}%
\AgdaSymbol{:}\AgdaSpace{}%
\AgdaSymbol{(}\AgdaBound{A}\AgdaSpace{}%
\AgdaBound{B}\AgdaSpace{}%
\AgdaSymbol{:}\AgdaSpace{}%
\AgdaPrimitive{Set}\AgdaSymbol{)}\AgdaSpace{}%
\AgdaSymbol{(}\AgdaBound{f}\AgdaSpace{}%
\AgdaSymbol{:}\AgdaSpace{}%
\AgdaBound{A}\AgdaSpace{}%
\AgdaSymbol{->}\AgdaSpace{}%
\AgdaBound{B}\AgdaSymbol{)}\AgdaSpace{}%
\AgdaSymbol{(}\AgdaBound{y}\AgdaSpace{}%
\AgdaSymbol{:}\AgdaSpace{}%
\AgdaBound{B}\AgdaSymbol{)}\AgdaSpace{}%
\AgdaSymbol{→}\AgdaSpace{}%
\AgdaPrimitive{Set}\<%
\\
\>[0]\AgdaFunction{fiber}\AgdaSpace{}%
\AgdaBound{A}\AgdaSpace{}%
\AgdaBound{B}\AgdaSpace{}%
\AgdaBound{f}\AgdaSpace{}%
\AgdaBound{y}\AgdaSpace{}%
\AgdaSymbol{=}\AgdaSpace{}%
\AgdaRecord{Σ}\AgdaSpace{}%
\AgdaBound{A}\AgdaSpace{}%
\AgdaSymbol{(λ}\AgdaSpace{}%
\AgdaBound{x}\AgdaSpace{}%
\AgdaSymbol{→}\AgdaSpace{}%
\AgdaBound{y}\AgdaSpace{}%
\AgdaOperator{\AgdaDatatype{≡}}\AgdaSpace{}%
\AgdaBound{f}\AgdaSpace{}%
\AgdaBound{x}\AgdaSymbol{)}\<%
\\
%
\\[\AgdaEmptyExtraSkip]%
\>[0]\AgdaFunction{isEquiv}\AgdaSpace{}%
\AgdaSymbol{:}\AgdaSpace{}%
\AgdaSymbol{(}\AgdaBound{A}\AgdaSpace{}%
\AgdaBound{B}\AgdaSpace{}%
\AgdaSymbol{:}\AgdaSpace{}%
\AgdaPrimitive{Set}\AgdaSymbol{)}\AgdaSpace{}%
\AgdaSymbol{→}\AgdaSpace{}%
\AgdaSymbol{(}\AgdaBound{f}\AgdaSpace{}%
\AgdaSymbol{:}\AgdaSpace{}%
\AgdaBound{A}\AgdaSpace{}%
\AgdaSymbol{→}\AgdaSpace{}%
\AgdaBound{B}\AgdaSymbol{)}\AgdaSpace{}%
\AgdaSymbol{→}\AgdaSpace{}%
\AgdaPrimitive{Set}\<%
\\
\>[0]\AgdaFunction{isEquiv}\AgdaSpace{}%
\AgdaBound{A}\AgdaSpace{}%
\AgdaBound{B}\AgdaSpace{}%
\AgdaBound{f}\AgdaSpace{}%
\AgdaSymbol{=}\AgdaSpace{}%
\AgdaSymbol{(}\AgdaBound{y}\AgdaSpace{}%
\AgdaSymbol{:}\AgdaSpace{}%
\AgdaBound{B}\AgdaSymbol{)}\AgdaSpace{}%
\AgdaSymbol{→}\AgdaSpace{}%
\AgdaFunction{iscontr}\AgdaSpace{}%
\AgdaSymbol{(}\AgdaFunction{fiber}\AgdaSpace{}%
\AgdaBound{A}\AgdaSpace{}%
\AgdaBound{B}\AgdaSpace{}%
\AgdaBound{f}\AgdaSpace{}%
\AgdaBound{y}\AgdaSymbol{)}\<%
\\
%
\\[\AgdaEmptyExtraSkip]%
\>[0]\AgdaFunction{isEquiv'}\AgdaSpace{}%
\AgdaSymbol{:}\AgdaSpace{}%
\AgdaSymbol{(}\AgdaBound{A}\AgdaSpace{}%
\AgdaBound{B}\AgdaSpace{}%
\AgdaSymbol{:}\AgdaSpace{}%
\AgdaPrimitive{Set}\AgdaSymbol{)}\AgdaSpace{}%
\AgdaSymbol{→}\AgdaSpace{}%
\AgdaSymbol{(}\AgdaBound{f}\AgdaSpace{}%
\AgdaSymbol{:}\AgdaSpace{}%
\AgdaBound{A}\AgdaSpace{}%
\AgdaSymbol{→}\AgdaSpace{}%
\AgdaBound{B}\AgdaSymbol{)}\AgdaSpace{}%
\AgdaSymbol{→}\AgdaSpace{}%
\AgdaPrimitive{Set}\<%
\\
\>[0]\AgdaFunction{isEquiv'}\AgdaSpace{}%
\AgdaBound{A}\AgdaSpace{}%
\AgdaBound{B}\AgdaSpace{}%
\AgdaBound{f}\AgdaSpace{}%
\AgdaSymbol{=}\AgdaSpace{}%
\AgdaSymbol{∀}\AgdaSpace{}%
\AgdaSymbol{(}\AgdaBound{y}\AgdaSpace{}%
\AgdaSymbol{:}\AgdaSpace{}%
\AgdaBound{B}\AgdaSymbol{)}\AgdaSpace{}%
\AgdaSymbol{→}\AgdaSpace{}%
\AgdaFunction{iscontr}\AgdaSpace{}%
\AgdaSymbol{(}\AgdaFunction{fiber'}\AgdaSpace{}%
\AgdaBound{y}\AgdaSymbol{)}\<%
\\
\>[0][@{}l@{\AgdaIndent{0}}]%
\>[2]\AgdaKeyword{where}\<%
\\
\>[2][@{}l@{\AgdaIndent{0}}]%
\>[4]\AgdaFunction{fiber'}\AgdaSpace{}%
\AgdaSymbol{:}\AgdaSpace{}%
\AgdaSymbol{(}\AgdaBound{y}\AgdaSpace{}%
\AgdaSymbol{:}\AgdaSpace{}%
\AgdaBound{B}\AgdaSymbol{)}\AgdaSpace{}%
\AgdaSymbol{→}\AgdaSpace{}%
\AgdaPrimitive{Set}\<%
\\
%
\>[4]\AgdaFunction{fiber'}\AgdaSpace{}%
\AgdaBound{y}\AgdaSpace{}%
\AgdaSymbol{=}\AgdaSpace{}%
\AgdaRecord{Σ}\AgdaSpace{}%
\AgdaBound{A}\AgdaSpace{}%
\AgdaSymbol{(λ}\AgdaSpace{}%
\AgdaBound{x}\AgdaSpace{}%
\AgdaSymbol{→}\AgdaSpace{}%
\AgdaBound{y}\AgdaSpace{}%
\AgdaOperator{\AgdaDatatype{≡}}\AgdaSpace{}%
\AgdaBound{f}\AgdaSpace{}%
\AgdaBound{x}\AgdaSymbol{)}\<%
\\
%
\\[\AgdaEmptyExtraSkip]%
\>[0]\AgdaComment{-- proof from Aarne}\<%
\\
\>[0]\AgdaFunction{idIsEquiv'}\AgdaSpace{}%
\AgdaSymbol{:}\AgdaSpace{}%
\AgdaSymbol{(}\AgdaBound{A}\AgdaSpace{}%
\AgdaSymbol{:}\AgdaSpace{}%
\AgdaPrimitive{Set}\AgdaSymbol{)}\AgdaSpace{}%
\AgdaSymbol{→}\AgdaSpace{}%
\AgdaFunction{isEquiv}\AgdaSpace{}%
\AgdaBound{A}\AgdaSpace{}%
\AgdaBound{A}\AgdaSpace{}%
\AgdaSymbol{(}\AgdaFunction{id}\AgdaSpace{}%
\AgdaSymbol{\{}\AgdaBound{A}\AgdaSymbol{\})}\<%
\\
\>[0]\AgdaFunction{idIsEquiv'}\AgdaSpace{}%
\AgdaBound{A}\AgdaSpace{}%
\AgdaBound{y}\AgdaSpace{}%
\AgdaSymbol{=}\AgdaSpace{}%
\AgdaFunction{ybar}\AgdaSpace{}%
\AgdaOperator{\AgdaInductiveConstructor{,}}\AgdaSpace{}%
\AgdaFunction{help}\<%
\\
\>[0][@{}l@{\AgdaIndent{0}}]%
\>[2]\AgdaKeyword{where}\<%
\\
\>[2][@{}l@{\AgdaIndent{0}}]%
\>[4]\AgdaFunction{fib'}\AgdaSpace{}%
\AgdaSymbol{:}\AgdaSpace{}%
\AgdaPrimitive{Set}\AgdaSpace{}%
\AgdaComment{-- \{y : A\}}\<%
\\
%
\>[4]\AgdaFunction{fib'}\AgdaSpace{}%
\AgdaSymbol{=}\AgdaSpace{}%
\AgdaFunction{fiber}\AgdaSpace{}%
\AgdaBound{A}\AgdaSpace{}%
\AgdaBound{A}\AgdaSpace{}%
\AgdaFunction{id}\AgdaSpace{}%
\AgdaBound{y}\<%
\\
%
\>[4]\AgdaFunction{ybar}\AgdaSpace{}%
\AgdaSymbol{:}\AgdaSpace{}%
\AgdaFunction{fib'}\<%
\\
%
\>[4]\AgdaFunction{ybar}\AgdaSpace{}%
\AgdaSymbol{=}\AgdaSpace{}%
\AgdaBound{y}\AgdaSpace{}%
\AgdaOperator{\AgdaInductiveConstructor{,}}\AgdaSpace{}%
\AgdaInductiveConstructor{r}\<%
\\
%
\>[4]\AgdaFunction{help}\AgdaSpace{}%
\AgdaSymbol{:}\AgdaSpace{}%
\AgdaSymbol{(}\AgdaBound{x}\AgdaSpace{}%
\AgdaSymbol{:}\AgdaSpace{}%
\AgdaFunction{fib'}\AgdaSymbol{)}\AgdaSpace{}%
\AgdaSymbol{→}\AgdaSpace{}%
\AgdaOperator{\AgdaDatatype{\AgdaUnderscore{}≡\AgdaUnderscore{}}}\AgdaSpace{}%
\AgdaSymbol{\{}\AgdaRecord{Σ}\AgdaSpace{}%
\AgdaBound{A}\AgdaSpace{}%
\AgdaSymbol{(}\AgdaOperator{\AgdaDatatype{\AgdaUnderscore{}≡\AgdaUnderscore{}}}\AgdaSpace{}%
\AgdaBound{y}\AgdaSymbol{)\}}\AgdaSpace{}%
\AgdaFunction{ybar}\AgdaSpace{}%
\AgdaBound{x}\<%
\\
%
\>[4]\AgdaFunction{help}\AgdaSpace{}%
\AgdaSymbol{=}\AgdaSpace{}%
\AgdaSymbol{λ}\AgdaSpace{}%
\AgdaSymbol{\{(}\AgdaBound{a}\AgdaSpace{}%
\AgdaOperator{\AgdaInductiveConstructor{,}}\AgdaSpace{}%
\AgdaInductiveConstructor{r}\AgdaSymbol{)}\AgdaSpace{}%
\AgdaSymbol{→}\AgdaSpace{}%
\AgdaInductiveConstructor{r}\AgdaSymbol{\}}\<%
\\
%
\\[\AgdaEmptyExtraSkip]%
\>[0]\AgdaFunction{equiv}\AgdaSpace{}%
\AgdaSymbol{:}\AgdaSpace{}%
\AgdaSymbol{(}\AgdaSpace{}%
\AgdaBound{a}\AgdaSpace{}%
\AgdaBound{b}\AgdaSpace{}%
\AgdaSymbol{:}\AgdaSpace{}%
\AgdaPrimitive{Set}\AgdaSpace{}%
\AgdaSymbol{)}\AgdaSpace{}%
\AgdaSymbol{→}\AgdaSpace{}%
\AgdaPrimitive{Set}\<%
\\
\>[0]\AgdaFunction{equiv}\AgdaSpace{}%
\AgdaBound{a}\AgdaSpace{}%
\AgdaBound{b}\AgdaSpace{}%
\AgdaSymbol{=}\AgdaSpace{}%
\AgdaRecord{Σ}\AgdaSpace{}%
\AgdaSymbol{(}\AgdaBound{a}\AgdaSpace{}%
\AgdaSymbol{→}\AgdaSpace{}%
\AgdaBound{b}\AgdaSymbol{)}\AgdaSpace{}%
\AgdaSymbol{λ}\AgdaSpace{}%
\AgdaBound{f}\AgdaSpace{}%
\AgdaSymbol{→}\AgdaSpace{}%
\AgdaFunction{isEquiv}\AgdaSpace{}%
\AgdaBound{a}\AgdaSpace{}%
\AgdaBound{b}\AgdaSpace{}%
\AgdaBound{f}\<%
\\
%
\\[\AgdaEmptyExtraSkip]%
\>[0]\AgdaFunction{equivId}\AgdaSpace{}%
\AgdaSymbol{:}\AgdaSpace{}%
\AgdaSymbol{(}\AgdaBound{x}\AgdaSpace{}%
\AgdaSymbol{:}\AgdaSpace{}%
\AgdaPrimitive{Set}\AgdaSymbol{)}\AgdaSpace{}%
\AgdaSymbol{→}\AgdaSpace{}%
\AgdaFunction{equiv}\AgdaSpace{}%
\AgdaBound{x}\AgdaSpace{}%
\AgdaBound{x}\<%
\\
\>[0]\AgdaFunction{equivId}\AgdaSpace{}%
\AgdaBound{x}\AgdaSpace{}%
\AgdaSymbol{=}\AgdaSpace{}%
\AgdaFunction{id}\AgdaSpace{}%
\AgdaOperator{\AgdaInductiveConstructor{,}}\AgdaSpace{}%
\AgdaSymbol{(}\AgdaFunction{idIsEquiv'}\AgdaSpace{}%
\AgdaBound{x}\AgdaSymbol{)}\<%
\\
%
\\[\AgdaEmptyExtraSkip]%
\>[0]\AgdaFunction{eqToIso}\AgdaSpace{}%
\AgdaSymbol{:}\AgdaSpace{}%
\AgdaSymbol{(}\AgdaSpace{}%
\AgdaBound{a}\AgdaSpace{}%
\AgdaBound{b}\AgdaSpace{}%
\AgdaSymbol{:}\AgdaSpace{}%
\AgdaPrimitive{Set}\AgdaSpace{}%
\AgdaSymbol{)}\AgdaSpace{}%
\AgdaSymbol{→}\AgdaSpace{}%
\AgdaOperator{\AgdaDatatype{\AgdaUnderscore{}≡\AgdaUnderscore{}}}\AgdaSpace{}%
\AgdaSymbol{\{}\AgdaPrimitive{Set}\AgdaSymbol{\}}\AgdaSpace{}%
\AgdaBound{a}\AgdaSpace{}%
\AgdaBound{b}\AgdaSpace{}%
\AgdaSymbol{→}\AgdaSpace{}%
\AgdaFunction{equiv}\AgdaSpace{}%
\AgdaBound{a}\AgdaSpace{}%
\AgdaBound{b}\<%
\\
\>[0]\AgdaFunction{eqToIso}\AgdaSpace{}%
\AgdaBound{a}\AgdaSpace{}%
\AgdaDottedPattern{\AgdaSymbol{.}}\AgdaDottedPattern{a}\AgdaSpace{}%
\AgdaInductiveConstructor{r}\AgdaSpace{}%
\AgdaSymbol{=}\AgdaSpace{}%
\AgdaFunction{equivId}\AgdaSpace{}%
\AgdaBound{a}\<%
\end{code}

Compared with the latex code

\begin{figure}[H]
 \textbf{Definition}:
 A type $A$ is contractible, if there is $a : A$, called the center of contraction, such that for all $x : A$, $\equalH {a}{x}$.

 \textbf{Definition}:
 A map $f : \arrowH {A}{B}$ is an equivalence, if for all $y : B$, its fiber, $\comprehensionH {x}{A}{\equalH {\appH {f}{x}}{y}}$, is contractible.
 We write $\equivalenceH {A}{B}$, if there is an equivalence $\arrowH {A}{B}$.

 \textbf{Lemma}:
 For each type $A$, the identity map, $\defineH {\idMapH {A}}{\typingH {\lambdaH {x}{A}{x}}{\arrowH {A}{A}}}$, is an equivalence.

 \textbf{Proof}:
 For each $y : A$, let $\defineH {\fiberH {y}{A}}{\comprehensionH {x}{A}{\equalH {x}{y}}}$ be its fiber with respect to $\idMapH {A}$ and let $\defineH {\barH {y}}{\typingH {\pairH {y}{\reflexivityH {A}{y}}}{\fiberH {y}{A}}}$.
 As for all $y : A$, $\equalH {\pairH {y}{\reflexivityH {A}{y}}}{y}$, we may apply Id-induction on $y$, $\typingH {x}{A}$ and $\typingH {z}{\idPropH {x}{y}}$ to get that \[\equalH {\pairH {x}{z}}{y}\].
 Hence, for $y : A$, we may apply $\Sigma$ -elimination on $\typingH {u}{\fiberH {y}{A}}$ to get that $\equalH {u}{y}$, so that $\fiberH {y}{A}$ is contractible.
 Thus, $\typingH {\idMapH {A}}{\arrowH {A}{A}}$ is an equivalence.
  $\Box$

 \textbf{Corollary}:
 If $U$ is a type universe, then, for $X , Y : U$, \[(*)\arrowH {\equalH {X}{Y}}{\equivalenceH {X}{Y}}\].

 \textbf{Proof}:
 We may apply the lemma to get that for $X : U$, $\equivalenceH {X}{X}$.
 Hence, we may apply Id-induction on $\typingH {X , Y}{U}$ to get that $(*)$.
  $\Box$


 \textbf{Definition}:
 A type universe $U$ is univalent, if for $X , Y : U$, the map $\equivalenceMapH {X}{Y}: \arrowH {\equalH {X}{Y}}{\equivalenceH {X}{Y}}$ in $(*)$ is an equivalence.
\end{figure}

cubicalTT parses the following.  We note an idealization : that agda supports ananymous pattern matching, so 
\codeword{\\ ( ( b , refl )}  would not typecheck in the original cubicalTT. Additionally, the reflexivity constructor is only present when the identity is inductively defined, as it is a primitive in cubical type theories.

\begin{figure}[H]
\begin{verbatim}
id ( a : Set ) : a -> a = \\ ( b : a ) -> b
isContr ( a : Set ) : Set = ( b : a ) * ( x : a ) -> a b == x
fiber ( a b  : Set ) ( f : a -> b ) ( y : b )  : Set 
  = ( x : a ) * ( x : a ) -> b y == ( f x )
isEquiv ( a b  : Set ) ( f : a -> b )   : Set 
  = ( y : b ) -> isContr ( fiber a b f y )
  where fiber ( a b  : Set ) ( f : a -> b ) ( y : b )  : Set 
    = ( x : a ) * ( x : a ) -> b y == ( f x )
equiv ( a b : Set ) : Set = ( f : a -> b ) * isEquiv a b f

idIsEquiv ( a : Set ) : isEquiv a a ( id a ) =  ( ybar , lemma0 )
  where
    idFib : Set = fiber a a id y
    ^ ybar : idFib = ( y , refl )
    ^ lemma0 ( x : idFib ) : ( ( p ) ybar == x ) 
      = \\ ( ( b , refl ) : ( c : a ) * ( a c == c ) ) -> refl

idIsEquiv ( x : Set ) : equiv x x = ( id , idIsEquiv x )
eqToIso ( a b : Set ) : ( Set a == b ) -> equiv a b 
  = split refl -> idIsEquiv a
\end{verbatim}
\end{figure}

We compare two abstract syntax trees side by side to show that they have quite different structures,

\begin{figure}
\centering
\begin{minipage}[t]{.5\textwidth}
\begin{verbatim}
Exp> 
* DeclDef
    * Contr
      ConsTele
        * TeleC
            * A
              BaseAIdent
              Univ
          BaseTele
      Univ
      NoWhere
        * Sigma
            * BasePTele
                * PTeleC
                    * Var
                        * B
                      Var
                        * A
              Pi
                * BasePTele
                    * PTeleC
                        * Var
                            * X
                          Var
                            * A
                  Id
                    * Var
                        * A
                      Var
                        * B
                      Var
                        * X
\end{verbatim}
\end{minipage}%
\begin{minipage}[t]{.55\textwidth}
\begin{verbatim}
* PredDefinition
    * type_Sort
      A_Var
      contractible_Pred
      ExistCalledProp
        * a_Var
          ExpSort
            * VarExp
                * A_Var
          FunInd
            * centre_of_contraction_Fun
          ForAllProp
            * allUnivPhrase
                * BaseVar
                    * x_Var
                  ExpSort
                    * VarExp
                        * A_Var
              ExpProp
                * DollarMathEnv
                  equalExp
                    * VarExp
                        * a_Var
                      VarExp
                        * x_Var
\end{verbatim}
\end{minipage}
\caption{Mathematical Assertions and Agda Judgements} \label{fig:I2}
\end{figure}

What we notice : 


\begin{figure}
\centering
\begin{minipage}[t]{.5\textwidth}
\begin{verbatim}
* DeclSplit
    * EqToIso
      ConsTele
        * TeleC
            * A
              ConsAIdent
                * B
                  BaseAIdent
              Univ
          BaseTele
      Fun
        * Id
            * Univ
              Var
                * A
              Var
                * B
          App
            * App
                * Var
                    * Equiv
                  Var
                    * A
              Var
                * B
      BaseBranch
        * OBranch
            * Refl
              BaseAIdent
              NoWhere
                * App
                    * Var
                        * IdIsEquiv
                      Var
                        * A
\end{verbatim}
\end{minipage}%
\begin{minipage}[t]{.55\textwidth}
\begin{verbatim}
3 PropParagraph
    * NoConclusionPhrase
      ForAllProp
        * if_thenUnivPhrase
            * BaseVar
                * U_Var
              type_universe_Sort
          ForAllProp
            * plainUnivPhrase
                * ConsVar
                    * X_Var
                      BaseVar
                        * Y_Var
                  ExpSort
                    * VarExp
                        * U_Var
              LabelledExpProp
                * DisplayMathEnv
                  StarLabel
                  mapExp
                    * equalExp
                        * VarExp
                            * X_Var
                          VarExp
                            * Y_Var
                      equivalenceExp
                        * VarExp
                            * X_Var
                          VarExp
                            * Y_Var
4 ConclusionParagraph
    1 NoConclusionPhrase
      ApplyLabelConclusion
        * the_lemma_Label
          BaseInd
          ForAllProp
            * plainUnivPhrase
                * BaseVar
                    * X_Var
                  ExpSort
                    * VarExp
                        * U_Var
              ExpProp
                * DollarMathEnv
                  equivalenceExp
                    * VarExp
                        * X_Var
                      VarExp
                        * X_Var
    2 henceConclusionPhrase
      ApplyLabelConclusion
        * id_induction_Label
          ConsInd
            * FunInd
                * ExpFun
                    * TypedExp
                        * ConsExp
                            * VarExp
                                * X_Var
                              BaseExp
                                * VarExp
                                    * Y_Var
                          VarExp
                            * U_Var
              BaseInd
          LabelProp
            * StarLabel
\end{verbatim}
\end{minipage}
\caption{Mathematical Assertions and Agda Judgements} \label{fig:I3}
\end{figure}

todo : refactor to have the final sections side-by-side, do a more "thorough analysis of the text fragment above"
namely - look at the redundancy, the intro of identity local to a definition (often having more than one proposition in a proposition) 
the failure in some instances to provide relevant info, etc.

also, refactor to have the sigma proof here

Exp> * DeclDef
    * IdIsEquiv
      ConsTele
        * TeleC
            * X
              BaseAIdent
              Univ
          BaseTele
      App
        * App
            * Var
                * Equiv
              Var
                * X
          Var
            * X
      NoWhere
        * Pair
            * Var
                * Identity
              App
                * Var
                    * IdIsEquiv
                  Var
                    * X





\subsection{HoTT Agda Corpus} \label{hottproofs}