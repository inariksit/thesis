\section{Appendix} \label{appendix}



\subsection{What is a Homomorphism?}

To get a feel for the syntactic paradigm we explore in this thesis, let us look at a basic mathematical
example: that of a group homomorphism as expressed in by a variety of somewhat
randomly sampled authors.  

% Wikipedia Defn:

\begin{definition}
In mathematics, given two groups, $(G, \ast)$ and $(H, \cdot)$, a group homomorphism from $(G, \ast)$ to $(H, \cdot)$ is a function $h : G \to H$ such that for all $u$ and $v$ in $G$ it holds that
  $$h(u \ast v) = h ( u ) \cdot h ( v )$$ 
\end{definition}

% http://math.mit.edu/~jwellens/Group%20Theory%20Forum.pdf

\begin{definition}
Let $G = (G,\cdot)$ and $G' = (G',\ast)$ be groups, and let $\phi : G \to G'$ be a map between them. We call $\phi$ a \textbf{homomorphism} if for every pair of elements $g, h \in G$, we have 
% \begin{center}
  $$\phi(g \ast h) = \phi ( g ) \cdot \phi ( h )$$ 
% \end{center}
\end{definition}

% http://www.maths.gla.ac.uk/~mwemyss/teaching/3alg1-7.pdf

\begin{definition}\label{def:def3}
Let $G$, $H$, be groups.  A map $\phi : G \to H$ is called a \emph{group homomorphism} if
  $$\phi(xy) = \phi ( x ) \phi ( y )\ for\ all\ x, y \in G$$ 
(Note that $xy$ on the left is formed using the group operation in $G$, whilst the product $\phi ( x ) \phi ( y )$ is formed using the group operation $H$.)
\end{definition}

% NLab:

\begin{definition}\label{def:def4}
Classically, a group is a monoid in which every element has an inverse (necessarily unique).
\end{definition}

We inquire the reader to pay attention to nuance and difference in presentation
that is normally ignored or taken for granted by the fluent mathematician, ask
which definitions feel better, and how the reader herself might present the
definition differently.

If one wants to distill the meaning of each of these presentations, there is a
significant amount of subliminal interpretation happening very much analogous to
our innate linguistic usage. The inverse and identity are discarded, even though
they are necessary data when defining a group. The order of presenting the
information is inconsistent, as well as the choice to use symbolic or natural
language information. In Definition~\ref{def:def3}, the group operation is used
implicitly, and its clarification a side remark. Details aside, these all mean
the same thing - don't they?


We now show yet another definition of a group homomorphism formalized in the
Agda programming language:

\begin{code}[hide]%
\>[0]\AgdaComment{--\{-\# OPTIONS --cubical \#-\}}\<%
\\
\>[0]\AgdaSymbol{\{-\#}\AgdaSpace{}%
\AgdaKeyword{OPTIONS}\AgdaSpace{}%
\AgdaPragma{--cubical}\AgdaSpace{}%
\AgdaPragma{--no-import-sorts}\AgdaSpace{}%
\AgdaPragma{--safe}\AgdaSpace{}%
\AgdaSymbol{\#-\}}\<%
\\
%
\\[\AgdaEmptyExtraSkip]%
\>[0]\AgdaKeyword{module}\AgdaSpace{}%
\AgdaModule{monoid}\AgdaSpace{}%
\AgdaKeyword{where}\<%
\\
%
\\[\AgdaEmptyExtraSkip]%
\>[0]\AgdaKeyword{module}\AgdaSpace{}%
\AgdaModule{Namespace1}\AgdaSpace{}%
\AgdaKeyword{where}\<%
\\
%
\\[\AgdaEmptyExtraSkip]%
\>[0][@{}l@{\AgdaIndent{0}}]%
\>[2]\AgdaKeyword{open}\AgdaSpace{}%
\AgdaKeyword{import}\AgdaSpace{}%
\AgdaModule{Cubical.Foundations.Prelude}\<%
\\
%
\>[2]\AgdaKeyword{open}\AgdaSpace{}%
\AgdaKeyword{import}\AgdaSpace{}%
\AgdaModule{Cubical.Foundations.Equiv}\<%
\\
%
\>[2]\AgdaKeyword{open}\AgdaSpace{}%
\AgdaKeyword{import}\AgdaSpace{}%
\AgdaModule{Cubical.Foundations.Structure}\<%
\\
%
\>[2]\AgdaKeyword{open}\AgdaSpace{}%
\AgdaKeyword{import}\AgdaSpace{}%
\AgdaModule{Cubical.Algebra.Group.Base}\<%
\\
%
\>[2]\AgdaKeyword{open}\AgdaSpace{}%
\AgdaKeyword{import}\AgdaSpace{}%
\AgdaModule{Cubical.Data.Sigma}\<%
\\
%
\\[\AgdaEmptyExtraSkip]%
%
\>[2]\AgdaKeyword{private}\<%
\\
\>[2][@{}l@{\AgdaIndent{0}}]%
\>[4]\AgdaKeyword{variable}\<%
\\
\>[4][@{}l@{\AgdaIndent{0}}]%
\>[6]\AgdaGeneralizable{ℓ}\AgdaSpace{}%
\AgdaGeneralizable{ℓ'}\AgdaSpace{}%
\AgdaGeneralizable{ℓ''}\AgdaSpace{}%
\AgdaGeneralizable{ℓ'''}\AgdaSpace{}%
\AgdaSymbol{:}\AgdaSpace{}%
\AgdaPostulate{Level}\<%
\end{code}
\begin{code}%
%
\>[2]\AgdaFunction{isGroupHom}\AgdaSpace{}%
\AgdaSymbol{:}\AgdaSpace{}%
\AgdaSymbol{(}\AgdaBound{G}\AgdaSpace{}%
\AgdaSymbol{:}\AgdaSpace{}%
\AgdaFunction{Group}\AgdaSpace{}%
\AgdaSymbol{\{}\AgdaGeneralizable{ℓ}\AgdaSymbol{\})}\AgdaSpace{}%
\AgdaSymbol{(}\AgdaBound{H}\AgdaSpace{}%
\AgdaSymbol{:}\AgdaSpace{}%
\AgdaFunction{Group}\AgdaSpace{}%
\AgdaSymbol{\{}\AgdaGeneralizable{ℓ'}\AgdaSymbol{\})}\AgdaSpace{}%
\AgdaSymbol{(}\AgdaBound{f}\AgdaSpace{}%
\AgdaSymbol{:}\AgdaSpace{}%
\AgdaOperator{\AgdaFunction{⟨}}\AgdaSpace{}%
\AgdaBound{G}\AgdaSpace{}%
\AgdaOperator{\AgdaFunction{⟩}}\AgdaSpace{}%
\AgdaSymbol{→}\AgdaSpace{}%
\AgdaOperator{\AgdaFunction{⟨}}\AgdaSpace{}%
\AgdaBound{H}\AgdaSpace{}%
\AgdaOperator{\AgdaFunction{⟩}}\AgdaSymbol{)}\AgdaSpace{}%
\AgdaSymbol{→}\AgdaSpace{}%
\AgdaPrimitive{Type}\AgdaSpace{}%
\AgdaSymbol{\AgdaUnderscore{}}\<%
\\
%
\>[2]\AgdaFunction{isGroupHom}\AgdaSpace{}%
\AgdaBound{G}\AgdaSpace{}%
\AgdaBound{H}\AgdaSpace{}%
\AgdaBound{f}\AgdaSpace{}%
\AgdaSymbol{=}\AgdaSpace{}%
\AgdaSymbol{(}\AgdaBound{x}\AgdaSpace{}%
\AgdaBound{y}\AgdaSpace{}%
\AgdaSymbol{:}\AgdaSpace{}%
\AgdaOperator{\AgdaFunction{⟨}}\AgdaSpace{}%
\AgdaBound{G}\AgdaSpace{}%
\AgdaOperator{\AgdaFunction{⟩}}\AgdaSymbol{)}\AgdaSpace{}%
\AgdaSymbol{→}\AgdaSpace{}%
\AgdaBound{f}\AgdaSpace{}%
\AgdaSymbol{(}\AgdaBound{x}\AgdaSpace{}%
\AgdaOperator{\AgdaFunction{G.+}}\AgdaSpace{}%
\AgdaBound{y}\AgdaSymbol{)}\AgdaSpace{}%
\AgdaOperator{\AgdaFunction{≡}}\AgdaSpace{}%
\AgdaSymbol{(}\AgdaBound{f}\AgdaSpace{}%
\AgdaBound{x}\AgdaSpace{}%
\AgdaOperator{\AgdaFunction{H.+}}\AgdaSpace{}%
\AgdaBound{f}\AgdaSpace{}%
\AgdaBound{y}\AgdaSymbol{)}\AgdaSpace{}%
\AgdaKeyword{where}\<%
\\
\>[2][@{}l@{\AgdaIndent{0}}]%
\>[4]\AgdaKeyword{module}\AgdaSpace{}%
\AgdaModule{G}\AgdaSpace{}%
\AgdaSymbol{=}\AgdaSpace{}%
\AgdaModule{GroupStr}\AgdaSpace{}%
\AgdaSymbol{(}\AgdaField{snd}\AgdaSpace{}%
\AgdaBound{G}\AgdaSymbol{)}\<%
\\
%
\>[4]\AgdaKeyword{module}\AgdaSpace{}%
\AgdaModule{H}\AgdaSpace{}%
\AgdaSymbol{=}\AgdaSpace{}%
\AgdaModule{GroupStr}\AgdaSpace{}%
\AgdaSymbol{(}\AgdaField{snd}\AgdaSpace{}%
\AgdaBound{H}\AgdaSymbol{)}\<%
\\
%
\\[\AgdaEmptyExtraSkip]%
%
\>[2]\AgdaKeyword{record}\AgdaSpace{}%
\AgdaRecord{GroupHom}\AgdaSpace{}%
\AgdaSymbol{(}\AgdaBound{G}\AgdaSpace{}%
\AgdaSymbol{:}\AgdaSpace{}%
\AgdaFunction{Group}\AgdaSpace{}%
\AgdaSymbol{\{}\AgdaGeneralizable{ℓ}\AgdaSymbol{\})}\AgdaSpace{}%
\AgdaSymbol{(}\AgdaBound{H}\AgdaSpace{}%
\AgdaSymbol{:}\AgdaSpace{}%
\AgdaFunction{Group}\AgdaSpace{}%
\AgdaSymbol{\{}\AgdaGeneralizable{ℓ'}\AgdaSymbol{\})}\AgdaSpace{}%
\AgdaSymbol{:}\AgdaSpace{}%
\AgdaPrimitive{Type}\AgdaSpace{}%
\AgdaSymbol{(}\AgdaPrimitive{ℓ-max}\AgdaSpace{}%
\AgdaBound{ℓ}\AgdaSpace{}%
\AgdaBound{ℓ'}\AgdaSymbol{)}\AgdaSpace{}%
\AgdaKeyword{where}\<%
\\
\>[2][@{}l@{\AgdaIndent{0}}]%
\>[4]\AgdaKeyword{constructor}\AgdaSpace{}%
\AgdaInductiveConstructor{grouphom}\<%
\\
%
\\[\AgdaEmptyExtraSkip]%
%
\>[4]\AgdaKeyword{field}\<%
\\
\>[4][@{}l@{\AgdaIndent{0}}]%
\>[6]\AgdaField{fun}\AgdaSpace{}%
\AgdaSymbol{:}\AgdaSpace{}%
\AgdaOperator{\AgdaFunction{⟨}}\AgdaSpace{}%
\AgdaBound{G}\AgdaSpace{}%
\AgdaOperator{\AgdaFunction{⟩}}\AgdaSpace{}%
\AgdaSymbol{→}\AgdaSpace{}%
\AgdaOperator{\AgdaFunction{⟨}}\AgdaSpace{}%
\AgdaBound{H}\AgdaSpace{}%
\AgdaOperator{\AgdaFunction{⟩}}\<%
\\
%
\>[6]\AgdaField{isHom}\AgdaSpace{}%
\AgdaSymbol{:}\AgdaSpace{}%
\AgdaFunction{isGroupHom}\AgdaSpace{}%
\AgdaBound{G}\AgdaSpace{}%
\AgdaBound{H}\AgdaSpace{}%
\AgdaField{fun}\<%
\end{code}
This actually \emph{was} the Cubical Agda implementation of a group homomorphism
sometime around the end of 2020. We see that, while a mathematician might be
able to infer the meaning of some of the syntax, the use of levels,
distinguising between isGroupHom and GroupHom itself, and many other details
might obscure what's going on.

We finally provide the current (May 2021) definition via Cubical Agda. One may
witness a significant number of differences from the previous version : concrete
syntax differences via changes in camel case, new uses of Group vs GroupStr, as
well as, most significantly, the identity and inverse preservation data not
appearing as corollaries, but part of the definition. Additionally, we had to
refactor the commented lines to those shown below to be compatible with our
outdated version of cubical. These changes reflect interesting syntactic
changes.

\begin{code}%
%
\>[2]\AgdaKeyword{record}\AgdaSpace{}%
\AgdaRecord{IsGroupHom}\AgdaSpace{}%
\AgdaSymbol{\{}\AgdaBound{A}\AgdaSpace{}%
\AgdaSymbol{:}\AgdaSpace{}%
\AgdaPrimitive{Type}\AgdaSpace{}%
\AgdaGeneralizable{ℓ}\AgdaSymbol{\}}\AgdaSpace{}%
\AgdaSymbol{\{}\AgdaBound{B}\AgdaSpace{}%
\AgdaSymbol{:}\AgdaSpace{}%
\AgdaPrimitive{Type}\AgdaSpace{}%
\AgdaGeneralizable{ℓ'}\AgdaSymbol{\}}\<%
\\
\>[2][@{}l@{\AgdaIndent{0}}]%
\>[4]\AgdaSymbol{(}\AgdaBound{M}\AgdaSpace{}%
\AgdaSymbol{:}\AgdaSpace{}%
\AgdaRecord{GroupStr}\AgdaSpace{}%
\AgdaBound{A}\AgdaSymbol{)}\AgdaSpace{}%
\AgdaSymbol{(}\AgdaBound{f}\AgdaSpace{}%
\AgdaSymbol{:}\AgdaSpace{}%
\AgdaBound{A}\AgdaSpace{}%
\AgdaSymbol{→}\AgdaSpace{}%
\AgdaBound{B}\AgdaSymbol{)}\AgdaSpace{}%
\AgdaSymbol{(}\AgdaBound{N}\AgdaSpace{}%
\AgdaSymbol{:}\AgdaSpace{}%
\AgdaRecord{GroupStr}\AgdaSpace{}%
\AgdaBound{B}\AgdaSymbol{)}\<%
\\
%
\>[4]\AgdaSymbol{:}\AgdaSpace{}%
\AgdaPrimitive{Type}\AgdaSpace{}%
\AgdaSymbol{(}\AgdaPrimitive{ℓ-max}\AgdaSpace{}%
\AgdaBound{ℓ}\AgdaSpace{}%
\AgdaBound{ℓ'}\AgdaSymbol{)}\<%
\\
%
\>[4]\AgdaKeyword{where}\<%
\\
%
\\[\AgdaEmptyExtraSkip]%
%
\>[4]\AgdaComment{-- Shorter qualified names}\<%
\\
%
\>[4]\AgdaKeyword{private}\<%
\\
\>[4][@{}l@{\AgdaIndent{0}}]%
\>[6]\AgdaKeyword{module}\AgdaSpace{}%
\AgdaModule{M}\AgdaSpace{}%
\AgdaSymbol{=}\AgdaSpace{}%
\AgdaModule{GroupStr}\AgdaSpace{}%
\AgdaBound{M}\<%
\\
%
\>[6]\AgdaKeyword{module}\AgdaSpace{}%
\AgdaModule{N}\AgdaSpace{}%
\AgdaSymbol{=}\AgdaSpace{}%
\AgdaModule{GroupStr}\AgdaSpace{}%
\AgdaBound{N}\<%
\\
%
\\[\AgdaEmptyExtraSkip]%
%
\>[4]\AgdaKeyword{field}\<%
\\
\>[4][@{}l@{\AgdaIndent{0}}]%
\>[6]\AgdaField{pres·}\AgdaSpace{}%
\AgdaSymbol{:}\AgdaSpace{}%
\AgdaSymbol{(}\AgdaBound{x}\AgdaSpace{}%
\AgdaBound{y}\AgdaSpace{}%
\AgdaSymbol{:}\AgdaSpace{}%
\AgdaBound{A}\AgdaSymbol{)}\AgdaSpace{}%
\AgdaSymbol{→}\AgdaSpace{}%
\AgdaBound{f}\AgdaSpace{}%
\AgdaSymbol{(}\AgdaOperator{\AgdaFunction{M.\AgdaUnderscore{}+\AgdaUnderscore{}}}\AgdaSpace{}%
\AgdaBound{x}\AgdaSpace{}%
\AgdaBound{y}\AgdaSymbol{)}\AgdaSpace{}%
\AgdaOperator{\AgdaFunction{≡}}\AgdaSpace{}%
\AgdaSymbol{(}\AgdaOperator{\AgdaFunction{N.\AgdaUnderscore{}+\AgdaUnderscore{}}}\AgdaSpace{}%
\AgdaSymbol{(}\AgdaBound{f}\AgdaSpace{}%
\AgdaBound{x}\AgdaSymbol{)}\AgdaSpace{}%
\AgdaSymbol{(}\AgdaBound{f}\AgdaSpace{}%
\AgdaBound{y}\AgdaSymbol{))}\<%
\\
%
\>[6]\AgdaField{pres1}\AgdaSpace{}%
\AgdaSymbol{:}\AgdaSpace{}%
\AgdaBound{f}\AgdaSpace{}%
\AgdaFunction{M.0g}\AgdaSpace{}%
\AgdaOperator{\AgdaFunction{≡}}\AgdaSpace{}%
\AgdaFunction{N.0g}\<%
\\
%
\>[6]\AgdaField{presinv}\AgdaSpace{}%
\AgdaSymbol{:}\AgdaSpace{}%
\AgdaSymbol{(}\AgdaBound{x}\AgdaSpace{}%
\AgdaSymbol{:}\AgdaSpace{}%
\AgdaBound{A}\AgdaSymbol{)}\AgdaSpace{}%
\AgdaSymbol{→}\AgdaSpace{}%
\AgdaBound{f}\AgdaSpace{}%
\AgdaSymbol{(}\AgdaOperator{\AgdaFunction{M.-\AgdaUnderscore{}}}\AgdaSpace{}%
\AgdaBound{x}\AgdaSymbol{)}\AgdaSpace{}%
\AgdaOperator{\AgdaFunction{≡}}\AgdaSpace{}%
\AgdaOperator{\AgdaFunction{N.-\AgdaUnderscore{}}}\AgdaSpace{}%
\AgdaSymbol{(}\AgdaBound{f}\AgdaSpace{}%
\AgdaBound{x}\AgdaSymbol{)}\<%
\\
%
\>[6]\AgdaComment{-- pres· : (x y : A) → f (x M.· y) ≡ f x N.· f y}\<%
\\
%
\>[6]\AgdaComment{-- pres1 : f M.1g ≡ N.1g}\<%
\\
%
\>[6]\AgdaComment{-- presinv : (x : A) → f (M.inv x) ≡ N.inv (f x)}\<%
\\
%
\\[\AgdaEmptyExtraSkip]%
%
\>[2]\AgdaFunction{GroupHom'}\AgdaSpace{}%
\AgdaSymbol{:}\AgdaSpace{}%
\AgdaSymbol{(}\AgdaBound{G}\AgdaSpace{}%
\AgdaSymbol{:}\AgdaSpace{}%
\AgdaFunction{Group}\AgdaSpace{}%
\AgdaSymbol{\{}\AgdaGeneralizable{ℓ}\AgdaSymbol{\})}\AgdaSpace{}%
\AgdaSymbol{(}\AgdaBound{H}\AgdaSpace{}%
\AgdaSymbol{:}\AgdaSpace{}%
\AgdaFunction{Group}\AgdaSpace{}%
\AgdaSymbol{\{}\AgdaGeneralizable{ℓ'}\AgdaSymbol{\})}\AgdaSpace{}%
\AgdaSymbol{→}\AgdaSpace{}%
\AgdaPrimitive{Type}\AgdaSpace{}%
\AgdaSymbol{(}\AgdaPrimitive{ℓ-max}\AgdaSpace{}%
\AgdaGeneralizable{ℓ}\AgdaSpace{}%
\AgdaGeneralizable{ℓ'}\AgdaSymbol{)}\<%
\\
%
\>[2]\AgdaComment{-- GroupHom' : (G : Group ℓ) (H : Group ℓ') → Type (ℓ-max ℓ ℓ')}\<%
\\
%
\>[2]\AgdaFunction{GroupHom'}\AgdaSpace{}%
\AgdaBound{G}\AgdaSpace{}%
\AgdaBound{H}\AgdaSpace{}%
\AgdaSymbol{=}\AgdaSpace{}%
\AgdaFunction{Σ[}\AgdaSpace{}%
\AgdaBound{f}\AgdaSpace{}%
\AgdaFunction{∈}\AgdaSpace{}%
\AgdaSymbol{(}\AgdaBound{G}\AgdaSpace{}%
\AgdaSymbol{.}\AgdaField{fst}\AgdaSpace{}%
\AgdaSymbol{→}\AgdaSpace{}%
\AgdaBound{H}\AgdaSpace{}%
\AgdaSymbol{.}\AgdaField{fst}\AgdaSymbol{)}\AgdaSpace{}%
\AgdaFunction{]}\AgdaSpace{}%
\AgdaRecord{IsGroupHom}\AgdaSpace{}%
\AgdaSymbol{(}\AgdaBound{G}\AgdaSpace{}%
\AgdaSymbol{.}\AgdaField{snd}\AgdaSymbol{)}\AgdaSpace{}%
\AgdaBound{f}\AgdaSpace{}%
\AgdaSymbol{(}\AgdaBound{H}\AgdaSpace{}%
\AgdaSymbol{.}\AgdaField{snd}\AgdaSymbol{)}\<%
\end{code}


While the last two definitions be somewhat compressible to a programmer
or mathematician not exposed to Agda, it is certainly comprehensible to a
computer : that is, the colors indicate it type-checks on a computer where Cubical Agda is installed.
While GF is designed for multilingual syntactic transformations and is targeted
for natural language translation, its underlying theory is largely based on
ideas from the compiler communities. A cousin of the BNF Converter (BNFC), GF is
fully capable of parsing programming languages like Agda! While the Agda
definitions are present another concrete presentation of a group
homomorphism, they are distinct in that they have inherent semantic content.

While this example may not exemplify the power of Agda's type-checker, it is of
considerable interest to many. The type-checker has merely assured us that
\term{GroupHom(')} are well-formed types - not that we have a canonical representation
of a group homomorphism.
% Aarne says cut

We note that the natural
language definitions of monoid differ in form, but also in pragmatic content.
How one expresses formalities in natural language is incredibly diverse, and
Definition~\ref{def:def4} as compared with the prior homomorphism definitions is
particularly poignant in demonstrating this. These differ very much in nature to
the Agda definitions - especially pragmatically.
The differences between the Cubical
Agda definitions may be loosely called pragmatic, in the sense that the choice
of definitions may have downstream effects on readability, maintainability, modularity, and other
considerations when trying to write good code, in a burgeoning area known as proof engineering.

% TODO section name


\subsection{Twin Primes Conjecture Revisited} \label{twin}
\begin{code}[hide]%
\>[0]\AgdaKeyword{module}\AgdaSpace{}%
\AgdaModule{twinsigma}\AgdaSpace{}%
\AgdaKeyword{where}\<%
\\
%
\\[\AgdaEmptyExtraSkip]%
\>[0]\AgdaKeyword{open}\AgdaSpace{}%
\AgdaKeyword{import}\AgdaSpace{}%
\AgdaModule{Data.Nat}\AgdaSpace{}%
\AgdaKeyword{renaming}\AgdaSpace{}%
\AgdaSymbol{(}\AgdaOperator{\AgdaPrimitive{\AgdaUnderscore{}+\AgdaUnderscore{}}}\AgdaSpace{}%
\AgdaSymbol{to}\AgdaSpace{}%
\AgdaOperator{\AgdaPrimitive{\AgdaUnderscore{}∔\AgdaUnderscore{}}}\AgdaSymbol{)}\<%
\\
\>[0]\AgdaKeyword{open}\AgdaSpace{}%
\AgdaKeyword{import}\AgdaSpace{}%
\AgdaModule{Data.Product}\AgdaSpace{}%
\AgdaKeyword{using}\AgdaSpace{}%
\AgdaSymbol{(}\AgdaRecord{Σ}\AgdaSymbol{;}\AgdaSpace{}%
\AgdaOperator{\AgdaFunction{\AgdaUnderscore{}×\AgdaUnderscore{}}}\AgdaSymbol{;}\AgdaSpace{}%
\AgdaOperator{\AgdaInductiveConstructor{\AgdaUnderscore{},\AgdaUnderscore{}}}\AgdaSymbol{;}\AgdaSpace{}%
\AgdaField{proj₁}\AgdaSymbol{;}\AgdaSpace{}%
\AgdaField{proj₂}\AgdaSymbol{;}\AgdaSpace{}%
\AgdaFunction{∃}\AgdaSymbol{;}\AgdaSpace{}%
\AgdaFunction{Σ-syntax}\AgdaSymbol{;}\AgdaSpace{}%
\AgdaFunction{∃-syntax}\AgdaSymbol{)}\<%
\\
\>[0]\AgdaKeyword{open}\AgdaSpace{}%
\AgdaKeyword{import}\AgdaSpace{}%
\AgdaModule{Data.Sum}\AgdaSpace{}%
\AgdaKeyword{renaming}\AgdaSpace{}%
\AgdaSymbol{(}\AgdaOperator{\AgdaDatatype{\AgdaUnderscore{}⊎\AgdaUnderscore{}}}\AgdaSpace{}%
\AgdaSymbol{to}\AgdaSpace{}%
\AgdaOperator{\AgdaDatatype{\AgdaUnderscore{}+\AgdaUnderscore{}}}\AgdaSymbol{)}\<%
\\
\>[0]\AgdaKeyword{import}\AgdaSpace{}%
\AgdaModule{Relation.Binary.PropositionalEquality}\AgdaSpace{}%
\AgdaSymbol{as}\AgdaSpace{}%
\AgdaModule{Eq}\<%
\\
\>[0]\AgdaKeyword{open}\AgdaSpace{}%
\AgdaModule{Eq}\AgdaSpace{}%
\AgdaKeyword{using}\AgdaSpace{}%
\AgdaSymbol{(}\AgdaOperator{\AgdaDatatype{\AgdaUnderscore{}≡\AgdaUnderscore{}}}\AgdaSymbol{;}\AgdaSpace{}%
\AgdaInductiveConstructor{refl}\AgdaSymbol{;}\AgdaSpace{}%
\AgdaFunction{trans}\AgdaSymbol{;}\AgdaSpace{}%
\AgdaFunction{sym}\AgdaSymbol{;}\AgdaSpace{}%
\AgdaFunction{cong}\AgdaSymbol{;}\AgdaSpace{}%
\AgdaFunction{cong-app}\AgdaSymbol{;}\AgdaSpace{}%
\AgdaFunction{subst}\AgdaSymbol{)}\<%
\\
\>[0]\AgdaKeyword{open}\AgdaSpace{}%
\AgdaModule{Eq.≡-Reasoning}\AgdaSpace{}%
\AgdaKeyword{using}\AgdaSpace{}%
\AgdaSymbol{(}\AgdaOperator{\AgdaFunction{begin\AgdaUnderscore{}}}\AgdaSymbol{;}\AgdaSpace{}%
\AgdaOperator{\AgdaFunction{\AgdaUnderscore{}≡⟨⟩\AgdaUnderscore{}}}\AgdaSymbol{;}\AgdaSpace{}%
\AgdaFunction{step-≡}\AgdaSymbol{;}\AgdaSpace{}%
\AgdaOperator{\AgdaFunction{\AgdaUnderscore{}∎}}\AgdaSymbol{)}\<%
\\
%
\\[\AgdaEmptyExtraSkip]%
\>[0]\AgdaOperator{\AgdaFunction{\AgdaUnderscore{}-\AgdaUnderscore{}}}\AgdaSpace{}%
\AgdaSymbol{:}\AgdaSpace{}%
\AgdaDatatype{ℕ}\AgdaSpace{}%
\AgdaSymbol{→}\AgdaSpace{}%
\AgdaDatatype{ℕ}\AgdaSpace{}%
\AgdaSymbol{→}\AgdaSpace{}%
\AgdaDatatype{ℕ}\<%
\\
\>[0]\AgdaBound{n}%
\>[6]\AgdaOperator{\AgdaFunction{-}}\AgdaSpace{}%
\AgdaInductiveConstructor{zero}\AgdaSpace{}%
\AgdaSymbol{=}\AgdaSpace{}%
\AgdaBound{n}\<%
\\
\>[0]\AgdaInductiveConstructor{zero}%
\>[6]\AgdaOperator{\AgdaFunction{-}}\AgdaSpace{}%
\AgdaInductiveConstructor{suc}\AgdaSpace{}%
\AgdaBound{m}\AgdaSpace{}%
\AgdaSymbol{=}\AgdaSpace{}%
\AgdaInductiveConstructor{zero}\<%
\\
\>[0]\AgdaInductiveConstructor{suc}\AgdaSpace{}%
\AgdaBound{n}\AgdaSpace{}%
\AgdaOperator{\AgdaFunction{-}}\AgdaSpace{}%
\AgdaInductiveConstructor{suc}\AgdaSpace{}%
\AgdaBound{m}\AgdaSpace{}%
\AgdaSymbol{=}\AgdaSpace{}%
\AgdaBound{n}\AgdaSpace{}%
\AgdaOperator{\AgdaFunction{-}}\AgdaSpace{}%
\AgdaBound{m}\<%
\\
%
\\[\AgdaEmptyExtraSkip]%
\>[0]\AgdaFunction{isPrime}\AgdaSpace{}%
\AgdaSymbol{:}\AgdaSpace{}%
\AgdaDatatype{ℕ}\AgdaSpace{}%
\AgdaSymbol{→}\AgdaSpace{}%
\AgdaPrimitive{Set}\<%
\\
\>[0]\AgdaFunction{isPrime}\AgdaSpace{}%
\AgdaBound{n}\AgdaSpace{}%
\AgdaSymbol{=}\<%
\\
\>[0][@{}l@{\AgdaIndent{0}}]%
\>[2]\AgdaSymbol{(}\AgdaBound{n}\AgdaSpace{}%
\AgdaOperator{\AgdaFunction{≥}}\AgdaSpace{}%
\AgdaNumber{2}\AgdaSymbol{)}\AgdaSpace{}%
\AgdaOperator{\AgdaFunction{×}}\<%
\\
%
\>[2]\AgdaSymbol{((}\AgdaBound{x}\AgdaSpace{}%
\AgdaBound{y}\AgdaSpace{}%
\AgdaSymbol{:}\AgdaSpace{}%
\AgdaDatatype{ℕ}\AgdaSymbol{)}\AgdaSpace{}%
\AgdaSymbol{→}\AgdaSpace{}%
\AgdaBound{x}\AgdaSpace{}%
\AgdaOperator{\AgdaPrimitive{*}}\AgdaSpace{}%
\AgdaBound{y}\AgdaSpace{}%
\AgdaOperator{\AgdaDatatype{≡}}\AgdaSpace{}%
\AgdaBound{n}\AgdaSpace{}%
\AgdaSymbol{→}\AgdaSpace{}%
\AgdaSymbol{(}\AgdaBound{x}\AgdaSpace{}%
\AgdaOperator{\AgdaDatatype{≡}}\AgdaSpace{}%
\AgdaNumber{1}\AgdaSymbol{)}\AgdaSpace{}%
\AgdaOperator{\AgdaDatatype{+}}\AgdaSpace{}%
\AgdaSymbol{(}\AgdaBound{x}\AgdaSpace{}%
\AgdaOperator{\AgdaDatatype{≡}}\AgdaSpace{}%
\AgdaBound{n}\AgdaSymbol{))}\<%
\end{code}
We now give the dependent uncurring from the functions from \ref{twinprime} We
note that this perhaps is a bit more linguistically natural, because we can
refer to definitions of a prime number, sucessive prime numbers, etc. We leave
it to the reader to decide which presentation would be better suited for
translation.
\begin{code}%
\>[0]\AgdaFunction{prime}\AgdaSpace{}%
\AgdaSymbol{=}\AgdaSpace{}%
\AgdaFunction{Σ[}\AgdaSpace{}%
\AgdaBound{p}\AgdaSpace{}%
\AgdaFunction{∈}\AgdaSpace{}%
\AgdaDatatype{ℕ}\AgdaSpace{}%
\AgdaFunction{]}\AgdaSpace{}%
\AgdaFunction{isPrime}\AgdaSpace{}%
\AgdaBound{p}\<%
\\
%
\\[\AgdaEmptyExtraSkip]%
\>[0]\AgdaFunction{isSuccessivePrime}\AgdaSpace{}%
\AgdaSymbol{:}\AgdaSpace{}%
\AgdaFunction{prime}\AgdaSpace{}%
\AgdaSymbol{→}\AgdaSpace{}%
\AgdaFunction{prime}\AgdaSpace{}%
\AgdaSymbol{→}\AgdaSpace{}%
\AgdaPrimitive{Set}\<%
\\
\>[0]\AgdaFunction{isSuccessivePrime}\AgdaSpace{}%
\AgdaSymbol{(}\AgdaBound{p}\AgdaSpace{}%
\AgdaOperator{\AgdaInductiveConstructor{,}}\AgdaSpace{}%
\AgdaBound{pIsPrime}\AgdaSymbol{)}\AgdaSpace{}%
\AgdaSymbol{(}\AgdaBound{p'}\AgdaSpace{}%
\AgdaOperator{\AgdaInductiveConstructor{,}}\AgdaSpace{}%
\AgdaBound{p'IsPrime}\AgdaSymbol{)}\AgdaSpace{}%
\AgdaSymbol{=}\<%
\\
\>[0][@{}l@{\AgdaIndent{0}}]%
\>[2]\AgdaSymbol{(}\AgdaBound{(}\AgdaBound{p''}\AgdaSpace{}%
\AgdaOperator{\AgdaInductiveConstructor{,}}\AgdaSpace{}%
\AgdaBound{p''IsPrime}\AgdaBound{)}\AgdaSpace{}%
\AgdaSymbol{:}\AgdaSpace{}%
\AgdaFunction{prime}\AgdaSymbol{)}\AgdaSpace{}%
\AgdaSymbol{→}\<%
\\
%
\>[2]\AgdaBound{p}\AgdaSpace{}%
\AgdaOperator{\AgdaDatatype{≤}}\AgdaSpace{}%
\AgdaBound{p'}\AgdaSpace{}%
\AgdaSymbol{→}\AgdaSpace{}%
\AgdaBound{p}\AgdaSpace{}%
\AgdaOperator{\AgdaDatatype{≤}}\AgdaSpace{}%
\AgdaBound{p''}\AgdaSpace{}%
\AgdaSymbol{→}\AgdaSpace{}%
\AgdaBound{p'}\AgdaSpace{}%
\AgdaOperator{\AgdaDatatype{≤}}\AgdaSpace{}%
\AgdaBound{p''}\<%
\\
%
\\[\AgdaEmptyExtraSkip]%
\>[0]\AgdaFunction{successivePrimes}\AgdaSpace{}%
\AgdaSymbol{=}\<%
\\
\>[0][@{}l@{\AgdaIndent{0}}]%
\>[2]\AgdaFunction{Σ[}\AgdaSpace{}%
\AgdaBound{p}\AgdaSpace{}%
\AgdaFunction{∈}\AgdaSpace{}%
\AgdaFunction{prime}\AgdaSpace{}%
\AgdaFunction{]}\AgdaSpace{}%
\AgdaFunction{Σ[}\AgdaSpace{}%
\AgdaBound{p'}\AgdaSpace{}%
\AgdaFunction{∈}\AgdaSpace{}%
\AgdaFunction{prime}\AgdaSpace{}%
\AgdaFunction{]}\AgdaSpace{}%
\AgdaFunction{isSuccessivePrime}\AgdaSpace{}%
\AgdaBound{p}\AgdaSpace{}%
\AgdaBound{p'}\<%
\\
%
\\[\AgdaEmptyExtraSkip]%
\>[0]\AgdaFunction{primeGap}\AgdaSpace{}%
\AgdaSymbol{:}\AgdaSpace{}%
\AgdaFunction{successivePrimes}\AgdaSpace{}%
\AgdaSymbol{→}\AgdaSpace{}%
\AgdaDatatype{ℕ}\<%
\\
\>[0]\AgdaFunction{primeGap}\AgdaSpace{}%
\AgdaSymbol{((}\AgdaBound{p}\AgdaSpace{}%
\AgdaOperator{\AgdaInductiveConstructor{,}}\AgdaSpace{}%
\AgdaBound{pIsPrime}\AgdaSymbol{)}\AgdaSpace{}%
\AgdaOperator{\AgdaInductiveConstructor{,}}\AgdaSpace{}%
\AgdaSymbol{(}\AgdaBound{p'}\AgdaSpace{}%
\AgdaOperator{\AgdaInductiveConstructor{,}}\AgdaSpace{}%
\AgdaBound{p'IsPrime}\AgdaSymbol{)}\AgdaSpace{}%
\AgdaOperator{\AgdaInductiveConstructor{,}}\AgdaSpace{}%
\AgdaBound{p'-is-after-px}\AgdaSymbol{)}\AgdaSpace{}%
\AgdaSymbol{=}\AgdaSpace{}%
\AgdaBound{p}\AgdaSpace{}%
\AgdaOperator{\AgdaFunction{-}}\AgdaSpace{}%
\AgdaBound{p'}\<%
\\
%
\\[\AgdaEmptyExtraSkip]%
\>[0]\AgdaFunction{nth-pletPrimes}\AgdaSpace{}%
\AgdaSymbol{:}\AgdaSpace{}%
\AgdaFunction{successivePrimes}\AgdaSpace{}%
\AgdaSymbol{→}\AgdaSpace{}%
\AgdaDatatype{ℕ}\AgdaSpace{}%
\AgdaSymbol{→}\AgdaSpace{}%
\AgdaPrimitive{Set}\<%
\\
\>[0]\AgdaFunction{nth-pletPrimes}\AgdaSpace{}%
\AgdaSymbol{(}\AgdaBound{p}\AgdaSpace{}%
\AgdaOperator{\AgdaInductiveConstructor{,}}\AgdaSpace{}%
\AgdaBound{p'}\AgdaSpace{}%
\AgdaOperator{\AgdaInductiveConstructor{,}}\AgdaSpace{}%
\AgdaBound{p'-is-after-p}\AgdaSymbol{)}\AgdaSpace{}%
\AgdaBound{n}\AgdaSpace{}%
\AgdaSymbol{=}\<%
\\
\>[0][@{}l@{\AgdaIndent{0}}]%
\>[2]\AgdaFunction{primeGap}\AgdaSpace{}%
\AgdaSymbol{(}\AgdaBound{p}\AgdaSpace{}%
\AgdaOperator{\AgdaInductiveConstructor{,}}\AgdaSpace{}%
\AgdaBound{p'}\AgdaSpace{}%
\AgdaOperator{\AgdaInductiveConstructor{,}}\AgdaSpace{}%
\AgdaBound{p'-is-after-p}\AgdaSymbol{)}\AgdaSpace{}%
\AgdaOperator{\AgdaDatatype{≡}}\AgdaSpace{}%
\AgdaBound{n}\<%
\\
%
\\[\AgdaEmptyExtraSkip]%
\>[0]\AgdaFunction{twinPrimes}\AgdaSpace{}%
\AgdaSymbol{:}\AgdaSpace{}%
\AgdaFunction{successivePrimes}\AgdaSpace{}%
\AgdaSymbol{→}%
\>[33]\AgdaPrimitive{Set}\<%
\\
\>[0]\AgdaFunction{twinPrimes}\AgdaSpace{}%
\AgdaBound{sucPrimes}\AgdaSpace{}%
\AgdaSymbol{=}\AgdaSpace{}%
\AgdaFunction{nth-pletPrimes}\AgdaSpace{}%
\AgdaBound{sucPrimes}\AgdaSpace{}%
\AgdaNumber{2}\<%
\\
%
\\[\AgdaEmptyExtraSkip]%
\>[0]\AgdaFunction{twinPrimeConjecture}\AgdaSpace{}%
\AgdaSymbol{:}\AgdaSpace{}%
\AgdaPrimitive{Set}\<%
\\
\>[0]\AgdaFunction{twinPrimeConjecture}\AgdaSpace{}%
\AgdaSymbol{=}\AgdaSpace{}%
\AgdaSymbol{(}\AgdaBound{n}\AgdaSpace{}%
\AgdaSymbol{:}\AgdaSpace{}%
\AgdaDatatype{ℕ}\AgdaSymbol{)}\AgdaSpace{}%
\AgdaSymbol{→}\<%
\\
\>[0][@{}l@{\AgdaIndent{0}}]%
\>[2]\AgdaFunction{Σ[}\AgdaSpace{}%
\AgdaBound{sprs}\AgdaSymbol{@((}\AgdaBound{p}\AgdaSpace{}%
\AgdaOperator{\AgdaInductiveConstructor{,}}\AgdaSpace{}%
\AgdaBound{p'}\AgdaSymbol{)}\AgdaOperator{\AgdaInductiveConstructor{,}}\AgdaSpace{}%
\AgdaBound{p'-after-p}\AgdaSymbol{)}\AgdaSpace{}%
\AgdaFunction{∈}\AgdaSpace{}%
\AgdaFunction{successivePrimes}\AgdaSpace{}%
\AgdaFunction{]}\<%
\\
\>[2][@{}l@{\AgdaIndent{0}}]%
\>[4]\AgdaSymbol{(}\AgdaBound{p}\AgdaSpace{}%
\AgdaOperator{\AgdaFunction{≥}}\AgdaSpace{}%
\AgdaBound{n}\AgdaSymbol{)}\<%
\\
%
\>[2]\AgdaOperator{\AgdaFunction{×}}\AgdaSpace{}%
\AgdaFunction{twinPrimes}\AgdaSpace{}%
\AgdaBound{sprs}\<%
\end{code}


\subsection{cubicalTT} \label{cubicaltt}
\begin{verbatim}
abstract Exp = {

flags startcat = Decl ;
      -- note, cubical tt doesn't support inductive families, and therefore the datatype (& labels) need to be modified

cat
  Comment ;
  Module  ;
  AIdent ;
  Imp ; --imports, add later
  Decl ;
  Exp ;
  ExpWhere ;
  Tele ;
  Branch ;
  PTele ;
  Label ;
  [AIdent]{0} ; -- "x y z"
  [Decl]{1} ;
  [Tele]{0} ;
  [Branch]{1} ;
  [Label]{1} ;
  [PTele]{1} ;
  -- [Exp]{1};

fun

  DeclDef : AIdent -> [Tele] -> Exp -> ExpWhere -> Decl ;
  -- foo ( b : bool ) : bool = b
  DeclData : AIdent -> [Tele] -> [Label] -> Decl ;
  -- data nat : Set where zero | suc ( n : nat )
  DeclSplit : AIdent -> [Tele] -> Exp -> [Branch] -> Decl ;
  -- caseBool ( x : Set ) ( y z : x ) : bool -> Set = split false -> y || true -> z
  DeclUndef : AIdent -> [Tele] -> Exp -> Decl ;
  -- funExt ( a : Set ) ( b : a -> Set ) ( f g : ( x : a ) -> b x ) ( p : ( x : a ) -> ( b x ) ( f x ) == ( g x ) ) : ( ( y : a ) -> b y ) f == g = undefined

  Where : Exp -> [Decl] -> ExpWhere ;
  -- foo ( b : bool ) : bool =
  -- f b where f : bool -> bool = negb
  NoWhere : Exp -> ExpWhere ;
  -- foo ( b : bool ) : bool =
  -- b

  Split : Exp -> [Branch] -> Exp ;
  --split@ ( nat -> bool ) with zero  -> true || suc n -> false
  Let : [Decl] -> Exp -> Exp ;
  -- foo ( b : bool ) : bool =
  -- let f : bool -> bool = negb in f b
  Lam : [PTele] -> Exp -> Exp ;
  -- \\ ( x : bool ) -> negb x
  -- todo, allow implicit typing
  Fun : Exp -> Exp -> Exp ;
  -- Set -> Set
  -- Set -> ( b : bool ) -> ( x : Set ) -> ( f x )
  Pi    : [PTele] -> Exp -> Exp ;
  --( f : bool -> Set ) -> ( b : bool ) -> ( x : Set ) -> ( f x )
  -- ( f : bool -> Set ) ( b : bool ) ( x : Set ) -> ( f x )
  Sigma : [PTele] -> Exp -> Exp ;
  -- ( f : bool -> Set ) ( b : bool ) ( x : Set ) * ( f x )
  App : Exp -> Exp -> Exp ;
  -- proj1 ( x , y )
  Id : Exp -> Exp -> Exp -> Exp ;
  -- Set bool == nat
  IdJ : Exp -> Exp -> Exp -> Exp -> Exp -> Exp ;
  -- J e d x y p
  Fst : Exp -> Exp ; -- "proj1 x"
  Snd : Exp -> Exp ; -- "proj2 x"
  -- Pair : Exp -> [Exp] -> Exp ;
  Pair : Exp -> Exp -> Exp ;
  -- ( x , y )
  Var : AIdent -> Exp ;
  -- x
  Univ : Exp ;
  -- Set
  Refl : AIdent ; -- Exp ;
  -- refl
  --Hole : HoleIdent -> Exp ; -- need to add holes

  OBranch :  AIdent -> [AIdent] -> ExpWhere -> Branch ;
  -- suc m -> split@ ( nat -> bool ) with zero -> false || suc n -> equalNat m n
  -- for splits

  OLabel : AIdent -> [Tele] -> Label ;
  -- suc ( n : nat )
  -- fora data types

  -- construct telescope
  TeleC : AIdent -> [AIdent] -> Exp -> Tele ;
  -- "( f g : ( x : a ) -> b x )"
  -- ( a : Set ) ( b : ( a ) -> ( Set ) ) ( f g : ( x : a ) -> ( ( b ) ( x ) ) ) ( p : ( x : a ) -> ( ( ( b ) ( x ) ) ( ( f ) ( x ) ) == ( ( g ) ( x ) ) ) )

  -- why does gr with this fail so epically?
  PTeleC : Exp -> Exp -> PTele ; 
  -- ( x : Set ) -- ( y : x -> Set )" -- ( x : f y z )"

  --everything below this is just strings

  Foo : AIdent ;
  A , B , C , D , E , F , G , H , I , J , K , L , M , N , O , P , Q , R , S , T , U , V , W , X , Y , Z : AIdent ;
  True , False , Bool : AIdent ;
  NegB : AIdent ;
  CaseBool : AIdent ;
  IndBool : AIdent ;
  FunExt : AIdent ;
  Nat : AIdent ;
  Zero : AIdent ;
  Suc : AIdent ;
  EqualNat : AIdent ;
  Unit : AIdent ;
  Top : AIdent ;
  Contr : AIdent ;
  Fiber : AIdent ;
  IsEquiv : AIdent ;
  IdIsEquiv : AIdent ;
  IdFun : AIdent ;
  ContrSingl : AIdent ;
  Equiv : AIdent ;
  EqToIso : AIdent ;
  Ybar : AIdent ;
  IdFib : AIdent ;
  Identity : AIdent ;
  Lemma0 : AIdent ;
}
\end{verbatim}
\begin{verbatim}
concrete ExpCubicalTT of Exp = open Prelude, FormalTwo in {

lincat 
  Comment,
  Module ,
  AIdent,
  Imp,
  Decl ,
  ExpWhere,
  Tele,
  Branch ,
  PTele,
  Label,
    -- = Str ;
  [AIdent],
  [Decl] ,
  -- [Exp],
  [Tele],
  [Branch] ,
  [PTele],
  [Label]
    -- = {hd,tl : Str} ;
    = Str ;
  Exp = TermPrec ;

lin

  DeclDef a lt e ew = a ++ lt ++ ":" ++ usePrec 0 e ++ "=" ++ ew ;
  DeclData a t d = "data" ++ a ++ t ++ ": Set where" ++ d ;
  DeclSplit ai lt e lb = ai ++ lt ++ ":" ++ usePrec 0 e ++ "= split" ++ lb ;
  DeclUndef a lt e = a ++ lt ++ ":" ++ usePrec 0 e ++ "= undefined" ; -- postulate in agda

  Where e ld = usePrec 0 e ++ "where" ++ ld ;
  NoWhere e = usePrec 0 e ;

  Let ld e = mkPrec 0 ("let" ++ ld ++ "in" ++ (usePrec 0 e)) ;
  Split e lb = mkPrec 0 ("split@" ++ usePrec 0 e ++ "with" ++ lb) ;
  Lam pt e = mkPrec 0 ("\\" ++ pt ++ "->" ++ usePrec 0 e) ;
  Fun = infixr 1 "->" ; -- A -> Set
  Pi pt e = mkPrec 1 (pt ++ "->" ++ usePrec 1 e) ;
  Sigma pt e = mkPrec 1 (pt ++ "*" ++ usePrec 1 e) ;
  App = infixl 2 "" ;
  Id e1 e2 e3 = mkPrec 3 (usePrec 4 e1 ++ usePrec 4 e2 ++ "==" ++ usePrec 3 e3) ;
-- for an explicit vs implicit use of parameters, may have to use expressions as records, with a parameter is_implicit
  IdJ e1 e2 e3 e4 e5 = mkPrec 3 ("J" ++ usePrec 4 e1 ++ usePrec 4 e2 ++ usePrec 4 e3 ++ usePrec 4 e4 ++ usePrec 4 e5) ;
  Fst e = mkPrec 4 ("fst" ++ usePrec 4 e) ;
  Snd e = mkPrec 4 ("snd" ++ usePrec 4 e) ;
  Pair e1 e2 = mkPrec 5 ("(" ++ usePrec 0 e1 ++ "," ++ usePrec 0 e2 ++ ")") ;
  Var a = constant a ;
  Univ = constant "Set" ;
  Refl = "refl" ; -- constant "refl" ;

  BaseAIdent = "" ;
  ConsAIdent x xs = x ++ xs ;

  -- [Decl] only used in ExpWhere
  BaseDecl x = x ;
  ConsDecl x xs = x ++ "^" ++ xs ;

  -- maybe accomodate so split on empty type just gives () 
  -- BaseBranch = "" ;
  BaseBranch x = x ;
  -- ConsBranch x xs = x ++ "\n" ++ xs ;
  ConsBranch x xs = x ++ "||" ++ xs ;

  -- for data constructors
  BaseLabel x = x ;
  ConsLabel x xs = x ++ "|" ++ xs ; 

  BasePTele x = x ;
  ConsPTele x xs = x ++ xs ;

  BaseTele = "" ;
  ConsTele x xs = x ++ xs ;

  OBranch a la ew = a ++ la ++ "->" ++ ew ;
  TeleC a la e = "(" ++ a ++ la ++ ":" ++ usePrec 0 e ++ ")" ;
  PTeleC e1 e2 = "(" ++ top e1 ++ ":" ++ top e2 ++ ")" ;

  OLabel a lt = a ++ lt ;

  --object language syntax, all variables for now
  Bool = "bool" ;
  True = "true" ;
  False = "false" ;
  CaseBool = "caseBool" ;
  IndBool = "indBool" ;
  FunExt = "funExt" ;
  Nat = "nat" ;
  Zero = "zero" ;
  Suc = "suc" ;
  EqualNat = "equalNat" ;
  Unit = "unit" ;
  Top = "top" ;
  Foo = "foo" ; 
  A = "a" ;
  B = "b" ;
  C = "c" ;
  D = "d" ;
  E = "e" ;
  F = "f" ;
  G = "g" ;
  H = "h" ;
  I = "i" ;
  J = "j" ;
  K = "k" ;
  L = "l" ;
  M = "m" ;
  N = "n" ;
  O = "o" ;
  P = "p" ;
  Q = "q" ;
  R = "r" ;
  S = "s" ;
  T = "t" ;
  U = "u" ;
  V = "v" ;
  W = "w" ;
  X = "x" ;
  Y = "y" ;
  Z = "z" ;
  NegB = "negb" ;
  -- everything below is for contractible proofs
  Contr = "isContr" ;
  Fiber = "fiber" ;
  IsEquiv = "isEquiv" ;
  IdIsEquiv = "idIsEquiv" ;
  IdFun = "idfun" ;
  ContrSingl = "contrSingl" ;
  Equiv = "equiv" ;
  EqToIso = "eqToIso" ;
  Identity = "id" ;
  Ybar = "ybar"  ;
  IdFib = "idFib"  ;
  Lemma0 = "lemma0" ;
}
\end{verbatim}
The resource FormalTwo.gf merely substitutes more precedences than Formal.gf
from the RGL, in the ideal case that we could scale the grammar to include
larger and more complicated fixity information.
\begin{verbatim}
resource FormalTwo = open Prelude in {

----Everything the same up until the definition of Prec in Formal.gf


    Prec : PType = Predef.Ints 9 ;

    highest = 9 ;

    lessPrec : Prec -> Prec -> Bool = \p,q ->
      case <<p,q> : Prec * Prec> of {
        <3,9> | <2,9> | <4,9> | <5,9> | <6,9> | <7,9> | <8,9> => True ;
        <3,8> | <2,8> | <4,8> | <5,8> | <6,8> | <7,8> => True ;
        <3,7> | <2,7> | <4,7> | <5,7> | <6,7> => True ;
        <3,6> | <2,6> | <4,6> | <5,6> => True ;
        <3,5> | <2,5> | <4,5> => True ;
        <3,4> | <2,3> | <2,4> => True ;
        <1,1> | <1,0> | <0,0> => False ;
        <1,_> | <0,_>         => True ;
        _ => False
      } ;

    nextPrec : Prec -> Prec = \p -> case <p : Prec> of {
      9 => 9 ;
      n => Predef.plus n 1
      } ;
\end{verbatim}


\subsection{Hott and cubicalTT Grammars} \label{comparison}

\begin{code}[hide]%
\>[0]\AgdaSymbol{\{-\#}\AgdaSpace{}%
\AgdaKeyword{OPTIONS}\AgdaSpace{}%
\AgdaPragma{--omega-in-omega}\AgdaSpace{}%
\AgdaPragma{--type-in-type}\AgdaSpace{}%
\AgdaSymbol{\#-\}}\<%
\\
%
\\[\AgdaEmptyExtraSkip]%
\>[0]\AgdaKeyword{module}\AgdaSpace{}%
\AgdaModule{compare}\AgdaSpace{}%
\AgdaKeyword{where}\<%
\\
%
\\[\AgdaEmptyExtraSkip]%
\>[0]\AgdaKeyword{open}\AgdaSpace{}%
\AgdaKeyword{import}\AgdaSpace{}%
\AgdaModule{Agda.Builtin.Sigma}\AgdaSpace{}%
\AgdaKeyword{public}\<%
\\
%
\\[\AgdaEmptyExtraSkip]%
\>[0]\AgdaKeyword{variable}\<%
\\
\>[0][@{}l@{\AgdaIndent{0}}]%
\>[2]\AgdaGeneralizable{A}\AgdaSpace{}%
\AgdaGeneralizable{B}\AgdaSpace{}%
\AgdaSymbol{:}\AgdaSpace{}%
\AgdaPrimitive{Set}\<%
\\
%
\\[\AgdaEmptyExtraSkip]%
\>[0]\AgdaKeyword{data}\AgdaSpace{}%
\AgdaOperator{\AgdaDatatype{\AgdaUnderscore{}≡\AgdaUnderscore{}}}\AgdaSpace{}%
\AgdaSymbol{\{}\AgdaBound{A}\AgdaSpace{}%
\AgdaSymbol{:}\AgdaSpace{}%
\AgdaPrimitive{Set}\AgdaSymbol{\}}\AgdaSpace{}%
\AgdaSymbol{(}\AgdaBound{a}\AgdaSpace{}%
\AgdaSymbol{:}\AgdaSpace{}%
\AgdaBound{A}\AgdaSymbol{)}\AgdaSpace{}%
\AgdaSymbol{:}\AgdaSpace{}%
\AgdaBound{A}\AgdaSpace{}%
\AgdaSymbol{→}\AgdaSpace{}%
\AgdaPrimitive{Set}\AgdaSpace{}%
\AgdaKeyword{where}\<%
\\
\>[0][@{}l@{\AgdaIndent{0}}]%
\>[2]\AgdaInductiveConstructor{r}\AgdaSpace{}%
\AgdaSymbol{:}\AgdaSpace{}%
\AgdaBound{a}\AgdaSpace{}%
\AgdaOperator{\AgdaDatatype{≡}}\AgdaSpace{}%
\AgdaBound{a}\<%
\\
%
\\[\AgdaEmptyExtraSkip]%
\>[0]\AgdaKeyword{infix}\AgdaSpace{}%
\AgdaNumber{20}\AgdaSpace{}%
\AgdaOperator{\AgdaDatatype{\AgdaUnderscore{}≡\AgdaUnderscore{}}}\<%
\end{code}
\begin{code}%
\>[0]\AgdaFunction{id}\AgdaSpace{}%
\AgdaSymbol{:}\AgdaSpace{}%
\AgdaGeneralizable{A}\AgdaSpace{}%
\AgdaSymbol{→}\AgdaSpace{}%
\AgdaGeneralizable{A}\<%
\\
\>[0]\AgdaFunction{id}\AgdaSpace{}%
\AgdaSymbol{=}\AgdaSpace{}%
\AgdaSymbol{λ}\AgdaSpace{}%
\AgdaBound{z}\AgdaSpace{}%
\AgdaSymbol{→}\AgdaSpace{}%
\AgdaBound{z}\<%
\\
%
\\[\AgdaEmptyExtraSkip]%
\>[0]\AgdaFunction{iscontr}\AgdaSpace{}%
\AgdaSymbol{:}\AgdaSpace{}%
\AgdaSymbol{(}\AgdaBound{A}\AgdaSpace{}%
\AgdaSymbol{:}\AgdaSpace{}%
\AgdaPrimitive{Set}\AgdaSymbol{)}\AgdaSpace{}%
\AgdaSymbol{→}\AgdaSpace{}%
\AgdaPrimitive{Set}\<%
\\
\>[0]\AgdaFunction{iscontr}\AgdaSpace{}%
\AgdaBound{A}\AgdaSpace{}%
\AgdaSymbol{=}%
\>[13]\AgdaRecord{Σ}\AgdaSpace{}%
\AgdaBound{A}\AgdaSpace{}%
\AgdaSymbol{λ}\AgdaSpace{}%
\AgdaBound{a}\AgdaSpace{}%
\AgdaSymbol{→}\AgdaSpace{}%
\AgdaSymbol{(}\AgdaBound{x}\AgdaSpace{}%
\AgdaSymbol{:}\AgdaSpace{}%
\AgdaBound{A}\AgdaSymbol{)}\AgdaSpace{}%
\AgdaSymbol{→}\AgdaSpace{}%
\AgdaSymbol{(}\AgdaBound{a}\AgdaSpace{}%
\AgdaOperator{\AgdaDatatype{≡}}\AgdaSpace{}%
\AgdaBound{x}\AgdaSymbol{)}\<%
\\
%
\\[\AgdaEmptyExtraSkip]%
\>[0]\AgdaFunction{fiber}\AgdaSpace{}%
\AgdaSymbol{:}\AgdaSpace{}%
\AgdaSymbol{(}\AgdaBound{A}\AgdaSpace{}%
\AgdaBound{B}\AgdaSpace{}%
\AgdaSymbol{:}\AgdaSpace{}%
\AgdaPrimitive{Set}\AgdaSymbol{)}\AgdaSpace{}%
\AgdaSymbol{(}\AgdaBound{f}\AgdaSpace{}%
\AgdaSymbol{:}\AgdaSpace{}%
\AgdaBound{A}\AgdaSpace{}%
\AgdaSymbol{->}\AgdaSpace{}%
\AgdaBound{B}\AgdaSymbol{)}\AgdaSpace{}%
\AgdaSymbol{(}\AgdaBound{y}\AgdaSpace{}%
\AgdaSymbol{:}\AgdaSpace{}%
\AgdaBound{B}\AgdaSymbol{)}\AgdaSpace{}%
\AgdaSymbol{→}\AgdaSpace{}%
\AgdaPrimitive{Set}\<%
\\
\>[0]\AgdaFunction{fiber}\AgdaSpace{}%
\AgdaBound{A}\AgdaSpace{}%
\AgdaBound{B}\AgdaSpace{}%
\AgdaBound{f}\AgdaSpace{}%
\AgdaBound{y}\AgdaSpace{}%
\AgdaSymbol{=}\AgdaSpace{}%
\AgdaRecord{Σ}\AgdaSpace{}%
\AgdaBound{A}\AgdaSpace{}%
\AgdaSymbol{(λ}\AgdaSpace{}%
\AgdaBound{x}\AgdaSpace{}%
\AgdaSymbol{→}\AgdaSpace{}%
\AgdaBound{y}\AgdaSpace{}%
\AgdaOperator{\AgdaDatatype{≡}}\AgdaSpace{}%
\AgdaBound{f}\AgdaSpace{}%
\AgdaBound{x}\AgdaSymbol{)}\<%
\\
%
\\[\AgdaEmptyExtraSkip]%
\>[0]\AgdaFunction{isEquiv}\AgdaSpace{}%
\AgdaSymbol{:}\AgdaSpace{}%
\AgdaSymbol{(}\AgdaBound{A}\AgdaSpace{}%
\AgdaBound{B}\AgdaSpace{}%
\AgdaSymbol{:}\AgdaSpace{}%
\AgdaPrimitive{Set}\AgdaSymbol{)}\AgdaSpace{}%
\AgdaSymbol{→}\AgdaSpace{}%
\AgdaSymbol{(}\AgdaBound{f}\AgdaSpace{}%
\AgdaSymbol{:}\AgdaSpace{}%
\AgdaBound{A}\AgdaSpace{}%
\AgdaSymbol{→}\AgdaSpace{}%
\AgdaBound{B}\AgdaSymbol{)}\AgdaSpace{}%
\AgdaSymbol{→}\AgdaSpace{}%
\AgdaPrimitive{Set}\<%
\\
\>[0]\AgdaFunction{isEquiv}\AgdaSpace{}%
\AgdaBound{A}\AgdaSpace{}%
\AgdaBound{B}\AgdaSpace{}%
\AgdaBound{f}\AgdaSpace{}%
\AgdaSymbol{=}\AgdaSpace{}%
\AgdaSymbol{(}\AgdaBound{y}\AgdaSpace{}%
\AgdaSymbol{:}\AgdaSpace{}%
\AgdaBound{B}\AgdaSymbol{)}\AgdaSpace{}%
\AgdaSymbol{→}\AgdaSpace{}%
\AgdaFunction{iscontr}\AgdaSpace{}%
\AgdaSymbol{(}\AgdaFunction{fiber}\AgdaSpace{}%
\AgdaBound{A}\AgdaSpace{}%
\AgdaBound{B}\AgdaSpace{}%
\AgdaBound{f}\AgdaSpace{}%
\AgdaBound{y}\AgdaSymbol{)}\<%
\\
%
\\[\AgdaEmptyExtraSkip]%
\>[0]\AgdaFunction{isEquiv'}\AgdaSpace{}%
\AgdaSymbol{:}\AgdaSpace{}%
\AgdaSymbol{(}\AgdaBound{A}\AgdaSpace{}%
\AgdaBound{B}\AgdaSpace{}%
\AgdaSymbol{:}\AgdaSpace{}%
\AgdaPrimitive{Set}\AgdaSymbol{)}\AgdaSpace{}%
\AgdaSymbol{→}\AgdaSpace{}%
\AgdaSymbol{(}\AgdaBound{f}\AgdaSpace{}%
\AgdaSymbol{:}\AgdaSpace{}%
\AgdaBound{A}\AgdaSpace{}%
\AgdaSymbol{→}\AgdaSpace{}%
\AgdaBound{B}\AgdaSymbol{)}\AgdaSpace{}%
\AgdaSymbol{→}\AgdaSpace{}%
\AgdaPrimitive{Set}\<%
\\
\>[0]\AgdaFunction{isEquiv'}\AgdaSpace{}%
\AgdaBound{A}\AgdaSpace{}%
\AgdaBound{B}\AgdaSpace{}%
\AgdaBound{f}\AgdaSpace{}%
\AgdaSymbol{=}\AgdaSpace{}%
\AgdaSymbol{∀}\AgdaSpace{}%
\AgdaSymbol{(}\AgdaBound{y}\AgdaSpace{}%
\AgdaSymbol{:}\AgdaSpace{}%
\AgdaBound{B}\AgdaSymbol{)}\AgdaSpace{}%
\AgdaSymbol{→}\AgdaSpace{}%
\AgdaFunction{iscontr}\AgdaSpace{}%
\AgdaSymbol{(}\AgdaFunction{fiber'}\AgdaSpace{}%
\AgdaBound{y}\AgdaSymbol{)}\<%
\\
\>[0][@{}l@{\AgdaIndent{0}}]%
\>[2]\AgdaKeyword{where}\<%
\\
\>[2][@{}l@{\AgdaIndent{0}}]%
\>[4]\AgdaFunction{fiber'}\AgdaSpace{}%
\AgdaSymbol{:}\AgdaSpace{}%
\AgdaSymbol{(}\AgdaBound{y}\AgdaSpace{}%
\AgdaSymbol{:}\AgdaSpace{}%
\AgdaBound{B}\AgdaSymbol{)}\AgdaSpace{}%
\AgdaSymbol{→}\AgdaSpace{}%
\AgdaPrimitive{Set}\<%
\\
%
\>[4]\AgdaFunction{fiber'}\AgdaSpace{}%
\AgdaBound{y}\AgdaSpace{}%
\AgdaSymbol{=}\AgdaSpace{}%
\AgdaRecord{Σ}\AgdaSpace{}%
\AgdaBound{A}\AgdaSpace{}%
\AgdaSymbol{(λ}\AgdaSpace{}%
\AgdaBound{x}\AgdaSpace{}%
\AgdaSymbol{→}\AgdaSpace{}%
\AgdaBound{y}\AgdaSpace{}%
\AgdaOperator{\AgdaDatatype{≡}}\AgdaSpace{}%
\AgdaBound{f}\AgdaSpace{}%
\AgdaBound{x}\AgdaSymbol{)}\<%
\\
%
\\[\AgdaEmptyExtraSkip]%
\>[0]\AgdaComment{-- proof from Aarne}\<%
\\
\>[0]\AgdaFunction{idIsEquiv'}\AgdaSpace{}%
\AgdaSymbol{:}\AgdaSpace{}%
\AgdaSymbol{(}\AgdaBound{A}\AgdaSpace{}%
\AgdaSymbol{:}\AgdaSpace{}%
\AgdaPrimitive{Set}\AgdaSymbol{)}\AgdaSpace{}%
\AgdaSymbol{→}\AgdaSpace{}%
\AgdaFunction{isEquiv}\AgdaSpace{}%
\AgdaBound{A}\AgdaSpace{}%
\AgdaBound{A}\AgdaSpace{}%
\AgdaSymbol{(}\AgdaFunction{id}\AgdaSpace{}%
\AgdaSymbol{\{}\AgdaBound{A}\AgdaSymbol{\})}\<%
\\
\>[0]\AgdaFunction{idIsEquiv'}\AgdaSpace{}%
\AgdaBound{A}\AgdaSpace{}%
\AgdaBound{y}\AgdaSpace{}%
\AgdaSymbol{=}\AgdaSpace{}%
\AgdaFunction{ybar}\AgdaSpace{}%
\AgdaOperator{\AgdaInductiveConstructor{,}}\AgdaSpace{}%
\AgdaFunction{help}\<%
\\
\>[0][@{}l@{\AgdaIndent{0}}]%
\>[2]\AgdaKeyword{where}\<%
\\
\>[2][@{}l@{\AgdaIndent{0}}]%
\>[4]\AgdaFunction{fib'}\AgdaSpace{}%
\AgdaSymbol{:}\AgdaSpace{}%
\AgdaPrimitive{Set}\AgdaSpace{}%
\AgdaComment{-- \{y : A\}}\<%
\\
%
\>[4]\AgdaFunction{fib'}\AgdaSpace{}%
\AgdaSymbol{=}\AgdaSpace{}%
\AgdaFunction{fiber}\AgdaSpace{}%
\AgdaBound{A}\AgdaSpace{}%
\AgdaBound{A}\AgdaSpace{}%
\AgdaFunction{id}\AgdaSpace{}%
\AgdaBound{y}\<%
\\
%
\>[4]\AgdaFunction{ybar}\AgdaSpace{}%
\AgdaSymbol{:}\AgdaSpace{}%
\AgdaFunction{fib'}\<%
\\
%
\>[4]\AgdaFunction{ybar}\AgdaSpace{}%
\AgdaSymbol{=}\AgdaSpace{}%
\AgdaBound{y}\AgdaSpace{}%
\AgdaOperator{\AgdaInductiveConstructor{,}}\AgdaSpace{}%
\AgdaInductiveConstructor{r}\<%
\\
%
\>[4]\AgdaFunction{help}\AgdaSpace{}%
\AgdaSymbol{:}\AgdaSpace{}%
\AgdaSymbol{(}\AgdaBound{x}\AgdaSpace{}%
\AgdaSymbol{:}\AgdaSpace{}%
\AgdaFunction{fib'}\AgdaSymbol{)}\AgdaSpace{}%
\AgdaSymbol{→}\AgdaSpace{}%
\AgdaOperator{\AgdaDatatype{\AgdaUnderscore{}≡\AgdaUnderscore{}}}\AgdaSpace{}%
\AgdaSymbol{\{}\AgdaRecord{Σ}\AgdaSpace{}%
\AgdaBound{A}\AgdaSpace{}%
\AgdaSymbol{(}\AgdaOperator{\AgdaDatatype{\AgdaUnderscore{}≡\AgdaUnderscore{}}}\AgdaSpace{}%
\AgdaBound{y}\AgdaSymbol{)\}}\AgdaSpace{}%
\AgdaFunction{ybar}\AgdaSpace{}%
\AgdaBound{x}\<%
\\
%
\>[4]\AgdaFunction{help}\AgdaSpace{}%
\AgdaSymbol{=}\AgdaSpace{}%
\AgdaSymbol{λ}\AgdaSpace{}%
\AgdaSymbol{\{(}\AgdaBound{a}\AgdaSpace{}%
\AgdaOperator{\AgdaInductiveConstructor{,}}\AgdaSpace{}%
\AgdaInductiveConstructor{r}\AgdaSymbol{)}\AgdaSpace{}%
\AgdaSymbol{→}\AgdaSpace{}%
\AgdaInductiveConstructor{r}\AgdaSymbol{\}}\<%
\\
%
\\[\AgdaEmptyExtraSkip]%
\>[0]\AgdaFunction{equiv}\AgdaSpace{}%
\AgdaSymbol{:}\AgdaSpace{}%
\AgdaSymbol{(}\AgdaSpace{}%
\AgdaBound{a}\AgdaSpace{}%
\AgdaBound{b}\AgdaSpace{}%
\AgdaSymbol{:}\AgdaSpace{}%
\AgdaPrimitive{Set}\AgdaSpace{}%
\AgdaSymbol{)}\AgdaSpace{}%
\AgdaSymbol{→}\AgdaSpace{}%
\AgdaPrimitive{Set}\<%
\\
\>[0]\AgdaFunction{equiv}\AgdaSpace{}%
\AgdaBound{a}\AgdaSpace{}%
\AgdaBound{b}\AgdaSpace{}%
\AgdaSymbol{=}\AgdaSpace{}%
\AgdaRecord{Σ}\AgdaSpace{}%
\AgdaSymbol{(}\AgdaBound{a}\AgdaSpace{}%
\AgdaSymbol{→}\AgdaSpace{}%
\AgdaBound{b}\AgdaSymbol{)}\AgdaSpace{}%
\AgdaSymbol{λ}\AgdaSpace{}%
\AgdaBound{f}\AgdaSpace{}%
\AgdaSymbol{→}\AgdaSpace{}%
\AgdaFunction{isEquiv}\AgdaSpace{}%
\AgdaBound{a}\AgdaSpace{}%
\AgdaBound{b}\AgdaSpace{}%
\AgdaBound{f}\<%
\\
%
\\[\AgdaEmptyExtraSkip]%
\>[0]\AgdaFunction{equivId}\AgdaSpace{}%
\AgdaSymbol{:}\AgdaSpace{}%
\AgdaSymbol{(}\AgdaBound{x}\AgdaSpace{}%
\AgdaSymbol{:}\AgdaSpace{}%
\AgdaPrimitive{Set}\AgdaSymbol{)}\AgdaSpace{}%
\AgdaSymbol{→}\AgdaSpace{}%
\AgdaFunction{equiv}\AgdaSpace{}%
\AgdaBound{x}\AgdaSpace{}%
\AgdaBound{x}\<%
\\
\>[0]\AgdaFunction{equivId}\AgdaSpace{}%
\AgdaBound{x}\AgdaSpace{}%
\AgdaSymbol{=}\AgdaSpace{}%
\AgdaFunction{id}\AgdaSpace{}%
\AgdaOperator{\AgdaInductiveConstructor{,}}\AgdaSpace{}%
\AgdaSymbol{(}\AgdaFunction{idIsEquiv'}\AgdaSpace{}%
\AgdaBound{x}\AgdaSymbol{)}\<%
\\
%
\\[\AgdaEmptyExtraSkip]%
\>[0]\AgdaFunction{eqToIso}\AgdaSpace{}%
\AgdaSymbol{:}\AgdaSpace{}%
\AgdaSymbol{(}\AgdaSpace{}%
\AgdaBound{a}\AgdaSpace{}%
\AgdaBound{b}\AgdaSpace{}%
\AgdaSymbol{:}\AgdaSpace{}%
\AgdaPrimitive{Set}\AgdaSpace{}%
\AgdaSymbol{)}\AgdaSpace{}%
\AgdaSymbol{→}\AgdaSpace{}%
\AgdaOperator{\AgdaDatatype{\AgdaUnderscore{}≡\AgdaUnderscore{}}}\AgdaSpace{}%
\AgdaSymbol{\{}\AgdaPrimitive{Set}\AgdaSymbol{\}}\AgdaSpace{}%
\AgdaBound{a}\AgdaSpace{}%
\AgdaBound{b}\AgdaSpace{}%
\AgdaSymbol{→}\AgdaSpace{}%
\AgdaFunction{equiv}\AgdaSpace{}%
\AgdaBound{a}\AgdaSpace{}%
\AgdaBound{b}\<%
\\
\>[0]\AgdaFunction{eqToIso}\AgdaSpace{}%
\AgdaBound{a}\AgdaSpace{}%
\AgdaDottedPattern{\AgdaSymbol{.}}\AgdaDottedPattern{a}\AgdaSpace{}%
\AgdaInductiveConstructor{r}\AgdaSpace{}%
\AgdaSymbol{=}\AgdaSpace{}%
\AgdaFunction{equivId}\AgdaSpace{}%
\AgdaBound{a}\<%
\end{code}

Compared with the latex code

\begin{figure}[H]
 \textbf{Definition}:
 A type $A$ is contractible, if there is $a : A$, called the center of contraction, such that for all $x : A$, $\equalH {a}{x}$.

 \textbf{Definition}:
 A map $f : \arrowH {A}{B}$ is an equivalence, if for all $y : B$, its fiber, $\comprehensionH {x}{A}{\equalH {\appH {f}{x}}{y}}$, is contractible.
 We write $\equivalenceH {A}{B}$, if there is an equivalence $\arrowH {A}{B}$.

 \textbf{Lemma}:
 For each type $A$, the identity map, $\defineH {\idMapH {A}}{\typingH {\lambdaH {x}{A}{x}}{\arrowH {A}{A}}}$, is an equivalence.

 \textbf{Proof}:
 For each $y : A$, let $\defineH {\fiberH {y}{A}}{\comprehensionH {x}{A}{\equalH {x}{y}}}$ be its fiber with respect to $\idMapH {A}$ and let $\defineH {\barH {y}}{\typingH {\pairH {y}{\reflexivityH {A}{y}}}{\fiberH {y}{A}}}$.
 As for all $y : A$, $\equalH {\pairH {y}{\reflexivityH {A}{y}}}{y}$, we may apply Id-induction on $y$, $\typingH {x}{A}$ and $\typingH {z}{\idPropH {x}{y}}$ to get that \[\equalH {\pairH {x}{z}}{y}\].
 Hence, for $y : A$, we may apply $\Sigma$ -elimination on $\typingH {u}{\fiberH {y}{A}}$ to get that $\equalH {u}{y}$, so that $\fiberH {y}{A}$ is contractible.
 Thus, $\typingH {\idMapH {A}}{\arrowH {A}{A}}$ is an equivalence.
  $\Box$

 \textbf{Corollary}:
 If $U$ is a type universe, then, for $X , Y : U$, \[(*)\arrowH {\equalH {X}{Y}}{\equivalenceH {X}{Y}}\].

 \textbf{Proof}:
 We may apply the lemma to get that for $X : U$, $\equivalenceH {X}{X}$.
 Hence, we may apply Id-induction on $\typingH {X , Y}{U}$ to get that $(*)$.
  $\Box$


 \textbf{Definition}:
 A type universe $U$ is univalent, if for $X , Y : U$, the map $\equivalenceMapH {X}{Y}: \arrowH {\equalH {X}{Y}}{\equivalenceH {X}{Y}}$ in $(*)$ is an equivalence.
\end{figure}

cubicalTT parses the following.  We note an idealization : that agda supports ananymous pattern matching, so 
\codeword{\\ ( ( b , refl )}  would not typecheck in the original cubicalTT. Additionally, the reflexivity constructor is only present when the identity is inductively defined, as it is a primitive in cubical type theories.

\begin{figure}[H]
\begin{verbatim}
id ( a : Set ) : a -> a = \\ ( b : a ) -> b
isContr ( a : Set ) : Set = ( b : a ) * ( x : a ) -> a b == x
fiber ( a b  : Set ) ( f : a -> b ) ( y : b )  : Set 
  = ( x : a ) * ( x : a ) -> b y == ( f x )
isEquiv ( a b  : Set ) ( f : a -> b )   : Set 
  = ( y : b ) -> isContr ( fiber a b f y )
  where fiber ( a b  : Set ) ( f : a -> b ) ( y : b )  : Set 
    = ( x : a ) * ( x : a ) -> b y == ( f x )
equiv ( a b : Set ) : Set = ( f : a -> b ) * isEquiv a b f

idIsEquiv ( a : Set ) : isEquiv a a ( id a ) =  ( ybar , lemma0 )
  where
    idFib : Set = fiber a a id y
    ^ ybar : idFib = ( y , refl )
    ^ lemma0 ( x : idFib ) : ( ( p ) ybar == x ) 
      = \\ ( ( b , refl ) : ( c : a ) * ( a c == c ) ) -> refl

idIsEquiv ( x : Set ) : equiv x x = ( id , idIsEquiv x )
eqToIso ( a b : Set ) : ( Set a == b ) -> equiv a b 
  = split refl -> idIsEquiv a
\end{verbatim}
\end{figure}

We compare two abstract syntax trees side by side to show that they have quite different structures,

\begin{figure}
\centering
\begin{minipage}[t]{.5\textwidth}
\begin{verbatim}
Exp> 
* DeclDef
    * Contr
      ConsTele
        * TeleC
            * A
              BaseAIdent
              Univ
          BaseTele
      Univ
      NoWhere
        * Sigma
            * BasePTele
                * PTeleC
                    * Var
                        * B
                      Var
                        * A
              Pi
                * BasePTele
                    * PTeleC
                        * Var
                            * X
                          Var
                            * A
                  Id
                    * Var
                        * A
                      Var
                        * B
                      Var
                        * X
\end{verbatim}
\end{minipage}%
\begin{minipage}[t]{.55\textwidth}
\begin{verbatim}
* PredDefinition
    * type_Sort
      A_Var
      contractible_Pred
      ExistCalledProp
        * a_Var
          ExpSort
            * VarExp
                * A_Var
          FunInd
            * centre_of_contraction_Fun
          ForAllProp
            * allUnivPhrase
                * BaseVar
                    * x_Var
                  ExpSort
                    * VarExp
                        * A_Var
              ExpProp
                * DollarMathEnv
                  equalExp
                    * VarExp
                        * a_Var
                      VarExp
                        * x_Var
\end{verbatim}
\end{minipage}
\caption{Mathematical Assertions and Agda Judgements} \label{fig:I2}
\end{figure}

What we notice : 


\begin{figure}
\centering
\begin{minipage}[t]{.5\textwidth}
\begin{verbatim}
* DeclSplit
    * EqToIso
      ConsTele
        * TeleC
            * A
              ConsAIdent
                * B
                  BaseAIdent
              Univ
          BaseTele
      Fun
        * Id
            * Univ
              Var
                * A
              Var
                * B
          App
            * App
                * Var
                    * Equiv
                  Var
                    * A
              Var
                * B
      BaseBranch
        * OBranch
            * Refl
              BaseAIdent
              NoWhere
                * App
                    * Var
                        * IdIsEquiv
                      Var
                        * A
\end{verbatim}
\end{minipage}%
\begin{minipage}[t]{.55\textwidth}
\begin{verbatim}
3 PropParagraph
    * NoConclusionPhrase
      ForAllProp
        * if_thenUnivPhrase
            * BaseVar
                * U_Var
              type_universe_Sort
          ForAllProp
            * plainUnivPhrase
                * ConsVar
                    * X_Var
                      BaseVar
                        * Y_Var
                  ExpSort
                    * VarExp
                        * U_Var
              LabelledExpProp
                * DisplayMathEnv
                  StarLabel
                  mapExp
                    * equalExp
                        * VarExp
                            * X_Var
                          VarExp
                            * Y_Var
                      equivalenceExp
                        * VarExp
                            * X_Var
                          VarExp
                            * Y_Var
4 ConclusionParagraph
    1 NoConclusionPhrase
      ApplyLabelConclusion
        * the_lemma_Label
          BaseInd
          ForAllProp
            * plainUnivPhrase
                * BaseVar
                    * X_Var
                  ExpSort
                    * VarExp
                        * U_Var
              ExpProp
                * DollarMathEnv
                  equivalenceExp
                    * VarExp
                        * X_Var
                      VarExp
                        * X_Var
    2 henceConclusionPhrase
      ApplyLabelConclusion
        * id_induction_Label
          ConsInd
            * FunInd
                * ExpFun
                    * TypedExp
                        * ConsExp
                            * VarExp
                                * X_Var
                              BaseExp
                                * VarExp
                                    * Y_Var
                          VarExp
                            * U_Var
              BaseInd
          LabelProp
            * StarLabel
\end{verbatim}
\end{minipage}
\caption{Mathematical Assertions and Agda Judgements} \label{fig:I3}
\end{figure}

todo : refactor to have the final sections side-by-side, do a more "thorough analysis of the text fragment above"
namely - look at the redundancy, the intro of identity local to a definition (often having more than one proposition in a proposition) 
the failure in some instances to provide relevant info, etc.

also, refactor to have the sigma proof here

Exp> * DeclDef
    * IdIsEquiv
      ConsTele
        * TeleC
            * X
              BaseAIdent
              Univ
          BaseTele
      App
        * App
            * Var
                * Equiv
              Var
                * X
          Var
            * X
      NoWhere
        * Pair
            * Var
                * Identity
              App
                * Var
                    * IdIsEquiv
                  Var
                    * X





\subsection{HoTT Agda Corpus} \label{hottproofs}