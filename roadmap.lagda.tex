\documentclass[11pt, a4paper]{article}

\usepackage{mlt-thesis-2015}

% With Xetex/Luatex this shouldn't be used
%\usepackage[utf8]{inputenc}

\usepackage[english]{babel}
\usepackage{graphicx}
\usepackage{setspace}

% from here

\usepackage{fontspec}
\usepackage{fullpage}
\usepackage{hyperref}
\usepackage{agda}

\usepackage{unicode-math}

%\usepackage{amssymb,amsmath,amsthm,stmaryrd,mathrsfs,wasysym}

%\usepackage{enumitem,mathtools,xspace}
\usepackage{amsfonts}
\usepackage{mathtools}
\usepackage{xspace}


\usepackage{enumitem}


\setmainfont{DejaVu Serif}
\setsansfont{DejaVu Sans}
\setmonofont{DejaVu Sans Mono}

% \setmonofont{Fira Mono}
% \setsansfont{Noto Sans}

\usepackage{newunicodechar}

\newunicodechar{ℓ}{\ensuremath{\mathnormal\ell}}
\newunicodechar{→}{\ensuremath{\mathnormal\rightarrow}}

\newtheorem{definition}{Definition}
\newtheorem{lem}{Lemma}
\newtheorem{proof}{Proof}
\newtheorem{thm}{Theorem}

\newcommand{\jdeq}{\equiv}      % An equality judgment
\newcommand{\refl}[1]{\ensuremath{\mathsf{refl}_{#1}}\xspace}
\newcommand{\define}[1]{\textbf{#1}}
\newcommand{\defeq}{\vcentcolon\equiv}  % A judgmental equality currently being defined

%\newcommand{\jdeq}{\equiv}      % An equality judgment
%\let\judgeq\jdeq


\newcommand{\ind}[1]{\mathsf{ind}_{#1}}
\newcommand{\indid}[1]{\ind{=_{#1}}} % (Martin-Lof) path induction principle for identity types

\newcommand{\blank}{\mathord{\hspace{1pt}\text{--}\hspace{1pt}}}

\newcommand{\opp}[1]{\mathord{{#1}^{-1}}}
\let\rev\opp

\newcommand{\id}[3][]{\ensuremath{#2 =_{#1} #3}\xspace}



\newcommand{\UU}{\ensuremath{\mathcal{U}}\xspace}
\let\bbU\UU
\let\type\UU


%\newcommand{\ct}{%
  %\mathchoice{\mathbin{\raisebox{0.5ex}{$\displaystyle\centerdot$}}}%
             %{\mathbin{\raisebox{0.5ex}{$\centerdot$}}}%
             %{\mathbin{\raisebox{0.25ex}{$\scriptstyle\,\centerdot\,$}}}%
             %{\mathbin{\raisebox{0.1ex}{$\scriptscriptstyle\,\centerdot\,$}}}
%}

% til here

\title{The grammar of proof}
% \subtitle{Subtitle}
\author{Warrick Macmillan}

\begin{document}

%% ============================================================================
%% Title page
%% ============================================================================
\begin{titlepage}

\maketitle

\vfill

\begingroup
\renewcommand*{\arraystretch}{1.2}
\begin{tabular}{l@{\hskip 20mm}l}
\hline
Master's Thesis: & 30 credits\\
Programme: & Master’s Programme in Language Technology\\
Level: & Advanced level \\
Semester and year: & Fall, 2021\\
Supervisor & Aarne Ranta\\
Examiner & (name of the examiner)\\
Report number & (number will be provided by the administrators) \\
Keywords &  Grammatical Framework, Natural Language Generation,\\
\end{tabular}
\endgroup

\thispagestyle{empty}
\end{titlepage}

%% ============================================================================
%% Abstract
%% ============================================================================
\newpage
\singlespacing
\section*{Abstract}

Brief summary of research question, background, method, results\ldots

\thispagestyle{empty}

%% ============================================================================
%% Preface
%% ============================================================================
\newpage
\section*{Preface}

Acknowledgements, etc.

\thispagestyle{empty}

%% ============================================================================
%% Contents
%% ============================================================================
\newpage

\begin{spacing}{0.0}
\tableofcontents
\end{spacing}

\thispagestyle{empty}

%% ============================================================================
%% Introduction
%% ============================================================================
\newpage
\setcounter{page}{1}

\section{Introduction}
\label{sec:intro}

The central concern of this thesis is the syntax of mathematics, programming
languages, and their respective mutual influence, as conceived and practiced by
mathematicians and computer scientists.  From one vantage point, the role of
syntax in mathematics may be regarded as a 2nd order concern, a topic for
discussion during a Fika, an artifact of ad hoc development by the working
mathematician whose real goals are producing genuine mathematical knowledge.
For the programmers and computer scientists, syntax may be regarding as a
matter of taste, with friendly debates recurring regarding the use of
semicolons, brackets, and white space.  Yet, when viewed through the lens of
the propositions-as-types paradigm, these discussions intersect in new and
interesting ways.  When one introduces a third paradigm through which to
analyze the use of syntax in mathematics and programming, namely Linguistics, I
propose what some may regard as superficial detail, indeed becomes a central
paradigm, with many interesting and important questions. 

To get a feel for this syntactic paradigm, let us look at a basic mathematical
example: that of a group homomorphism, as expressed in a variety of sources.  

% Wikipedia Defn:

\begin{definition}
In mathematics, given two groups, $(G, \ast)$ and $(H, \cdot)$, a group homomorphism from $(G, \ast)$ to $(H, \cdot)$ is a function $h : G \to H$ such that for all $u$ and $v$ in $G$ it holds that

\begin{center}
  $h(u \ast v) = h ( u ) \cdot h ( v )$ 
\end{center}
\end{definition}

% http://math.mit.edu/~jwellens/Group%20Theory%20Forum.pdf

\begin{definition}
Let $G = (G,\cdot)$ and $G' = (G',\ast)$ be groups, and let $\phi : G \to G'$ be a map between them. We call $\phi$ a \textbf{homomorphism} if for every pair of elements $g, h \in G$, we have 
\begin{center}
  $\phi(g \ast h) = \phi ( g ) \cdot \phi ( h )$ 
\end{center}
\end{definition}

% http://www.maths.gla.ac.uk/~mwemyss/teaching/3alg1-7.pdf

\begin{definition}
Let $G$, $H$, be groups.  A map $\phi : G \to H$ is called a \emph{group homomorphism} if
\begin{center}
  $\phi(xy) = \phi ( x ) \phi ( y )$ for all $x, y \in G$ 
\end{center}
(Note that $xy$ on the left is formed using the group operation in $G$, whilst the product $\phi ( x ) \phi ( y )$ is formed using the group operation $H$.)
\end{definition}

% NLab:

\begin{definition}
Classically, a group is a monoid in which every element has an inverse (necessarily unique).
\end{definition}

We inquire the reader to pay attention to nuance and difference in presentation
that is normally ignored or taken for granted by the fluent mathematician.

If one want to distill the meaning of each of these presentations, there is a
significant amount of subliminal interpretation happening very much analagous
to our innate lingusitic ussage.  The inverse and identity are discarded, even
though they are necessary data when defning a group. The order of presentation
of information is incostent, as well as the choice to use symbolic or natural
language information. In (3), the group operation is used implicitly, and its
clarification a side remark.

Details aside, these all mean the same thing--don't they?  This thesis seeks to provide an
abstract framework to determine whether two lingusitically nuanced presenations
mean the same thing via their syntactic transformations. 

These syntactic transformations come in two flavors : parsing and
linearization, and are natively handled by a Logical Framework (LF) for
specifying grammars : Grammatical Framework (GF).

\begin{code}[hide]
{-# OPTIONS --cubical #-}

module roadmap where

\end{code}

\begin{code}[hide]

module Monoid where

module Namespace1 where

  open import Cubical.Core.Everything
  open import Cubical.Foundations.Prelude renaming (_∙_ to _∙''_)
  open import Cubical.Foundations.Isomorphism

  private
    variable
      ℓ : Level

  is-left-unit-for : {A : Type ℓ} → A → (A → A → A) → Type ℓ
  is-left-unit-for {A = A} e _⋆_ = (x : A) → e ⋆ x ≡ x

  is-right-unit-for : {A : Type ℓ} → A → (A → A → A) → Type ℓ
  is-right-unit-for {A = A} e _⋆_ = (x : A) → x ⋆ e ≡ x

  is-assoc : {A : Type ℓ} → (A → A → A) → Type ℓ
  is-assoc {A = A} _⋆_ = (x y z : A) → (x ⋆ y) ⋆ z ≡ x ⋆ (y ⋆ z)

  record MonoidStrRec (A : Type ℓ) : Type ℓ where
    constructor
      monoid
    field
      ε   : A
      _∙_ : A → A → A

      left-unit  : is-left-unit-for ε _∙_
      right-unit : is-right-unit-for ε _∙_
      assoc      : is-assoc _∙_

      carrier-set : isSet A

  record Monoid' : Type (ℓ-suc ℓ) where
    constructor
      monoid'
    field
      A : Type ℓ
      ε   : A
      _∙_ : A → A → A

      left-unit  : is-left-unit-for ε _∙_
      right-unit : is-right-unit-for ε _∙_
      assoc      : is-assoc _∙_

      carrier-set : isSet A

\end{code}

We now show yet another definition of a group homomorphism formalized in the
Agda programming language:

[TODO: replace monoidhom with grouphom]

\begin{code}
  monoidHom : {ℓ : Level}
            → ((monoid' a _ _ _ _ _ _) (monoid' a' _ _ _ _ _ _) : Monoid' {ℓ} )
            → (a → a') → Type ℓ
  monoidHom
    (monoid' A ε _∙_ left-unit right-unit assoc carrier-set)
    (monoid' A₁ ε₁ _∙₁_ left-unit₁ right-unit₁ assoc₁ carrier-set₁)
    f
    = (m1 m2 : A) → f (m1 ∙ m2) ≡ (f m1) ∙₁ (f m2)
\end{code}

While the first three definitions above are should be linguistically
comprehensible to a non-mathematician, this last definition is most certainly
not.  While may carry degree of comprehension to a programmer or mathematician
not exposed to Agda, it is certainly comprehensible to a computer : that is, it
typechecks on a computer where Cubical Agda is installed. While GF is designed
for multilingual syntactic transformations and is targeted for natural language
translation, it's underlying theory is largely based on ideas from the compiler
communities. A cousin of the BNF Converter (BNFC), GF is fully capable of
parsing progamming languages like Agda! And while the above definition is just
another concrete syntactic presentation of a group homomorphism, it is distinct
from the natural language presentations above in that the colors indicate it
has indeed type checked. 

While this example may not exemplify the power of Agda's type checker, it is of
considerable interest to many. The typechecker has merely assured us that
monoidHom, is a well-formed type.  The type-checker is much more useful than is
immediately evident: it delegates the work of verifying that a proof is
correct, that is, the work of judging whether a term has a type, to the
computer. While it's of practical concern is immediate to any exploited grad
student grading papers late on a Sunday night, its theoretical concern has led
to many recent developments in modern mathematics. Thomas Hales solution to the
Kepler Conjecture was seen as unverifiable by those reviewing it. This led to
Hales outsourcing the verification to Interactive Theorem provers HOL Light and
Isabelle, during which led to many minor corrections in the original proof
which were never spotted due to human oversight.

Fields Medalist Vladimir Voevodsky, had the experience of being told one day
his proof of the Milnor conjecture was fatally flawed. Although the leak in the
proof was patched, this experience of temporarily believing much of his life's
work invalid led him to investigate proof assintants as a tool for future
thought. Indeed, this proof verification error was a key event that led to the
Univalent Foundations
Project~\cite{theunivalentfoundationsprogram-homotopytypetheory-2013}.

While Agda and other programming languages are capable of encoding definitions,
theorems, and proofs, they have so far seen little adoption, and in some cases
treated with suspicion and scorn by many mathematicians.  This isn't entirely
unfounded : it's a lot of work to learn how to use Agda or Coq, software
updates may cause proofs to break, and the inevitable errors we humans are
instilled in these Theorem Provers. And that's not to mention that Martin-Löf
Type Theory, the constructive foundational project which underlies these proof
assistants, is rejected by those who dogmatically accept the law of the
excluded middle and ZFC as the word of God.

It should be noted, the constructivist rejects neither the law of the excluded
middle nor ZFC. She merely observes them, and admits their handiness in certain
cituations. Excluded middle is indeed a helpful tool, as many mathematicians
may attest. The contention is that it should be avoided whenever possible -
proofs which don't rely on it, or it's corallary of proof by contradction, are
much more ameanable to formalization in systems with decideable type checking.
And ZFC, while serving the mathematicians of the early 20th century, is 
lacking when it comes to the higher dimensional structure of n-categories and
infinity groupoids.

What these theorem provers give the mathematician is confidence that her work
is correct, and even more importantly, that the work which she takes for
granted and references in her work is also correct. The task before us is then
one of religious conversion. And one doesn't undertake a conversion by simply
by preaching. Foundational details aside, this thesis is meant to provide a
blueprint for the syntactic reformation that must take place.  

It doesn't ask the mathematician to relinquish the beautiful language she has
come to love in expressing her ideas. Rather, it asks her to make a compromise
for the time being, and use a Controlled Natural Language (CNL) to develop her
work. In exchange she'll get the confidence that Agda provides. Not only that,
she'll be able to search through a library, to see who else has possibly already
postulated and proved her conjecture. A version of this grandiose vision is
proposed via The Formal Abstracts Project.

\begin{code}[hide]%
\>[0]\AgdaComment{--\{-\# OPTIONS --cubical \#-\}}\<%
\\
\>[0]\AgdaSymbol{\{-\#}\AgdaSpace{}%
\AgdaKeyword{OPTIONS}\AgdaSpace{}%
\AgdaPragma{--cubical}\AgdaSpace{}%
\AgdaPragma{--no-import-sorts}\AgdaSpace{}%
\AgdaPragma{--safe}\AgdaSpace{}%
\AgdaSymbol{\#-\}}\<%
\\
%
\\[\AgdaEmptyExtraSkip]%
\>[0]\AgdaKeyword{module}\AgdaSpace{}%
\AgdaModule{monoid}\AgdaSpace{}%
\AgdaKeyword{where}\<%
\\
%
\\[\AgdaEmptyExtraSkip]%
\>[0]\AgdaKeyword{module}\AgdaSpace{}%
\AgdaModule{Namespace1}\AgdaSpace{}%
\AgdaKeyword{where}\<%
\\
%
\\[\AgdaEmptyExtraSkip]%
\>[0][@{}l@{\AgdaIndent{0}}]%
\>[2]\AgdaKeyword{open}\AgdaSpace{}%
\AgdaKeyword{import}\AgdaSpace{}%
\AgdaModule{Cubical.Foundations.Prelude}\<%
\\
%
\>[2]\AgdaKeyword{open}\AgdaSpace{}%
\AgdaKeyword{import}\AgdaSpace{}%
\AgdaModule{Cubical.Foundations.Equiv}\<%
\\
%
\>[2]\AgdaKeyword{open}\AgdaSpace{}%
\AgdaKeyword{import}\AgdaSpace{}%
\AgdaModule{Cubical.Foundations.Structure}\<%
\\
%
\>[2]\AgdaKeyword{open}\AgdaSpace{}%
\AgdaKeyword{import}\AgdaSpace{}%
\AgdaModule{Cubical.Algebra.Group.Base}\<%
\\
%
\>[2]\AgdaKeyword{open}\AgdaSpace{}%
\AgdaKeyword{import}\AgdaSpace{}%
\AgdaModule{Cubical.Data.Sigma}\<%
\\
%
\\[\AgdaEmptyExtraSkip]%
%
\>[2]\AgdaKeyword{private}\<%
\\
\>[2][@{}l@{\AgdaIndent{0}}]%
\>[4]\AgdaKeyword{variable}\<%
\\
\>[4][@{}l@{\AgdaIndent{0}}]%
\>[6]\AgdaGeneralizable{ℓ}\AgdaSpace{}%
\AgdaGeneralizable{ℓ'}\AgdaSpace{}%
\AgdaGeneralizable{ℓ''}\AgdaSpace{}%
\AgdaGeneralizable{ℓ'''}\AgdaSpace{}%
\AgdaSymbol{:}\AgdaSpace{}%
\AgdaPostulate{Level}\<%
\end{code}
\begin{code}%
%
\>[2]\AgdaFunction{isGroupHom}\AgdaSpace{}%
\AgdaSymbol{:}\AgdaSpace{}%
\AgdaSymbol{(}\AgdaBound{G}\AgdaSpace{}%
\AgdaSymbol{:}\AgdaSpace{}%
\AgdaFunction{Group}\AgdaSpace{}%
\AgdaSymbol{\{}\AgdaGeneralizable{ℓ}\AgdaSymbol{\})}\AgdaSpace{}%
\AgdaSymbol{(}\AgdaBound{H}\AgdaSpace{}%
\AgdaSymbol{:}\AgdaSpace{}%
\AgdaFunction{Group}\AgdaSpace{}%
\AgdaSymbol{\{}\AgdaGeneralizable{ℓ'}\AgdaSymbol{\})}\AgdaSpace{}%
\AgdaSymbol{(}\AgdaBound{f}\AgdaSpace{}%
\AgdaSymbol{:}\AgdaSpace{}%
\AgdaOperator{\AgdaFunction{⟨}}\AgdaSpace{}%
\AgdaBound{G}\AgdaSpace{}%
\AgdaOperator{\AgdaFunction{⟩}}\AgdaSpace{}%
\AgdaSymbol{→}\AgdaSpace{}%
\AgdaOperator{\AgdaFunction{⟨}}\AgdaSpace{}%
\AgdaBound{H}\AgdaSpace{}%
\AgdaOperator{\AgdaFunction{⟩}}\AgdaSymbol{)}\AgdaSpace{}%
\AgdaSymbol{→}\AgdaSpace{}%
\AgdaPrimitive{Type}\AgdaSpace{}%
\AgdaSymbol{\AgdaUnderscore{}}\<%
\\
%
\>[2]\AgdaFunction{isGroupHom}\AgdaSpace{}%
\AgdaBound{G}\AgdaSpace{}%
\AgdaBound{H}\AgdaSpace{}%
\AgdaBound{f}\AgdaSpace{}%
\AgdaSymbol{=}\AgdaSpace{}%
\AgdaSymbol{(}\AgdaBound{x}\AgdaSpace{}%
\AgdaBound{y}\AgdaSpace{}%
\AgdaSymbol{:}\AgdaSpace{}%
\AgdaOperator{\AgdaFunction{⟨}}\AgdaSpace{}%
\AgdaBound{G}\AgdaSpace{}%
\AgdaOperator{\AgdaFunction{⟩}}\AgdaSymbol{)}\AgdaSpace{}%
\AgdaSymbol{→}\AgdaSpace{}%
\AgdaBound{f}\AgdaSpace{}%
\AgdaSymbol{(}\AgdaBound{x}\AgdaSpace{}%
\AgdaOperator{\AgdaFunction{G.+}}\AgdaSpace{}%
\AgdaBound{y}\AgdaSymbol{)}\AgdaSpace{}%
\AgdaOperator{\AgdaFunction{≡}}\AgdaSpace{}%
\AgdaSymbol{(}\AgdaBound{f}\AgdaSpace{}%
\AgdaBound{x}\AgdaSpace{}%
\AgdaOperator{\AgdaFunction{H.+}}\AgdaSpace{}%
\AgdaBound{f}\AgdaSpace{}%
\AgdaBound{y}\AgdaSymbol{)}\AgdaSpace{}%
\AgdaKeyword{where}\<%
\\
\>[2][@{}l@{\AgdaIndent{0}}]%
\>[4]\AgdaKeyword{module}\AgdaSpace{}%
\AgdaModule{G}\AgdaSpace{}%
\AgdaSymbol{=}\AgdaSpace{}%
\AgdaModule{GroupStr}\AgdaSpace{}%
\AgdaSymbol{(}\AgdaField{snd}\AgdaSpace{}%
\AgdaBound{G}\AgdaSymbol{)}\<%
\\
%
\>[4]\AgdaKeyword{module}\AgdaSpace{}%
\AgdaModule{H}\AgdaSpace{}%
\AgdaSymbol{=}\AgdaSpace{}%
\AgdaModule{GroupStr}\AgdaSpace{}%
\AgdaSymbol{(}\AgdaField{snd}\AgdaSpace{}%
\AgdaBound{H}\AgdaSymbol{)}\<%
\\
%
\\[\AgdaEmptyExtraSkip]%
%
\>[2]\AgdaKeyword{record}\AgdaSpace{}%
\AgdaRecord{GroupHom}\AgdaSpace{}%
\AgdaSymbol{(}\AgdaBound{G}\AgdaSpace{}%
\AgdaSymbol{:}\AgdaSpace{}%
\AgdaFunction{Group}\AgdaSpace{}%
\AgdaSymbol{\{}\AgdaGeneralizable{ℓ}\AgdaSymbol{\})}\AgdaSpace{}%
\AgdaSymbol{(}\AgdaBound{H}\AgdaSpace{}%
\AgdaSymbol{:}\AgdaSpace{}%
\AgdaFunction{Group}\AgdaSpace{}%
\AgdaSymbol{\{}\AgdaGeneralizable{ℓ'}\AgdaSymbol{\})}\AgdaSpace{}%
\AgdaSymbol{:}\AgdaSpace{}%
\AgdaPrimitive{Type}\AgdaSpace{}%
\AgdaSymbol{(}\AgdaPrimitive{ℓ-max}\AgdaSpace{}%
\AgdaBound{ℓ}\AgdaSpace{}%
\AgdaBound{ℓ'}\AgdaSymbol{)}\AgdaSpace{}%
\AgdaKeyword{where}\<%
\\
\>[2][@{}l@{\AgdaIndent{0}}]%
\>[4]\AgdaKeyword{constructor}\AgdaSpace{}%
\AgdaInductiveConstructor{grouphom}\<%
\\
%
\\[\AgdaEmptyExtraSkip]%
%
\>[4]\AgdaKeyword{field}\<%
\\
\>[4][@{}l@{\AgdaIndent{0}}]%
\>[6]\AgdaField{fun}\AgdaSpace{}%
\AgdaSymbol{:}\AgdaSpace{}%
\AgdaOperator{\AgdaFunction{⟨}}\AgdaSpace{}%
\AgdaBound{G}\AgdaSpace{}%
\AgdaOperator{\AgdaFunction{⟩}}\AgdaSpace{}%
\AgdaSymbol{→}\AgdaSpace{}%
\AgdaOperator{\AgdaFunction{⟨}}\AgdaSpace{}%
\AgdaBound{H}\AgdaSpace{}%
\AgdaOperator{\AgdaFunction{⟩}}\<%
\\
%
\>[6]\AgdaField{isHom}\AgdaSpace{}%
\AgdaSymbol{:}\AgdaSpace{}%
\AgdaFunction{isGroupHom}\AgdaSpace{}%
\AgdaBound{G}\AgdaSpace{}%
\AgdaBound{H}\AgdaSpace{}%
\AgdaField{fun}\<%
\end{code}
This actually \emph{was} the Cubical Agda implementation of a group homomorphism
sometime around the end of 2020. We see that, while a mathematician might be
able to infer the meaning of some of the syntax, the use of levels,
distinguising between isGroupHom and GroupHom itself, and many other details
might obscure what's going on.

We finally provide the current (May 2021) definition via Cubical Agda. One may
witness a significant number of differences from the previous version : concrete
syntax differences via changes in camel case, new uses of Group vs GroupStr, as
well as, most significantly, the identity and inverse preservation data not
appearing as corollaries, but part of the definition. Additionally, we had to
refactor the commented lines to those shown below to be compatible with our
outdated version of cubical. These changes reflect interesting syntactic
changes.

\begin{code}%
%
\>[2]\AgdaKeyword{record}\AgdaSpace{}%
\AgdaRecord{IsGroupHom}\AgdaSpace{}%
\AgdaSymbol{\{}\AgdaBound{A}\AgdaSpace{}%
\AgdaSymbol{:}\AgdaSpace{}%
\AgdaPrimitive{Type}\AgdaSpace{}%
\AgdaGeneralizable{ℓ}\AgdaSymbol{\}}\AgdaSpace{}%
\AgdaSymbol{\{}\AgdaBound{B}\AgdaSpace{}%
\AgdaSymbol{:}\AgdaSpace{}%
\AgdaPrimitive{Type}\AgdaSpace{}%
\AgdaGeneralizable{ℓ'}\AgdaSymbol{\}}\<%
\\
\>[2][@{}l@{\AgdaIndent{0}}]%
\>[4]\AgdaSymbol{(}\AgdaBound{M}\AgdaSpace{}%
\AgdaSymbol{:}\AgdaSpace{}%
\AgdaRecord{GroupStr}\AgdaSpace{}%
\AgdaBound{A}\AgdaSymbol{)}\AgdaSpace{}%
\AgdaSymbol{(}\AgdaBound{f}\AgdaSpace{}%
\AgdaSymbol{:}\AgdaSpace{}%
\AgdaBound{A}\AgdaSpace{}%
\AgdaSymbol{→}\AgdaSpace{}%
\AgdaBound{B}\AgdaSymbol{)}\AgdaSpace{}%
\AgdaSymbol{(}\AgdaBound{N}\AgdaSpace{}%
\AgdaSymbol{:}\AgdaSpace{}%
\AgdaRecord{GroupStr}\AgdaSpace{}%
\AgdaBound{B}\AgdaSymbol{)}\<%
\\
%
\>[4]\AgdaSymbol{:}\AgdaSpace{}%
\AgdaPrimitive{Type}\AgdaSpace{}%
\AgdaSymbol{(}\AgdaPrimitive{ℓ-max}\AgdaSpace{}%
\AgdaBound{ℓ}\AgdaSpace{}%
\AgdaBound{ℓ'}\AgdaSymbol{)}\<%
\\
%
\>[4]\AgdaKeyword{where}\<%
\\
%
\\[\AgdaEmptyExtraSkip]%
%
\>[4]\AgdaComment{-- Shorter qualified names}\<%
\\
%
\>[4]\AgdaKeyword{private}\<%
\\
\>[4][@{}l@{\AgdaIndent{0}}]%
\>[6]\AgdaKeyword{module}\AgdaSpace{}%
\AgdaModule{M}\AgdaSpace{}%
\AgdaSymbol{=}\AgdaSpace{}%
\AgdaModule{GroupStr}\AgdaSpace{}%
\AgdaBound{M}\<%
\\
%
\>[6]\AgdaKeyword{module}\AgdaSpace{}%
\AgdaModule{N}\AgdaSpace{}%
\AgdaSymbol{=}\AgdaSpace{}%
\AgdaModule{GroupStr}\AgdaSpace{}%
\AgdaBound{N}\<%
\\
%
\\[\AgdaEmptyExtraSkip]%
%
\>[4]\AgdaKeyword{field}\<%
\\
\>[4][@{}l@{\AgdaIndent{0}}]%
\>[6]\AgdaField{pres·}\AgdaSpace{}%
\AgdaSymbol{:}\AgdaSpace{}%
\AgdaSymbol{(}\AgdaBound{x}\AgdaSpace{}%
\AgdaBound{y}\AgdaSpace{}%
\AgdaSymbol{:}\AgdaSpace{}%
\AgdaBound{A}\AgdaSymbol{)}\AgdaSpace{}%
\AgdaSymbol{→}\AgdaSpace{}%
\AgdaBound{f}\AgdaSpace{}%
\AgdaSymbol{(}\AgdaOperator{\AgdaFunction{M.\AgdaUnderscore{}+\AgdaUnderscore{}}}\AgdaSpace{}%
\AgdaBound{x}\AgdaSpace{}%
\AgdaBound{y}\AgdaSymbol{)}\AgdaSpace{}%
\AgdaOperator{\AgdaFunction{≡}}\AgdaSpace{}%
\AgdaSymbol{(}\AgdaOperator{\AgdaFunction{N.\AgdaUnderscore{}+\AgdaUnderscore{}}}\AgdaSpace{}%
\AgdaSymbol{(}\AgdaBound{f}\AgdaSpace{}%
\AgdaBound{x}\AgdaSymbol{)}\AgdaSpace{}%
\AgdaSymbol{(}\AgdaBound{f}\AgdaSpace{}%
\AgdaBound{y}\AgdaSymbol{))}\<%
\\
%
\>[6]\AgdaField{pres1}\AgdaSpace{}%
\AgdaSymbol{:}\AgdaSpace{}%
\AgdaBound{f}\AgdaSpace{}%
\AgdaFunction{M.0g}\AgdaSpace{}%
\AgdaOperator{\AgdaFunction{≡}}\AgdaSpace{}%
\AgdaFunction{N.0g}\<%
\\
%
\>[6]\AgdaField{presinv}\AgdaSpace{}%
\AgdaSymbol{:}\AgdaSpace{}%
\AgdaSymbol{(}\AgdaBound{x}\AgdaSpace{}%
\AgdaSymbol{:}\AgdaSpace{}%
\AgdaBound{A}\AgdaSymbol{)}\AgdaSpace{}%
\AgdaSymbol{→}\AgdaSpace{}%
\AgdaBound{f}\AgdaSpace{}%
\AgdaSymbol{(}\AgdaOperator{\AgdaFunction{M.-\AgdaUnderscore{}}}\AgdaSpace{}%
\AgdaBound{x}\AgdaSymbol{)}\AgdaSpace{}%
\AgdaOperator{\AgdaFunction{≡}}\AgdaSpace{}%
\AgdaOperator{\AgdaFunction{N.-\AgdaUnderscore{}}}\AgdaSpace{}%
\AgdaSymbol{(}\AgdaBound{f}\AgdaSpace{}%
\AgdaBound{x}\AgdaSymbol{)}\<%
\\
%
\>[6]\AgdaComment{-- pres· : (x y : A) → f (x M.· y) ≡ f x N.· f y}\<%
\\
%
\>[6]\AgdaComment{-- pres1 : f M.1g ≡ N.1g}\<%
\\
%
\>[6]\AgdaComment{-- presinv : (x : A) → f (M.inv x) ≡ N.inv (f x)}\<%
\\
%
\\[\AgdaEmptyExtraSkip]%
%
\>[2]\AgdaFunction{GroupHom'}\AgdaSpace{}%
\AgdaSymbol{:}\AgdaSpace{}%
\AgdaSymbol{(}\AgdaBound{G}\AgdaSpace{}%
\AgdaSymbol{:}\AgdaSpace{}%
\AgdaFunction{Group}\AgdaSpace{}%
\AgdaSymbol{\{}\AgdaGeneralizable{ℓ}\AgdaSymbol{\})}\AgdaSpace{}%
\AgdaSymbol{(}\AgdaBound{H}\AgdaSpace{}%
\AgdaSymbol{:}\AgdaSpace{}%
\AgdaFunction{Group}\AgdaSpace{}%
\AgdaSymbol{\{}\AgdaGeneralizable{ℓ'}\AgdaSymbol{\})}\AgdaSpace{}%
\AgdaSymbol{→}\AgdaSpace{}%
\AgdaPrimitive{Type}\AgdaSpace{}%
\AgdaSymbol{(}\AgdaPrimitive{ℓ-max}\AgdaSpace{}%
\AgdaGeneralizable{ℓ}\AgdaSpace{}%
\AgdaGeneralizable{ℓ'}\AgdaSymbol{)}\<%
\\
%
\>[2]\AgdaComment{-- GroupHom' : (G : Group ℓ) (H : Group ℓ') → Type (ℓ-max ℓ ℓ')}\<%
\\
%
\>[2]\AgdaFunction{GroupHom'}\AgdaSpace{}%
\AgdaBound{G}\AgdaSpace{}%
\AgdaBound{H}\AgdaSpace{}%
\AgdaSymbol{=}\AgdaSpace{}%
\AgdaFunction{Σ[}\AgdaSpace{}%
\AgdaBound{f}\AgdaSpace{}%
\AgdaFunction{∈}\AgdaSpace{}%
\AgdaSymbol{(}\AgdaBound{G}\AgdaSpace{}%
\AgdaSymbol{.}\AgdaField{fst}\AgdaSpace{}%
\AgdaSymbol{→}\AgdaSpace{}%
\AgdaBound{H}\AgdaSpace{}%
\AgdaSymbol{.}\AgdaField{fst}\AgdaSymbol{)}\AgdaSpace{}%
\AgdaFunction{]}\AgdaSpace{}%
\AgdaRecord{IsGroupHom}\AgdaSpace{}%
\AgdaSymbol{(}\AgdaBound{G}\AgdaSpace{}%
\AgdaSymbol{.}\AgdaField{snd}\AgdaSymbol{)}\AgdaSpace{}%
\AgdaBound{f}\AgdaSpace{}%
\AgdaSymbol{(}\AgdaBound{H}\AgdaSpace{}%
\AgdaSymbol{.}\AgdaField{snd}\AgdaSymbol{)}\<%
\end{code}


multiple defs?

\begin{code}[hide]%
\>[0]\AgdaComment{--\{-\# OPTIONS --cubical \#-\}}\<%
\\
\>[0]\AgdaSymbol{\{-\#}\AgdaSpace{}%
\AgdaKeyword{OPTIONS}\AgdaSpace{}%
\AgdaPragma{--cubical}\AgdaSpace{}%
\AgdaPragma{--no-import-sorts}\AgdaSpace{}%
\AgdaPragma{--safe}\AgdaSpace{}%
\AgdaSymbol{\#-\}}\<%
\\
%
\\[\AgdaEmptyExtraSkip]%
\>[0]\AgdaKeyword{module}\AgdaSpace{}%
\AgdaModule{monoid}\AgdaSpace{}%
\AgdaKeyword{where}\<%
\\
%
\\[\AgdaEmptyExtraSkip]%
\>[0]\AgdaKeyword{module}\AgdaSpace{}%
\AgdaModule{Namespace1}\AgdaSpace{}%
\AgdaKeyword{where}\<%
\\
%
\\[\AgdaEmptyExtraSkip]%
\>[0][@{}l@{\AgdaIndent{0}}]%
\>[2]\AgdaKeyword{open}\AgdaSpace{}%
\AgdaKeyword{import}\AgdaSpace{}%
\AgdaModule{Cubical.Foundations.Prelude}\<%
\\
%
\>[2]\AgdaKeyword{open}\AgdaSpace{}%
\AgdaKeyword{import}\AgdaSpace{}%
\AgdaModule{Cubical.Foundations.Equiv}\<%
\\
%
\>[2]\AgdaKeyword{open}\AgdaSpace{}%
\AgdaKeyword{import}\AgdaSpace{}%
\AgdaModule{Cubical.Foundations.Structure}\<%
\\
%
\>[2]\AgdaKeyword{open}\AgdaSpace{}%
\AgdaKeyword{import}\AgdaSpace{}%
\AgdaModule{Cubical.Algebra.Group.Base}\<%
\\
%
\>[2]\AgdaKeyword{open}\AgdaSpace{}%
\AgdaKeyword{import}\AgdaSpace{}%
\AgdaModule{Cubical.Data.Sigma}\<%
\\
%
\\[\AgdaEmptyExtraSkip]%
%
\>[2]\AgdaKeyword{private}\<%
\\
\>[2][@{}l@{\AgdaIndent{0}}]%
\>[4]\AgdaKeyword{variable}\<%
\\
\>[4][@{}l@{\AgdaIndent{0}}]%
\>[6]\AgdaGeneralizable{ℓ}\AgdaSpace{}%
\AgdaGeneralizable{ℓ'}\AgdaSpace{}%
\AgdaGeneralizable{ℓ''}\AgdaSpace{}%
\AgdaGeneralizable{ℓ'''}\AgdaSpace{}%
\AgdaSymbol{:}\AgdaSpace{}%
\AgdaPostulate{Level}\<%
\end{code}
\begin{code}%
%
\>[2]\AgdaFunction{isGroupHom}\AgdaSpace{}%
\AgdaSymbol{:}\AgdaSpace{}%
\AgdaSymbol{(}\AgdaBound{G}\AgdaSpace{}%
\AgdaSymbol{:}\AgdaSpace{}%
\AgdaFunction{Group}\AgdaSpace{}%
\AgdaSymbol{\{}\AgdaGeneralizable{ℓ}\AgdaSymbol{\})}\AgdaSpace{}%
\AgdaSymbol{(}\AgdaBound{H}\AgdaSpace{}%
\AgdaSymbol{:}\AgdaSpace{}%
\AgdaFunction{Group}\AgdaSpace{}%
\AgdaSymbol{\{}\AgdaGeneralizable{ℓ'}\AgdaSymbol{\})}\AgdaSpace{}%
\AgdaSymbol{(}\AgdaBound{f}\AgdaSpace{}%
\AgdaSymbol{:}\AgdaSpace{}%
\AgdaOperator{\AgdaFunction{⟨}}\AgdaSpace{}%
\AgdaBound{G}\AgdaSpace{}%
\AgdaOperator{\AgdaFunction{⟩}}\AgdaSpace{}%
\AgdaSymbol{→}\AgdaSpace{}%
\AgdaOperator{\AgdaFunction{⟨}}\AgdaSpace{}%
\AgdaBound{H}\AgdaSpace{}%
\AgdaOperator{\AgdaFunction{⟩}}\AgdaSymbol{)}\AgdaSpace{}%
\AgdaSymbol{→}\AgdaSpace{}%
\AgdaPrimitive{Type}\AgdaSpace{}%
\AgdaSymbol{\AgdaUnderscore{}}\<%
\\
%
\>[2]\AgdaFunction{isGroupHom}\AgdaSpace{}%
\AgdaBound{G}\AgdaSpace{}%
\AgdaBound{H}\AgdaSpace{}%
\AgdaBound{f}\AgdaSpace{}%
\AgdaSymbol{=}\AgdaSpace{}%
\AgdaSymbol{(}\AgdaBound{x}\AgdaSpace{}%
\AgdaBound{y}\AgdaSpace{}%
\AgdaSymbol{:}\AgdaSpace{}%
\AgdaOperator{\AgdaFunction{⟨}}\AgdaSpace{}%
\AgdaBound{G}\AgdaSpace{}%
\AgdaOperator{\AgdaFunction{⟩}}\AgdaSymbol{)}\AgdaSpace{}%
\AgdaSymbol{→}\AgdaSpace{}%
\AgdaBound{f}\AgdaSpace{}%
\AgdaSymbol{(}\AgdaBound{x}\AgdaSpace{}%
\AgdaOperator{\AgdaFunction{G.+}}\AgdaSpace{}%
\AgdaBound{y}\AgdaSymbol{)}\AgdaSpace{}%
\AgdaOperator{\AgdaFunction{≡}}\AgdaSpace{}%
\AgdaSymbol{(}\AgdaBound{f}\AgdaSpace{}%
\AgdaBound{x}\AgdaSpace{}%
\AgdaOperator{\AgdaFunction{H.+}}\AgdaSpace{}%
\AgdaBound{f}\AgdaSpace{}%
\AgdaBound{y}\AgdaSymbol{)}\AgdaSpace{}%
\AgdaKeyword{where}\<%
\\
\>[2][@{}l@{\AgdaIndent{0}}]%
\>[4]\AgdaKeyword{module}\AgdaSpace{}%
\AgdaModule{G}\AgdaSpace{}%
\AgdaSymbol{=}\AgdaSpace{}%
\AgdaModule{GroupStr}\AgdaSpace{}%
\AgdaSymbol{(}\AgdaField{snd}\AgdaSpace{}%
\AgdaBound{G}\AgdaSymbol{)}\<%
\\
%
\>[4]\AgdaKeyword{module}\AgdaSpace{}%
\AgdaModule{H}\AgdaSpace{}%
\AgdaSymbol{=}\AgdaSpace{}%
\AgdaModule{GroupStr}\AgdaSpace{}%
\AgdaSymbol{(}\AgdaField{snd}\AgdaSpace{}%
\AgdaBound{H}\AgdaSymbol{)}\<%
\\
%
\\[\AgdaEmptyExtraSkip]%
%
\>[2]\AgdaKeyword{record}\AgdaSpace{}%
\AgdaRecord{GroupHom}\AgdaSpace{}%
\AgdaSymbol{(}\AgdaBound{G}\AgdaSpace{}%
\AgdaSymbol{:}\AgdaSpace{}%
\AgdaFunction{Group}\AgdaSpace{}%
\AgdaSymbol{\{}\AgdaGeneralizable{ℓ}\AgdaSymbol{\})}\AgdaSpace{}%
\AgdaSymbol{(}\AgdaBound{H}\AgdaSpace{}%
\AgdaSymbol{:}\AgdaSpace{}%
\AgdaFunction{Group}\AgdaSpace{}%
\AgdaSymbol{\{}\AgdaGeneralizable{ℓ'}\AgdaSymbol{\})}\AgdaSpace{}%
\AgdaSymbol{:}\AgdaSpace{}%
\AgdaPrimitive{Type}\AgdaSpace{}%
\AgdaSymbol{(}\AgdaPrimitive{ℓ-max}\AgdaSpace{}%
\AgdaBound{ℓ}\AgdaSpace{}%
\AgdaBound{ℓ'}\AgdaSymbol{)}\AgdaSpace{}%
\AgdaKeyword{where}\<%
\\
\>[2][@{}l@{\AgdaIndent{0}}]%
\>[4]\AgdaKeyword{constructor}\AgdaSpace{}%
\AgdaInductiveConstructor{grouphom}\<%
\\
%
\\[\AgdaEmptyExtraSkip]%
%
\>[4]\AgdaKeyword{field}\<%
\\
\>[4][@{}l@{\AgdaIndent{0}}]%
\>[6]\AgdaField{fun}\AgdaSpace{}%
\AgdaSymbol{:}\AgdaSpace{}%
\AgdaOperator{\AgdaFunction{⟨}}\AgdaSpace{}%
\AgdaBound{G}\AgdaSpace{}%
\AgdaOperator{\AgdaFunction{⟩}}\AgdaSpace{}%
\AgdaSymbol{→}\AgdaSpace{}%
\AgdaOperator{\AgdaFunction{⟨}}\AgdaSpace{}%
\AgdaBound{H}\AgdaSpace{}%
\AgdaOperator{\AgdaFunction{⟩}}\<%
\\
%
\>[6]\AgdaField{isHom}\AgdaSpace{}%
\AgdaSymbol{:}\AgdaSpace{}%
\AgdaFunction{isGroupHom}\AgdaSpace{}%
\AgdaBound{G}\AgdaSpace{}%
\AgdaBound{H}\AgdaSpace{}%
\AgdaField{fun}\<%
\end{code}
This actually \emph{was} the Cubical Agda implementation of a group homomorphism
sometime around the end of 2020. We see that, while a mathematician might be
able to infer the meaning of some of the syntax, the use of levels,
distinguising between isGroupHom and GroupHom itself, and many other details
might obscure what's going on.

We finally provide the current (May 2021) definition via Cubical Agda. One may
witness a significant number of differences from the previous version : concrete
syntax differences via changes in camel case, new uses of Group vs GroupStr, as
well as, most significantly, the identity and inverse preservation data not
appearing as corollaries, but part of the definition. Additionally, we had to
refactor the commented lines to those shown below to be compatible with our
outdated version of cubical. These changes reflect interesting syntactic
changes.

\begin{code}%
%
\>[2]\AgdaKeyword{record}\AgdaSpace{}%
\AgdaRecord{IsGroupHom}\AgdaSpace{}%
\AgdaSymbol{\{}\AgdaBound{A}\AgdaSpace{}%
\AgdaSymbol{:}\AgdaSpace{}%
\AgdaPrimitive{Type}\AgdaSpace{}%
\AgdaGeneralizable{ℓ}\AgdaSymbol{\}}\AgdaSpace{}%
\AgdaSymbol{\{}\AgdaBound{B}\AgdaSpace{}%
\AgdaSymbol{:}\AgdaSpace{}%
\AgdaPrimitive{Type}\AgdaSpace{}%
\AgdaGeneralizable{ℓ'}\AgdaSymbol{\}}\<%
\\
\>[2][@{}l@{\AgdaIndent{0}}]%
\>[4]\AgdaSymbol{(}\AgdaBound{M}\AgdaSpace{}%
\AgdaSymbol{:}\AgdaSpace{}%
\AgdaRecord{GroupStr}\AgdaSpace{}%
\AgdaBound{A}\AgdaSymbol{)}\AgdaSpace{}%
\AgdaSymbol{(}\AgdaBound{f}\AgdaSpace{}%
\AgdaSymbol{:}\AgdaSpace{}%
\AgdaBound{A}\AgdaSpace{}%
\AgdaSymbol{→}\AgdaSpace{}%
\AgdaBound{B}\AgdaSymbol{)}\AgdaSpace{}%
\AgdaSymbol{(}\AgdaBound{N}\AgdaSpace{}%
\AgdaSymbol{:}\AgdaSpace{}%
\AgdaRecord{GroupStr}\AgdaSpace{}%
\AgdaBound{B}\AgdaSymbol{)}\<%
\\
%
\>[4]\AgdaSymbol{:}\AgdaSpace{}%
\AgdaPrimitive{Type}\AgdaSpace{}%
\AgdaSymbol{(}\AgdaPrimitive{ℓ-max}\AgdaSpace{}%
\AgdaBound{ℓ}\AgdaSpace{}%
\AgdaBound{ℓ'}\AgdaSymbol{)}\<%
\\
%
\>[4]\AgdaKeyword{where}\<%
\\
%
\\[\AgdaEmptyExtraSkip]%
%
\>[4]\AgdaComment{-- Shorter qualified names}\<%
\\
%
\>[4]\AgdaKeyword{private}\<%
\\
\>[4][@{}l@{\AgdaIndent{0}}]%
\>[6]\AgdaKeyword{module}\AgdaSpace{}%
\AgdaModule{M}\AgdaSpace{}%
\AgdaSymbol{=}\AgdaSpace{}%
\AgdaModule{GroupStr}\AgdaSpace{}%
\AgdaBound{M}\<%
\\
%
\>[6]\AgdaKeyword{module}\AgdaSpace{}%
\AgdaModule{N}\AgdaSpace{}%
\AgdaSymbol{=}\AgdaSpace{}%
\AgdaModule{GroupStr}\AgdaSpace{}%
\AgdaBound{N}\<%
\\
%
\\[\AgdaEmptyExtraSkip]%
%
\>[4]\AgdaKeyword{field}\<%
\\
\>[4][@{}l@{\AgdaIndent{0}}]%
\>[6]\AgdaField{pres·}\AgdaSpace{}%
\AgdaSymbol{:}\AgdaSpace{}%
\AgdaSymbol{(}\AgdaBound{x}\AgdaSpace{}%
\AgdaBound{y}\AgdaSpace{}%
\AgdaSymbol{:}\AgdaSpace{}%
\AgdaBound{A}\AgdaSymbol{)}\AgdaSpace{}%
\AgdaSymbol{→}\AgdaSpace{}%
\AgdaBound{f}\AgdaSpace{}%
\AgdaSymbol{(}\AgdaOperator{\AgdaFunction{M.\AgdaUnderscore{}+\AgdaUnderscore{}}}\AgdaSpace{}%
\AgdaBound{x}\AgdaSpace{}%
\AgdaBound{y}\AgdaSymbol{)}\AgdaSpace{}%
\AgdaOperator{\AgdaFunction{≡}}\AgdaSpace{}%
\AgdaSymbol{(}\AgdaOperator{\AgdaFunction{N.\AgdaUnderscore{}+\AgdaUnderscore{}}}\AgdaSpace{}%
\AgdaSymbol{(}\AgdaBound{f}\AgdaSpace{}%
\AgdaBound{x}\AgdaSymbol{)}\AgdaSpace{}%
\AgdaSymbol{(}\AgdaBound{f}\AgdaSpace{}%
\AgdaBound{y}\AgdaSymbol{))}\<%
\\
%
\>[6]\AgdaField{pres1}\AgdaSpace{}%
\AgdaSymbol{:}\AgdaSpace{}%
\AgdaBound{f}\AgdaSpace{}%
\AgdaFunction{M.0g}\AgdaSpace{}%
\AgdaOperator{\AgdaFunction{≡}}\AgdaSpace{}%
\AgdaFunction{N.0g}\<%
\\
%
\>[6]\AgdaField{presinv}\AgdaSpace{}%
\AgdaSymbol{:}\AgdaSpace{}%
\AgdaSymbol{(}\AgdaBound{x}\AgdaSpace{}%
\AgdaSymbol{:}\AgdaSpace{}%
\AgdaBound{A}\AgdaSymbol{)}\AgdaSpace{}%
\AgdaSymbol{→}\AgdaSpace{}%
\AgdaBound{f}\AgdaSpace{}%
\AgdaSymbol{(}\AgdaOperator{\AgdaFunction{M.-\AgdaUnderscore{}}}\AgdaSpace{}%
\AgdaBound{x}\AgdaSymbol{)}\AgdaSpace{}%
\AgdaOperator{\AgdaFunction{≡}}\AgdaSpace{}%
\AgdaOperator{\AgdaFunction{N.-\AgdaUnderscore{}}}\AgdaSpace{}%
\AgdaSymbol{(}\AgdaBound{f}\AgdaSpace{}%
\AgdaBound{x}\AgdaSymbol{)}\<%
\\
%
\>[6]\AgdaComment{-- pres· : (x y : A) → f (x M.· y) ≡ f x N.· f y}\<%
\\
%
\>[6]\AgdaComment{-- pres1 : f M.1g ≡ N.1g}\<%
\\
%
\>[6]\AgdaComment{-- presinv : (x : A) → f (M.inv x) ≡ N.inv (f x)}\<%
\\
%
\\[\AgdaEmptyExtraSkip]%
%
\>[2]\AgdaFunction{GroupHom'}\AgdaSpace{}%
\AgdaSymbol{:}\AgdaSpace{}%
\AgdaSymbol{(}\AgdaBound{G}\AgdaSpace{}%
\AgdaSymbol{:}\AgdaSpace{}%
\AgdaFunction{Group}\AgdaSpace{}%
\AgdaSymbol{\{}\AgdaGeneralizable{ℓ}\AgdaSymbol{\})}\AgdaSpace{}%
\AgdaSymbol{(}\AgdaBound{H}\AgdaSpace{}%
\AgdaSymbol{:}\AgdaSpace{}%
\AgdaFunction{Group}\AgdaSpace{}%
\AgdaSymbol{\{}\AgdaGeneralizable{ℓ'}\AgdaSymbol{\})}\AgdaSpace{}%
\AgdaSymbol{→}\AgdaSpace{}%
\AgdaPrimitive{Type}\AgdaSpace{}%
\AgdaSymbol{(}\AgdaPrimitive{ℓ-max}\AgdaSpace{}%
\AgdaGeneralizable{ℓ}\AgdaSpace{}%
\AgdaGeneralizable{ℓ'}\AgdaSymbol{)}\<%
\\
%
\>[2]\AgdaComment{-- GroupHom' : (G : Group ℓ) (H : Group ℓ') → Type (ℓ-max ℓ ℓ')}\<%
\\
%
\>[2]\AgdaFunction{GroupHom'}\AgdaSpace{}%
\AgdaBound{G}\AgdaSpace{}%
\AgdaBound{H}\AgdaSpace{}%
\AgdaSymbol{=}\AgdaSpace{}%
\AgdaFunction{Σ[}\AgdaSpace{}%
\AgdaBound{f}\AgdaSpace{}%
\AgdaFunction{∈}\AgdaSpace{}%
\AgdaSymbol{(}\AgdaBound{G}\AgdaSpace{}%
\AgdaSymbol{.}\AgdaField{fst}\AgdaSpace{}%
\AgdaSymbol{→}\AgdaSpace{}%
\AgdaBound{H}\AgdaSpace{}%
\AgdaSymbol{.}\AgdaField{fst}\AgdaSymbol{)}\AgdaSpace{}%
\AgdaFunction{]}\AgdaSpace{}%
\AgdaRecord{IsGroupHom}\AgdaSpace{}%
\AgdaSymbol{(}\AgdaBound{G}\AgdaSpace{}%
\AgdaSymbol{.}\AgdaField{snd}\AgdaSymbol{)}\AgdaSpace{}%
\AgdaBound{f}\AgdaSpace{}%
\AgdaSymbol{(}\AgdaBound{H}\AgdaSpace{}%
\AgdaSymbol{.}\AgdaField{snd}\AgdaSymbol{)}\<%
\end{code}


but no inputs from included files ? why. seems to be working now

\section{Introduction}
\label{sec:intro}

The central concern of this thesis is the syntax of mathematics, programming
languages, and their respective mutual influence, as conceived and practiced by
mathematicians and computer scientists.  From one vantage point, the role of
syntax in mathematics may be regarded as a 2nd order concern, a topic for
discussion during a Fika, an artifact of ad hoc development by the working
mathematician whose real goals are producing genuine mathematical knowledge.
For the programmers and computer scientists, syntax may be regarding as a
matter of taste, with friendly debates recurring regarding the use of
semicolons, brackets, and white space.  Yet, when viewed through the lens of
the propositions-as-types paradigm, these discussions intersect in new and
interesting ways.  When one introduces a third paradigm through which to
analyze the use of syntax in mathematics and programming, namely linguistics, I
propose what some may regard as superficial detail, indeed becomes a central
paradigm raising many interesting and important questions. 


\subsection{Beyond Computational Trinitarianism}

\begin{displayquote}

The doctrine of computational trinitarianism holds that computation manifests
itself in three forms: proofs of propositions, programs of a type, and mappings
between structures. These three aspects give rise to three sects of worship:
Logic, which gives primacy to proofs and propositions; Languages, which gives
primacy to programs and types; Categories, which gives primacy to mappings and
structures.\cite{harperTrinity}
\end{displayquote}

We begin this discussion of the three relationships between three respective
fields, mathematics, computer science, and logic. The aptly named 
trinity, shown in \autoref{fig:M1}, are related via both \emph{formal} and \emph{informal}
methods. The propositions as types paradigm, for example, is a heuristic. Yet
it also offers many examples of successful ideas translating between the domains.
Alternatively, the interpretation of a Type Theory(TT) into a category theory is
incredibly \emph{formal}.


\begin{figure}[H]
\centering
\begin{tikzcd}
                                                                            &  &  & Logic \arrow[llldddd, "Denotational\ Semantics" description] \arrow[rrrdddd, "Include\ Terms" description] &  &  &                                                                                                       \\
                                                                            &  &  &                                                                                                            &  &  &                                                                                                       \\
                                                                            &  &  &                                                                                                            &  &  &                                                                                                       \\
                                                                            &  &  &                                                                                                            &  &  &                                                                                                       \\
Math \arrow[rrruuuu, "Embedded\ in\ FOL", bend left] \arrow[rrrrrr, "ITP"'] &  &  &                                                                                                            &  &  & CS \arrow[llllll, "Denotational\ Semantics", bend left] \arrow[llluuuu, "Remove\ Terms"', bend right]
\end{tikzcd}
\caption{The Holy Trinity} \label{fig:M1}
\end{figure}

We hope this thesis will help clarify another possible dimension in this
diagram, that of Linguistics, and call it the ``holy tetrahedron". The different
vertices also resemble
religions in their own right, with communities convinced that they have a
canonical perspective on foundations and the essence of mathematics. Questioning the holy trinity is an act of a heresy, and
it is the goal of this thesis to be a bit heretical by including a much less well understood 
perspective which provides additional challenges and
insights into the trinity.

\begin{figure}[H]
\centering
\begin{tikzcd}
     &  &  & Logic                                                                                                                     &  &  &            \\
     &  &  &                                                                                                                           &  &  &            \\
     &  &  & Linguistics \arrow[uu, "Montague\ Semantics"'] \arrow[llldd, "Distributional\ Semantics"'] \arrow[rrrdd, "TT\ Semantics"] &  &  &            \\
     &  &  &                                                                                                                           &  &  &            \\
Math &  &  &                                                                                                                           &  &  & CS\ (MLTT)
\end{tikzcd}
\caption{Formal Semantics} \label{fig:M2}
\end{figure}

One may see how the trinity give rise to \emph{formal} semantic interpretations
of natural language in \autoref{fig:M2}. Semantics is just one possible
linguistic phenomenon worth investigating in these domains, and could be
replaced by other linguistic paradigms. This thesis is alternatively concerned
with syntax.

Finally, as in \autoref{fig:M3}, we can ask : how does the trinity embed into
natural language? These are the most \emph{informal} arrows of tetrahedron, or
at least one reading of it. One can analyze mathematics using linguistic
methods, or try to give a natural language justification of Intuitionistic Type
Theory (ITT) using Martin-Löf's meaning explanations.

\begin{figure}[H]
\centering
\begin{tikzcd}
                                                &  &  & Logic \arrow[dd, "Embedding"] &  &  &                               \\
                                                &  &  &                               &  &  &                               \\
                                                &  &  & Linguistics                   &  &  &                               \\
                                                &  &  &                               &  &  &                               \\
Math \arrow[rrruu, "Language\ Of\ Mathematics"] &  &  &
&  &  & CS\ (MLTT) \arrow[llluu, "Meaning\ Explanations"]
\end{tikzcd}
\caption{Interpretations of Natural Language} \label{fig:M3}
\end{figure}

In this work, we will see that there are multiple GF grammars which model some
subset of each member of the trinity. Constructing these grammars, and asking
how they can be used in applications for mathematicians, logicians, and computer
scientists is an important practical and philosophical question. Therefore we
hope this attempt at giving the language of mathematics, in particular how
propositions and proofs are expressed and thought about in that language, a
stronger foundation.

\subsection{What is a Homomorophism?}

To get a feel for the syntactic paradigm we explore in this thesis, let us look at a basic mathematical
example: that of a group homomorphism as expressed in by a variety of somewhat
randomly sampled authors.  

% Wikipedia Defn:

\begin{definition}
In mathematics, given two groups, $(G, \ast)$ and $(H, \cdot)$, a group homomorphism from $(G, \ast)$ to $(H, \cdot)$ is a function $h : G \to H$ such that for all $u$ and $v$ in $G$ it holds that
  $$h(u \ast v) = h ( u ) \cdot h ( v )$$ 
\end{definition}

% http://math.mit.edu/~jwellens/Group%20Theory%20Forum.pdf

\begin{definition}
Let $G = (G,\cdot)$ and $G' = (G',\ast)$ be groups, and let $\phi : G \to G'$ be a map between them. We call $\phi$ a \textbf{homomorphism} if for every pair of elements $g, h \in G$, we have 
% \begin{center}
  $$\phi(g \ast h) = \phi ( g ) \cdot \phi ( h )$$ 
% \end{center}
\end{definition}

% http://www.maths.gla.ac.uk/~mwemyss/teaching/3alg1-7.pdf

\begin{definition}\label{def:def3}
Let $G$, $H$, be groups.  A map $\phi : G \to H$ is called a \emph{group homomorphism} if
  $$\phi(xy) = \phi ( x ) \phi ( y )$ for all $x, y \in G$$ 
(Note that $xy$ on the left is formed using the group operation in $G$, whilst the product $\phi ( x ) \phi ( y )$ is formed using the group operation $H$.)
\end{definition}

% NLab:

\begin{definition}\label{def:def4}
Classically, a group is a monoid in which every element has an inverse (necessarily unique).
\end{definition}

We inquire the reader to pay attention to nuance and difference in presentation
that is normally ignored or taken for granted by the fluent mathematician, ask
which definitions feel better, and how the reader herself might present the
definition differently.

If one want to distill the meaning of each of these presentations, there is a
significant amount of subliminal interpretation happening very much analogous to
our innate lingusitic ussage. The inverse and identity are discarded, even
though they are necessary data when defning a group. The order of presentation
of information is inconsistent, as well as the choice to use symbolic or natural
language information. In Definition~\ref{def:def3}, the group operation is used
implicitly, and its clarification a side remark.

Details aside, these all mean the same thing - don't they?  This thesis seeks to provide an
abstract framework to determine whether two lingusitically nuanced presenations
mean the same thing via their syntactic transformations. Obviously these
meanings  are not resolvable in any kind of absolute sense, but at least from a
translational sense. These syntactic transformations come in two flavors : parsing and
linearization, and are natively handled by a Logical Framework (LF) for
specifying grammars : Grammatical Framework (GF).

We now show yet another definition of a group homomorphism formalized in the
Agda programming language:

\begin{code}[hide]%
\>[0]\AgdaComment{--\{-\# OPTIONS --cubical \#-\}}\<%
\\
\>[0]\AgdaSymbol{\{-\#}\AgdaSpace{}%
\AgdaKeyword{OPTIONS}\AgdaSpace{}%
\AgdaPragma{--cubical}\AgdaSpace{}%
\AgdaPragma{--no-import-sorts}\AgdaSpace{}%
\AgdaPragma{--safe}\AgdaSpace{}%
\AgdaSymbol{\#-\}}\<%
\\
%
\\[\AgdaEmptyExtraSkip]%
\>[0]\AgdaKeyword{module}\AgdaSpace{}%
\AgdaModule{monoid}\AgdaSpace{}%
\AgdaKeyword{where}\<%
\\
%
\\[\AgdaEmptyExtraSkip]%
\>[0]\AgdaKeyword{module}\AgdaSpace{}%
\AgdaModule{Namespace1}\AgdaSpace{}%
\AgdaKeyword{where}\<%
\\
%
\\[\AgdaEmptyExtraSkip]%
\>[0][@{}l@{\AgdaIndent{0}}]%
\>[2]\AgdaKeyword{open}\AgdaSpace{}%
\AgdaKeyword{import}\AgdaSpace{}%
\AgdaModule{Cubical.Foundations.Prelude}\<%
\\
%
\>[2]\AgdaKeyword{open}\AgdaSpace{}%
\AgdaKeyword{import}\AgdaSpace{}%
\AgdaModule{Cubical.Foundations.Equiv}\<%
\\
%
\>[2]\AgdaKeyword{open}\AgdaSpace{}%
\AgdaKeyword{import}\AgdaSpace{}%
\AgdaModule{Cubical.Foundations.Structure}\<%
\\
%
\>[2]\AgdaKeyword{open}\AgdaSpace{}%
\AgdaKeyword{import}\AgdaSpace{}%
\AgdaModule{Cubical.Algebra.Group.Base}\<%
\\
%
\>[2]\AgdaKeyword{open}\AgdaSpace{}%
\AgdaKeyword{import}\AgdaSpace{}%
\AgdaModule{Cubical.Data.Sigma}\<%
\\
%
\\[\AgdaEmptyExtraSkip]%
%
\>[2]\AgdaKeyword{private}\<%
\\
\>[2][@{}l@{\AgdaIndent{0}}]%
\>[4]\AgdaKeyword{variable}\<%
\\
\>[4][@{}l@{\AgdaIndent{0}}]%
\>[6]\AgdaGeneralizable{ℓ}\AgdaSpace{}%
\AgdaGeneralizable{ℓ'}\AgdaSpace{}%
\AgdaGeneralizable{ℓ''}\AgdaSpace{}%
\AgdaGeneralizable{ℓ'''}\AgdaSpace{}%
\AgdaSymbol{:}\AgdaSpace{}%
\AgdaPostulate{Level}\<%
\end{code}
\begin{code}%
%
\>[2]\AgdaFunction{isGroupHom}\AgdaSpace{}%
\AgdaSymbol{:}\AgdaSpace{}%
\AgdaSymbol{(}\AgdaBound{G}\AgdaSpace{}%
\AgdaSymbol{:}\AgdaSpace{}%
\AgdaFunction{Group}\AgdaSpace{}%
\AgdaSymbol{\{}\AgdaGeneralizable{ℓ}\AgdaSymbol{\})}\AgdaSpace{}%
\AgdaSymbol{(}\AgdaBound{H}\AgdaSpace{}%
\AgdaSymbol{:}\AgdaSpace{}%
\AgdaFunction{Group}\AgdaSpace{}%
\AgdaSymbol{\{}\AgdaGeneralizable{ℓ'}\AgdaSymbol{\})}\AgdaSpace{}%
\AgdaSymbol{(}\AgdaBound{f}\AgdaSpace{}%
\AgdaSymbol{:}\AgdaSpace{}%
\AgdaOperator{\AgdaFunction{⟨}}\AgdaSpace{}%
\AgdaBound{G}\AgdaSpace{}%
\AgdaOperator{\AgdaFunction{⟩}}\AgdaSpace{}%
\AgdaSymbol{→}\AgdaSpace{}%
\AgdaOperator{\AgdaFunction{⟨}}\AgdaSpace{}%
\AgdaBound{H}\AgdaSpace{}%
\AgdaOperator{\AgdaFunction{⟩}}\AgdaSymbol{)}\AgdaSpace{}%
\AgdaSymbol{→}\AgdaSpace{}%
\AgdaPrimitive{Type}\AgdaSpace{}%
\AgdaSymbol{\AgdaUnderscore{}}\<%
\\
%
\>[2]\AgdaFunction{isGroupHom}\AgdaSpace{}%
\AgdaBound{G}\AgdaSpace{}%
\AgdaBound{H}\AgdaSpace{}%
\AgdaBound{f}\AgdaSpace{}%
\AgdaSymbol{=}\AgdaSpace{}%
\AgdaSymbol{(}\AgdaBound{x}\AgdaSpace{}%
\AgdaBound{y}\AgdaSpace{}%
\AgdaSymbol{:}\AgdaSpace{}%
\AgdaOperator{\AgdaFunction{⟨}}\AgdaSpace{}%
\AgdaBound{G}\AgdaSpace{}%
\AgdaOperator{\AgdaFunction{⟩}}\AgdaSymbol{)}\AgdaSpace{}%
\AgdaSymbol{→}\AgdaSpace{}%
\AgdaBound{f}\AgdaSpace{}%
\AgdaSymbol{(}\AgdaBound{x}\AgdaSpace{}%
\AgdaOperator{\AgdaFunction{G.+}}\AgdaSpace{}%
\AgdaBound{y}\AgdaSymbol{)}\AgdaSpace{}%
\AgdaOperator{\AgdaFunction{≡}}\AgdaSpace{}%
\AgdaSymbol{(}\AgdaBound{f}\AgdaSpace{}%
\AgdaBound{x}\AgdaSpace{}%
\AgdaOperator{\AgdaFunction{H.+}}\AgdaSpace{}%
\AgdaBound{f}\AgdaSpace{}%
\AgdaBound{y}\AgdaSymbol{)}\AgdaSpace{}%
\AgdaKeyword{where}\<%
\\
\>[2][@{}l@{\AgdaIndent{0}}]%
\>[4]\AgdaKeyword{module}\AgdaSpace{}%
\AgdaModule{G}\AgdaSpace{}%
\AgdaSymbol{=}\AgdaSpace{}%
\AgdaModule{GroupStr}\AgdaSpace{}%
\AgdaSymbol{(}\AgdaField{snd}\AgdaSpace{}%
\AgdaBound{G}\AgdaSymbol{)}\<%
\\
%
\>[4]\AgdaKeyword{module}\AgdaSpace{}%
\AgdaModule{H}\AgdaSpace{}%
\AgdaSymbol{=}\AgdaSpace{}%
\AgdaModule{GroupStr}\AgdaSpace{}%
\AgdaSymbol{(}\AgdaField{snd}\AgdaSpace{}%
\AgdaBound{H}\AgdaSymbol{)}\<%
\\
%
\\[\AgdaEmptyExtraSkip]%
%
\>[2]\AgdaKeyword{record}\AgdaSpace{}%
\AgdaRecord{GroupHom}\AgdaSpace{}%
\AgdaSymbol{(}\AgdaBound{G}\AgdaSpace{}%
\AgdaSymbol{:}\AgdaSpace{}%
\AgdaFunction{Group}\AgdaSpace{}%
\AgdaSymbol{\{}\AgdaGeneralizable{ℓ}\AgdaSymbol{\})}\AgdaSpace{}%
\AgdaSymbol{(}\AgdaBound{H}\AgdaSpace{}%
\AgdaSymbol{:}\AgdaSpace{}%
\AgdaFunction{Group}\AgdaSpace{}%
\AgdaSymbol{\{}\AgdaGeneralizable{ℓ'}\AgdaSymbol{\})}\AgdaSpace{}%
\AgdaSymbol{:}\AgdaSpace{}%
\AgdaPrimitive{Type}\AgdaSpace{}%
\AgdaSymbol{(}\AgdaPrimitive{ℓ-max}\AgdaSpace{}%
\AgdaBound{ℓ}\AgdaSpace{}%
\AgdaBound{ℓ'}\AgdaSymbol{)}\AgdaSpace{}%
\AgdaKeyword{where}\<%
\\
\>[2][@{}l@{\AgdaIndent{0}}]%
\>[4]\AgdaKeyword{constructor}\AgdaSpace{}%
\AgdaInductiveConstructor{grouphom}\<%
\\
%
\\[\AgdaEmptyExtraSkip]%
%
\>[4]\AgdaKeyword{field}\<%
\\
\>[4][@{}l@{\AgdaIndent{0}}]%
\>[6]\AgdaField{fun}\AgdaSpace{}%
\AgdaSymbol{:}\AgdaSpace{}%
\AgdaOperator{\AgdaFunction{⟨}}\AgdaSpace{}%
\AgdaBound{G}\AgdaSpace{}%
\AgdaOperator{\AgdaFunction{⟩}}\AgdaSpace{}%
\AgdaSymbol{→}\AgdaSpace{}%
\AgdaOperator{\AgdaFunction{⟨}}\AgdaSpace{}%
\AgdaBound{H}\AgdaSpace{}%
\AgdaOperator{\AgdaFunction{⟩}}\<%
\\
%
\>[6]\AgdaField{isHom}\AgdaSpace{}%
\AgdaSymbol{:}\AgdaSpace{}%
\AgdaFunction{isGroupHom}\AgdaSpace{}%
\AgdaBound{G}\AgdaSpace{}%
\AgdaBound{H}\AgdaSpace{}%
\AgdaField{fun}\<%
\end{code}
This actually \emph{was} the Cubical Agda implementation of a group homomorphism
sometime around the end of 2020. We see that, while a mathematician might be
able to infer the meaning of some of the syntax, the use of levels,
distinguising between isGroupHom and GroupHom itself, and many other details
might obscure what's going on.

We finally provide the current (May 2021) definition via Cubical Agda. One may
witness a significant number of differences from the previous version : concrete
syntax differences via changes in camel case, new uses of Group vs GroupStr, as
well as, most significantly, the identity and inverse preservation data not
appearing as corollaries, but part of the definition. Additionally, we had to
refactor the commented lines to those shown below to be compatible with our
outdated version of cubical. These changes reflect interesting syntactic
changes.

\begin{code}%
%
\>[2]\AgdaKeyword{record}\AgdaSpace{}%
\AgdaRecord{IsGroupHom}\AgdaSpace{}%
\AgdaSymbol{\{}\AgdaBound{A}\AgdaSpace{}%
\AgdaSymbol{:}\AgdaSpace{}%
\AgdaPrimitive{Type}\AgdaSpace{}%
\AgdaGeneralizable{ℓ}\AgdaSymbol{\}}\AgdaSpace{}%
\AgdaSymbol{\{}\AgdaBound{B}\AgdaSpace{}%
\AgdaSymbol{:}\AgdaSpace{}%
\AgdaPrimitive{Type}\AgdaSpace{}%
\AgdaGeneralizable{ℓ'}\AgdaSymbol{\}}\<%
\\
\>[2][@{}l@{\AgdaIndent{0}}]%
\>[4]\AgdaSymbol{(}\AgdaBound{M}\AgdaSpace{}%
\AgdaSymbol{:}\AgdaSpace{}%
\AgdaRecord{GroupStr}\AgdaSpace{}%
\AgdaBound{A}\AgdaSymbol{)}\AgdaSpace{}%
\AgdaSymbol{(}\AgdaBound{f}\AgdaSpace{}%
\AgdaSymbol{:}\AgdaSpace{}%
\AgdaBound{A}\AgdaSpace{}%
\AgdaSymbol{→}\AgdaSpace{}%
\AgdaBound{B}\AgdaSymbol{)}\AgdaSpace{}%
\AgdaSymbol{(}\AgdaBound{N}\AgdaSpace{}%
\AgdaSymbol{:}\AgdaSpace{}%
\AgdaRecord{GroupStr}\AgdaSpace{}%
\AgdaBound{B}\AgdaSymbol{)}\<%
\\
%
\>[4]\AgdaSymbol{:}\AgdaSpace{}%
\AgdaPrimitive{Type}\AgdaSpace{}%
\AgdaSymbol{(}\AgdaPrimitive{ℓ-max}\AgdaSpace{}%
\AgdaBound{ℓ}\AgdaSpace{}%
\AgdaBound{ℓ'}\AgdaSymbol{)}\<%
\\
%
\>[4]\AgdaKeyword{where}\<%
\\
%
\\[\AgdaEmptyExtraSkip]%
%
\>[4]\AgdaComment{-- Shorter qualified names}\<%
\\
%
\>[4]\AgdaKeyword{private}\<%
\\
\>[4][@{}l@{\AgdaIndent{0}}]%
\>[6]\AgdaKeyword{module}\AgdaSpace{}%
\AgdaModule{M}\AgdaSpace{}%
\AgdaSymbol{=}\AgdaSpace{}%
\AgdaModule{GroupStr}\AgdaSpace{}%
\AgdaBound{M}\<%
\\
%
\>[6]\AgdaKeyword{module}\AgdaSpace{}%
\AgdaModule{N}\AgdaSpace{}%
\AgdaSymbol{=}\AgdaSpace{}%
\AgdaModule{GroupStr}\AgdaSpace{}%
\AgdaBound{N}\<%
\\
%
\\[\AgdaEmptyExtraSkip]%
%
\>[4]\AgdaKeyword{field}\<%
\\
\>[4][@{}l@{\AgdaIndent{0}}]%
\>[6]\AgdaField{pres·}\AgdaSpace{}%
\AgdaSymbol{:}\AgdaSpace{}%
\AgdaSymbol{(}\AgdaBound{x}\AgdaSpace{}%
\AgdaBound{y}\AgdaSpace{}%
\AgdaSymbol{:}\AgdaSpace{}%
\AgdaBound{A}\AgdaSymbol{)}\AgdaSpace{}%
\AgdaSymbol{→}\AgdaSpace{}%
\AgdaBound{f}\AgdaSpace{}%
\AgdaSymbol{(}\AgdaOperator{\AgdaFunction{M.\AgdaUnderscore{}+\AgdaUnderscore{}}}\AgdaSpace{}%
\AgdaBound{x}\AgdaSpace{}%
\AgdaBound{y}\AgdaSymbol{)}\AgdaSpace{}%
\AgdaOperator{\AgdaFunction{≡}}\AgdaSpace{}%
\AgdaSymbol{(}\AgdaOperator{\AgdaFunction{N.\AgdaUnderscore{}+\AgdaUnderscore{}}}\AgdaSpace{}%
\AgdaSymbol{(}\AgdaBound{f}\AgdaSpace{}%
\AgdaBound{x}\AgdaSymbol{)}\AgdaSpace{}%
\AgdaSymbol{(}\AgdaBound{f}\AgdaSpace{}%
\AgdaBound{y}\AgdaSymbol{))}\<%
\\
%
\>[6]\AgdaField{pres1}\AgdaSpace{}%
\AgdaSymbol{:}\AgdaSpace{}%
\AgdaBound{f}\AgdaSpace{}%
\AgdaFunction{M.0g}\AgdaSpace{}%
\AgdaOperator{\AgdaFunction{≡}}\AgdaSpace{}%
\AgdaFunction{N.0g}\<%
\\
%
\>[6]\AgdaField{presinv}\AgdaSpace{}%
\AgdaSymbol{:}\AgdaSpace{}%
\AgdaSymbol{(}\AgdaBound{x}\AgdaSpace{}%
\AgdaSymbol{:}\AgdaSpace{}%
\AgdaBound{A}\AgdaSymbol{)}\AgdaSpace{}%
\AgdaSymbol{→}\AgdaSpace{}%
\AgdaBound{f}\AgdaSpace{}%
\AgdaSymbol{(}\AgdaOperator{\AgdaFunction{M.-\AgdaUnderscore{}}}\AgdaSpace{}%
\AgdaBound{x}\AgdaSymbol{)}\AgdaSpace{}%
\AgdaOperator{\AgdaFunction{≡}}\AgdaSpace{}%
\AgdaOperator{\AgdaFunction{N.-\AgdaUnderscore{}}}\AgdaSpace{}%
\AgdaSymbol{(}\AgdaBound{f}\AgdaSpace{}%
\AgdaBound{x}\AgdaSymbol{)}\<%
\\
%
\>[6]\AgdaComment{-- pres· : (x y : A) → f (x M.· y) ≡ f x N.· f y}\<%
\\
%
\>[6]\AgdaComment{-- pres1 : f M.1g ≡ N.1g}\<%
\\
%
\>[6]\AgdaComment{-- presinv : (x : A) → f (M.inv x) ≡ N.inv (f x)}\<%
\\
%
\\[\AgdaEmptyExtraSkip]%
%
\>[2]\AgdaFunction{GroupHom'}\AgdaSpace{}%
\AgdaSymbol{:}\AgdaSpace{}%
\AgdaSymbol{(}\AgdaBound{G}\AgdaSpace{}%
\AgdaSymbol{:}\AgdaSpace{}%
\AgdaFunction{Group}\AgdaSpace{}%
\AgdaSymbol{\{}\AgdaGeneralizable{ℓ}\AgdaSymbol{\})}\AgdaSpace{}%
\AgdaSymbol{(}\AgdaBound{H}\AgdaSpace{}%
\AgdaSymbol{:}\AgdaSpace{}%
\AgdaFunction{Group}\AgdaSpace{}%
\AgdaSymbol{\{}\AgdaGeneralizable{ℓ'}\AgdaSymbol{\})}\AgdaSpace{}%
\AgdaSymbol{→}\AgdaSpace{}%
\AgdaPrimitive{Type}\AgdaSpace{}%
\AgdaSymbol{(}\AgdaPrimitive{ℓ-max}\AgdaSpace{}%
\AgdaGeneralizable{ℓ}\AgdaSpace{}%
\AgdaGeneralizable{ℓ'}\AgdaSymbol{)}\<%
\\
%
\>[2]\AgdaComment{-- GroupHom' : (G : Group ℓ) (H : Group ℓ') → Type (ℓ-max ℓ ℓ')}\<%
\\
%
\>[2]\AgdaFunction{GroupHom'}\AgdaSpace{}%
\AgdaBound{G}\AgdaSpace{}%
\AgdaBound{H}\AgdaSpace{}%
\AgdaSymbol{=}\AgdaSpace{}%
\AgdaFunction{Σ[}\AgdaSpace{}%
\AgdaBound{f}\AgdaSpace{}%
\AgdaFunction{∈}\AgdaSpace{}%
\AgdaSymbol{(}\AgdaBound{G}\AgdaSpace{}%
\AgdaSymbol{.}\AgdaField{fst}\AgdaSpace{}%
\AgdaSymbol{→}\AgdaSpace{}%
\AgdaBound{H}\AgdaSpace{}%
\AgdaSymbol{.}\AgdaField{fst}\AgdaSymbol{)}\AgdaSpace{}%
\AgdaFunction{]}\AgdaSpace{}%
\AgdaRecord{IsGroupHom}\AgdaSpace{}%
\AgdaSymbol{(}\AgdaBound{G}\AgdaSpace{}%
\AgdaSymbol{.}\AgdaField{snd}\AgdaSymbol{)}\AgdaSpace{}%
\AgdaBound{f}\AgdaSpace{}%
\AgdaSymbol{(}\AgdaBound{H}\AgdaSpace{}%
\AgdaSymbol{.}\AgdaField{snd}\AgdaSymbol{)}\<%
\end{code}

While the last two definitions may carry degree of comprehension to a programmer
or mathematician not exposed to Agda, it is certainly comprehensible to a
computer : that is, it typechecks on a computer where Cubical Agda is installed.
While GF is designed for multilingual syntactic transformations and is targeted
for natural language translation, it's underlying theory is largely based on
ideas from the compiler communities. A cousin of the BNF Converter (BNFC), GF is
fully capable of parsing programming languages like Agda! And while the Agda
definitions are just another concrete syntactic presentation of a group
homomorphism, they are distinct from the natural language presentations above in
that the colors indicate it has indeed type checked.

While this example may not exemplify the power of Agda's type-checker, it is of
considerable interest to many. The type-checker has merely assured us that
\term{GroupHom(')} are well-formed types - not that we have a canonical representation
of a group homomorphism. The type-checker is much more useful than is
immediately evident: it delegates the work of verifying that a proof is correct,
that is, the work of judging whether a term has a type, to the computer. While
it's of practical concern is immediate to any exploited grad student grading
papers late on a Sunday night, its theoretical concern has led to many recent
developments in modern mathematics. Thomas Hales solution to the Kepler
Conjecture was seen as unverifiable by those reviewing it, and this led to Hales
outsourcing the verification to Interactive Theorem Provers (ITPs) HOL Light and
Isabelle. This computer delegated verification phase led to many minor
corrections in the original proof which were never spotted due to human
oversight.

Fields medalist Vladimir Voevodsky had the experience of being told one day
his proof of the Milnor conjecture was fatally flawed. Although the leak in the
proof was patched, this experience of temporarily believing much of his life's
work invalidated led him to investigate proof assintants as a tool for future
thought. Indeed, this proof verification error was a key event that led to the
Univalent Foundations
Project~\cite{theunivalentfoundationsprogram-homotopytypetheory-2013}.

While Agda and other programming languages are capable of encoding definitions,
theorems, and proofs, they have so far seen little adoption. In some cases they
have been treated with suspicion and scorn by many mathematicians. This isn't
entirely unfounded : it's a lot of work to learn how to use Agda or Coq,
software updates may cause proofs to break, and the inevitable imperfections we
humans are prone to instilled in these tools . Besides, Martin-Löf Type Theory,
the constructive foundational project which underlies these proof assistants, is
often misunderstood by those who dogmatically accept the law of the excluded
middle as the word of God.

It should be noted, the constructivist rejects neither the law of the excluded
middle, nor ZFC. She merely observes them, and admits their handiness in certain
citations. Excluded middle is indeed a helpful tool as many mathematicians
may attest. The contention is that it should be avoided whenever possible -
proofs which don't rely on it, or it's corallary of proof by contradction, are
much more ameanable to formalization in systems with decideable type checking.
And ZFC, while serving the mathematicians of the early 20th century, is 
lacking when it comes to the higher dimensional structure of n-categories and
infinity groupoids.

What these theorem provers give the mathematician is confidence that her work
is correct, and even more importantly, that the work which she takes for
granted and references in her work is also correct. The task before us is then
one of religious conversion. And one doesn't undertake a conversion by simply
by preaching. Foundational details aside, this thesis is meant to provide a
blueprint for the syntactic reformation that must take place.  

We don't insist a mathematician relinquish the beautiful language she has
come to love in expressing her ideas.  Rather, it asks her to make a
hypothetical compromise
for the time being, and use a Controlled Natural Language (CNL) to develop her
work. In exchange she'll get the confidence that Agda provides. Not only that,
she'll be able to search through a library, to see who else has possibly
already postulated and proved her conjecture. A version of this grandiose vision is 
explored in The Formal Abstracts Project \cite{halesCNL}, and it should
practically motivate work.  

Practicalities aside, this work also attempts to offer a nuanced philosophical
perspective on the matter by exploring why translation of mathematical language,
despite it's seemingly structured form, is difficult. We note that the natural
language definitions of monoid differ in form, but also in pragmatic content.
How one expresses formalities in natural language is incredibly diverse, and
Definition~\ref{def:def4} as compared with the prior homomorphism definitions is
particularly poignant in demonstrating this. These differ very much in nature to
the Agda definitions - especially pragmatically. The differences between the Cubical
Agda definitions may be loosely called pragmatic, in the sense that the choice
of definitions may have downstream effects on readability, maintainability, modularity, and other
considerations when trying to write good code, in a burgeoning area known as proof engineering.

A pragmatic treatment of the language of mathematics is the golden egg if one
wishes to articulate the nuance in how the notions proposition, proof, and
judgment are understood by humans. Nonetheless, this problem is just now seeing
attention. We hope that the treatment of syntax in this thesis, while a long
ways away from giving a pragmatic account of mathematics, will help pave the way
there.


% \include{intro.lagda.tex}
\section{Perspectives}

\begin{displayquote}

...when it comes to understanding the power of mathematical language to guide our
thought and help us reason well, formal mathematical languages like the ones
used by interactive proof assistants provide informative models of informal
mathematical language. The formal languages un- derlying foundational frameworks
such as set theory and type theory were designed to provide an account of the
correct rules of mathematical reasoning, and, as Gödel observed, they do a
remarkably good job. But correctness isn’t everything: we want our mathematical
languages to enable us to reason efficiently and effectively as well. To that
end, we need not just accounts as to what makes a mathematical argument correct,
but also accounts of the structural features of our theorizing that help us
manage mathematical complexity.\cite{avigad2015mathematics}

\end{displayquote}

The key development of this thesis it to explore the formal and informal
distinction of presenting mathematics as understood by mathematicians and computer
scientists by means of rule-based, syntax oriented machine translation.

Computational linguistics, particularly those in the tradition of type
theoretical semantics\cite{ranta1994type}, gives one a way of comparing natural
and programming languages. Type theoretical semantics it is concerned with the
semantics of natural language in the logical tradition of Montague, who
synthesized work in the shadows of Chomsky \cite{Chomsky57} and Frege
\cite{frege79}. This work ended up inspiring the GF system, a side effect of
which was to realize that machine translation was possible as a side effect of
this abstracted view of natural language semantics. Indeed, one such description
of GF is that it is a compiler tool applied to domain specific machine
translation. We may compare the ``compiler view" of PLs and the ``linguistics view"
of NLs, and interpolate this comparison to other general phenomenon in the
respective domains.

We will reference these programming language and linguistic abstraction ladders,
and after viewing \autoref{fig:M1}, the reader should examine this
comparison with her own knowledge and expertise in mind. These respective
ladders are perhaps the most important or lens one should keep in mind while
reading this thesis. Importantly, we should observe that the PL dimension, the
left diagram, represents synthetic processes, those which we design, make
decisions about, and describe formally. Alternatively, the NL abstractions on
the right represent analytic observations. They are therefore are subject to
different, in some ways orthogonal, constraints.

The linguistic abstractions are subject to empirical observations and
constraints, and this diagram only serves as an atlas for the different
abstractions and relations between these abstractions, which may be subject to
modifications depending on the linguist or philosopher investigating such
matters. The PL abstractions as represented, while also an approximations,
serves as an actual high altitude blueprint for the design of programming
languages. While the devil is in the details and this view is greatly
simplified, the representation of PL design is unlikely to create angst in the
computer science communities. The linguistic abstractions are at the
intersection of many fascinating debates between linguists, and there is
certainly nothing close to any type of consensus among linguists which
linguistic abstractions, as well as their hierarchical arrangement, are more
practically useful, theoretically compelling, or empirically testable.

\begin{figure}
\centering
\begin{tikzcd}
Strings \ar[d,"Lexical\ Analysis"] \ar[dd,bend right=+90.0, swap,"Front\ End"]
&[5m]
\\ Lexemes \ar[d,"Parsing"] &[5em]
\\ ASTs \ar[d,"Type\ Checker"] &[5m]
\\ Typed\ ASTs
  \ar[dd, bend left, "Code\ Generator"] 
  \ar[dd, bend right, swap, "Interpreter"] &[5m]
\\ ...
\\ Target\ Language
\end{tikzcd}
\hspace{1cm}
\begin{tikzcd}
  Phonemes \arrow[d, "Morhphophonological
  \\ Anaylsis" description]
  \\ Morphemes \arrow[d, "Parse"]
  \\ \{\ Syntactic\ Representation\ \} \arrow[d, "Montague"', bend right=49]
    \arrow[d, "Ranta", bend left=49] \arrow[d, "..." description]
  \\ {\{\ STLC,\ ...\ ,\ DTT\ \}} \arrow[d, "?" description]
  \\ {\{\ Nearal Encoding\ ,\ ...\ Internal\ Language\ \}} \arrow[d, "?" description]
  \\ Phonemes
\end{tikzcd}
\caption{PL (left) and NL (right) Abstraction Ladders} \label{fig:M1}
\end{figure}


There are also many relevant concerns not addressed in either abstraction chain
that are necessary to give a more comprehsive snapshot. For instance, we may
consider intrinsic and extrensic abstractions that diverge from the idealized
picture. In PL extrensic domain, we can inquire about 

\begin{itemize}

\item systems with multiple interactive programming language 
\item how the programming languages behave with respect to given programs
\item embedding programming languages into one another

\end{itemize}

Alternatively, intrinsic to a given PL, there picture is also not so clear.
Agda, for example, requires the evaluation of terms during typechecking. It is
implemented with 4.5 different stages between the syntax written by the
programmers and the ``fully reflected Abstract Syntax Tree (AST)''. But this
example is perhaps an outlier, because Agda's typechecker is so powerful that
the design, implemenation, and use of Agda revolves around its type-checker,
(which, ironically, is already called during the parsing phase). It is not
anticipated that floating point computation, for instance, would ever be
considered when implementing new features of Agda, at least not for the
foreseeable future. Indeed, the ways Agda represents ASTs were an obstacle
encountered doing this work, because deciding which parsing stage one should connect
to the Portable Grammar Format (PGF) embedding is nontrivial.

% \begin{displayquote}
% \begin{enumerate}
% \item unicode: before parsing
% \item Concrete: after happy parsing Parser/Parser.y that does a little desugaring already---[but ideally shouldn't].  Expressions are not parsed yet.
% \item Concrete.Definitions: the "nice" syntax after the nicifier, preparing for the scope checker
% \item Abstract: after scope checking, mostly desugared, expressions parsed
% \item Internal: after type checking, fully desugared
% \item Reflected: quoted from Internal
% \begin{end}
% \end{displayquote}


Let's zoom in a little and observe the so-called front-end part of the compiler.
Displayed in \autoref{fig:M2} is the highest possible overview of GF. This is
deceptively simple depiction of such a powerful and intricate system.

\begin{figure}
\centering
\begin{tikzcd}
Strings \ar[r,"Lexical\ Analysis"] \ar[rr,bend right,"GF\ Parser"'] &[10em] Lexemes
\ar[r,"Parsing"] &[10em] ASTs \ar[ll,bend right, "GF\ Linearization"] 
\end{tikzcd}
\caption{GF in a nutshell} \label{fig:M2}
\end{figure}

What makes GF so compelling is its ability to translate between
inductively defined languages that type theorists define and relatively
expressive fragments of natural languages, via the composition GF's parsing and
linearization capabilities. It is in some sense the attempt to overlay the
abstraction ladders that led to this development. 

For natural language, some intrinsic properties might take place, if one
chooses, at the neurological level, where one somehow can contrast the internal
language, mechanism of externalization (generally speech), as proposed by
Chomsky. Extrinsically to the linguistic abstractions depicted, pragmatics is
absent, but of incredible interest in the field generally.

The point is to recognize their are stark differences between natural languages
and programming languages which are even more apparent when one gets to certain
abstractions. Classifying both programming languages and natural languages as
languages is best read as an incomplete (and even sometimes contradictory)
metaphor, due to perceived similarities (of which their are ample).

Nonetheless, the point of this thesis is to take a crack at that exact question
: how can one compare programming and natural languages, in the sense that a
natural language, when restricted to a small enough (and presumably
well-behaved) domain, behaves as a programming language. Simultaneously, we
probe the topic of Natural Language Generation (NLG). Given a
logic or type system with some theory inside (say arithmetic over the naturals),
how do we not just find a natural language representation which interprets our
expressions,but also does so in a way that is linguistically coherent in a
sense that a competent speaker can read such an expression and make sense of it.

The specific linguistic domain we focus on, that of mathematics, is a particular
sweet spot at the intersection of these natural and formal language spaces. It should be
noted that this problem, that of translating between formal (in a PL or logic)
and informal (in linguistic sense) mathematics as stated, is both vague and
difficult. It is difficult in both the practical sense, that it may be
either of infeasible complexity or even perhaps undecidable. But it is also
difficult in the philosophical sense, that it a question which one may entertain
arguments which a priori may deflate its effectiveness or meaningfulness.
Like all collective human endeavors, mathematics is a historical
construction - that is, its conventions, notations, understanding,
methodologies, and means of publication and distribution have all been in a
constant flux. While in some sense the mathematics today can be seen today as a
much refined version of whatever the Greeks or Egyptians were doing, there is no
consensus on what mathematics is, how it is to be done, and most relevant for
this treatise, how it is to be expressed.

We present a sketch of the difference of this so-called formal/informal
distinction. Mathematics, that is mathematical constructions like numbers and
geometrical figures, arose out of ad-hoc needs as humans cultures grew and
evolved over the millennia. Indeed, just like many of the most interesting human
developments of which there is a sparsely documented record until relatively
recently, it is likely to remain a mystery what the long historical arc of
mathematics could have looked like in the context of human evolution. And while
mathematical intuitions precede mathematical constructions (the spherical planet
precedes the human use of a ruler compass construction to generate a circle), we
should take it as a starting point that mathematics arises naturally out of our
linguistic capacity. This may very well not be the case, or at least not
universally so, but it is impossible to imagine humans developing mathematical
constructions elaborating anything particularly general without linguistic
faculties. Despite whatever empirical or philosophical dispute one takes with
this linguistic view of mathematical abilities, we seek to make a first order
approximation of our linguistic view for the sake of this work. The discussion around
mathematics relation to linguistics generally, regardless of the stance
one takes, should benefit from this work.

\subsection{Formalization and Informalization}

Formalization is the process of taking an informal piece of natural language
mathematics, embedding it in into a theorem prover, constructing a model, ,
working with types instead of sets. This often requires significant amounts of
work, and if the proof is big enough, warrants the title \emph{proof
engineering}. We note some interesting artifacts about a piece of mathematics
being formalized:

\begin{itemize}

\item it may be formalized differently by two different people in many different ways
\item it may have to be modified, to include hidden lemmas, to correct of an
  error, or other bureaucratic obstacles
\item it may not type check, and only be presumed hypothetically to be 'a
  correct formalization' given evidence 

\end{itemize}

Informalization, on the other hand is a process of taking a piece formal syntax, and turning it into a natural
language utterance, along with commentary motivating and or relating it to other
mathematics. It is a clarification of the meaning of a piece of
code, suppressing certain details and  sometimes
redundantly reiterating other details.

Mathematicians working in either direction know this is a respectable task,
often leading to new methods, abstractions, and research altogether. And just as
any type of machine translation, rule-based or statistical, for a Virginia Woolf
novel or Emily Dickinson poem from English to Mandarin would be entirely
laughable, so-to would the pretense that the methods we explore here using GF
could actually match the competence of mathematicians translating work between a
computer a book. Despite the futility of beating a mathematician at proof
translation, it shouldn't deter those so inspired to try.

\subsection{Syntactic Completeness and Semantic Adequacy}

The GF pipeline, that of bidirectional translation through an intermediary
abstract syntax representation, has, two fundamental criteria that must be
assessed for one to judge the success of an approach over both formalization and
informalization.


The first criterion mentioned above, which we'll call \emph{syntactic completeness}, asks the
following : given an utterance or natural language expression that a
mathematician might understand, does the GF grammar emit a well-formed syntactic
expression in the target logic or programming language? [Example?]

This problem is certain to be infeasible in many cases - a mathematician might
not be able to reconstruct the unstated syntactic details of a proof in an
discipline outside her expertise, it is at worthy pursuit to ask why it is so
difficult! Additionally, one does not know a priori that the generated
expression in the logic has its intended meaning, other than through some meta
verification procedure.

Conversely, given a well formed syntactic expression in, for instance, Agda, one
can ask if the resulting English expression generated by GF is
\emph{semantically adequate}. This notion of semantic adequacy is also delicate,
as mathematicians themselves may dispute, for instance, the proof of a given
proposition or the correct definition of some notion. However, if it is doubtful
that there would be many mathematicians who would not understand some standard
theorem statement and proof in an arbitrary introductory analysis text, even if
one may dispute it's presentation, clarity, pedagogy, or other pedantic details.
Whether one asks that semantic adequacy means some kind of sociological
consensus among those with relevant expertise, or a more relaxed criterion that
some expert herself understands the argument, a dubious perspective in
scientific circles, semantic adequacy should appease at least one and
potentially more mathematicians.

We introduce these terms, syntactic completeness and semantic adequacy, to
highlight a perspective and insight that seems to underlie the biggest
differences between informal and formal mathematics. We claim that mathematics,
as done on a theorem prover, is a syntax oriented endeavor, whereas mathematics,
as practiced by mathematicians, prioritizes semantic understanding. Developing a
system which is able to formalize and informalize utterances which preserve 
syntactic completeness and semantic adequacy, respectively, is
probably infeasible. Even introducing objective criteria to really judge these
terms is likely to infeasible.

This perspective represents an observation, and not intended to judge whether
the syntactic or semantic perspective on mathematics is better - there is a
dialectical phenomena between the two. Let's highlight some advantages both
provide, and try to distinguish more precisely what a syntactic and semantic
perspective may be.

When the Agda user builds her proof, she is outsourcing much of the bookkeeping
to the type-checker. This isn't purely a mechanical process though, she often
does have to think, how her definitions will interact with downstream programs,
as well as whether they are even sensible to begin with (i.e. does this have a
proof). The syntactic side is expressly clear from the readers perspective as
well. If Agda proofs were semantically coherent, one would only need to look at
code, with perhaps a few occasional remarks about various intentions and
conclusions, to understand the mathematics being expressed. Yet, papers are
often written exclusively in Latex, where Agda proofs have had to be reverse
engineered, preserving only semantic details and forsaking syntactic nuance.
Oftentimes the code is kept in the appendix so as to provide a complete
syntactic blueprint. But the act of writing an Agda proof and reading them are
often orthogonal, as the term somehow shadows the application of typing rules
which enable its construction. In some sense, the construction of the proof is
entirely engaged with the types, whereas the human witness of a large term is
either lost as to why it fulfills the typing judgment, they have to reexamine
parts of the proof reasoning in their head or perhaps, again, try to build it
interactively with Agda's help.

Even in cases where Agda code is included in a paper, it is most often the types
which are emphasized and produced. Complex proof terms are seldom to be read on
their own terms. The natural language description and commentary is still
largely necessary to convey whatever results, regardless if the Agda code is
self-contained. And while literate Agda is some type of bridge, it is still the
commentary which in some sense unfolds the code and ultimately makes the Agda
code legible.

This is particularly pronounced in the Coq programming language, where proof
terms are built using Ltac, which can be seen as some kind of imperative
syntactic metaprogramming over the core language, Gallina. The user rarely sees
the internal proof tree that one becomes familiar with in Agda. The tactics are
not typed, often feel very adhoc, and tacticals, sequences of tactics, may carry
very little semantic value (or even possibly muddy one's understanding when
reading proofs with unknown tactics). Indeed, since Ltac isn't itself typed, it
often descends into the sorrows of so-called untyped languages (which are really
uni-typed), and there are recent attempts to change this \cite{mtac2}
\cite{ltac2}. From our perspective, the use of tactics is an additional syntactic obfuscation
of what a proof should look like from the mathematicians perspective - and
attempt to remedy this is. Alecytron is one impressive development in giving Coq
proofs more readability through a interactive back-end which shows the proof
state, and offers other semantically appealing models like interactive graphics
\cite{coqAlec}. This kind of system could and should inspire other proof
assistants to allow for experimentation with syntactic alternative to code.

Tactics obviously have their uses, and sometimes enhance high level proof
understanding, as tactics like \emph{ring} or \emph{omega} often save the reader overhead
of parsing pedantic and uninformative details. And for certain proofs,
especially those involving many hundreds of cases, the metaprogramming
facilities actually give one exclusive advantages not offered to the classical
mathematician using pen and paper. Nonetheless, the dependent type theorist's
dream that all mathematicians begin using theorem provers in their everyday work
is largely just a dream, and with relatively little mainstream adoption by
mathematicians, the future is all but clear.

Mathematicians may indeed like some of the facilities theorem provers provide,
but ultimately, they may not see that as the "essence" of what they are doing.
What is this essence? We will try to shine a small light on perhaps the most
fundamental question in mathematics.

\subsection{What is a proof?}

\begin{displayquote}

A proof is what makes a judgment evident \cite{mlMeanings}.

\end{displayquote}

The proofs of Agda, and any programming language supporting proof development,
are \emph{formal proofs}. Formal proofs have no holes, and while there may very
well be bugs in the underlying technologies supporting these proofs, formal
proofs are seen as some kind of immutable form of data. One could say they
provide \cite{objective evidence} for judgments, which themselves are objective
entities when encoded on a computer. What we call
formal proofs might provide a science fiction writer an interesting thought
experiment as regards communicating mathematics with an alien species incapable
of understanding our language otherwise. Formal proofs, however, certainly don't appease all
mathematicians writing for other mathematicians.

Mathematics, and the act of proving theorems, according to Brouwer is a social
process. And because social processes between humans involve our linguistic
faculties, a we hope to elucidates what a proof with a simplified description.
Suppose we have two humans, $h_1$ and $h_2$. If $h_1$ claims to have a proof
$p_1$, and elaborates it to $p_2$ who claims she can either verify $p_1$ or
reproduce and re-articulate it via $p_1'$, such that $h1$ and $h2$ agree that
$p1$ and $p_1'$ are equivalent, then they have discovered some mathematics. In
fact, in this guise mathematics, can be viewed as a science, even if in fact it
is constructed instead of discovered.

An apt comparison is to see the mathematician is architect, whereas the computer
scientist responsible for formalizing the mathematics is a civil engineer. The
mathematics is the building which, like all human endeavors, is created via
resources and labor of many people. The role of the architect is to envision the
facade, the exterior layer directly perceived by others, giving a building its
character, purpose, and function. The engineer is on the other hand, tasked with
assuring the building gets built, doesn't collapse, and functions with many
implicit features which the user of the building may not notice : the running
water, insulation, and electricity. Whereas the architect is responsible for the
building's \emph{specification}, the engineer is tasked with its
\emph{implementation}.  Informal proofs are specifications and formal proofs are implementations.
Importantly, we are just now at a point historically where our technologies support the
implementations, and modern theorem provers like Coq and Agda provide an incredible
amount of help in realizing the them.

Mathematicians seek model independence in their results (i.e., they don't need a
direct encoding of Fermat's last theorem in set theory in order to trust its
validity). This is one possible reason why there is so much reluctance to adopt
proof assistant, because the implementation of a result in Coq, Agda, or HOL4
may lead to many permutations of the same result, each presumably representing
the same piece of knowledge with. It's also noted a proof doesn't obey the same
universality that it does when it's on paper or verbalized - that Agda 2.6.2,
and its standard library, when updated in the future, may break current
developments, and proofs. While this is a unanimous problem with all software,
we believe the GF approach offers at least a vision of not only linguistic, but
also foundation independence with respect to mathematics.

This thesis examines not just a practical problem, but touches many deep issues in
some space in the intersection of the foundations, both practical and
philosophical, of mathematics, logic, computer science, and their relations
studied via linguistic formalisms. These subjects, and their various relations,
are the subject of countless hours of work and consideration by many great
minds. We barely scratches the surface of a few of these developments,
but it nonetheless, it is hoped, provides a nontrivial perspective at many
important issues.

Recapitulating much of what was said, we hope that the following questions may
have a new perspective :

\begin{itemize}

\item what are mathematical objects
\item   how do their encodings in different foundational formalisms affect their interpretations
\item how does is mathematics develop as a social process
\item how does what mathematics is and how it is done rely on given technologies of a given historical era.
  
\end{itemize}

While various branches of linguistics has seen rapid evolution due to, in large
part, their adoption of mathematical tools, the dual application of linguistic
tools to mathematics is quite sparse and open terrain. We hope the reader can
walk away with an new appreciation to some of these questions and topics after
reading this.

These nuances we will not explore here, but do intend to further elaborate in the
future and and more importantly, inspire other readers to respond accordingly.

Although not given in specific detail, the view of what mathematics is, in both
a philosophical and mathematical sense, as well as from the view of what a
foundational perspective, requires deep consideration in its relation to
linguistics. And while this work is perhaps just a finer grain of sandpaper on
an incomplete and primordial marble sculpture, it is hoped that the sculptor's
own reflection is a little bit more clear after we polish it here.



\section{Previous Work}

The prior exploration of these interleaving subjects is vast, and we can only
sample the available literature here.  

\section{Preliminaries}

We give brief but relevant overviews of the background ideas and tools that went
into the generation of this thesis. 


% from blog post
\subsection{Martin-Löf Type Theory}
\subsubsection{Judgments}

\begin{displayquote}

With Kant, something important happened, namely, that the term judgement, Ger.
Urteil, came to be used instead of proposition \cite{mlMeanings}.

\end{displayquote}

A central contribution of Per Martin-Löf in the development of type theory was
the recognition of the centrality of judgments in logic. Many mathematicians
aren't familiar with the spectrum of judgments available, and merely believe
they are concerned with \emph{the} notion of truth, namely \emph{the truth} of a
mathematical proposition or theorem. There are many judgments one can make which
most mathematicians aren't aware of or at least never mention. Examples of both familiar
and unfamiliar judgments include,

\begin{itemize}

\item $A$ is true
\item $A$ is a proposition
\item $A$ is possible
\item $A$ is necessarily true
\item $A$ is true at time $t$

\end{itemize}

These judgments are understood not in the object language in which we state our
propositions, possibilities, or probabilities, but as assertions in the
metalanguage which require evidence for us to know and believe them. Most
mathematicians may reach for their wallets if I come in and give a talk saying
it is possible that the Riemann Hypothesis is true, partially because they
already know that, and partially because it doesn't seem particularly
interesting to say that something is possible, in the same way that a physicist
may flinch if you say alchemy is possible. Most mathematicians, however, would
agree that $P = NP$ is a proposition, and it is also possible, but isn't true.

For the logician these judgments may well be interesting because their may be
logics in which the discussion of possibility or necessity is even more
interesting than the discussion of truth. And for the type theorist interested
in designing and building programming languages over many various logics, these
judgments become a prime focus. The role of the type-checker in a programming
language is to present evidence for, or decide the validity of the judgments.
The four main judgments of type theory are given in natural language on the left
and symbolically on the right.

\begin{multicols}{2}
\begin{itemize}
\item $T$ is a type
\item $T$ and $T'$ are equal types
\item $t$ is a term of type $T$
\item $t$ and $t'$ are equal terms of type $T$
\item $\vdash T \; {\rm type}$
\item $\vdash T = T'$
\item $\vdash t:T$
\item $\vdash t = t':T$
\end{itemize}
\end{multicols}

Frege's turnstile, $\vdash$, denotes a judgment.

These judgments become much more interesting when we add the ability for them to
be interpreted in a some context with judgment hypotheses. Given a series of
judgments $J_1,...,J_n$, denoted $\Gamma$, where $J_i$ can depend on previously
listed $J's$, we can make judgment $J$ under the hypotheses, e.g. $J_1,...,J_n
\vdash J$. Often these hypotheses $J_i$, alternatively called \emph{antecedents},
denote variables which may occur freely in the *consequent* judgment $J$. For
instance, the antecedent, $x : \mathbb{R}$ occurs freely in the syntactic
expression $\sin x$, a which is given meaning in the judgment $\vdash \sin x { :
} \mathbb{R}$. We write our hypothetical judgement as follows :

$$x : \mathbb{R} \vdash \sin x : \mathbb{R}$$



\subsubsection{Rules}

Martin-Löf systematically used the four fundamental judgments in the proof
theoretic style of Pragwitz. To this end, the intuitionistic formulation of the
logical connectives just gives rules which admit an immediate computational
interpretation. The main types of rules are type formation, introduction,
elimination, and computation rules. The introduction rules for a type admit an
induction principle derivable from that type's signature. Additionally, the
$\beta$ and $\eta$ computation rules are derivable via the composition of
introduction and elimination rules, which, if correctly formulated, should
satisfy a relation known as harmony.

The fundamental notion of the lambda calculus, the function, is 
abstracted over a variable and returns a term of some type when applied to an
argument which is subsequently reduced via the computational rules.
Dependent Type Theory (DTT) generalizes this to allow the return type be
parameterized by the variable being abstracted over. The dependent function
forms the basis of the LF which underlies Agda and GF. 

One reason why hypothetical judgments are so interesting is we can devise rules
which allow us to translate from the metalanguage to the object language using
lambda expressions. These play the role of a function in mathematics and
implication in logic. This comes out in the following introduction rule :

% $$ \frac{\Gamma, x : A \vdash b : B} {\Gamma \vdash \lambda x. b : A \rightarrow
% B} $$

Using this rule, we now see a typical judgment, typical in a field like from
real analysis,

$\vdash \lambda x. \sin x : \R \rightarrow \R$

Equality :

Mathematicians denote this judgement
\begin{align*} f {:} \mathbb{R} &\rightarrow \mathbb{R}\\ x &\mapsto \sin ( x )
\end{align*}

\subsection{Propositions, Sets, and Types}

While the rules of type theory have been well-articulated elsewhere, we provide
briefly compare the syntax of mathematical constructions in FOL, one possible
natural language use \cite{rantaLog}, and MLTT. From this vantage, these look
like simple symbolic manipulations, and in some sense, one doesn't need a the
expressive power of system like GF to parse these to the same form.

% \begin{figure}
% \centering
% \begin{tabular}{|c|c|c|c|} \hline
%   FOL & MLTT & NL FOL & NL MLTT \\ \hline
%   $\forall\ x\ P(x)$ & $\Pi x : \tau.\ P(x)$     & $for\ all\ x,\ p$  & $the product over x in\ p$ \\ hline
%   $\exists\ x\ P(x)$ & $\Sigma x : \tau.\ P(x)$  & $there\ exists\ an\ x\ such\ that\ p$ & $there\ exists\ an\ x\ \in \tau \such \that p$ \\ hline 
%   $p\ \supset\ q$    & $p\ \to\ q$               & $if\ p\ then\ q$   & $p to q$ \\ hline
%   $p\ \wedge\ q$     & $p\ \times\ q$            & $p\ and\ q$        & $the product of p and q$ \\ hline
%   $p\ \lor\ q$       & $p\ +\ q$                 & $p\ or\ q$         & $the coproduct of p and q$ \\ hline
%   $\neg\ p$          & $\neg\ p$                 & $it\ is\ not\ the\ case\ that\ p$ \\ hline  
%   $\top$             & $\top$                    & $true$             & $top$ \\ hline
%   $\bot$             & $\bot$                    & $false$            & $bottom$ \\ hline
%   $p\ =\ q$          & $p\ \equiv\ q$            & $p\ equals\ q$     & $definitionally equal$ \\ hline
% \end{tabular}
% \caption{FOL vs MLTT} \label{fig:M5}
% \end{figure}


% \begin{figure}
% \centering
% \begin{tabular}{|c|c|c|c|} \hline
%   FOL & MLTT & NL FOL & NL MLTT \\ \hline
%   $\forall\ x\ P(x)$ & $\Pi x : \tau.\ P(x)$     & $for\ all\ x,\ p$  & $the\  product\  over\  x\  in\ p$ \\ 
%   $\exists\ x\ P(x)$ & $\Sigma x : \tau.\ P(x)$  & $there\ exists\ an\ x\ such\ that\ p$ & $there\ exists\ an\ x\ in\ \tau such\ that\ p$ \\ 
%   $p\ \supset\ q$    & $p\ \to\ q$               & $if\ p\ then\ q$   & $p\  to\  q$ \\ 
%   $p\ \wedge\ q$     & $p\ \times\ q$            & $p\ and\ q$        & $the\  product\  of\  p\  and\  q$ \\ 
%   $p\ \lor\ q$       & $p\ +\ q$                 & $p\ or\ q$         & $the\  coproduct\  of\  p\  and\  q$ \\ 
%   $\neg\ p$          & $\neg\ p$                 & $it\ is\ not\ the\ case\ that\ p$ & $not\ p$ \\ 
%   $\top$             & $\top$                    & $true$             & $top$ \\ 
%   $\bot$             & $\bot$                    & $false$            & $bottom$ \\ 
%   $p\ =\ q$          & $p\ \equiv\ q$            & $p\ equals\ q$     & $definitionally\  equal$ \\ 
% \end{tabular}
% \caption{FOL vs MLTT} \label{fig:M5}
% \end{figure}


% \begin{multicols}{3}
%   \begin{itemize}
%     \item $\forall\ x\ P(x)$
%     \item $\exists\ x\ P(x)$
%     \item $p\ \supset\ q$
%     \item $p\ \wedge\ q$
%     \item $p\ \lor\ q$
%     \item $\neg\ p$
%     \item $\top$
%     \item $\bot$
%     \item $p\ =\ q$
%     \item $\Pi x : \tau.\ P(x)$
%     \item $\Sigma x : \tau.\ P(x)$
%     \item $p\ \to\ q$
%     \item $p\ \times\ q$
%     \item $p\ +\ q$
%     \item $\neg\ p$
%     \item $\top$
%     \item $\bot$
%     \item $p\ \equiv\ q$
%     \item $for\ all\ x,\ p$
%     \item $there\ exists\ an\ x\ such\ that\ p$
%     \item $if\ p\ then\ q$
%     \item $p\ and\ q$
%     \item $p\ or\ q$
%     \item $it\ is\ not\ the\ case\ that\ p$
%     \item $true$
%     \item $false$
%     \item $p\ equals\ q$
%   \end{itemize}
% \end{multicols}



Additionally, it is worth comparing the type theoretic and natural language
syntax with set theory. Now we bear witness to some deeper cracks than were
visible above. We note that the type theoretic syntax is \emph{the same} in both
tables, whereas the set theoretic and logical syntax shares no overlap. This is
because set theory and first order logic are treated as distinct domains,
whereas in vanilla MLTT, there is no distinguishing mathematical types from
logical types - everything is a type.

\begin{figure}[H]
\centering
\begin{tabular}{|c|c|c|c|} \hline
  FOL & MLTT & NL FOL & NL MLTT \\ \hline
  $\forall\ x\ P(x)$ & $\Pi x : \tau.\ P(x)$     & $for\ all\ x,\ p$  & $the\  product\  over\  x\  in\ p$ \\ 
  $\exists\ x\ P(x)$ & $\Sigma x : \tau.\ P(x)$  & $there\ exists\ an\ x\ such\ that\ p$ & $there\ exists\ an\ x\ in\ \tau such\ that\ p$ \\ 
  $p\ \supset\ q$    & $p\ \to\ q$               & $if\ p\ then\ q$   & $p\  to\  q$ \\ 
  $p\ \wedge\ q$     & $p\ \times\ q$            & $p\ and\ q$        & $the\  product\  of\  p\  and\  q$ \\ 
  $p\ \lor\ q$       & $p\ +\ q$                 & $p\ or\ q$         & $the\  coproduct\  of\  p\  and\  q$ \\ 
  $\neg\ p$          & $\neg\ p$                 & $it\ is\ not\ the\ case\ that\ p$ & $not\ p$ \\ 
  $\top$             & $\top$                    & $true$             & $top$ \\ 
  $\bot$             & $\bot$                    & $false$            & $bottom$ \\ 
  $p\ =\ q$          & $p\ \equiv\ q$            & $p\ equals\ q$     & $definitionally\  equal$ \\ \hline
\end{tabular}
\caption{FOL vs MLTT} \label{fig:M5}
\end{figure}


\begin{figure}[H]
\centering
\begin{tabular}{|c|c|c|c|} \hline
 Set Theory & MLTT & NL Set Theory & NL MLTT \\ \hline
 $S$          & $\tau$                 & $the\ set\ S$                     & $the\ type\ \tau$ \\ 
 $\mathbb{N}$ & $Nat$                  & $the\ set\ of\ natural\ numbers$  & $the\ type\ nat$ \\
 $S \times T$ & $S \times T$           & $the\ product\ of\ S\ and\ T$     & $the\  product\  of\  S\  and\  T$ \\
 $S \to T$    & $S \to T$              & $the\ function\ \from\ S\ to\ T$  & $p\  to\  q$ \\
 $\{x|P(x)\}$ & $\Sigma x : \tau.\ P(x)$ & $the\ set\ of\ x\ such\ that\ P$  & $there\ exists\ an\ x\ in\ \tau such\ that\ p$ \\
 $\emptyset$  & $\bot$                 & $the\ empty\ set$                 & $bottom$ \\
 $?$          & $\top$                 & $?$                             & $top$ \\
 $S \cup T$   & $?$                    & $the\ union\ of\ S\ and\ T$       & $?$ \\
 $S \subset T$ & $S <: T$              & $the\ subset\ S\ of\ T$          & $S\ is\ a\ subtype\ of\ T$ \\
 $?$          & $U_1$                  & $?$ & $the\ second\ Universe$        \\ \hline 
\end{tabular}
\caption{Sets vs MLTT} \label{fig:M6}
\end{figure}



% \begin{multicols}{3}
%   \begin{itemize}
%     \item $S$
%     \item $\mathbb{N}$
%     \item $S \times T$
%     \item $S \to T$
%     \item $\{x|P(x)\}$
%     \item $\emptyset$
%     \item $?$
%     \item $S \cup T$
%     \item $S \subset T$
%     \item $?$
%     \item $\tau$
%     \item $Nat$
%     \item $S \times T$
%     \item $S \to T$
%     \item $\Sigma x : \_ . P(x)$
%     \item $\bot$
%     \item $\top$
%     \item $?$
%     \item $S <: T$
%     \item $U_1$
%     \item $the\ set\ S$
%     \item $the\ set\ of\ natural\ numbers$
%     \item $the\ product\ of\ N\ and\ N$
%     \item $the\ function\ \from\ S\ to\ T$
%     \item $the\ set\ of\ x\ such\ that\ P$
%     \item $the\ empty\ set$
%     \item $top$
%     \item $the\ union\ of\ S\ and\ T$
%     \item $the\ subset\ S\ of\ T$
%     \item $the\ second\ Universe$
%   \end{itemize}
% \end{multicols}

The basic types are sometimes simpler to work with, because they are incredibly
expressive, but it also comes at a cost. The union of two sets simply gives a
predicate over the members of the sets, whereas union and intersection types are
often not considered ``core" to type theory, with multiple possible ways of
interpreting how to treat this set-theoretic concept. The behavior of subtypes
and subsets, while related in some ways, also represents a semantic departure
from sets and types. For example, while there can be a greatest type in some
sub-typing schema, there is no notion of a top set. This is why we use the type
theoretic NL syntax when there are question marks in the set theory column.

We also note that pragmatically, type theorists often interchange the logical,
set theoretic, and type theoretic lexicons when describing types. Because the
types were developed to overcome shortcomings of set theory and classical logic,
the lexicons of all three ended up being blended, and in some sense, the type
theorist can substitute certain words that a classical mathematician
wouldn't.  Whereas $p\ implies\ q$ and $function\ from\ X\ to\ Y$ are not to
be mixed, the type theorist may in some sense default to either.
Nontheless, pragmatically speaking, one would never catch a type theorist
saying $Nat implies Nat$ when expressing $Nat\ \to\ Nat$.


Continuing with sets, we compare elements with terms, this time, via examples.


\centering
\begin{tabular}{|c|c|c|c|} \hline
  Set Theory & MLTT & NL Set Theory & NL MLTT \\ \hline

\begin{multicols}{2}
  \begin{itemize}

  \end{itemize}
\end{multicols}

Mathemacians and T


% \begin{columns}

% \begin{column}{0.4 \textwidth}
% \begin{exampleblock}{Sets}
%   \begin{itemize}
%     \item $1$
%     \item $(1,0)$
%   \end{itemize}
% \end{exampleblock}
% \end{column}

% \begin{column}{0.4 \textwidth}
% \begin{block}{Programs}
%   \begin{itemize}
%     \item $suc\ zero$
%     \item $(suc\ zero, zero)$
%   \end{itemize}
% \end{block}

Nonetheless,
there are many nuances this side-by-side comparison doesn't offer. First  

While these differences may 



\section{Grammatical Framework}

\subsection{Thinking about GF}

A grammar specification in GF is actually just an abstract syntax. With an abstract
syntax specified, one can then define various linearization rules which
compositionally evaluate to strings. An Abstract Syntax Tree (AST) may then be
linearized to various strings admitted by different concrete syntaxes.
Conversely, given a string admitted by the language being defined, GF's powerful
parser will generate all the ASTs which linearize to that tree.

When defining a GF pipeline, one merely to construct an abstract syntax file and a
concrete syntax file such that they are coherent. In the abstract, one
specifies the \emph{semantics} of the domain one wants to translate over, which
is ironic, because we normally associate abstract syntax with \emph{just syntax}.
However, because GF was intended for implementing the natural language
phenomena, the types of semantic categories (or sorts) can grow much bigger than is
desirable in a programming language, where minimalism is generally favored.  The
\emph{foods grammar} is the \emph{hello world} of GF, and should be referred to
for those interested in example of how the abstract syntax serves as a semantic
space in non-formal NL applications \cite{ranta2011grammatical}.

Let us revisit the ``tetrahedral doctrine", now restricting our attention to the
subset of linguistics which GF occupies. We first examine how GF fits into the
trinity, as seen in  \autoref{fig:G1}. Immediately, GF abstract syntax with
dependent types can just be seen as an
implementation of MLTT with the added bonus of a parser.  Additionally, GF is a
relatively tame Type Theory, and therefore it would be easy to construct a model
in a general purpose programming language, like Agda.  Embeddings of GF already
exist in Coq [cite FraCoq], Haskell [cite pgf], and MMT [cite mmt],  These applications allow one to use GF's
parser so that a GF AST may be transformed into some kind of
inductively defined tree these languages all support. Future work could involve
modeling GF in Agda would allow one to prove things about GF
meta-theorems about soundness and termination, or perhaps statements about
specific grammars, such as one being unambiguous.

From the logical side, we note that GF's parser specification was done using
inference rules [cite krasimir]. Given the coupling of Context-Free Grammars
(CFGs) and operads (also known as multicategories) [cite lambek, etc], one could use much more
advanced mathematical machinery to articulate and understand GF.  We sketch this
briefly below [refer].

\begin{figure}[H]
\centering
\begin{tikzcd}
     &  &  & Logic                                                                                                                                             &  &  &            \\
     &  &  &                                                                                                                                                   &  &  &            \\
     &  &  & GF \arrow[uu, "GF\ Parser\ Specification"'] \arrow[llldd, "Theory\ of\ Operads"']
     \arrow[rrrdd, "Implementation\ of", bend left] \arrow[rrrdd, "Agda\ Embedding", bend right] &  &  &            \\
     &  &  &                                                                                                                                                   &  &  &            \\
Math &  &  &                                                                                                                                                   &  &  & CS\ (MLTT)
\end{tikzcd}
\caption{Models of GF} \label{fig:G1}
\end{figure}

One can additionally model these domains in GF, which is obviously the main
focus of this work. In \autoref{fig:G2}, we see that there are 3 grammars which
give allow one to translate in these domains. Ranta's grammar from CADE 2011,
built a propositional framework with a core grammar extended with other
categories to capture syntactic nuance. Ranta's grammar from the Stockholm University
mathematics seminar in 2014 took verbatim text from a publication of Peter Aczel
and sought to show that all the syntactic nuance by constructing a grammar
capable of NL translation. Finally, our work takes a BNFC grammar for a real
programming language cubicaltt [cite], GFifies it, producing an unambiguous
grammar.

\begin{figure}[H]
\centering
\begin{tikzcd}
                                              &  &  & Logic \arrow[dd, "Ranta\
                                              Logic\ (CADE\ 11)"] &  &  &                                       \\
                                              &  &  &                                          &  &  &                                       \\
                                              &  &  & GF                                       &  &  &                                       \\
                                              &  &  &                                          &  &  &                                       \\
Math \arrow[rrruu, "Ranta\ (HoTT\ 14)"] &  &  &                                          &  &  & CS\ (MLTT) \arrow[llluu, "cubicalTT"]
\end{tikzcd}
\caption{Trinitarian Grammars} \label{fig:G2}
\end{figure}

While these three grammars offer the most poignant points of comparison between
the computational, logical, and mathematical phenomena they attempt to capture,
we also note that there were many other smaller grammars developed during the
course of this work to supplement and experiment with various ideas presented.
Importantly, the ``Trinitarian Grammars" do not only model these different
domains, but they each do so in a unique way, making compromises and capturing
various linguistic and formal phenomena. The phenomena should be seen on a
spectrum of \emph{semantic adequacy} and \emph{syntactic completeness}, as in
autoref{fig:G3} .  


\begin{figure}[H]
\centering
\begin{tikzcd}
Lexicon\ Size                                                                                                                                          &  &  & Syntactic\ Completeness \\
                                                                                                                                                       &  &  & {}                      \\
                                                                                                                                                       &  &  &                         \\
Spectrum\ of\ GF \arrow[uuu, "Statistical\ Methods?"] \arrow[rrr, "Ranta\ HoTT\ '14"'] \arrow[rrruuu, "cubicalTT"] \arrow[rrruu, "Ranta\ Logic\ '14"'] &  &  & Semantic\ Adequacy
\end{tikzcd}
\caption{The Grammatical Dimension} \label{fig:G3}
\end{figure}

The cubicalTT grammar, seeking syntacitic completeness, only has a pidgin
English syntax, and therefore is only capable of parsing a programming language.
Ranta's HoTT grammar on the other hand, while capable of presenting a
quasi-logical form, would require extensive refactoring in order to transform
the ASTs to something that resembles the ASTs of a programming language. The
Logic grammar, which produces logically coherent and linguistically nuanced
expressions, does not yet cover proofs, and therefore would require additional extensions
to actually express anything a computer might understand, or, alternatively,
theorems capable of impressing a mathematician. Finally, we note that large-scale
coverage of linguistic phenomena for any of these grammars will additionally
need to incorporate statistical methods in some way. 

Before providing perspectives on the grammar design process, it is alo 
When designing grammars, the foremost question one should ask
A few remarks on designing GF grammars should be noted as well. 

The PMCFG class of languages is still quite tame when compared with, for
instance, Turing complete languages. Thus, the `abstract` and `concrete`
coupling tight, the evaluation is quite simple, and the programs tend to write
themselves once the correct types are chosen. This is not to say GF programming
is easier than in other languages, because often there are unforeseen
constraints that the programmer must get used to, limiting the palette available
when writing code. These constraints allow for fast parsing, but greatly limit
the types of programs one often thinks of writing.

\subsection{A Brief Introduction to GF}

GF is a very powerful yet simple system.  While learning the basics may not be
to difficult for the experienced programmer, GF requires the programmer to work
with, in some sense, an incredibly stiff set of constraints compared to general
purpose languages, and therefore its lack of expressiveness requires a different
way of thinking about programming.

The two functions displayed in \autoref{fig:N2}, $Parse : \{Strings\}
\rightarrow \{\{ASTs\}\}$ and $Linearize : \{ASTs\} \rightarrow \{Strings\}$, obey
the important property that :

 $$\forall s \in \{Strings\} \forall x \in (Parse(s)), Linearize(x) \equiv s$$

This seems somewhat natural from the programmers perspective. The limitation on
ASTs to linearize uniquely is actually a benefit, because it saves the user
having to make a choice about a translation (although, again, a statistical
mechanism could alleviate this constraint). We also want our translations to be
well-behaved mathematically, i.e. composing $Linearize$ and $Parse$ ad
infinitum should presumably not diverge.

GF captures languages more expressive than Chomsky's original 
CFG [cite] but is still remains decidable, with parsing in polynomial
time. Which polynomial depends on the grammar [cite krasimir]. 
It comes equipped with 6 basic judgments:

\begin{itemize}[noitemsep]
  \item Abstract : `cat, fun`
  \item Concrete : `lincat, lin, param`
  \item Auxiliary : `oper`
\end{itemize}

There are two judgments in an abstract file, for categories and named functions
defined over those categories, namely \term{cat} and \term{fun}. The categories
are just (succinct) names, and while GF allows dependent types, e.g. categories
which are parameterized over other categories and thereby allow for more
fine-grained semantic distinctions. We will leave these details aside, but do
note that GF's dependent types can be used to implement a programming language
which only parses well-typed terms (and can actually compute with them using
auxiliary declarations).

In a simply typed programming language we can choose categories, for
variables, types and expressions, or what might \term{Var}, \term{Typ}, and
\term{Exp} respectively. One can then define the functions for the simply typed
lambda calculus extended with natural numbers, known as Gödel's T.

\begin{verbatim} 
cat
  Typ ; Exp ; Var ;
fun
  Tarr : Typ -> Typ -> Typ ;
  Tnat : Typ ;

  Evar : Var -> Exp ;
  Elam : Var -> Typ -> Exp -> Exp ;
  Eapp : Exp -> Exp -> Exp ;

  Ezer : Exp ;
  Esuc : Exp -> Exp ;
  Enatrec : Exp -> Exp -> Exp ->  Exp ;

  X : Var ;
  Y : Var ;
  F : Var ;
  IntV : Int -> Var ;
\end{verbatim}

So far we have specified how to form expressions : types built out of
possibly higher order functions
between natural numbers, and expressions built out of lambda and
natural number terms. The variables are kept as a separate syntactic category,
and integers, \term{Int}, are predefined via GF's internals and simply allow one
to parse numeric expressions. One may then define a functional which takes a
function over the natural numbers and returns that function applied to $1$ - the
AST for this expression is :

\begin{verbatim} 
Elam
    F
    Tarr
        Tnat Tnat
      Eapp
        Evar
            F
        Evar
            IntV
                1
\end{verbatim} 

Dual to the abstract syntax there are parallel judgments when defining a concrete
syntax in GF, \term{lincat} and \term{lin} corresponding to \term{cat} and
\term{fun}, respectively. Wher the AST is the specification, the concrete
form is its implementation in a given lanaguage. The \term{lincat} serves to
give \emph{linearization types} which are quite simply either strings, records (or products
which support sub-typing and named fields), or tables (or coproducts) which can
make choices when computing with arbitrarily named parameters, which are
naturally isomorphic to the sets of some finite cardinality. The tables are
actually derivable from the records and their projections, which is how PGF is
defined internally, but they are so fundamental to GF programming and
expressiveness that they merit syntactic distincion.  The \term{lin}
is a term which matches the type signature of the \term{fun} with which it
shares a name. The \term{lincat} constrains the concrete types of the arguments,
and therefore subjects the GF user to how they are used. 

If we assume we are just working with strings, then we can simply define the
functions as recursively concatenating \term{++} strings. The lambda function
for pidgin English then has, as its linearization form as follows :

\begin{verbatim}
lin 
  Elam v t e = "function taking" ++ v ++ "in" ++ t ++ "to" ++ e ;
\end{verbatim}

Once all the relevant functions are giving correct linearizations, one can now
parse and linearize to the abstract syntax tree above the to string ``function
taking f in the natural numbers to the natural numbers to apply f to 1". This is
clearly unnatural for a variety of reasons, but it's an approximation of what
a computer scientist might say. Suppose instead, we choose to linearize this same
expression to a pidgin expression modeled off Haskell's syntax, ``\\ ( f
: nat -> nat ) -> f 1". We should notice the absence of parentheses for
application suggest something more subtle is happening with the linearization
process, for normally programming languages use fixity declarations to avoid
lispy looking code. Here are the linearization functions which allow for
linearization from the above AST :

\begin{verbatim}
lincat
  Typ = TermPrec ;
  Exp = TermPrec ;
lin
  Elam v t e = 
    mkPrec 0 ("\\" ++ parenth (v ++ ":" ++ usePrec 0 t) ++ "->" ++ usePrec 0 e) ;
  Eapp = infixl 2 "" ;
\end{verbatim}

Where did \term{TermPrec}, \term{infixl}, \term{parenth}, \term{mkPrec}, and
\term{usePrec} come from? These are all functions defined in the RGL. We show a
few of them below, thereby introducing the final, main GF judgments \term{param}
and \term{oper} for parameters and operators.

\begin{verbatim}
param 
  Bool = True | False ;
oper
  TermPrec : Type = {s : Str ; p : Prec} ;
  usePrec : Prec -> TermPrec -> Str = \p,x ->
    case lessPrec x.p p of {
      True => parenth x.s ;
      False => parenthOpt x.s
    } ;
  parenth : Str -> Str = \s -> "(" ++ s ++ ")" ;
  parenthOpt : Str -> Str = \s -> variants {s ; "(" ++ s ++ ")"} ;
\end{verbatim}

Parameters in GF, to a first approximation, are simply data types of unary
constructors with finite cardinality. Operators, on the other hand, encode the
logic of GF linearization rules. They are an unnecessary part of the language
because they don't introduce new logical content, but they do allow one to
abstract the function bodies of \term{lin}'s so that one may keep the actual
linearization rules looking clean. Since GF also support \term{oper}
overloading, one can often get away with often deceptively sleek looking
linearizations, and this is a key feature of the RGL. The variants is one of the
ways to encode multiple linearizations forms for a given tree, so here, for
example, we're breaking the nice property from above.

This more or less resembles a typical programming language, with very little
deviation from what when would expect specifying something in twelf.
Nonetheless, because this is both meant to somehow capture the logical form in
addition to the surface appearance of a language, the separation of concerns
leaves the user with an important decision to make regarding how one couples the
linear and abstract syntaxes. There are in some sense two extremes one can take
to get a well performing GF grammar.

Suppose you have a page of text from some random source of length $l$, and you
take it as an exercise to build a GF grammar which translates it. The first
extreme approach you could take would be to give each word in the text to a
unique category, a unique function for each category bearing the word's name,
along with a single really function with $l$ arguments for the whole sequence of
words in the text. One could then verbatim copy the words as just strings with
their corresponding names in the concrete syntax. This overfitted grammar would
fail : it wouldn't scale to other languages, wouldn't cover any texts other than
the one given to it, and wouldn't be at all informative. Alternatively, one
could create a grammar of a two categories $c$ and $s$ with two functions, $f_0
: c$ and $f_1 : c \rightarrow s$, whereby c would be given $n$ fields, each
strings, with the string given at position $i$ in $f_0$ matching $word_i$ from
the text. $f_1$ would merely concatenate it all. This grammar would be similarly
degenerate, despite also parsing the page of text.

This seemingly silly example highlights the most blatant tension the GF grammar
writer will face : how to balance syntactic and semantic content of the grammar
in between the concrete and the abstract syntax. It is also highly
relevant as concerns the domain of translation, for a programming language
with minimal syntax and the mathematicians language in expressing her ideas are
on vastly different sides of this issue.

We claim that syntactically complete grammars are much more easily dealt with
simple abstract syntax. However, to take allow a syntactically complete grammar
to capture semantic nuance and neutrality then humans requires immensely more
work on the concrete side. Semantically adequate grammars on the other hand,
require significantly more attention on the abstract side, because semantically
meaningful expressions often don't generalize - each part of an expressions
exhibits unique behaviors which can't be abstracted to apply to other parts of
the expression. Therefore, producing a syntactically complete expressions which
doesn't overgenerate parses also requires a lot work from the grammar writer.

We hope the subsequent examples will illuminate this tension. The problem with
treating a syntactically oriented domain like type theory with and a semantically
oriented one like mathematics with the same abstract syntax poses very serious
problems, but also highlights the power of other features of GF, like the RGL [cite]
and Haskell embedding PGF [cite].

The GF RGL is a very robust library for parsing grammatically coherent language.
It exists for many different natural languages with a core abstract syntax
shared by all of them. The API allows one to easily construct, sentence level
phrases once the lexicon has been defined, which are also greatly facilitated by
the API. 

PGF, is an embedding of a GF abstract syntax into Haskell, where the categories
are given ``shadow types", so that one can build turn an abstract syntax into (a
possibly massive) Generalized Algebraic Data Type (GADT) \term{Tree} with kind
\codeword{* -> *} where all the functions serve as constructors. If function
\codeword{h} returns category \codeword{c}, the Haskell constructor
\codeword{Gh} returns \codeword{Tree c}.

The PGF API also allows for the Haskell user to call the parse and linearization
functions, so that once the grammar is built, one can use Haskell as an
interface with the outside world. While GF originally was conceived as allowing
computation with ASTs, using a semantic computation judgment \term{def}, this
has approach has largely been overshadowed by Haskell. Once a grammar is
embedded in Haskell, one can use general recursion, monads, and all other types
of bells and whistles produced by the functional programming community to
compute with the embedded ASTs.

We note that this further muddies the water of
what syntax and semantics refer to in the GF lexicon. Although a GF
abstract syntax somehow represents the programmers idealized semantic domain,
once embedded in PGF the trees now may represent syntactic objects to be
evaluated or transformed to some other semantic domain which may or may not
eventually be linked back to a GF linearization.

These are all the main ingredients a GF user will hopefully need to understand
the grammars hereby elaborated, and hopefully these examples will showcase the
full potential of GF for the problem of mathematical translations.

% \subsection{Mathematical Model of GF}
% Note on the construction of free monoids

% Consider a language $L$ we want to represent, and we come up with a model that we
% build as a set of categories and functions over those categories.  Let $Cat(L)$,
% denote the categories.  Also suppose we define functions such that, given an
% ordered list $x_1,...,x_n;y \in Cat(L)$ we define a set of functions,
% $Fun_L(x_1,...,x_n;y)$ defined over the categories. In gf, a function can be
% denoted something like $\phi : x_1 \rightarrow ... \rightarrow x_n$. We may compose these based
% off their arities. So, given a function $\psi \in Fun_L(y_1,,...,y_n;z)$,
% functions $\phi_1,...\phi_n$ such that $\phi_i \in Fun_L(x_{i,1},...,x_{i,m};y_i)}$ 
%  we can plug these functions in together, or nest them such that
% $$\psi \circ (\phi_1,...,\phi_n) : \rightarrow (x_{i,j}) \rightarrow (y_{i})
% \rightarrow Z$$ 

% This is how abstract syntax trees are formed. It is also worth noting that they
% obey an associativity property, namely that 

% \begin{align*}
% &\theta \circ (\psi_1 \circ (\phi_{1,1},...,\phi_{1,k_1}),...,\psi_n \circ
% (\phi_{n,1},...,\phi_{n,k_n}))\\ = &(\theta \circ \psi_1,...\psi_n) \circ (\phi_{1,1},...,\phi_{1,k_1},...,\phi_{n,1},...,\phi_{n,k_n})
% \end{align*}

% This means that trees in GF are invariant as to how they are built - we
% can build a tree from the leaves to the root or vice versa.

% Example : consider the arithmetic grammar of exponentiation, multiplication, and
% addition defined over a single category of natural number expressions, whereby
% the function symbol is to be interpreted as a string and the tensor product,
% $\otimes$ as the concatenation during evaluation. 

% $$\_\^{}\_ : \mathds{N} \to \mathds{N} \to \mathds{N}$$
% $$\_*\_ : \mathds{N} \to \mathds{N} \to \mathds{N}$$
% $$\_+\_ : \mathds{N} \to \mathds{N} \to \mathds{N}$$

% We can think of constructing the trees by partial application, i.e., 

% $(\lambda x.\: 2 \otimes \^{} \otimes x) : \mathds{N} -> \mathds{N}$

% Lets try see the constructions yielding the string $(1 + 2) \^{} (3 * 4)$.

% We can either (i) construct this as the exponent of two fully formed expressions,
% namely a sum and a product applied to some numbers, or we can first apply the
% exponent to the two binary functions, yielding a quaternary function .

% $x ++ y$
% $x \doubleplus y$
% $``x \doubleplus y"$

% \begin{align*}
% &(\lambda x,y.\: x \otimes \^{} \otimes y)\\
% &\hspace{1cm} ((\lambda x,y.\:x \otimes + \otimes y)\; 1\; 2)\\
% &\hspace{1cm} ((\lambda x,y.\: x \otimes * \otimes y)\; 3\; 4) \\
% \mapsto\; &(\lambda x,y.\: x \otimes \^{} \otimes y)\\
% &\hspace{1cm} (1 + 2)\\
% &\hspace{1cm} (3 * 4))\\
% \mapsto\; &((1 + 2) \^{} (3 * 4))\\
% \end{align}

% \begin{align*}
% &((\lambda x,y.\: x \otimes \^{} \otimes y)\\
% &\hspace{1cm} (\lambda x,y.\:x \otimes + \otimes y)\\
% &\hspace{1cm} (\lambda x,y.\: x \otimes * \otimes y)) \\
% &\hspace{1cm} 1\; 2; 3; 4; \\
% \end{align}

% (1 + 2) \^{} (3 * 4)
  

% ((\lambda x,y. x \^{} y)
%   (\lambda x,y. x + y) 
%   (\lambda x,y. x * y))
%     1 2 3 4

% ((\lambda x,y. x + y) \^{} (\lambda x,y. x * y)) 1 2 3 4
% ((\lambda x,y. x + y) \^{} (\lambda x,y. x * y)) 1 2 3 4

% (1 + 2) \^{} (3 * 4)

% and then say
% (\lambda x. 2 \^{} x) (1 + 3) * (4 + 5)
% = 
% (\lambda x. 2 \^{} x) (1 + 3) * (4 + 5)

% $(\lambda x. 2 \wedge x) : \mathds{N} -> \mathds{N}$

% and then apply it to a complex arguement, say 
% (1 + 3) * (4 + 5)
% (\lambda x. 2 ^ x) : Nat -> Nat

% where 


% \lambda y : Pow y 1 : Nat -> Nat

% (times (plus 2 3) (plus 4 5))
% (Pow \circ (1,times)) : Nat -> Nat -> Nat

% (plus 2 3) (plus 4 5)

% can either be 

% 2^(1+3)*(4+5)


%   % \sin {:} \mathbb{R} &\rightarrow \mathbb{R}\\ x &\mapsto \sin ( x )
% % \circ (\phi_1,...,\phi_n) : \rightarrow (x_{i,j}) \rightarrow (y_{i})
% % \rightarrow Z$$ 

% The two functions displayed in, \autoref{fig:N2}.  If we can loosely call String
% the set of strings freely generated osome acan be 

% for now given a single linear presentation $C^{AST}$ , where

% AST_L String_L0 denote the sets GF ASTs and Strings in the languages generated
% by the rules of L's abstract syntax and L0s compositional representation.

% $$Parse : String -> {AST}$$
% $$Linearize : AST -> String$$

% with the important property that given a string s,


% And given an AST a, we can Parse . Linearize a belongs to {AST}

% Now we should explore why the linearizations are interesting. In part, this is
% because they have arisen from the role of grammars have played in the
% intersection and interaction between computer science and linguistics at least
% since Chomsky in the 50s, and they have different understandings and utilities
% in the respective disciplines. These two discplines converge in GF, which allows
% us to talk about natural languages (NLs) from programming languages (PLs)
% perspective.




\subsection{Agda}

Agda is an attempt to formalize Martin-Löf's intensional type theory (reference 1984).
For our purposes, we will only look at what can in some sense be seen as the
kernel of Agda.

\subsection{Natural Language and Mathematics}

...

\section{HoTT Proofs}

\subsection{Why HoTT for natural language?}

We note that all natural language definitions, theorems, and proofs are copied
here verbatim from the HoTT book.  This decision is admittedly arbitrary, but
does have some benefits.  We list some here : 

\begin{itemize}[noitemsep]

\item As the HoTT book was a collaborative effort, it mixes the language of
many individuals and editors, and can be seen as more ``linguistically
neutral''

\item By its very nature HoTT is interdiscplinary, conceived and constructed by
mathematicians, logicians, and computer scientists. It therefore is meant to
interface with all these discplines, and much of the book was indeed formalized
before it was written

\item It has become canonical reference in the field, and therefore benefits
from wide familiarity

\item It is open source, with publically available Latex files free for
modification and distribution

\end{itemize}

The genisis of higher type theory is a somewhat elementary observation : that
the identity type, parameterized by an arbitrary type $A$ and indexed by
elements of $A$, can actually be built iteratively from previous identities.
That is, $A$ may actually already be an identity defined over another type
$A'$, namely $A \defeq x=_{A'} y$ where $x,y:A'$. The key idea is that this
iterating identities admits a homotpical interpretation : 

\begin{itemize}[noitemsep]

\item Types are topological spaces
\item Terms are points in these space

\item Equality types $x=_{A} y$ are paths in $A$ with endpoints $x$ and $y$ in
$A$

\item Iterated equality types are paths between paths, or continuous path
deformations in some higher path space. This is, intuitively, what
mathematicians call a homotopy.

\end{itemize}

To be explicit, given a type $A$, we can form the homotopy $p=_{x=_{A} y}q$
with endpoints $p$ and $q$ inhabiting the path space $x=_{A} y$.

Let's start out by examining the inductive definition of the identity type.  We
present this definition as it appears in section 1.12 of the HoTT book.

\begin{definition}
  The formation rule says that given a type $A:\UU$ and two elements $a,b:A$, we can form the type $(\id[A]{a}{b}):\UU$ in the same universe.
  The basic way to construct an element of $\id{a}{b}$ is to know that $a$ and $b$ are the same.
  Thus, the introduction rule is a dependent function
  \[\refl{} : \prod_{a:A} (\id[A]{a}{a}) \]
  called \define{reflexivity},
  which says that every element of $A$ is equal to itself (in a specified way).  We regard $\refl{a}$ as being the
  constant path %path\indexdef{path!constant}\indexsee{loop!constant}{path, constant}
  at the point $a$.
\end{definition}

We recapitulate this definition in Agda, and treat : 

\begin{code}[hide]

module Id where

\end{code}
\begin{code}

  data _≡'_ {A : Set} : (a b : A) → Set where
    r : (a : A) → a ≡' a

\end{code}

\subsection{An introduction to equality}

There is already some tension brewing : most mathematicains have an intuition
for equality, that of an identitfication between two pieces of information
which intuitively must be the same thing, i.e. $2+2=4$. They might ask, what
does it mean to ``construct an element of $\id{a}{b}$''? For the mathematician
use to thinking in terms of sets $\{\id{a}{b} \mid a,b \in \mathbb{N} \}$ isn't
a well-defined notion. Due to its use of the axiom of extensionality, the set
theoretic notion of equality is, no suprise, extensional.  This means that sets
are identified when they have the same elements, and equality is therefore
external to the notion of set. To inhabit a type means to provide evidence for
that inhabitation. The reflexivity constructor is therefore a means of
providing evidence of an equality. This evidence approach is disctinctly
constructive, and a big reason why classical and constructive mathematics,
especially when treated in an intuitionistic type theory suitable for a
programming language implementation, are such different beasts.

In Martin-Löf Type Theory, there are two fundamental notions of equality,
propositional and definitional.  While propositional equality is inductively
defined (as above) as a type which may have possibly more than one inhabitant,
definitional equality, denoted $-\equiv -$ and perhaps more aptly named
computational equality, is familiarly what most people think of as equality.
Namely, two terms which compute to the same canonical form are computationally
equal. In intensional type theory, propositional equality is a weaker notion
than computational equality : all propositionally equal terms are
computationally equal. However, computational equality does not imply
propistional equality - if it does, then one enters into the space of
extensional type theory. 

Prior to the homotopical interpretation of identity types, debates about
extensional and intensional type theories centred around two features or bugs :
extensional type theory sacrificed decideable type checking, while intensional
type theories required extra beauracracy when dealing with equality in proofs.
One approach in intensional type theories treated types as setoids, therefore
leading to so-called ``Setoid Hell''. These debates reflected Martin-Löf's
flip-flopping on the issue. His seminal 1979 Constructive Mathematics and
Computer Programming, which took an extensional view, was soon betrayed by
lectures he gave soon thereafter in Padova in 1980.  Martin-Löf was a born
again intensional type theorist.  These Padova lectures were later published in
the "Bibliopolis Book", and went on to inspire the European (and Gothenburg in
particular) approach to implementing proof assitants, whereas the
extensionalists were primarily eminating from Robert Constable's group at
Cornell. 

This tension has now been at least partially resolved, or at the very least
clarified, by an insight Voevodsky was apparently most proud of : the
introduction of h-levels. We'll delegate these details for a later section, it
is mentioned here to indicate that extensional type theory was really ``set
theory'' in disguise, in that it collapses the higher path structure of
identity types. The work over the past 10 years has elucidated the intensional
and extensional positions. HoTT, by allowing higher paths, is unashamedly
intentional, and admits a collapse into the extensional universe if so desired.
We now the examine the structure induced by this propositional equality.

\subsection{All about Identity}

We start with a slight reformulation of the identity type, where the element
determining the equality is treated as a parameter rather than an index. This
is a matter of convenience more than taste, as it delegates work for Agda's
typechecker that the programmer may find a distraction. The reflexivity terms
can generally have their endpoints inferred, and therefore cuts down on the
beauracry which often obscures code. 

\begin{code}

  data _≡_ {A : Set} (a : A) : A → Set where
    r : a ≡ a

  infix 20 _≡_

\end{code}

It is of particular concern in this thesis, because it hightlights a
fundamental difference between the lingusitic and the formal approach to proof
presentation.  While the mathematician can whimsically choose to include the
reflexivity arguement or ignore it if she believes it can be inferred, the
programmer can't afford such a laxidasical attitude. Once the type has been
defined, the arguement strcuture is fixed, all future references to the
definition carefully adhere to its specification. The advantage that the
programmer does gain however, that of Agda's powerful inferential abilities,
allows for the insides to be seen via interaction windown. 

Perhaps not of much interest up front, this is incredibly important detail
which the mathematician never has to deal with explicity, but can easily make
type and term translation infeasible due to the fast and loose nature of the
mathematician's writing. Conversely, it may make Natural Language Generation
(NLG) incredibly clunky, adhering to strict rules when created sentences out of
programs. 

[ToDo, give a GF example]

A prime source of beauty in constructive mathematics arises from Gentzen's
recognition of a natural duality in the rules for introducing and using logical
connectives. The mutually coherence between introduction and elmination rules
form the basis of what has since been labeled harmony in a deductive system.
This harmony isn't just an artifact of beauty, it forms the basis for cuts in
proof normalization, and correspondingly, evaluation of terms in a programming
langauge. 

The idea is simple, each new connective, or type former, needs a means of
constructing its terms from its constiutuent parts, yielding introduction
rules. This however, isn't enough - we need a way of dissecting and using the
terms we construct. This yields an elimantion rule which can be uniquely
derived from an inductively defined type. These elimination forms yield
induction principles, or a general notion of proof by induction, when given an
interpration in mathematics. In the non-depedent case, this is known as a
recursion principle, and corresponds to recursion known by programmers far and
wide.  The proof by induction over natural numbers familiar to mathematicians
is just one special case of this induction principle at work--the power of
induction has been recognized and brought to the fore by computer scientists.

We now elaborate the most important induction principle in HoTT, namely, the
induction of an identity type.

\begin{definition}[Version 1]

Moreover, one of the amazing things about homotopy type theory is that all of the basic constructions and axioms---all of the
higher groupoid structure---arises automatically from the induction
principle for identity types.
Recall from [section 1.12]  that this says that if % \cref{sec:identity-types}
  \begin{itemize}[noitemsep]
    \item for every $x,y:A$ and every $p:\id[A]xy$ we have a type $D(x,y,p)$, and
    \item for every $a:A$ we have an element $d(a):D(a,a,\refl a)$,
  \end{itemize}
then
  \begin{itemize}[noitemsep]
    \item there exists an element $\indid{A}(D,d,x,y,p):D(x,y,p)$ for \emph{every}
    two elements $x,y:A$ and $p:\id[A]xy$, such that $\indid{A}(D,d,a,a,\refl a)
    \jdeq d(a)$.
  \end{itemize}
\end{definition}
The book then reiterates this definition, with basically no natural language,
essentially in the raw logical framework devoid of anything but dependent
function types.
\begin{definition}[Version 2]
In other words, given dependent functions
\begin{align*}
  D & :\prod_{(x,y:A)}(x= y) \; \to \; \type\\
  d & :\prod_{a:A} D(a,a,\refl{a})
\end{align*}
there is a dependent function
\[\indid{A}(D,d):\prod_{(x,y:A)}\prod_{(p:\id{x}{y})} D(x,y,p)\]
such that
\begin{equation}\label{eq:Jconv}
\indid{A}(D,d,a,a,\refl{a})\jdeq d(a)
\end{equation}
for every $a:A$.
Usually, every time we apply this induction rule we will either not care about the specific function being defined, or we will immediately give it a different name.

\end{definition}
Again, we define this, in Agda, staying as true to the syntax as possible.
\begin{code}

  J : {A : Set}
      → (D : (x y : A) → (x ≡ y) →  Set)
      → ((a : A) → (D a a r )) -- → (d : (a : A) → (D a a r ))
      → (x y : A)
      → (p : x ≡ y)
      ------------------------------------
      → D x y p
  J D d x .x r = d x

\end{code}

It should be noted that, for instance, we can choose to leave out the $d$ label
on the third line. Indeed minimizing the amount of dependent typing and using
vanilla function types when dependency is not necessary, is generally
considered ``best practice'' Agda, because it will get desugared by the time it
typechecks anyways. For the writer of the text; however, it was convenient to
define $d$ once, as there are not the same constraints on a mathematician
writing in latex. It will again, serve as a nontrivial exercise to deal with
when specifying the grammar, and will be dealt with later [ToDo add section].
It is also of note that we choose to include Martin-Löf's original name $J$, as
this is more common in the computer science literature.

Once the identity type has been defined, it is natural to develop an ``equality
calculus'',  so that we can actually use it in proof's, as well as develop the
higher groupoid structure of types. The first fact, that propositional equality
is an equivalence relation, is well motivated by needs of practical theorem
proving in Agda and the more homotopically minded mathematician. First, we show the symmetry of equality--that paths are reversible.

\begin{lem}\label{lem:opp}
  For every type $A$ and every $x,y:A$ there is a function
  \begin{equation*}
    (x= y)\to(y= x)
  \end{equation*}
  denoted $p\mapsto \opp{p}$, such that $\opp{\refl{x}}\jdeq\refl{x}$ for each $x:A$.
  We call $\opp{p}$ the \define{inverse} of $p$.
  %\indexdef{path!inverse}%
  %\indexdef{inverse!of path}%
  %\index{equality!symmetry of}%a
  %\index{symmetry!of equality}%
\end{lem}

\begin{proof}[First proof]
  Assume given $A:\UU$, and
  let $D:{\textstyle\prod_{(x,y:A)}}(x= y) \; \to \; \type$ be the type family defined by $D(x,y,p)\defeq (y= x)$.
  %$\prod_{(x:A)} \prod_{y:B}$
  In other words, $D$ is a function assigning to any $x,y:A$ and $p:x=y$ a type, namely the type $y=x$.
  Then we have an element
  \begin{equation*}
    d\defeq \lambda x.\ \refl{x}:\prod_{x:A} D(x,x,\refl{x}).
  \end{equation*}
  Thus, the induction principle for identity types gives us an element
  $\indid{A}(D,d,x,y,p): (y= x)$
  for each $p:(x= y)$.
  We can now define the desired function $\opp{(\blank)}$ to be 
  $\lambda p.\ \indid{A}(D,d,x,y,p)$, 
  i.e.\ we set 
  $\opp{p} \defeq \indid{A}(D,d,x,y,p)$.
  The conversion rule [missing reference] %rule~\eqref{eq:Jconv} 
  gives $\opp{\refl{x}}\jdeq \refl{x}$, as required.
\end{proof}
The Agda code is certainly more brief: 
\begin{code}

  _⁻¹ : {A : Set} {x y : A} → x ≡ y → y ≡ x
  _⁻¹ {A} {x} {y} p = J D d x y p
    where
      D : (x y : A) → x ≡ y → Set
      D x y p = y ≡ x
      d : (a : A) → D a a r
      d a = r

  infixr 50 _⁻¹

\end{code}

While first encountering induction principles can be scary, they are actually
more mechanical than one may think. This is due to the the fact that they
uniquely compliment the introduction rules of an an inductive type, and are
simply a means of showing one can ``map out'', or derive an arbitrary type
dependent on the type which has been inductively defined. The mechanical nature
is what allows for Coq's induction tactic, and perhaps even more elegantly,
Agda's pattern matching capabilities. It is always easier to use pattern
matching for the novice Agda programmer, which almost feels like magic.
Nonetheless, for completeness sake, the book uses the induction principle for
much of Chapter 2. And pattern matching is unique to programming languages,
its elegance isn't matched in the mathematicians' lexicon.

Here is the same proof via ``natural language pattern matching'' and Agda
pattern matching:

\begin{proof}[Second proof]
  We want to construct, for each $x,y:A$ and $p:x=y$, an element $\opp{p}:y=x$.
  By induction, it suffices to do this in the case when $y$ is $x$ and $p$ is $\refl{x}$.
  But in this case, the type $x=y$ of $p$ and the type $y=x$ in which we are trying to construct $\opp{p}$ are both simply $x=x$.
  Thus, in the ``reflexivity case'', we can define $\opp{\refl{x}}$ to be simply $\refl{x}$.
  The general case then follows by the induction principle, and the conversion rule $\opp{\refl{x}}\jdeq\refl{x}$ is precisely the proof in the reflexivity case that we gave.
\end{proof}

\begin{code}

  _⁻¹' : {A : Set} {x y : A} → x ≡ y → y ≡ x
  _⁻¹' {A} {x} {y} r = r

\end{code}

Next is trasitivity--concatenation of paths--and we omit the natural language
presentation, which is a slightly more sophisticated arguement than for
symmetry.  


\begin{code}
  _∙_ : {A : Set} → {x y : A} → (p : x ≡ y) → {z : A} → (q : y ≡ z) → x ≡ z
  _∙_ {A} {x} {y} p {z} q = J D d x y p z q
      where
      D : (x₁ y₁ : A) → x₁ ≡ y₁ → Set
      D x y p = (z : A) → (q : y ≡ z) → x ≡ z
      d : (z₁ : A) → D z₁ z₁ r
      d = λ v z q → q

  infixl 40 _∙_
\end{code}

Putting on our spectacles, the reflexivity term serves as evidence of a
constant path for any given point of any given type. To the category theorist,
this makes up the data of an identity map. Likewise, conctanation of paths
starts to look like function composition. This, along with the identity laws
and associativity as proven below, gives us the data of a category. And we have
not only have a category, but the symmetry allows us to prove all paths are
isomorphisms, giving us a groupoid. This isn't a coincedence, it's a very deep
and fascinating articulation of power of the machinery we've so far built. The
fact the path space over a type naturally must satisfies coherence laws in an
even higher path space gives leads to this notion of types as higher groupoids.  

As regards the natural language--at this point, the bookkeeping starts to get hairy.  Paths between paths, and paths between paths between paths, these ideas start to lose geometric inutiotion. And the mathematician often fails to express, when writing $p= q$, that she is already reasoning in a path space. While clever, our brains aren't wired to do too much book-keeping.  Fortunately Agda does this for us, and we can use implicit arguements to avoid our code getting too messy.  [ToDo, add example]

We now proceed to show that we have a groupoid, where the objects are points,
the morphisms are paths. The isomorphisms arise from the path reversal.  Many
of the proofs beyond this point are either routinely made via the induction
principle, or even more routinely by just pattern matching on equality paths,
we omit the details which can be found in the HoTT book, but it is expected
that the GF parser will soon cover such examples.

\begin{code}
  iₗ : {A : Set} {x y : A} (p : x ≡ y) → p ≡ r ∙ p
  iₗ {A} {x} {y} p = J D d x y p 
    where
      D : (x y : A) → x ≡ y → Set
      D x y p = p ≡ r ∙ p
      d : (a : A) → D a a r
      d a = r

  iᵣ : {A : Set} {x y : A} (p : x ≡ y) → p ≡ p ∙ r
  iᵣ {A} {x} {y} p = J D d x y p 
    where
      D : (x y : A) → x ≡ y → Set
      D x y p = p ≡ p ∙ r
      d : (a : A) → D a a r
      d a = r

  leftInverse : {A : Set} {x y : A} (p : x ≡ y) → p ⁻¹ ∙ p ≡ r 
  leftInverse {A} {x} {y} p = J D d x y p
    where
      D : (x y : A) → x ≡ y → Set
      D x y p = p ⁻¹ ∙ p ≡ r
      d : (x : A) → D x x r
      d x = r

  rightInverse : {A : Set} {x y : A} (p : x ≡ y) → p ∙ p ⁻¹ ≡ r 
  rightInverse {A} {x} {y} p = J D d x y p
    where
      D : (x y : A) → x ≡ y → Set
      D x y p = p ∙ p ⁻¹ ≡ r
      d : (a : A) → D a a r
      d a = r

  doubleInv : {A : Set} {x y : A} (p : x ≡ y) → p ⁻¹ ⁻¹ ≡ p
  doubleInv {A} {x} {y} p = J D d x y p
    where
      D : (x y : A) → x ≡ y → Set
      D x y p = p ⁻¹ ⁻¹ ≡ p
      d : (a : A) → D a a r
      d a = r

  associativity :{A : Set} {x y z w : A} (p : x ≡ y) (q : y ≡ z) (r' : z ≡ w ) → p ∙ (q ∙ r') ≡ p ∙ q ∙ r'
  associativity {A} {x} {y} {z} {w} p q r' = J D₁ d₁ x y p z w q r'
    where
      D₁ : (x y : A) → x ≡ y → Set
      D₁ x y p = (z w : A) (q : y ≡ z) (r' : z ≡ w ) → p ∙ (q ∙ r') ≡ p ∙ q ∙ r'
      -- d₁ : (x : A) → D₁ x x r 
      -- d₁ x z w q r' = r -- why can it infer this 
      D₂ : (x z : A) → x ≡ z → Set
      D₂ x z q = (w : A) (r' : z ≡ w ) → r ∙ (q ∙ r') ≡ r ∙ q ∙ r'
      D₃ : (x w : A) → x ≡ w → Set
      D₃ x w r' =  r ∙ (r ∙ r') ≡ r ∙ r ∙ r'
      d₃ : (x : A) → D₃ x x r
      d₃ x = r
      d₂ : (x : A) → D₂ x x r
      d₂ x w r' = J D₃ d₃ x w r' 
      d₁ : (x : A) → D₁ x x r
      d₁ x z w q r' = J D₂ d₂ x z q w r'

\end{code}

When one starts to look at structure above the groupoid level, i.e., the paths between paths between paths level, some interesting and nonintuitive results emerge. If one defines a path space that is seemingly trivial, namely, taking the same starting and end points, the higherdimensional structure yields non-trivial structure. 
We now arrive at the first ``interesting'' result in this book, the Eckmann-Hilton Arguement. It says that composition on the loop space of a loop space, the second loop space, is commutitive.



\begin{definition}

Thus, given a type $A$ with a point $a:A$, we define its \define{loop space}
\index{loop space}%
$\Omega(A,a)$ to be the type $\id[A]{a}{a}$.
We may sometimes write simply $\Omega A$ if the point $a$ is understood from context.

\end {definition}


\begin{definition}
It can also be useful to consider the loop space\index{loop space!iterated}\index{iterated loop space} \emph{of} the loop space of $A$, which is the space of 2-dimensional loops on the identity loop at $a$.
This is written $\Omega^2(A,a)$ and represented in type theory by the type $\id[({\id[A]{a}{a}})]{\refl{a}}{\refl{a}}$.
\end {definition}

\begin{thm}[Eckmann--Hilton]%\label{thm:EckmannHilton}
  The composition operation on the second loop space
  %
  \begin{equation*}
    \Omega^2(A)\times \Omega^2(A)\to \Omega^2(A)
  \end{equation*}
  is commutative: $\alpha\cdot\beta = \beta\cdot\alpha$, for any $\alpha, \beta:\Omega^2(A)$.
  %\index{Eckmann--Hilton argument}%
\end{thm}

\begin{proof}
First, observe that the composition of $1$-loops $\Omega A\times \Omega A\to \Omega A$ induces an operation
\[
\star : \Omega^2(A)\times \Omega^2(A)\to \Omega^2(A)
\]
as follows: consider elements $a, b, c : A$ and 1- and 2-paths,
%
\begin{align*}
 p &: a = b,       &       r &: b = c \\
 q &: a = b,       &       s &: b = c \\
 \alpha &: p = q,  &   \beta &: r = s
\end{align*}
%
as depicted in the following diagram (with paths drawn as arrows).

[TODO Finish Eckmann Hilton Arguement]
%\[
 %\xymatrix@+5em{
   %{a} \rtwocell<10>^p_q{\alpha}
   %&
   %{b} \rtwocell<10>^r_s{\beta}
   %&
   %{c}
 %}
%\]
%Composing the upper and lower 1-paths, respectively, we get two paths $p\ct r,\ q\ct s : a = c$, and there is then a ``horizontal composition''
%%
%\begin{equation*}
  %\alpha\hct\beta : p\ct r = q\ct s
%\end{equation*}
%%
%between them, defined as follows.
%First, we define $\alpha \rightwhisker r : p\ct r = q\ct r$ by path induction on $r$, so that
%\[ \alpha \rightwhisker \refl{b} \jdeq \opp{\mathsf{ru}_p} \ct \alpha \ct \mathsf{ru}_q \]
%where $\mathsf{ru}_p : p = p \ct \refl{b}$ is the right unit law from \cref{thm:omg}\ref{item:omg1}.
%We could similarly define $\rightwhisker$ by induction on $\alpha$, or on all paths in sight, resulting in different judgmental equalities, but for present purposes the definition by induction on $r$ will make things simpler.
%Similarly, we define $q\leftwhisker \beta : q\ct r = q\ct s$ by induction on $q$, so that
%\[ \refl{b} \leftwhisker \beta \jdeq \opp{\mathsf{lu}_r} \ct \beta \ct \mathsf{lu}_s \]
%where $\mathsf{lu}_r$ denotes the left unit law.
%The operations $\leftwhisker$ and $\rightwhisker$ are called \define{whiskering}\indexdef{whiskering}.
%Next, since $\alpha \rightwhisker r$ and $q\leftwhisker \beta$ are composable 2-paths, we can define the \define{horizontal composition}
%\indexdef{horizontal composition!of paths}%
%\indexdef{composition!of paths!horizontal}%
%by:
%\[
%\alpha\hct\beta\ \defeq\ (\alpha\rightwhisker r) \ct (q\leftwhisker \beta).
%\]
%Now suppose that $a \jdeq  b \jdeq  c$, so that all the 1-paths $p$, $q$, $r$, and $s$ are elements of $\Omega(A,a)$, and assume moreover that $p\jdeq q \jdeq r \jdeq s\jdeq \refl{a}$, so that $\alpha:\refl{a} = \refl{a}$ and $\beta:\refl{a} = \refl{a}$ are composable in both orders.
%In that case, we have
%\begin{align*}
  %\alpha\hct\beta
  %&\jdeq (\alpha\rightwhisker\refl{a}) \ct (\refl{a}\leftwhisker \beta)\\
  %&= \opp{\mathsf{ru}_{\refl{a}}} \ct \alpha \ct \mathsf{ru}_{\refl{a}} \ct \opp{\mathsf{lu}_{\refl a}} \ct \beta \ct \mathsf{lu}_{\refl{a}}\\
  %&\jdeq \opp{\refl{\refl{a}}} \ct \alpha \ct \refl{\refl{a}} \ct \opp{\refl{\refl a}} \ct \beta \ct \refl{\refl{a}}\\
  %&= \alpha \ct \beta.
%\end{align*}
%(Recall that $\mathsf{ru}_{\refl{a}} \jdeq \mathsf{lu}_{\refl{a}} \jdeq \refl{\refl{a}}$, by the computation rule for path induction.)
%On the other hand, we can define another horizontal composition analogously by
%\[
%\alpha\hct'\beta\ \defeq\ (p\leftwhisker \beta)\ct (\alpha\rightwhisker s)
%\]
%and we similarly learn that
%\[
%\alpha\hct'\beta = \beta\ct\alpha.
%\]
%\index{interchange law}%
%But, in general, the two ways of defining horizontal composition agree, $\alpha\hct\beta = \alpha\hct'\beta$, as we can see by induction on $\alpha$ and $\beta$ and then on the two remaining 1-paths, to reduce everything to reflexivity.
%Thus we have
%\[\alpha \ct \beta = \alpha\hct\beta = \alpha\hct'\beta = \beta\ct\alpha.
%\qedhere
%\]
\end{proof}


[Todo, clean up code so that its more tightly correspondent to the book proof]
The corresponding agda code is below :

\begin{code}

  -- whiskering
  _∙ᵣ_ : {A : Set} → {b c : A} {a : A} {p q : a ≡ b} (α : p ≡ q) (r' : b ≡ c) → p ∙ r' ≡ q ∙ r'
  _∙ᵣ_ {A} {b} {c} {a} {p} {q} α r' = J D d b c r' a α
    where
      D : (b c : A) → b ≡ c → Set
      D b c r' = (a : A) {p q : a ≡ b} (α : p ≡ q) → p ∙ r' ≡ q ∙ r'
      d : (a : A) → D a a r
      d a a' {p} {q} α = iᵣ p ⁻¹ ∙ α ∙ iᵣ q

  -- iᵣ == ruₚ

  _∙ₗ_ : {A : Set} → {a b : A} (q : a ≡ b) {c : A} {r' s : b ≡ c} (β : r' ≡ s) → q ∙ r' ≡ q ∙ s
  _∙ₗ_ {A} {a} {b} q {c} {r'} {s} β = J D d a b q c β
    where
      D : (a b : A) → a ≡ b → Set
      D a b q = (c : A) {r' s : b ≡ c} (β : r' ≡ s) → q ∙ r' ≡ q ∙ s
      d : (a : A) → D a a r
      d a a' {r'} {s} β = iₗ r' ⁻¹ ∙ β ∙ iₗ s

  _⋆_ : {A : Set} → {a b c : A} {p q : a ≡ b} {r' s : b ≡ c} (α : p ≡ q) (β : r' ≡ s) → p ∙ r' ≡ q ∙ s
  _⋆_ {A} {q = q} {r' = r'} α β = (α ∙ᵣ r') ∙ (q ∙ₗ β)

  _⋆'_ : {A : Set} → {a b c : A} {p q : a ≡ b} {r' s : b ≡ c} (α : p ≡ q) (β : r' ≡ s) → p ∙ r' ≡ q ∙ s
  _⋆'_ {A} {p = p} {s = s} α β =  (p ∙ₗ β) ∙ (α ∙ᵣ s)

  Ω : {A : Set} (a : A) → Set
  Ω {A} a = a ≡ a

  Ω² : {A : Set} (a : A) → Set
  Ω² {A} a = _≡_ {a ≡ a} r r 

  lem1 : {A : Set} → (a : A) → (α β : Ω² {A} a) → (α ⋆ β) ≡ (iᵣ r ⁻¹ ∙ α ∙ iᵣ r) ∙ (iₗ r ⁻¹ ∙ β ∙ iₗ r)
  lem1 a α β = r

  lem1' : {A : Set} → (a : A) → (α β : Ω² {A} a) → (α ⋆' β) ≡  (iₗ r ⁻¹ ∙ β ∙ iₗ r) ∙ (iᵣ r ⁻¹ ∙ α ∙ iᵣ r)
  lem1' a α β = r

  -- ap\_
  apf : {A B : Set} → {x y : A} → (f : A → B) → (x ≡ y) → f x ≡ f y
  apf {A} {B} {x} {y} f p = J D d x y p
    where
      D : (x y : A) → x ≡ y → Set
      D x y p = {f : A → B} → f x ≡ f y
      d : (x : A) → D x x r
      d = λ x → r 

  ap : {A B : Set} → {x y : A} → (f : A → B) → (x ≡ y) → f x ≡ f y
  ap f r = r

  lem20 : {A : Set} → {a : A} → (α : Ω² {A} a) → (iᵣ r ⁻¹ ∙ α ∙ iᵣ r) ≡ α
  lem20 α = iᵣ (α) ⁻¹

  lem21 : {A : Set} → {a : A} → (β : Ω² {A} a) → (iₗ r ⁻¹ ∙ β ∙ iₗ r) ≡ β
  lem21 β = iᵣ (β) ⁻¹

  lem2 : {A : Set} → (a : A) → (α β : Ω² {A} a) → (iᵣ r ⁻¹ ∙ α ∙ iᵣ r) ∙ (iₗ r ⁻¹ ∙ β ∙ iₗ r) ≡ (α ∙ β)
  lem2 {A} a α β = apf (λ - → - ∙ (iₗ r ⁻¹ ∙ β ∙ iₗ r) ) (lem20 α) ∙ apf (λ - → α ∙ -) (lem21 β)

  lem2' : {A : Set} → (a : A) → (α β : Ω² {A} a) → (iₗ r ⁻¹ ∙ β ∙ iₗ r) ∙ (iᵣ r ⁻¹ ∙ α ∙ iᵣ r) ≡ (β ∙ α )
  lem2' {A} a α β =  apf  (λ - → - ∙ (iᵣ r ⁻¹ ∙ α ∙ iᵣ r)) (lem21 β) ∙ apf (λ - → β ∙ -) (lem20 α)
  -- apf (λ - → - ∙ (iₗ r ⁻¹ ∙ β ∙ iₗ r) ) (lem20 α) ∙ apf (λ - → α ∙ -) (lem21 β)

  ⋆≡∙ : {A : Set} → (a : A) → (α β : Ω² {A} a) → (α ⋆ β) ≡ (α ∙ β)
  ⋆≡∙ a α β = lem1 a α β ∙ lem2 a α β

  -- proven similairly to above 
  ⋆'≡∙ : {A : Set} → (a : A) → (α β : Ω² {A} a) → (α ⋆' β) ≡ (β ∙ α)
  ⋆'≡∙ a α β = lem1' a α β ∙ lem2' a α β


  --eckmannHilton : {A : Set} → (a : A) → (α β : Ω² {A} a) → α ∙ β ≡ β ∙ α 
  --eckmannHilton a r r = r

\end{code}

[TODO, fix without k errors]

\section{GF Grammar for types}

We now discuss the GF implementation, capable of parsing both natural language
and Agda syntax. The parser was appropriated from the cubicaltt BNFC parser,
de-cubified and then gf-ified. The languages are tightly coupled, so the
translation is actually quite simple. Some main differences are:

\begin{itemize}[noitemsep]

\item GF treats abstract and concrete syntax seperately. This allows GF to
support many concrete syntax implementation of a given grammar

\item Fixity is dealt with at the concrete syntax layer in GF.  This allows for
more refined control of fixity, but also results in difficulties : during
linearization their can be the insertion of extra parens.

\item GF supports dependent hypes and higher order abstract syntax, which makes
it suitable to typecheck at the parsing stage. It would very interesting to see
if this is interoperable with the current version of this work in later
iterations [Todo - add github link referncing work I've done in this direction]

\item GF also is enhanced by a PGF back-end, allowing an embedding of grammars
into, among other languages, Haskell.

\end{itemize}

While GF is targeted towards natural language translation, there's nothing
stopping it from being used as a PL tool as well, like, for instance, the
front-end of a compiler. The innovation of this thesis is to combine both uses,
thereby allowing translation between Controlled Natural Languages and
programming languages.

Example expressions the grammar can parse are seen below, which have been
verified by hand to be isomorphic to the corresponding cubicaltt BNFC trees:

\begin{verbatim}

data bool : Set where true | false 
data nat : Set where zero | suc ( n : nat )  
caseBool ( x : Set ) ( y z : x ) : bool -> Set = split false -> y || true -> z
indBool ( x : bool -> Set ) ( y : x false ) ( z : x true ) : ( b : bool ) -> x b = split false -> y || true  -> z
funExt  ( a : Set )   ( b : a -> Set )   ( f g :  ( x : a )  -> b x )   ( p :  ( x : a )  -> ( b x )   ( f x ) == ( g x )  )  : (  ( y : a )  -> b y )  f == g = undefined
foo ( b : bool ) : bool = b

\end{verbatim}

[Todo] add use cases

\section{Goals and Challenges}

The parser is still quite primitive, and needs to be extended extensively to
support natural language ambiguity in mathematics as well as other linguistic
nuance that GF captures well, like tense and aspect. This can follow a method
expored in Aarne's paper : "Translating between Language and Logic: What Is
Easy and What Is Difficult" where one develops a denotational semantics for
translating between natural language expressions with the desired AST. The bulk
of this work will be writing a Haskell back-end implementing this AST
transformation. The extended syntax, designed for linguistic nuance, will be
filtered into the core syntax, which is essentially what I have done.

The Resource Grammar Library (RGL) is designed for out-of-the box grammar
writing, and therefore much of the linearization nuance can be outsourced to
this robust and well-studied library. Nonetheless, each application grammar
brings its own unique challenges, and the RGL will only get one so far. My
linearization may require extensive tweaking.

Thus far, our parser is only able to parse non-cubical fragments of the
cubicalTT standard library. Dealing with Agda pattern matching, it was
realized, is outside the theoretical boundaries of GF (at least, if one were to
do it in a non ad-hoc way) due to its inability to pass arbitrary strings down
the syntax tree nodes during linearization. Pattern matching therefore needs to be dealt
with via pre and post processing.  Additionally, cubicaltt is weaker at
dealing with telescopes than Agda, and so a full generalization to Agda is not
yet possible. Universes are another feature future iterations of this Grammar
would need to deal with, but as they aren't present in most mathematician's
vernacular, it is not seen as relevant for the state of this project.

Records should also be added, but because this grammar supports sigma types,
there is no rush. The Identity type is so far deeply embedded in our grammar,
so the first code fragment may just be for explanatory purposes.  The degree to
which the library is extended to cover domain specific information is up to
debate, but for now the grammar is meant to be kept as minimal as possible.

One interesting extension, time dependnet, would be to allow for a bidrectional
feedback between GF and Agda : thereby allowing ad hoc extensions to GF's ASTs
to allow for newly defined Agda functions to be treated with more care, i.e.
have an arguement structure rather than just treating everything as variables.
This may be too ambitious for the time being.

\section{Code}

\subsection{GF Parser}

\subsection{Additional Agda Hott Code}

Two citation examples: 
\cite{dunning1993} introduced a well-known method for extracting
collocations. Bilingual data can be used to train part-of-speech
taggers \citep{das2011}. Another one: \citep{cortes2014}

Testing Unicode: Göteborgs universitet

\textit{Testing} \textbf{testing} \textsc{testing} some font series.

Testing a formula:
\[
P(X) = \sum_{i=1}^N P(A_i) P(X|A_i)
\]

Testing a table:
\begin{table}[htbp]
\begin{center}
\begin{tabular}{c|c}
cell 1 & cell 2 \\
\hline
cell 3 & cell 4
\end{tabular}
\caption{This is a table.}
\end{center}
\end{table}

\newpage

\section{Example section (heading level 1)}

Text

\subsection{Example subsection (heading level 2)}

xxx

\subsubsection{Example subsubsection (heading level 3)}

xxx

\newpage

\addcontentsline{toc}{section}{References}
\bibliographystyle{plain}
\bibliography{example_bibliography}


\newpage
\section{Appendices}

\end{document}


