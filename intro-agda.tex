\subsection{Agda}

\subsubsection{Overview}
Agda is an attempt to faithfully formalize Martin-Löf's intensional type theory
\cite{ml1984}. Referencing our previous distinction, one can think of
Martin-Löf's original work as a specification, and Agda as one possible
implementation.

Agda is a functionaly programming language which, through an interactive
environment, allows one to iteratively apply rules and develop constructive
mathematics. It's current incarnation, Agda2 (but just called Agda), was
preceded by ALF, Cayenne, and Alfa, and the Agda1. On top of the basic MLTT,
Agda incorporates dependent records, inductive definitions, pattern matching, a
versatile module system, and a myriad of other bells and whistles which are of
interest generally and in various states of development but not relevant to this
work.

For our purposes, we will only look at what can in some sense be seen as the
kernel of Agda. Developing a full-blown GF grammar to incorporate more
advanced Agda features would require efforts beyond the scope of this work.

Agda's purpose is to manifest the propositions-as-types paradigm in a practical
and useable programming language. And while there are still many reasons one may
wish to use other programming languages, or just pen and paper to do her work,
there is a sense of purity one gets when writing Agda code. There are many good
resources for learning Agda \cite{Bove2009} \cite{stump} \cite{ulf}
\cite{wadler} so we'll only give a cursory overview of what is relevant for this
thesis, with a particular emphasis on the syntax.

\begin{code}[hide]%
\>[0]\<%
\\
\>[0]\AgdaComment{-- \{-\# OPTIONS --omega-in-omega --type-in-type \#-\}}\<%
\\
%
\\[\AgdaEmptyExtraSkip]%
\>[0]\AgdaKeyword{module}\AgdaSpace{}%
\AgdaModule{ex}\AgdaSpace{}%
\AgdaKeyword{where}\<%
\\
%
\\[\AgdaEmptyExtraSkip]%
\>[0]\AgdaKeyword{data}\AgdaSpace{}%
\AgdaDatatype{aℕ}\AgdaSpace{}%
\AgdaSymbol{:}\AgdaSpace{}%
\AgdaPrimitive{Set}\AgdaSpace{}%
\AgdaKeyword{where}\<%
\\
\>[0][@{}l@{\AgdaIndent{0}}]%
\>[2]\AgdaInductiveConstructor{zero'}\AgdaSpace{}%
\AgdaSymbol{:}\AgdaSpace{}%
\AgdaDatatype{aℕ}\<%
\\
%
\\[\AgdaEmptyExtraSkip]%
\>[0]\AgdaKeyword{variable}\<%
\\
\>[0][@{}l@{\AgdaIndent{0}}]%
\>[2]\AgdaGeneralizable{A}\AgdaSpace{}%
\AgdaSymbol{:}\AgdaSpace{}%
\AgdaPrimitive{Set}\<%
\\
%
\>[2]\AgdaGeneralizable{D}\AgdaSpace{}%
\AgdaSymbol{:}\AgdaSpace{}%
\AgdaPrimitive{Set}\<%
\\
%
\>[2]\AgdaGeneralizable{stuff}\AgdaSpace{}%
\AgdaSymbol{:}\AgdaSpace{}%
\AgdaPrimitive{Set}\<%
\\
%
\\[\AgdaEmptyExtraSkip]%
\>[0]\AgdaFunction{definition-body}\AgdaSpace{}%
\AgdaSymbol{=}\AgdaSpace{}%
\AgdaDatatype{aℕ}\<%
\\
%
\\[\AgdaEmptyExtraSkip]%
\>[0]\AgdaFunction{T}\AgdaSpace{}%
\AgdaSymbol{=}\AgdaSpace{}%
\AgdaDatatype{aℕ}\AgdaSpace{}%
\AgdaSymbol{→}\AgdaSpace{}%
\AgdaDatatype{aℕ}\<%
\\
\>[0]\AgdaFunction{L}\AgdaSpace{}%
\AgdaSymbol{=}\AgdaSpace{}%
\AgdaDatatype{aℕ}\<%
\\
\>[0]\AgdaFunction{E}\AgdaSpace{}%
\AgdaSymbol{=}\AgdaSpace{}%
\AgdaDatatype{aℕ}\<%
\\
\>[0]\AgdaFunction{C}\AgdaSpace{}%
\AgdaSymbol{=}\AgdaSpace{}%
\AgdaDatatype{aℕ}\<%
\\
%
\\[\AgdaEmptyExtraSkip]%
\>[0]\AgdaFunction{proof}\AgdaSpace{}%
\AgdaSymbol{:}\AgdaSpace{}%
\AgdaFunction{L}\<%
\\
\>[0]\AgdaFunction{proof}\AgdaSpace{}%
\AgdaSymbol{=}\AgdaSpace{}%
\AgdaInductiveConstructor{zero'}\<%
\\
%
\\[\AgdaEmptyExtraSkip]%
\>[0]\AgdaFunction{corollaryStuff}\AgdaSpace{}%
\AgdaSymbol{=}\AgdaSpace{}%
\AgdaDatatype{aℕ}\<%
\\
%
\\[\AgdaEmptyExtraSkip]%
\>[0]\AgdaFunction{proofNeedingLemma}\AgdaSpace{}%
\AgdaSymbol{:}\AgdaSpace{}%
\AgdaDatatype{aℕ}\AgdaSpace{}%
\AgdaSymbol{→}\AgdaSpace{}%
\AgdaDatatype{aℕ}\AgdaSpace{}%
\AgdaSymbol{→}\AgdaSpace{}%
\AgdaDatatype{aℕ}\<%
\\
\>[0]\AgdaFunction{proofNeedingLemma}\AgdaSpace{}%
\AgdaBound{x}\AgdaSpace{}%
\AgdaSymbol{=}\AgdaSpace{}%
\AgdaSymbol{λ}\AgdaSpace{}%
\AgdaBound{x₁}\AgdaSpace{}%
\AgdaSymbol{→}\AgdaSpace{}%
\AgdaInductiveConstructor{zero'}\<%
\\
\>[0]\<%
\end{code}

\subsubsection{Agda Programming}

Listed is the syntax Agda uses for judgements: \term{T} : \term{Set} means
\term{T} is a type, \term{t} : \term{T} means a term \term{t} has type \term{T},
and \term{t} = \term{t'} means \term{t} is defined to be judgmentally equal to
\term{t'}. Once one has made this equality judgement, Agda can normalize the
definitionally equal terms to the same normal form. Let's compare these Agda
judgements to those keywords ubiquitous in mathematics:

\begin{figure}
\centering
\begin{minipage}[t]{.3\textwidth}
\vspace{2cm}
\begin{itemize}
\item Axiom
\item Definition
\item Lemma
\item Theorem
\item Proof
\item Corollary
\item Example
\end{itemize}
\end{minipage}%
\begin{minipage}[t]{.55\textwidth}
\begin{code}%
\>[0]\AgdaKeyword{postulate}%
\>[12]\AgdaComment{-- Axiom}\<%
\\
\>[0][@{}l@{\AgdaIndent{0}}]%
\>[2]\AgdaPostulate{axiom}\AgdaSpace{}%
\AgdaSymbol{:}\AgdaSpace{}%
\AgdaGeneralizable{A}\<%
\\
%
\\[\AgdaEmptyExtraSkip]%
\>[0]\AgdaFunction{definition}\AgdaSpace{}%
\AgdaSymbol{:}\AgdaSpace{}%
\AgdaGeneralizable{stuff}\AgdaSpace{}%
\AgdaSymbol{→}\AgdaSpace{}%
\AgdaPrimitive{Set}\AgdaSpace{}%
\AgdaComment{--Definition}\<%
\\
\>[0]\AgdaFunction{definition}\AgdaSpace{}%
\AgdaBound{s}\AgdaSpace{}%
\AgdaSymbol{=}\AgdaSpace{}%
\AgdaFunction{definition-body}\<%
\\
%
\\[\AgdaEmptyExtraSkip]%
\>[0]\AgdaFunction{theorem}\AgdaSpace{}%
\AgdaSymbol{:}\AgdaSpace{}%
\AgdaFunction{T}%
\>[16]\AgdaComment{-- Theorem Statement}\<%
\\
\>[0]\AgdaFunction{theorem}\AgdaSpace{}%
\AgdaSymbol{=}\AgdaSpace{}%
\AgdaFunction{proofNeedingLemma}\AgdaSpace{}%
\AgdaFunction{lemma}\AgdaSpace{}%
\AgdaComment{-- Proof}\<%
\\
\>[0][@{}l@{\AgdaIndent{0}}]%
\>[2]\AgdaKeyword{where}\<%
\\
\>[2][@{}l@{\AgdaIndent{0}}]%
\>[4]\AgdaFunction{lemma}\AgdaSpace{}%
\AgdaSymbol{:}\AgdaSpace{}%
\AgdaFunction{L}%
\>[18]\AgdaComment{-- Lemma Statement}\<%
\\
%
\>[4]\AgdaFunction{lemma}\AgdaSpace{}%
\AgdaSymbol{=}\AgdaSpace{}%
\AgdaFunction{proof}\<%
\\
%
\\[\AgdaEmptyExtraSkip]%
\>[0]\AgdaFunction{corollary}\AgdaSpace{}%
\AgdaSymbol{:}\AgdaSpace{}%
\AgdaFunction{corollaryStuff}\AgdaSpace{}%
\AgdaSymbol{→}\AgdaSpace{}%
\AgdaFunction{C}\<%
\\
\>[0]\AgdaFunction{corollary}\AgdaSpace{}%
\AgdaBound{coro-term}\AgdaSpace{}%
\AgdaSymbol{=}\AgdaSpace{}%
\AgdaFunction{theorem}\AgdaSpace{}%
\AgdaBound{coro-term}\<%
\\
%
\\[\AgdaEmptyExtraSkip]%
\>[0]\AgdaFunction{example}\AgdaSpace{}%
\AgdaSymbol{:}\AgdaSpace{}%
\AgdaFunction{E}%
\>[16]\AgdaComment{-- Example Statement}\<%
\\
\>[0]\AgdaFunction{example}\AgdaSpace{}%
\AgdaSymbol{=}\AgdaSpace{}%
\AgdaFunction{proof}\<%
\end{code}
\end{minipage}
\caption{Mathematical Assertions and Agda Judgements} \label{fig:O1}
\end{figure}

Formation rules are given by the first line of the data declaration, followed
by some number of constructors which correspond to the introduction forms of the
type being defined. Therefore, to define a type for Booleans, $𝔹$, we present
these rules both in the proof theoretic and Agda syntax. We note that the
context $\Gamma$ is not present in Agda.

\begin{minipage}[t]{.4\textwidth}
\vspace{3mm}
\[
  \begin{prooftree}
    \infer1[]{ \vdash 𝔹 : {\rm type}}
  \end{prooftree}
\]
\[
  \begin{prooftree}
    \infer1[]{ \Gamma \vdash true : 𝔹  }
  \end{prooftree}
  \quad \quad
  \begin{prooftree}
    \infer1[]{ \Gamma \vdash false : 𝔹  }
  \end{prooftree}
\]
\end{minipage}
\begin{minipage}[t]{.3\textwidth}
\begin{code}%
\>[0]\AgdaKeyword{data}\AgdaSpace{}%
\AgdaDatatype{𝔹}\AgdaSpace{}%
\AgdaSymbol{:}\AgdaSpace{}%
\AgdaPrimitive{Set}\AgdaSpace{}%
\AgdaKeyword{where}\AgdaSpace{}%
\AgdaComment{-- formation rule}\<%
\\
\>[0][@{}l@{\AgdaIndent{0}}]%
\>[2]\AgdaInductiveConstructor{true}%
\>[8]\AgdaSymbol{:}\AgdaSpace{}%
\AgdaDatatype{𝔹}\AgdaSpace{}%
\AgdaComment{-- introduction rule}\<%
\\
%
\>[2]\AgdaInductiveConstructor{false}\AgdaSpace{}%
\AgdaSymbol{:}\AgdaSpace{}%
\AgdaDatatype{𝔹}\<%
\end{code}
\end{minipage}

The elimination forms are deriveable from the introduction rules, and the
computation rules can then be extracted by via the harmonious relationship
between the introduction and elmination forms \cite{pfenningHar}. Agda's pattern
matching is equivalent to the deriveable dependently typed elimination forms
\cite{coqPat}, and one can simply pattern match on a boolean, producing multiple
lines for each constructor of the variable's type, to extract the classic
recursion principle for Booleans. The \term{if then else} statement shown below
is really just the boolean elimination form. It is not standard to include the
premises of the eqaulity rules.

\begin{minipage}[t]{.4\textwidth}
\[
  \begin{prooftree}
    \hypo{̌\Gamma \vdash A : {\rm type} }
    \hypo{\Gamma \vdash b : 𝔹 }
    \hypo{\Gamma \vdash a1 : A}
    \hypo{\Gamma \vdash a2 : A }
    \infer4[]{\Gamma \vdash boolrec\{a1;a2\}(b) : A }
  \end{prooftree}
\]
$$\Gamma \vdash boolrec\{a1;a2\}(true) \equiv a1 : A$$
$$\Gamma \vdash boolrec\{a1;a2\}(false) \equiv a2 : A$$
\end{minipage}
\hfill
\begin{minipage}[t]{.5\textwidth}
\begin{code}%
\>[0]\AgdaOperator{\AgdaFunction{if\AgdaUnderscore{}then\AgdaUnderscore{}else\AgdaUnderscore{}}}\AgdaSpace{}%
\AgdaSymbol{:}\<%
\\
\>[0][@{}l@{\AgdaIndent{0}}]%
\>[2]\AgdaSymbol{\{}\AgdaBound{A}\AgdaSpace{}%
\AgdaSymbol{:}\AgdaSpace{}%
\AgdaPrimitive{Set}\AgdaSymbol{\}}\AgdaSpace{}%
\AgdaSymbol{→}\AgdaSpace{}%
\AgdaDatatype{𝔹}\AgdaSpace{}%
\AgdaSymbol{→}\AgdaSpace{}%
\AgdaBound{A}\AgdaSpace{}%
\AgdaSymbol{→}\AgdaSpace{}%
\AgdaBound{A}\AgdaSpace{}%
\AgdaSymbol{→}\AgdaSpace{}%
\AgdaBound{A}\<%
\\
\>[0]\AgdaOperator{\AgdaFunction{if}}\AgdaSpace{}%
\AgdaInductiveConstructor{true}\AgdaSpace{}%
\AgdaOperator{\AgdaFunction{then}}\AgdaSpace{}%
\AgdaBound{a1}\AgdaSpace{}%
\AgdaOperator{\AgdaFunction{else}}\AgdaSpace{}%
\AgdaBound{a2}\AgdaSpace{}%
\AgdaSymbol{=}\AgdaSpace{}%
\AgdaBound{a1}\<%
\\
\>[0]\AgdaOperator{\AgdaFunction{if}}\AgdaSpace{}%
\AgdaInductiveConstructor{false}\AgdaSpace{}%
\AgdaOperator{\AgdaFunction{then}}\AgdaSpace{}%
\AgdaBound{a1}\AgdaSpace{}%
\AgdaOperator{\AgdaFunction{else}}\AgdaSpace{}%
\AgdaBound{a2}\AgdaSpace{}%
\AgdaSymbol{=}\AgdaSpace{}%
\AgdaBound{a2}\<%
\end{code}
\end{minipage}

When using Agda one is interactively building a proof via holes. There is an
Agda Emacs mode which enables this. Glossing over many details, we show sample
code in the proof development state prior to pattern matching on \codeword{b}.
We have a hole, \codeword{{ b }0}, and the proof state is displayed to the
right. It shows both the current context with \codeword{A, b, a1, a2}, the goal
which is something of type \codeword{A}, and what we have, \codeword{B},
represents the type of the variable in the hole.

\hfill
\begin{minipage}[t]{.4\textwidth}
\begin{verbatim}
if_then_else_ :
  {A : Set} → B → A → A → A
if b then a1 else a2 = { b }0
\end{verbatim}
\end{minipage}
\hfill
\begin{minipage}[t]{.5\textwidth}
\begin{verbatim}
Goal: A
Have: B
———————————————
a2 : A
a1 : A
b  : B
A  : Set   (not in scope)
\end{verbatim}
\end{minipage}

The interactivity is performed via emacs commands, and every time one updates
the hole with a new term, we can immediately view the next goal with an updated
context. The underscore in \term{if_then_else_} denotes the placement of the
arguements, as Agda allows mixfix operations. Agda allows for more nuanced
syntacic features like unicode. This is interesting from the \emph{concrete
syntax} perspective as the arguement placement and symbolic expressiveness makes
Agda's syntax feel more familiar to the mathematician. We also observe the use
of parametric polymorphism, namely, that we can extract a member of some
arbtitrary type \term{A} from a boolean value given two members of \term{A}.

This polymorphism allows one to implement simple programs like boolean negation,
\term{~}, and more interestingly, \term{functionalNegation}, where one can use
functions as arguements. \term{functionalNegation} is a functional, or higher
order functions, which take functions as arguements and return functions. We
also notice in \term{functionalNegation} that one can work directly with a
built-in $\lambda$ to ensure the correct return type.

\begin{code}%
\>[0]\AgdaFunction{\textasciitilde{}}\AgdaSpace{}%
\AgdaSymbol{:}\AgdaSpace{}%
\AgdaDatatype{𝔹}\AgdaSpace{}%
\AgdaSymbol{→}\AgdaSpace{}%
\AgdaDatatype{𝔹}\<%
\\
\>[0]\AgdaFunction{\textasciitilde{}}\AgdaSpace{}%
\AgdaBound{b}\AgdaSpace{}%
\AgdaSymbol{=}\AgdaSpace{}%
\AgdaOperator{\AgdaFunction{if}}\AgdaSpace{}%
\AgdaBound{b}\AgdaSpace{}%
\AgdaOperator{\AgdaFunction{then}}\AgdaSpace{}%
\AgdaInductiveConstructor{false}\AgdaSpace{}%
\AgdaOperator{\AgdaFunction{else}}\AgdaSpace{}%
\AgdaInductiveConstructor{true}\<%
\\
%
\\[\AgdaEmptyExtraSkip]%
\>[0]\AgdaFunction{functionalNegation}\AgdaSpace{}%
\AgdaSymbol{:}\AgdaSpace{}%
\AgdaDatatype{𝔹}\AgdaSpace{}%
\AgdaSymbol{→}\AgdaSpace{}%
\AgdaSymbol{(}\AgdaDatatype{𝔹}\AgdaSpace{}%
\AgdaSymbol{→}\AgdaSpace{}%
\AgdaDatatype{𝔹}\AgdaSymbol{)}\AgdaSpace{}%
\AgdaSymbol{→}\AgdaSpace{}%
\AgdaSymbol{(}\AgdaDatatype{𝔹}\AgdaSpace{}%
\AgdaSymbol{→}\AgdaSpace{}%
\AgdaDatatype{𝔹}\AgdaSymbol{)}\<%
\\
\>[0]\AgdaFunction{functionalNegation}\AgdaSpace{}%
\AgdaBound{b}\AgdaSpace{}%
\AgdaBound{f}\AgdaSpace{}%
\AgdaSymbol{=}\AgdaSpace{}%
\AgdaOperator{\AgdaFunction{if}}\AgdaSpace{}%
\AgdaBound{b}\AgdaSpace{}%
\AgdaOperator{\AgdaFunction{then}}\AgdaSpace{}%
\AgdaBound{f}\AgdaSpace{}%
\AgdaOperator{\AgdaFunction{else}}\AgdaSpace{}%
\AgdaSymbol{λ}\AgdaSpace{}%
\AgdaBound{b'}\AgdaSpace{}%
\AgdaSymbol{→}\AgdaSpace{}%
\AgdaBound{f}\AgdaSpace{}%
\AgdaSymbol{(}\AgdaFunction{\textasciitilde{}}\AgdaSpace{}%
\AgdaBound{b'}\AgdaSymbol{)}\<%
\end{code}

This simple example leads us to one of the domains our subsequent grammars will
describe, like arithmetic (see \ref{npf}). We show how to inductively define
natural numbers in Agda, with the formation and introduction rules included
beside for contrast.

\begin{minipage}[t]{.4\textwidth}
\vspace{3mm}
\[
  \begin{prooftree}
    \infer1[]{ \vdash ℕ : {\rm type}}
  \end{prooftree}
\]
\[
  \begin{prooftree}
    \infer1[]{ \Gamma \vdash 0 : ℕ  }
  \end{prooftree}
  \quad \quad
  \begin{prooftree}
    \hypo{\Gamma \vdash n : ℕ}
    \infer1[]{ \Gamma \vdash (suc\ n) : ℕ  }
  \end{prooftree}
\]
\end{minipage}
\begin{minipage}[t]{.3\textwidth}
\begin{code}%
\>[0]\AgdaKeyword{data}\AgdaSpace{}%
\AgdaDatatype{ℕ}\AgdaSpace{}%
\AgdaSymbol{:}\AgdaSpace{}%
\AgdaPrimitive{Set}\AgdaSpace{}%
\AgdaKeyword{where}\<%
\\
\>[0][@{}l@{\AgdaIndent{0}}]%
\>[2]\AgdaInductiveConstructor{zero}\AgdaSpace{}%
\AgdaSymbol{:}\AgdaSpace{}%
\AgdaDatatype{ℕ}\<%
\\
%
\>[2]\AgdaInductiveConstructor{suc}%
\>[7]\AgdaSymbol{:}\AgdaSpace{}%
\AgdaDatatype{ℕ}\AgdaSpace{}%
\AgdaSymbol{→}\AgdaSpace{}%
\AgdaDatatype{ℕ}\<%
\end{code}
\end{minipage}

This is a recursive type, whereby pattern matching over $ℕ$ allows one to use an
induction hypothesis over the subtree and gurantee termination when making
recurive calls on the function being defined. We can define a recursion
principle for $ℕ$, which gives one the power to build iterators.
Again, we include the elimination and equality rules for syntactic
juxtaposition.

\[
  \begin{prooftree}
    \hypo{̌\Gamma \vdash X : {\rm type} }
    \hypo{\Gamma \vdash n : ℕ }
    \hypo{\Gamma \vdash e₀ : X}
    \hypo{\Gamma, x : ℕ, y : X \vdash e₁ : X }
    \infer4[]{\Gamma \vdash natrec\{e\;x.y.e₁\}(n) : X }
  \end{prooftree}
\]
$$\Gamma \vdash natrec\{e₀;x.y.e₁\}(n) \equiv e₀ : X$$
$$\Gamma \vdash natrec\{e₀;x.y.e₁\}(suc\ n) \equiv e₁[x := n,y := natrec\{e₀;x.y.e₁\}(n)] : X$$
\begin{code}%
\>[0]\AgdaFunction{natrec}\AgdaSpace{}%
\AgdaSymbol{:}\AgdaSpace{}%
\AgdaSymbol{\{}\AgdaBound{X}\AgdaSpace{}%
\AgdaSymbol{:}\AgdaSpace{}%
\AgdaPrimitive{Set}\AgdaSymbol{\}}\AgdaSpace{}%
\AgdaSymbol{→}\AgdaSpace{}%
\AgdaDatatype{ℕ}\AgdaSpace{}%
\AgdaSymbol{→}\AgdaSpace{}%
\AgdaBound{X}\AgdaSpace{}%
\AgdaSymbol{→}\AgdaSpace{}%
\AgdaSymbol{(}\AgdaDatatype{ℕ}\AgdaSpace{}%
\AgdaSymbol{→}\AgdaSpace{}%
\AgdaBound{X}\AgdaSpace{}%
\AgdaSymbol{→}\AgdaSpace{}%
\AgdaBound{X}\AgdaSymbol{)}\AgdaSpace{}%
\AgdaSymbol{→}\AgdaSpace{}%
\AgdaBound{X}\<%
\\
\>[0]\AgdaFunction{natrec}\AgdaSpace{}%
\AgdaInductiveConstructor{zero}\AgdaSpace{}%
\AgdaBound{e₀}\AgdaSpace{}%
\AgdaBound{e₁}\AgdaSpace{}%
\AgdaSymbol{=}\AgdaSpace{}%
\AgdaBound{e₀}\<%
\\
\>[0]\AgdaFunction{natrec}\AgdaSpace{}%
\AgdaSymbol{(}\AgdaInductiveConstructor{suc}\AgdaSpace{}%
\AgdaBound{n}\AgdaSymbol{)}\AgdaSpace{}%
\AgdaBound{e₀}\AgdaSpace{}%
\AgdaBound{e₁}\AgdaSpace{}%
\AgdaSymbol{=}\AgdaSpace{}%
\AgdaBound{e₁}\AgdaSpace{}%
\AgdaBound{n}\AgdaSpace{}%
\AgdaSymbol{(}\AgdaFunction{natrec}\AgdaSpace{}%
\AgdaBound{n}\AgdaSpace{}%
\AgdaBound{e₀}\AgdaSpace{}%
\AgdaBound{e₁}\AgdaSymbol{)}\<%
\end{code}

Since we are in a dependently typed setting, however, we prove theorems as well
as write programs. Therefore, we can see this recursion principle as a special
case of the induction principle \term{natind}, which represents the by induction
for the natural numbers. One may notice that while the types are different, the
programs \term{natrec} and \term{natind} are actually the same, up to
α-equivalence. One can therefore, as a corollary, actually just include the type
infomation and Agda can infer the speciliazation for you, as seen in
\term{natrec'} below.

\[
  \begin{prooftree}
    \hypo{̌\Gamma, x : ℕ \vdash X : {\rm type} }
    \hypo{\Gamma \vdash n : ℕ }
    \hypo{\Gamma \vdash e₀ : X[x := 0] }
    \hypo{\Gamma, y : ℕ, z : X[x := y] \vdash e₁ : X[x := suc\ y]}
    \infer4[]{\Gamma \vdash natind\{e₀,\;x.y.e₁\}(n) : X[x := n]}
  \end{prooftree}
\]
$$\Gamma \vdash natind\{e₀;x.y.e₁\}(n) \equiv e₀ : X[x := 0]$$
$$\Gamma \vdash natind\{e₀;x.y.e₁\}(suc\ n) \equiv e₁[x := n,y := natind\{e₀;x.y.e₁\}(n)] : X[x := suc\ n]$$
\begin{code}%
\>[0]\AgdaFunction{natind}\AgdaSpace{}%
\AgdaSymbol{:}\AgdaSpace{}%
\AgdaSymbol{\{}\AgdaBound{X}\AgdaSpace{}%
\AgdaSymbol{:}\AgdaSpace{}%
\AgdaDatatype{ℕ}\AgdaSpace{}%
\AgdaSymbol{→}\AgdaSpace{}%
\AgdaPrimitive{Set}\AgdaSymbol{\}}\AgdaSpace{}%
\AgdaSymbol{→}\AgdaSpace{}%
\AgdaSymbol{(}\AgdaBound{n}\AgdaSpace{}%
\AgdaSymbol{:}\AgdaSpace{}%
\AgdaDatatype{ℕ}\AgdaSymbol{)}\AgdaSpace{}%
\AgdaSymbol{→}\AgdaSpace{}%
\AgdaBound{X}\AgdaSpace{}%
\AgdaInductiveConstructor{zero}\AgdaSpace{}%
\AgdaSymbol{→}\AgdaSpace{}%
\AgdaSymbol{((}\AgdaBound{n}\AgdaSpace{}%
\AgdaSymbol{:}\AgdaSpace{}%
\AgdaDatatype{ℕ}\AgdaSymbol{)}\AgdaSpace{}%
\AgdaSymbol{→}\AgdaSpace{}%
\AgdaBound{X}\AgdaSpace{}%
\AgdaBound{n}\AgdaSpace{}%
\AgdaSymbol{→}\AgdaSpace{}%
\AgdaBound{X}\AgdaSpace{}%
\AgdaSymbol{(}\AgdaInductiveConstructor{suc}\AgdaSpace{}%
\AgdaBound{n}\AgdaSymbol{))}\AgdaSpace{}%
\AgdaSymbol{→}\AgdaSpace{}%
\AgdaBound{X}\AgdaSpace{}%
\AgdaBound{n}\<%
\\
\>[0]\AgdaFunction{natind}\AgdaSpace{}%
\AgdaInductiveConstructor{zero}\AgdaSpace{}%
\AgdaBound{base}\AgdaSpace{}%
\AgdaBound{step}\AgdaSpace{}%
\AgdaSymbol{=}\AgdaSpace{}%
\AgdaBound{base}\<%
\\
\>[0]\AgdaFunction{natind}\AgdaSpace{}%
\AgdaSymbol{(}\AgdaInductiveConstructor{suc}\AgdaSpace{}%
\AgdaBound{n}\AgdaSymbol{)}\AgdaSpace{}%
\AgdaBound{base}\AgdaSpace{}%
\AgdaBound{step}\AgdaSpace{}%
\AgdaSymbol{=}\AgdaSpace{}%
\AgdaBound{step}\AgdaSpace{}%
\AgdaBound{n}\AgdaSpace{}%
\AgdaSymbol{(}\AgdaFunction{natind}\AgdaSpace{}%
\AgdaBound{n}\AgdaSpace{}%
\AgdaBound{base}\AgdaSpace{}%
\AgdaBound{step}\AgdaSymbol{)}\<%
\\
%
\\[\AgdaEmptyExtraSkip]%
\>[0]\AgdaFunction{natrec'}\AgdaSpace{}%
\AgdaSymbol{:}\AgdaSpace{}%
\AgdaSymbol{\{}\AgdaBound{X}\AgdaSpace{}%
\AgdaSymbol{:}\AgdaSpace{}%
\AgdaPrimitive{Set}\AgdaSymbol{\}}\AgdaSpace{}%
\AgdaSymbol{→}\AgdaSpace{}%
\AgdaDatatype{ℕ}\AgdaSpace{}%
\AgdaSymbol{→}\AgdaSpace{}%
\AgdaBound{X}\AgdaSpace{}%
\AgdaSymbol{→}\AgdaSpace{}%
\AgdaSymbol{(}\AgdaDatatype{ℕ}\AgdaSpace{}%
\AgdaSymbol{→}\AgdaSpace{}%
\AgdaBound{X}\AgdaSpace{}%
\AgdaSymbol{→}\AgdaSpace{}%
\AgdaBound{X}\AgdaSymbol{)}\AgdaSpace{}%
\AgdaSymbol{→}\AgdaSpace{}%
\AgdaBound{X}\<%
\\
\>[0]\AgdaFunction{natrec'}\AgdaSpace{}%
\AgdaSymbol{=}\AgdaSpace{}%
\AgdaFunction{natind}\<%
\end{code}
We will defer the details of using induction and recursion principles for later
when we actually give examples of pidgin proofs some of our grammars can
handle.


\begin{code}[hide]%
\>[0]\AgdaKeyword{module}\AgdaSpace{}%
\AgdaModule{twin-primes}\AgdaSpace{}%
\AgdaKeyword{where}\<%
\\
%
\\[\AgdaEmptyExtraSkip]%
\>[0]\AgdaKeyword{open}\AgdaSpace{}%
\AgdaKeyword{import}\AgdaSpace{}%
\AgdaModule{Data.Nat}\AgdaSpace{}%
\AgdaKeyword{renaming}\AgdaSpace{}%
\AgdaSymbol{(}\AgdaOperator{\AgdaPrimitive{\AgdaUnderscore{}+\AgdaUnderscore{}}}\AgdaSpace{}%
\AgdaSymbol{to}\AgdaSpace{}%
\AgdaOperator{\AgdaPrimitive{\AgdaUnderscore{}∔\AgdaUnderscore{}}}\AgdaSymbol{)}\<%
\\
\>[0]\AgdaKeyword{open}\AgdaSpace{}%
\AgdaKeyword{import}\AgdaSpace{}%
\AgdaModule{Data.Product}\AgdaSpace{}%
\AgdaKeyword{using}\AgdaSpace{}%
\AgdaSymbol{(}\AgdaRecord{Σ}\AgdaSymbol{;}\AgdaSpace{}%
\AgdaOperator{\AgdaFunction{\AgdaUnderscore{}×\AgdaUnderscore{}}}\AgdaSymbol{;}\AgdaSpace{}%
\AgdaOperator{\AgdaInductiveConstructor{\AgdaUnderscore{},\AgdaUnderscore{}}}\AgdaSymbol{;}\AgdaSpace{}%
\AgdaField{proj₁}\AgdaSymbol{;}\AgdaSpace{}%
\AgdaField{proj₂}\AgdaSymbol{;}\AgdaSpace{}%
\AgdaFunction{∃}\AgdaSymbol{;}\AgdaSpace{}%
\AgdaFunction{Σ-syntax}\AgdaSymbol{;}\AgdaSpace{}%
\AgdaFunction{∃-syntax}\AgdaSymbol{)}\<%
\\
\>[0]\AgdaKeyword{open}\AgdaSpace{}%
\AgdaKeyword{import}\AgdaSpace{}%
\AgdaModule{Data.Sum}\AgdaSpace{}%
\AgdaKeyword{renaming}\AgdaSpace{}%
\AgdaSymbol{(}\AgdaOperator{\AgdaDatatype{\AgdaUnderscore{}⊎\AgdaUnderscore{}}}\AgdaSpace{}%
\AgdaSymbol{to}\AgdaSpace{}%
\AgdaOperator{\AgdaDatatype{\AgdaUnderscore{}+\AgdaUnderscore{}}}\AgdaSymbol{)}\<%
\\
\>[0]\AgdaKeyword{import}\AgdaSpace{}%
\AgdaModule{Relation.Binary.PropositionalEquality}\AgdaSpace{}%
\AgdaSymbol{as}\AgdaSpace{}%
\AgdaModule{Eq}\<%
\\
\>[0]\AgdaKeyword{open}\AgdaSpace{}%
\AgdaModule{Eq}\AgdaSpace{}%
\AgdaKeyword{using}\AgdaSpace{}%
\AgdaSymbol{(}\AgdaOperator{\AgdaDatatype{\AgdaUnderscore{}≡\AgdaUnderscore{}}}\AgdaSymbol{;}\AgdaSpace{}%
\AgdaInductiveConstructor{refl}\AgdaSymbol{;}\AgdaSpace{}%
\AgdaFunction{trans}\AgdaSymbol{;}\AgdaSpace{}%
\AgdaFunction{sym}\AgdaSymbol{;}\AgdaSpace{}%
\AgdaFunction{cong}\AgdaSymbol{;}\AgdaSpace{}%
\AgdaFunction{cong-app}\AgdaSymbol{;}\AgdaSpace{}%
\AgdaFunction{subst}\AgdaSymbol{)}\<%
\\
\>[0]\AgdaKeyword{open}\AgdaSpace{}%
\AgdaModule{Eq.≡-Reasoning}\AgdaSpace{}%
\AgdaKeyword{using}\AgdaSpace{}%
\AgdaSymbol{(}\AgdaOperator{\AgdaFunction{begin\AgdaUnderscore{}}}\AgdaSymbol{;}\AgdaSpace{}%
\AgdaOperator{\AgdaFunction{\AgdaUnderscore{}≡⟨⟩\AgdaUnderscore{}}}\AgdaSymbol{;}\AgdaSpace{}%
\AgdaFunction{step-≡}\AgdaSymbol{;}\AgdaSpace{}%
\AgdaOperator{\AgdaFunction{\AgdaUnderscore{}∎}}\AgdaSymbol{)}\<%
\\
%
\\[\AgdaEmptyExtraSkip]%
\>[0]\AgdaOperator{\AgdaFunction{\AgdaUnderscore{}-\AgdaUnderscore{}}}\AgdaSpace{}%
\AgdaSymbol{:}\AgdaSpace{}%
\AgdaDatatype{ℕ}\AgdaSpace{}%
\AgdaSymbol{→}\AgdaSpace{}%
\AgdaDatatype{ℕ}\AgdaSpace{}%
\AgdaSymbol{→}\AgdaSpace{}%
\AgdaDatatype{ℕ}\<%
\\
\>[0]\AgdaBound{n}%
\>[6]\AgdaOperator{\AgdaFunction{-}}\AgdaSpace{}%
\AgdaInductiveConstructor{zero}\AgdaSpace{}%
\AgdaSymbol{=}\AgdaSpace{}%
\AgdaBound{n}\<%
\\
\>[0]\AgdaInductiveConstructor{zero}%
\>[6]\AgdaOperator{\AgdaFunction{-}}\AgdaSpace{}%
\AgdaInductiveConstructor{suc}\AgdaSpace{}%
\AgdaBound{m}\AgdaSpace{}%
\AgdaSymbol{=}\AgdaSpace{}%
\AgdaInductiveConstructor{zero}\<%
\\
\>[0]\AgdaInductiveConstructor{suc}\AgdaSpace{}%
\AgdaBound{n}\AgdaSpace{}%
\AgdaOperator{\AgdaFunction{-}}\AgdaSpace{}%
\AgdaInductiveConstructor{suc}\AgdaSpace{}%
\AgdaBound{m}\AgdaSpace{}%
\AgdaSymbol{=}\AgdaSpace{}%
\AgdaBound{n}\AgdaSpace{}%
\AgdaOperator{\AgdaFunction{-}}\AgdaSpace{}%
\AgdaBound{m}\<%
\end{code}
\subsubsection{Formalizing The Twin Prime Conjecture}

Inspired by Escardos's formalization of the twin primes conjecture \cite{escardó2020introduction}, we intend to
demonstrate that while formalizing mathematics can be rewarding, it can also
create immense difficulties, especially if one wishes to do it in a way that
prioritizes natural language. The conjecture is incredibly compact

\begin{lem}
There are infinitely many twin primes.
\end{lem}

Somebody reading for the first time might then pose the immediate question : what is a twin prime?

\begin{definition}\label{def:def10}
A \emph{twin prime} is a prime number that is either 2 less or 2 more than another prime number
\end{definition}

Below Escardo's code is reproduced.
\begin{code}%
\>[0]\AgdaFunction{isPrime}\AgdaSpace{}%
\AgdaSymbol{:}\AgdaSpace{}%
\AgdaDatatype{ℕ}\AgdaSpace{}%
\AgdaSymbol{→}\AgdaSpace{}%
\AgdaPrimitive{Set}\<%
\\
\>[0]\AgdaFunction{isPrime}\AgdaSpace{}%
\AgdaBound{n}\AgdaSpace{}%
\AgdaSymbol{=}\<%
\\
\>[0][@{}l@{\AgdaIndent{0}}]%
\>[2]\AgdaSymbol{(}\AgdaBound{n}\AgdaSpace{}%
\AgdaOperator{\AgdaFunction{≥}}\AgdaSpace{}%
\AgdaNumber{2}\AgdaSymbol{)}\AgdaSpace{}%
\AgdaOperator{\AgdaFunction{×}}\<%
\\
%
\>[2]\AgdaSymbol{((}\AgdaBound{x}\AgdaSpace{}%
\AgdaBound{y}\AgdaSpace{}%
\AgdaSymbol{:}\AgdaSpace{}%
\AgdaDatatype{ℕ}\AgdaSymbol{)}\AgdaSpace{}%
\AgdaSymbol{→}\AgdaSpace{}%
\AgdaBound{x}\AgdaSpace{}%
\AgdaOperator{\AgdaPrimitive{*}}\AgdaSpace{}%
\AgdaBound{y}\AgdaSpace{}%
\AgdaOperator{\AgdaDatatype{≡}}\AgdaSpace{}%
\AgdaBound{n}\AgdaSpace{}%
\AgdaSymbol{→}\AgdaSpace{}%
\AgdaSymbol{(}\AgdaBound{x}\AgdaSpace{}%
\AgdaOperator{\AgdaDatatype{≡}}\AgdaSpace{}%
\AgdaNumber{1}\AgdaSymbol{)}\AgdaSpace{}%
\AgdaOperator{\AgdaDatatype{+}}\AgdaSpace{}%
\AgdaSymbol{(}\AgdaBound{x}\AgdaSpace{}%
\AgdaOperator{\AgdaDatatype{≡}}\AgdaSpace{}%
\AgdaBound{n}\AgdaSymbol{))}\<%
\\
%
\\[\AgdaEmptyExtraSkip]%
\>[0]\AgdaFunction{twinPrimeConjecture}\AgdaSpace{}%
\AgdaSymbol{:}\AgdaSpace{}%
\AgdaPrimitive{Set}\<%
\\
\>[0]\AgdaFunction{twinPrimeConjecture}\AgdaSpace{}%
\AgdaSymbol{=}\AgdaSpace{}%
\AgdaSymbol{(}\AgdaBound{n}\AgdaSpace{}%
\AgdaSymbol{:}\AgdaSpace{}%
\AgdaDatatype{ℕ}\AgdaSymbol{)}\AgdaSpace{}%
\AgdaSymbol{→}\AgdaSpace{}%
\AgdaFunction{Σ[}\AgdaSpace{}%
\AgdaBound{p}\AgdaSpace{}%
\AgdaFunction{∈}\AgdaSpace{}%
\AgdaDatatype{ℕ}\AgdaSpace{}%
\AgdaFunction{]}\AgdaSpace{}%
\AgdaSymbol{(}\AgdaBound{p}\AgdaSpace{}%
\AgdaOperator{\AgdaFunction{≥}}\AgdaSpace{}%
\AgdaBound{n}\AgdaSymbol{)}\<%
\\
\>[0][@{}l@{\AgdaIndent{0}}]%
\>[2]\AgdaOperator{\AgdaFunction{×}}\AgdaSpace{}%
\AgdaFunction{isPrime}\AgdaSpace{}%
\AgdaBound{p}\<%
\\
%
\>[2]\AgdaOperator{\AgdaFunction{×}}\AgdaSpace{}%
\AgdaFunction{isPrime}\AgdaSpace{}%
\AgdaSymbol{(}\AgdaBound{p}\AgdaSpace{}%
\AgdaOperator{\AgdaPrimitive{∔}}\AgdaSpace{}%
\AgdaNumber{2}\AgdaSymbol{)}\<%
\end{code}

We note there are some both subtle and big differences, between the natural
language claim. First, twin prime is defined implicitly via a product
expression, \term{×}. Additionally, the ``either 2 less or 2 more" clause is
oringially read as being interpreted as having ``2 more". This reading ignores
the symmetry of products, however, and both ``p or (p ∔ 2)" could be interpreted
as the twin prime. This phenomenon makes translation highly nontrivial; however,
we will later see that PGF is capable of adding a semantic layer where the
theorem can be evaluated during the translation. Finally, this theorem doesn't
say what it is to be infinite in general, because such a definition would
require a proving a bijection with the real numbers. In this case however, we
can rely on the order of the natural numbers, to simply state what it means to
have infinitely many primes.

Despite the beauty of this, mathematicians always look for alternative, more
general ways of stating things. Generalizing the notion of a twin prime is a
prime gap. And then one immediately has to ask what is a prime gap?

\begin{definition}\label{def:def11}
A \emph{twin prime} is a prime that has a prime gap of two.
\end{definition}
\begin{definition}\label{def:def12}
A \emph{prime gap} is the difference between two successive prime numbers.
\end{definition}

Now we're stuck, at least if you want to scour the internet for the definition
of ``two successive prime numbers". That is because any mathematician will take
for granted what it means, and it would be considered a waste of time and space
to include something \emph{everyone} alternatively knows. Agda, however, must
know in order to typecheck. Below we offer a presentation which suits Agda's
needs, and matches the number theorists presentation of twin prime.

\begin{code}%
\>[0]\AgdaFunction{isSuccessivePrime}\AgdaSpace{}%
\AgdaSymbol{:}\AgdaSpace{}%
\AgdaSymbol{(}\AgdaBound{p}\AgdaSpace{}%
\AgdaBound{p'}\AgdaSpace{}%
\AgdaSymbol{:}\AgdaSpace{}%
\AgdaDatatype{ℕ}\AgdaSymbol{)}\AgdaSpace{}%
\AgdaSymbol{→}\AgdaSpace{}%
\AgdaFunction{isPrime}\AgdaSpace{}%
\AgdaBound{p}\AgdaSpace{}%
\AgdaSymbol{→}\AgdaSpace{}%
\AgdaFunction{isPrime}\AgdaSpace{}%
\AgdaBound{p'}\AgdaSpace{}%
\AgdaSymbol{→}\AgdaSpace{}%
\AgdaPrimitive{Set}\<%
\\
\>[0]\AgdaFunction{isSuccessivePrime}\AgdaSpace{}%
\AgdaBound{p}\AgdaSpace{}%
\AgdaBound{p'}\AgdaSpace{}%
\AgdaBound{x}\AgdaSpace{}%
\AgdaBound{x₁}\AgdaSpace{}%
\AgdaSymbol{=}\<%
\\
\>[0][@{}l@{\AgdaIndent{0}}]%
\>[2]\AgdaSymbol{(}\AgdaBound{p''}\AgdaSpace{}%
\AgdaSymbol{:}\AgdaSpace{}%
\AgdaDatatype{ℕ}\AgdaSymbol{)}\AgdaSpace{}%
\AgdaSymbol{→}\AgdaSpace{}%
\AgdaSymbol{(}\AgdaFunction{isPrime}\AgdaSpace{}%
\AgdaBound{p''}\AgdaSymbol{)}\AgdaSpace{}%
\AgdaSymbol{→}\<%
\\
%
\>[2]\AgdaBound{p}\AgdaSpace{}%
\AgdaOperator{\AgdaDatatype{≤}}\AgdaSpace{}%
\AgdaBound{p'}\AgdaSpace{}%
\AgdaSymbol{→}\AgdaSpace{}%
\AgdaBound{p}\AgdaSpace{}%
\AgdaOperator{\AgdaDatatype{≤}}\AgdaSpace{}%
\AgdaBound{p''}\AgdaSpace{}%
\AgdaSymbol{→}\AgdaSpace{}%
\AgdaBound{p'}\AgdaSpace{}%
\AgdaOperator{\AgdaDatatype{≤}}\AgdaSpace{}%
\AgdaBound{p''}\<%
\\
%
\\[\AgdaEmptyExtraSkip]%
\>[0]\AgdaFunction{primeGap}\AgdaSpace{}%
\AgdaSymbol{:}\<%
\\
\>[0][@{}l@{\AgdaIndent{0}}]%
\>[2]\AgdaSymbol{(}\AgdaBound{p}\AgdaSpace{}%
\AgdaBound{p'}\AgdaSpace{}%
\AgdaSymbol{:}\AgdaSpace{}%
\AgdaDatatype{ℕ}\AgdaSymbol{)}\AgdaSpace{}%
\AgdaSymbol{(}\AgdaBound{pIsPrime}\AgdaSpace{}%
\AgdaSymbol{:}\AgdaSpace{}%
\AgdaFunction{isPrime}\AgdaSpace{}%
\AgdaBound{p}\AgdaSymbol{)}\AgdaSpace{}%
\AgdaSymbol{(}\AgdaBound{p'IsPrime}\AgdaSpace{}%
\AgdaSymbol{:}\AgdaSpace{}%
\AgdaFunction{isPrime}\AgdaSpace{}%
\AgdaBound{p'}\AgdaSymbol{)}\AgdaSpace{}%
\AgdaSymbol{→}\<%
\\
%
\>[2]\AgdaSymbol{(}\AgdaFunction{isSuccessivePrime}\AgdaSpace{}%
\AgdaBound{p}\AgdaSpace{}%
\AgdaBound{p'}\AgdaSpace{}%
\AgdaBound{pIsPrime}\AgdaSpace{}%
\AgdaBound{p'IsPrime}\AgdaSymbol{)}\AgdaSpace{}%
\AgdaSymbol{→}\<%
\\
%
\>[2]\AgdaDatatype{ℕ}\<%
\\
\>[0]\AgdaFunction{primeGap}\AgdaSpace{}%
\AgdaBound{p}\AgdaSpace{}%
\AgdaBound{p'}\AgdaSpace{}%
\AgdaBound{pIsPrime}\AgdaSpace{}%
\AgdaBound{p'IsPrime}\AgdaSpace{}%
\AgdaBound{p'-is-after-p}\AgdaSpace{}%
\AgdaSymbol{=}\AgdaSpace{}%
\AgdaBound{p}\AgdaSpace{}%
\AgdaOperator{\AgdaFunction{-}}\AgdaSpace{}%
\AgdaBound{p'}\<%
\\
%
\\[\AgdaEmptyExtraSkip]%
\>[0]\AgdaFunction{twinPrime}\AgdaSpace{}%
\AgdaSymbol{:}\AgdaSpace{}%
\AgdaSymbol{(}\AgdaBound{p}\AgdaSpace{}%
\AgdaSymbol{:}\AgdaSpace{}%
\AgdaDatatype{ℕ}\AgdaSymbol{)}\AgdaSpace{}%
\AgdaSymbol{→}\AgdaSpace{}%
\AgdaPrimitive{Set}\<%
\\
\>[0]\AgdaFunction{twinPrime}\AgdaSpace{}%
\AgdaBound{p}\AgdaSpace{}%
\AgdaSymbol{=}\<%
\\
\>[0][@{}l@{\AgdaIndent{0}}]%
\>[2]\AgdaSymbol{(}\AgdaBound{pIsPrime}\AgdaSpace{}%
\AgdaSymbol{:}\AgdaSpace{}%
\AgdaFunction{isPrime}\AgdaSpace{}%
\AgdaBound{p}\AgdaSymbol{)}\AgdaSpace{}%
\AgdaSymbol{(}\AgdaBound{p'}\AgdaSpace{}%
\AgdaSymbol{:}\AgdaSpace{}%
\AgdaDatatype{ℕ}\AgdaSymbol{)}\AgdaSpace{}%
\AgdaSymbol{(}\AgdaBound{p'IsPrime}\AgdaSpace{}%
\AgdaSymbol{:}\AgdaSpace{}%
\AgdaFunction{isPrime}\AgdaSpace{}%
\AgdaBound{p'}\AgdaSymbol{)}\<%
\\
%
\>[2]\AgdaSymbol{(}\AgdaBound{p'-is-after-p}\AgdaSpace{}%
\AgdaSymbol{:}\AgdaSpace{}%
\AgdaFunction{isSuccessivePrime}\AgdaSpace{}%
\AgdaBound{p}\AgdaSpace{}%
\AgdaBound{p'}\AgdaSpace{}%
\AgdaBound{pIsPrime}\AgdaSpace{}%
\AgdaBound{p'IsPrime}\AgdaSymbol{)}\AgdaSpace{}%
\AgdaSymbol{→}\<%
\\
%
\>[2]\AgdaSymbol{(}\AgdaFunction{primeGap}\AgdaSpace{}%
\AgdaBound{p}\AgdaSpace{}%
\AgdaBound{p'}\AgdaSpace{}%
\AgdaBound{pIsPrime}\AgdaSpace{}%
\AgdaBound{p'IsPrime}\AgdaSpace{}%
\AgdaBound{p'-is-after-p}\AgdaSymbol{)}\AgdaSpace{}%
\AgdaOperator{\AgdaDatatype{≡}}\AgdaSpace{}%
\AgdaNumber{2}\<%
\\
%
\\[\AgdaEmptyExtraSkip]%
\>[0]\AgdaFunction{twinPrimeConjecture'}\AgdaSpace{}%
\AgdaSymbol{:}\AgdaSpace{}%
\AgdaPrimitive{Set}\<%
\\
\>[0]\AgdaFunction{twinPrimeConjecture'}\AgdaSpace{}%
\AgdaSymbol{=}\AgdaSpace{}%
\AgdaSymbol{(}\AgdaBound{n}\AgdaSpace{}%
\AgdaSymbol{:}\AgdaSpace{}%
\AgdaDatatype{ℕ}\AgdaSymbol{)}\AgdaSpace{}%
\AgdaSymbol{→}\AgdaSpace{}%
\AgdaFunction{Σ[}\AgdaSpace{}%
\AgdaBound{p}\AgdaSpace{}%
\AgdaFunction{∈}\AgdaSpace{}%
\AgdaDatatype{ℕ}\AgdaSpace{}%
\AgdaFunction{]}\AgdaSpace{}%
\AgdaSymbol{(}\AgdaBound{p}\AgdaSpace{}%
\AgdaOperator{\AgdaFunction{≥}}\AgdaSpace{}%
\AgdaBound{n}\AgdaSymbol{)}\<%
\\
\>[0][@{}l@{\AgdaIndent{0}}]%
\>[2]\AgdaOperator{\AgdaFunction{×}}\AgdaSpace{}%
\AgdaFunction{twinPrime}\AgdaSpace{}%
\AgdaBound{p}\<%
\end{code}


We see that \term{isSuccessivePrime} captures this meaning, interpreting
``successive" as the type of suprema in the prime number ordering. We also see that all the primality proofs must be given explicitly.

The term \term{primeGap} then has to reference this successive prime data, even
though most of it is discarded and unused in the actual program returning a
number. One could keep these unused arguements around via extra record fields,
to anticipate future programs calling \term{primeGap}, but ultimately the developer has to
decide what is relevant. A GF translation would ideally be kept as simple as possible. We also use propositional equality here, which is
another departure from classical mathematics, as will be elaborated later.

Finally, \{twinPrime} is a specialized version of \term{primeGap} to 2. ``has a
prime gap of two`` needs to be interpreted ``whose prime gap is equal to two",
and writing a GF grammar capable of disambiguating \emph{has} in mathematics
generally is likely impossible. One can also uncurry much of the above code to
make it more readable, which we include in the appendix \ref{twin}.

While working on this example, I tried to prove that 2 is prime in Agda with
this defintion. It turned out to be nontrivial. When I told this to an analyst
(in the mathematical sense) he remarked that couldn't possibly be the case
because it's something which a simple algorithm can compute (or generate). This
exchange was incredibly stimulating, for the mathematian didn't know about the
\emph{propositions as types} principle, and was simply taking for granted his
internal computational capacity to confuse it for proof, especially in a
constructive setting. He also seemed perplexed that anyone would find it
interesting to prove that 2 is prime. The proof that 2 is prime, via Agda's
standard libary, is done via reflection - a way of quoting a term into in
abstract syntax tree and then performing some kind of metacomputation. While
elegant, this obviously requires a lot of machinery, none of which would be easy
to communicate to a mathematician who doesn't know much about coding. As is
hopefully revealed by this discussion, seemingly trivial things, when treated by
the type theorist or linguist, can become wonderful areas of exploration.


