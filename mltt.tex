% from blog post
\subsection{Martin-Löf Type Theory}
\subsubsection{Judgments}

A central contribution of Per Martin-Löf in the development of type theory was
the recognition of the centrality of judgments in logic. Many mathematicians
aren't familiar with the spectrum of judgments available, and merely believe
they are concerned with *the* notion of truth, namely *the truth* of a
mathematical proposition or theorem. There are many judgments one can make which
most mathematicians aren't aware of or at least never mention. These include,
for instance,

\begin{itemize}[noitemsep]

\item $A$ is a proposition
\item $A$ is possible
\item $A$ is probable

\end{itemize}

These judgments are understood not in the object language in which we state our
propositions, possibilities, or probabilities, but as assertions in the
metalanguage which require evidence for us to know and believe them. Most
mathematicians may reach for their wallets if I come in and give a talk saying
it is possible that the Riemann Hypothesis is true, partially because they
already know that, and partially because it doesn't seem particularly
interesting to say that something is possible, in the same way that a physicist
may flinch if you say alchemy is possible. Most mathematicians, however, would
agree that $P = NP$ is possible but isn't probable.

For the logician these judgments may well be interesting because their may be
logics in which the discussion of possibility or probability is even more
interesting than the discussion of truth. And for the type theorist, interested
in designing and building programming languages over many various logics, these
judgments become a prime focus. The role of the type-checker in a programming
language is to present evidence for, or decide the validity of the judgments.
The four main judgments of type theory are :

\begin{itemize}[noitemsep]
\item $T$ is a type
\item $T$ and $T'$ are equal types
\item $t$ is a term of type $T$
\item $t$ and $t'$ are equal terms of type $T$
\end{itemize}


We succinctly present these in a mathematical notation where Frege's turnstile,
$\vdash$, denotes a judgment :

\begin{itemize}[noitemsep]
\item $\vdash T \; {\rm type}$
\item $\vdash T = T'$
\item $\vdash t:T$
\item $\vdash t = t':T$
\end{itemize}

These judgments become much more interesting when we add the ability for them to
be interpreted in a some context with judgment hypotheses. Given a series of
judgments $J_1,...,J_n$, denoted $\Gamma$, where $J_i$ can depend on previously
listed $J's$, we can make judgment $J$ under the hypotheses, e.g. $J_1,...,J_n
\vdash J$. Often these hypotheses $J_i$, alternatively called *antecedents*,
denote variables which may occur freely in the *consequent* judgment $J$. For
instance, the antecedent, $x : \mathbb{R}$ occurs freely in the syntactic
expression $\sin x$, a which is given meaning in the judgment $\vdash \sin x { :
} \mathbb{R}$. We write our hypothetical judgement as follows :

$x : \mathbb{R} \vdash \sin x : \mathbb{R}$

One reason why hypothetical judgments are so interesting is we can devise rules
which allow us to translate from the metalanguage to the object language using
lambda expressions. These play the role of a function in mathematics and
implication in logic. This comes out in the following introduction rule :

$ \frac{\Gamma, x : A \vdash b : B} {\Gamma \vdash \lambda x. b : A \rightarrow
B} $

Using this rule, we now see a typical judgment, typical in a field like from
real analysis,

$\vdash \lambda x. \sin x : \R \rightarrow \R$

Equality :

Mathematicians denote this judgement
\begin{align*} f {:} \mathbb{R} &\rightarrow \mathbb{R}\\ x &\mapsto \sin ( x )
\end{align*}


