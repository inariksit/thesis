\begin{code}[hide]%
\>[0]\<%
\\
\>[0]\AgdaComment{-- \{-\# OPTIONS --omega-in-omega --type-in-type \#-\}}\<%
\\
%
\\[\AgdaEmptyExtraSkip]%
\>[0]\AgdaKeyword{module}\AgdaSpace{}%
\AgdaModule{ex}\AgdaSpace{}%
\AgdaKeyword{where}\<%
\\
%
\\[\AgdaEmptyExtraSkip]%
\>[0]\AgdaKeyword{data}\AgdaSpace{}%
\AgdaDatatype{aℕ}\AgdaSpace{}%
\AgdaSymbol{:}\AgdaSpace{}%
\AgdaPrimitive{Set}\AgdaSpace{}%
\AgdaKeyword{where}\<%
\\
\>[0][@{}l@{\AgdaIndent{0}}]%
\>[2]\AgdaInductiveConstructor{zero'}\AgdaSpace{}%
\AgdaSymbol{:}\AgdaSpace{}%
\AgdaDatatype{aℕ}\<%
\\
%
\\[\AgdaEmptyExtraSkip]%
\>[0]\AgdaKeyword{variable}\<%
\\
\>[0][@{}l@{\AgdaIndent{0}}]%
\>[2]\AgdaGeneralizable{A}\AgdaSpace{}%
\AgdaSymbol{:}\AgdaSpace{}%
\AgdaPrimitive{Set}\<%
\\
%
\>[2]\AgdaGeneralizable{D}\AgdaSpace{}%
\AgdaSymbol{:}\AgdaSpace{}%
\AgdaPrimitive{Set}\<%
\\
%
\>[2]\AgdaGeneralizable{stuff}\AgdaSpace{}%
\AgdaSymbol{:}\AgdaSpace{}%
\AgdaPrimitive{Set}\<%
\\
%
\\[\AgdaEmptyExtraSkip]%
\>[0]\AgdaFunction{definition-body}\AgdaSpace{}%
\AgdaSymbol{=}\AgdaSpace{}%
\AgdaDatatype{aℕ}\<%
\\
%
\\[\AgdaEmptyExtraSkip]%
\>[0]\AgdaFunction{T}\AgdaSpace{}%
\AgdaSymbol{=}\AgdaSpace{}%
\AgdaDatatype{aℕ}\AgdaSpace{}%
\AgdaSymbol{→}\AgdaSpace{}%
\AgdaDatatype{aℕ}\<%
\\
\>[0]\AgdaFunction{L}\AgdaSpace{}%
\AgdaSymbol{=}\AgdaSpace{}%
\AgdaDatatype{aℕ}\<%
\\
\>[0]\AgdaFunction{E}\AgdaSpace{}%
\AgdaSymbol{=}\AgdaSpace{}%
\AgdaDatatype{aℕ}\<%
\\
\>[0]\AgdaFunction{C}\AgdaSpace{}%
\AgdaSymbol{=}\AgdaSpace{}%
\AgdaDatatype{aℕ}\<%
\\
%
\\[\AgdaEmptyExtraSkip]%
\>[0]\AgdaFunction{proof}\AgdaSpace{}%
\AgdaSymbol{:}\AgdaSpace{}%
\AgdaFunction{L}\<%
\\
\>[0]\AgdaFunction{proof}\AgdaSpace{}%
\AgdaSymbol{=}\AgdaSpace{}%
\AgdaInductiveConstructor{zero'}\<%
\\
%
\\[\AgdaEmptyExtraSkip]%
\>[0]\AgdaFunction{corollaryStuff}\AgdaSpace{}%
\AgdaSymbol{=}\AgdaSpace{}%
\AgdaDatatype{aℕ}\<%
\\
%
\\[\AgdaEmptyExtraSkip]%
\>[0]\AgdaFunction{proofNeedingLemma}\AgdaSpace{}%
\AgdaSymbol{:}\AgdaSpace{}%
\AgdaDatatype{aℕ}\AgdaSpace{}%
\AgdaSymbol{→}\AgdaSpace{}%
\AgdaDatatype{aℕ}\AgdaSpace{}%
\AgdaSymbol{→}\AgdaSpace{}%
\AgdaDatatype{aℕ}\<%
\\
\>[0]\AgdaFunction{proofNeedingLemma}\AgdaSpace{}%
\AgdaBound{x}\AgdaSpace{}%
\AgdaSymbol{=}\AgdaSpace{}%
\AgdaSymbol{λ}\AgdaSpace{}%
\AgdaBound{x₁}\AgdaSpace{}%
\AgdaSymbol{→}\AgdaSpace{}%
\AgdaInductiveConstructor{zero'}\<%
\\
\>[0]\<%
\end{code}

\subsubsection{Agda Programming}

To give a brief overview of the syntax Agda uses for judgements, namely \term{T}
: \term{Set} means \term{T} is a type, \term{t} : \term{T} means a term \term{t}
has type \term{T}, and \term{t} = \term{t'} means \term{t} is defined to be
judgmentally equal to \term{t'}. Once one has made this equality judgement, agda
can normalize the definitionally equal terms to the same normal form in
downstream programs. Let's compare it these judgements to those keywords ubiquitous in
mathematics, and show how those are represented in Agda directly below.

\begin{figure}
\centering
\begin{minipage}[t]{.3\textwidth}
\vspace{2cm}
\begin{itemize}
\item Axiom
\item Definition
\item Lemma
\item Theorem
\item Proof
\item Corollary
\item Example
\end{itemize}
\end{minipage}%
\begin{minipage}[t]{.55\textwidth}
\begin{code}%
\>[0]\AgdaKeyword{postulate}%
\>[12]\AgdaComment{-- Axiom}\<%
\\
\>[0][@{}l@{\AgdaIndent{0}}]%
\>[2]\AgdaPostulate{axiom}\AgdaSpace{}%
\AgdaSymbol{:}\AgdaSpace{}%
\AgdaGeneralizable{A}\<%
\\
%
\\[\AgdaEmptyExtraSkip]%
\>[0]\AgdaFunction{definition}\AgdaSpace{}%
\AgdaSymbol{:}\AgdaSpace{}%
\AgdaGeneralizable{stuff}\AgdaSpace{}%
\AgdaSymbol{→}\AgdaSpace{}%
\AgdaPrimitive{Set}\AgdaSpace{}%
\AgdaComment{--Definition}\<%
\\
\>[0]\AgdaFunction{definition}\AgdaSpace{}%
\AgdaBound{s}\AgdaSpace{}%
\AgdaSymbol{=}\AgdaSpace{}%
\AgdaFunction{definition-body}\<%
\\
%
\\[\AgdaEmptyExtraSkip]%
\>[0]\AgdaFunction{theorem}\AgdaSpace{}%
\AgdaSymbol{:}\AgdaSpace{}%
\AgdaFunction{T}%
\>[16]\AgdaComment{-- Theorem Statement}\<%
\\
\>[0]\AgdaFunction{theorem}\AgdaSpace{}%
\AgdaSymbol{=}\AgdaSpace{}%
\AgdaFunction{proofNeedingLemma}\AgdaSpace{}%
\AgdaFunction{lemma}\AgdaSpace{}%
\AgdaComment{-- Proof}\<%
\\
\>[0][@{}l@{\AgdaIndent{0}}]%
\>[2]\AgdaKeyword{where}\<%
\\
\>[2][@{}l@{\AgdaIndent{0}}]%
\>[4]\AgdaFunction{lemma}\AgdaSpace{}%
\AgdaSymbol{:}\AgdaSpace{}%
\AgdaFunction{L}%
\>[18]\AgdaComment{-- Lemma Statement}\<%
\\
%
\>[4]\AgdaFunction{lemma}\AgdaSpace{}%
\AgdaSymbol{=}\AgdaSpace{}%
\AgdaFunction{proof}\<%
\\
%
\\[\AgdaEmptyExtraSkip]%
\>[0]\AgdaFunction{corollary}\AgdaSpace{}%
\AgdaSymbol{:}\AgdaSpace{}%
\AgdaFunction{corollaryStuff}\AgdaSpace{}%
\AgdaSymbol{→}\AgdaSpace{}%
\AgdaFunction{C}\<%
\\
\>[0]\AgdaFunction{corollary}\AgdaSpace{}%
\AgdaBound{coro-term}\AgdaSpace{}%
\AgdaSymbol{=}\AgdaSpace{}%
\AgdaFunction{theorem}\AgdaSpace{}%
\AgdaBound{coro-term}\<%
\\
%
\\[\AgdaEmptyExtraSkip]%
\>[0]\AgdaFunction{example}\AgdaSpace{}%
\AgdaSymbol{:}\AgdaSpace{}%
\AgdaFunction{E}%
\>[16]\AgdaComment{-- Example Statement}\<%
\\
\>[0]\AgdaFunction{example}\AgdaSpace{}%
\AgdaSymbol{=}\AgdaSpace{}%
\AgdaFunction{proof}\<%
\end{code}
\end{minipage}
\caption{Mathematical Assertions and Agda Judgements} \label{fig:O1}
\end{figure}

Formation rules, are given by the first line of the data declaration, followed
by some number of constructors which correspond to the introduction forms of the
type being defined. Therefore, to define a type for Booleans, \term{𝔹}, we present
these rules both in the proof theoretic and Agda syntax.

\begin{minipage}[t]{.4\textwidth}
\vspace{3mm}
\[
  \begin{prooftree}
    \infer1[]{ \vdash 𝔹 : {\rm type}}
  \end{prooftree}
\]
\[
  \begin{prooftree}
    \infer1[]{ \Gamma \vdash true : 𝔹  }
  \end{prooftree}
  \quad \quad
  \begin{prooftree}
    \infer1[]{ \Gamma \vdash false : 𝔹  }
  \end{prooftree}
\]
\end{minipage}
\begin{minipage}[t]{.3\textwidth}
\begin{code}%
\>[0]\AgdaKeyword{data}\AgdaSpace{}%
\AgdaDatatype{𝔹}\AgdaSpace{}%
\AgdaSymbol{:}\AgdaSpace{}%
\AgdaPrimitive{Set}\AgdaSpace{}%
\AgdaKeyword{where}\AgdaSpace{}%
\AgdaComment{-- formation rule}\<%
\\
\>[0][@{}l@{\AgdaIndent{0}}]%
\>[2]\AgdaInductiveConstructor{true}%
\>[8]\AgdaSymbol{:}\AgdaSpace{}%
\AgdaDatatype{𝔹}\AgdaSpace{}%
\AgdaComment{-- introduction rule}\<%
\\
%
\>[2]\AgdaInductiveConstructor{false}\AgdaSpace{}%
\AgdaSymbol{:}\AgdaSpace{}%
\AgdaDatatype{𝔹}\<%
\end{code}
\end{minipage}

As the elimination forms are deriveable from the introduction rules, the
computation rules can then be extracted by via the harmonious relationship
between the introduction and elmination forms \cite{pfenningHar}. Agda's pattern
matching is equivalent to the deriveable dependently typed elimination forms
\cite{coqPat}, and one can simply pattern match on a boolean, producing multiple
lines for each constructor of the variable's type, to extract the classic
recursion principle for Booleans.

When using Agda one is working interactively via holes in the emacs mode, and
that once one plays around with it, one recognizes both the beauty and elegance
in how Agda facilitates programming. We don't include the eqaulity rules as
rules because they redundantly use the same premises as the typing judgment.
Below we show the elimination and equality rules alongside the Agda version.

\begin{minipage}[t]{.4\textwidth}
\[
  \begin{prooftree}
    \hypo{̌\Gamma \vdash A : {\rm type} }
    \hypo{\Gamma \vdash b : 𝔹 }
    \hypo{\Gamma \vdash a1 : A}
    \hypo{\Gamma \vdash a2 : A }
    \infer4[]{\Gamma \vdash boolrec\{a1;a2\}(b) : A }
  \end{prooftree}
\]
$$\Gamma \vdash boolrec\{a1;a2\}(true) \equiv a1 : A$$
$$\Gamma \vdash boolrec\{a1;a2\}(false) \equiv a2 : A$$
\end{minipage}
\hfill
\begin{minipage}[t]{.5\textwidth}
\begin{code}%
\>[0]\AgdaOperator{\AgdaFunction{if\AgdaUnderscore{}then\AgdaUnderscore{}else\AgdaUnderscore{}}}\AgdaSpace{}%
\AgdaSymbol{:}\<%
\\
\>[0][@{}l@{\AgdaIndent{0}}]%
\>[2]\AgdaSymbol{\{}\AgdaBound{A}\AgdaSpace{}%
\AgdaSymbol{:}\AgdaSpace{}%
\AgdaPrimitive{Set}\AgdaSymbol{\}}\AgdaSpace{}%
\AgdaSymbol{→}\AgdaSpace{}%
\AgdaDatatype{𝔹}\AgdaSpace{}%
\AgdaSymbol{→}\AgdaSpace{}%
\AgdaBound{A}\AgdaSpace{}%
\AgdaSymbol{→}\AgdaSpace{}%
\AgdaBound{A}\AgdaSpace{}%
\AgdaSymbol{→}\AgdaSpace{}%
\AgdaBound{A}\<%
\\
\>[0]\AgdaOperator{\AgdaFunction{if}}\AgdaSpace{}%
\AgdaInductiveConstructor{true}\AgdaSpace{}%
\AgdaOperator{\AgdaFunction{then}}\AgdaSpace{}%
\AgdaBound{a1}\AgdaSpace{}%
\AgdaOperator{\AgdaFunction{else}}\AgdaSpace{}%
\AgdaBound{a2}\AgdaSpace{}%
\AgdaSymbol{=}\AgdaSpace{}%
\AgdaBound{a1}\<%
\\
\>[0]\AgdaOperator{\AgdaFunction{if}}\AgdaSpace{}%
\AgdaInductiveConstructor{false}\AgdaSpace{}%
\AgdaOperator{\AgdaFunction{then}}\AgdaSpace{}%
\AgdaBound{a1}\AgdaSpace{}%
\AgdaOperator{\AgdaFunction{else}}\AgdaSpace{}%
\AgdaBound{a2}\AgdaSpace{}%
\AgdaSymbol{=}\AgdaSpace{}%
\AgdaBound{a2}\<%
\end{code}
\end{minipage}

The underscore denotes the placement of the arguement, as Agda allows mixfix
operations. \term{if_then_else_} function allows for more nuanced syntacic
features out of the box than most programming languages provide, like unicode.
This is interesting from the \emph{concrete syntax} perspective as the arguement
placement, and symbolic expressiveness gives Agda a syntax more familiar to the
mathematician. We also observe the use of parametric polymorphism, namely, that
we can extract a member of some arbtitrary type \term{A} from a boolean value
given two members of \term{A}.

This polymorphism allows one to implement simple programs like the two
equivalent boolean negation function, \term{~-elimRule} and
\term{~-patternMatch}. More interestingly, one can work with functionals, or
higher order functions which take functions as arguements and return functions
as well. We also notice in \term{functionalExample} below that one can work
directly with lambda's if the typechecker infers a function type for a hole.

\begin{code}%
\>[0]\AgdaFunction{\textasciitilde{}-elimRule}\AgdaSpace{}%
\AgdaSymbol{:}\AgdaSpace{}%
\AgdaDatatype{𝔹}\AgdaSpace{}%
\AgdaSymbol{→}\AgdaSpace{}%
\AgdaDatatype{𝔹}\<%
\\
\>[0]\AgdaFunction{\textasciitilde{}-elimRule}\AgdaSpace{}%
\AgdaBound{b}\AgdaSpace{}%
\AgdaSymbol{=}\AgdaSpace{}%
\AgdaOperator{\AgdaFunction{if}}\AgdaSpace{}%
\AgdaBound{b}\AgdaSpace{}%
\AgdaOperator{\AgdaFunction{then}}\AgdaSpace{}%
\AgdaInductiveConstructor{false}\AgdaSpace{}%
\AgdaOperator{\AgdaFunction{else}}\AgdaSpace{}%
\AgdaInductiveConstructor{true}\<%
\\
%
\\[\AgdaEmptyExtraSkip]%
\>[0]\AgdaFunction{\textasciitilde{}-patternMatch}\AgdaSpace{}%
\AgdaSymbol{:}\AgdaSpace{}%
\AgdaDatatype{𝔹}\AgdaSpace{}%
\AgdaSymbol{→}\AgdaSpace{}%
\AgdaDatatype{𝔹}\<%
\\
\>[0]\AgdaFunction{\textasciitilde{}-patternMatch}\AgdaSpace{}%
\AgdaInductiveConstructor{true}\AgdaSpace{}%
\AgdaSymbol{=}\AgdaSpace{}%
\AgdaInductiveConstructor{false}\<%
\\
\>[0]\AgdaFunction{\textasciitilde{}-patternMatch}\AgdaSpace{}%
\AgdaInductiveConstructor{false}\AgdaSpace{}%
\AgdaSymbol{=}\AgdaSpace{}%
\AgdaInductiveConstructor{true}\<%
\\
%
\\[\AgdaEmptyExtraSkip]%
\>[0]\AgdaFunction{functionalNegation}\AgdaSpace{}%
\AgdaSymbol{:}\AgdaSpace{}%
\AgdaDatatype{𝔹}\AgdaSpace{}%
\AgdaSymbol{→}\AgdaSpace{}%
\AgdaSymbol{(}\AgdaDatatype{𝔹}\AgdaSpace{}%
\AgdaSymbol{→}\AgdaSpace{}%
\AgdaDatatype{𝔹}\AgdaSymbol{)}\AgdaSpace{}%
\AgdaSymbol{→}\AgdaSpace{}%
\AgdaSymbol{(}\AgdaDatatype{𝔹}\AgdaSpace{}%
\AgdaSymbol{→}\AgdaSpace{}%
\AgdaDatatype{𝔹}\AgdaSymbol{)}\<%
\\
\>[0]\AgdaFunction{functionalNegation}\AgdaSpace{}%
\AgdaBound{b}\AgdaSpace{}%
\AgdaBound{f}\AgdaSpace{}%
\AgdaSymbol{=}\AgdaSpace{}%
\AgdaOperator{\AgdaFunction{if}}\AgdaSpace{}%
\AgdaBound{b}\AgdaSpace{}%
\AgdaOperator{\AgdaFunction{then}}\AgdaSpace{}%
\AgdaBound{f}\AgdaSpace{}%
\AgdaOperator{\AgdaFunction{else}}\AgdaSpace{}%
\AgdaSymbol{λ}\AgdaSpace{}%
\AgdaBound{b'}\AgdaSpace{}%
\AgdaSymbol{→}\AgdaSpace{}%
\AgdaBound{f}\AgdaSpace{}%
\AgdaSymbol{(}\AgdaFunction{\textasciitilde{}-patternMatch}\AgdaSpace{}%
\AgdaBound{b'}\AgdaSymbol{)}\<%
\end{code}

This simple example leads us to one of the domains our subsequent grammars will describe, arithmetic. We show how to inductively define natural numbers in Agda, with the formation and introduction rules included beside for contrast.

\begin{minipage}[t]{.4\textwidth}
\vspace{3mm}
\[
  \begin{prooftree}
    \infer1[]{ \vdash ℕ : {\rm type}}
  \end{prooftree}
\]
\[
  \begin{prooftree}
    \infer1[]{ \Gamma \vdash 0 : ℕ  }
  \end{prooftree}
  \quad \quad
  \begin{prooftree}
    \hypo{\Gamma \vdash n : ℕ}
    \infer1[]{ \Gamma \vdash (suc n) : ℕ  }
  \end{prooftree}
\]
\end{minipage}
\begin{minipage}[t]{.3\textwidth}
\begin{code}%
\>[0]\AgdaKeyword{data}\AgdaSpace{}%
\AgdaDatatype{ℕ}\AgdaSpace{}%
\AgdaSymbol{:}\AgdaSpace{}%
\AgdaPrimitive{Set}\AgdaSpace{}%
\AgdaKeyword{where}\<%
\\
\>[0][@{}l@{\AgdaIndent{0}}]%
\>[2]\AgdaInductiveConstructor{zero}\AgdaSpace{}%
\AgdaSymbol{:}\AgdaSpace{}%
\AgdaDatatype{ℕ}\<%
\\
%
\>[2]\AgdaInductiveConstructor{suc}%
\>[7]\AgdaSymbol{:}\AgdaSpace{}%
\AgdaDatatype{ℕ}\AgdaSpace{}%
\AgdaSymbol{→}\AgdaSpace{}%
\AgdaDatatype{ℕ}\<%
\end{code}
\end{minipage}

This is our first observation of a recursive type, whereby the pattern matching
over ℕ allows one to use an induction hypothesis over the subtree and gurantee
termination when making recurive calls on the function being defined. We can
define a recursion principle for ℕ, which essentially gives one the power to
build iterators, i.e. for-loops. Again, we include the recursion rule
elimination and equality rules for syntactic juxtaposition.

\[
  \begin{prooftree}
    \hypo{̌\Gamma \vdash X : {\rm type} }
    \hypo{\Gamma \vdash n : ℕ }
    \hypo{\Gamma \vdash e₀ : X}
    \hypo{\Gamma, x : ℕ, y : X \vdash e₁ : X }
    \infer4[]{\Gamma \vdash natrec\{e\;x.y.e₁\}(n) : X }
  \end{prooftree}
\]
$$\Gamma \vdash natrec\{e₀;x.y.e₁\}(n) \equiv e₀ : X$$
$$\Gamma \vdash natrec\{e₀;x.y.e₁\}(suc\ n) \equiv e₁[x := n,y := natrec\{e₀;x.y.e₁\}(n)] : X$$
\begin{code}%
\>[0]\AgdaFunction{natrec}\AgdaSpace{}%
\AgdaSymbol{:}\AgdaSpace{}%
\AgdaSymbol{\{}\AgdaBound{X}\AgdaSpace{}%
\AgdaSymbol{:}\AgdaSpace{}%
\AgdaPrimitive{Set}\AgdaSymbol{\}}\AgdaSpace{}%
\AgdaSymbol{→}\AgdaSpace{}%
\AgdaDatatype{ℕ}\AgdaSpace{}%
\AgdaSymbol{→}\AgdaSpace{}%
\AgdaBound{X}\AgdaSpace{}%
\AgdaSymbol{→}\AgdaSpace{}%
\AgdaSymbol{(}\AgdaDatatype{ℕ}\AgdaSpace{}%
\AgdaSymbol{→}\AgdaSpace{}%
\AgdaBound{X}\AgdaSpace{}%
\AgdaSymbol{→}\AgdaSpace{}%
\AgdaBound{X}\AgdaSymbol{)}\AgdaSpace{}%
\AgdaSymbol{→}\AgdaSpace{}%
\AgdaBound{X}\<%
\\
\>[0]\AgdaFunction{natrec}\AgdaSpace{}%
\AgdaInductiveConstructor{zero}\AgdaSpace{}%
\AgdaBound{e₀}\AgdaSpace{}%
\AgdaBound{e₁}\AgdaSpace{}%
\AgdaSymbol{=}\AgdaSpace{}%
\AgdaBound{e₀}\<%
\\
\>[0]\AgdaFunction{natrec}\AgdaSpace{}%
\AgdaSymbol{(}\AgdaInductiveConstructor{suc}\AgdaSpace{}%
\AgdaBound{n}\AgdaSymbol{)}\AgdaSpace{}%
\AgdaBound{e₀}\AgdaSpace{}%
\AgdaBound{e₁}\AgdaSpace{}%
\AgdaSymbol{=}\AgdaSpace{}%
\AgdaBound{e₁}\AgdaSpace{}%
\AgdaBound{n}\AgdaSpace{}%
\AgdaSymbol{(}\AgdaFunction{natrec}\AgdaSpace{}%
\AgdaBound{n}\AgdaSpace{}%
\AgdaBound{e₀}\AgdaSpace{}%
\AgdaBound{e₁}\AgdaSymbol{)}\<%
\end{code}
Since we are in a dependently typed setting, however, we prove theorems as well
as write programs. Therefore, we can see this recursion principle as a special
case of the induction principle, which is the classic proof by induction for
natural numbers. One may notice that while the types are different, the programs
\term{natrec} and \term{natind} are actually the same, up to α-equivalence. One
can therefore, as a corollary, actually just include the type infomation and
Agda can infer the speciliazation for you, as seen in \term{natrec'} below.

\[
  \begin{prooftree}
    \hypo{̌\Gamma, x : ℕ \vdash X : {\rm type} }
    \hypo{\Gamma \vdash n : ℕ }
    \hypo{\Gamma \vdash e₀ : X[x := 0] }
    \hypo{\Gamma, y : ℕ, z : X[x := y] \vdash e₁ : X[x := suc\ y]}
    \infer4[]{\Gamma \vdash natind\{e₀,\;x.y.e₁\}(n) : X[x := n]}
  \end{prooftree}
\]
$$\Gamma \vdash natind{e₀;x.y.e₁\}(n) \equiv e₀ : X[x := 0]$$
$$\Gamma \vdash natind{e₀;x.y.e₁\}(suc\ n) \equiv e₁[x := n,y := natind\{e₀;x.y.e₁\}(n)] : X[x := suc\ n]$$
\begin{code}%
\>[0]\AgdaFunction{natind}\AgdaSpace{}%
\AgdaSymbol{:}\AgdaSpace{}%
\AgdaSymbol{\{}\AgdaBound{X}\AgdaSpace{}%
\AgdaSymbol{:}\AgdaSpace{}%
\AgdaDatatype{ℕ}\AgdaSpace{}%
\AgdaSymbol{→}\AgdaSpace{}%
\AgdaPrimitive{Set}\AgdaSymbol{\}}\AgdaSpace{}%
\AgdaSymbol{→}\AgdaSpace{}%
\AgdaSymbol{(}\AgdaBound{n}\AgdaSpace{}%
\AgdaSymbol{:}\AgdaSpace{}%
\AgdaDatatype{ℕ}\AgdaSymbol{)}\AgdaSpace{}%
\AgdaSymbol{→}\AgdaSpace{}%
\AgdaBound{X}\AgdaSpace{}%
\AgdaInductiveConstructor{zero}\AgdaSpace{}%
\AgdaSymbol{→}\AgdaSpace{}%
\AgdaSymbol{((}\AgdaBound{n}\AgdaSpace{}%
\AgdaSymbol{:}\AgdaSpace{}%
\AgdaDatatype{ℕ}\AgdaSymbol{)}\AgdaSpace{}%
\AgdaSymbol{→}\AgdaSpace{}%
\AgdaBound{X}\AgdaSpace{}%
\AgdaBound{n}\AgdaSpace{}%
\AgdaSymbol{→}\AgdaSpace{}%
\AgdaBound{X}\AgdaSpace{}%
\AgdaSymbol{(}\AgdaInductiveConstructor{suc}\AgdaSpace{}%
\AgdaBound{n}\AgdaSymbol{))}\AgdaSpace{}%
\AgdaSymbol{→}\AgdaSpace{}%
\AgdaBound{X}\AgdaSpace{}%
\AgdaBound{n}\<%
\\
\>[0]\AgdaFunction{natind}\AgdaSpace{}%
\AgdaInductiveConstructor{zero}\AgdaSpace{}%
\AgdaBound{base}\AgdaSpace{}%
\AgdaBound{step}\AgdaSpace{}%
\AgdaSymbol{=}\AgdaSpace{}%
\AgdaBound{base}\<%
\\
\>[0]\AgdaFunction{natind}\AgdaSpace{}%
\AgdaSymbol{(}\AgdaInductiveConstructor{suc}\AgdaSpace{}%
\AgdaBound{n}\AgdaSymbol{)}\AgdaSpace{}%
\AgdaBound{base}\AgdaSpace{}%
\AgdaBound{step}\AgdaSpace{}%
\AgdaSymbol{=}\AgdaSpace{}%
\AgdaBound{step}\AgdaSpace{}%
\AgdaBound{n}\AgdaSpace{}%
\AgdaSymbol{(}\AgdaFunction{natind}\AgdaSpace{}%
\AgdaBound{n}\AgdaSpace{}%
\AgdaBound{base}\AgdaSpace{}%
\AgdaBound{step}\AgdaSymbol{)}\<%
\\
%
\\[\AgdaEmptyExtraSkip]%
\>[0]\AgdaFunction{natrec'}\AgdaSpace{}%
\AgdaSymbol{:}\AgdaSpace{}%
\AgdaSymbol{\{}\AgdaBound{X}\AgdaSpace{}%
\AgdaSymbol{:}\AgdaSpace{}%
\AgdaPrimitive{Set}\AgdaSymbol{\}}\AgdaSpace{}%
\AgdaSymbol{→}\AgdaSpace{}%
\AgdaDatatype{ℕ}\AgdaSpace{}%
\AgdaSymbol{→}\AgdaSpace{}%
\AgdaBound{X}\AgdaSpace{}%
\AgdaSymbol{→}\AgdaSpace{}%
\AgdaSymbol{(}\AgdaDatatype{ℕ}\AgdaSpace{}%
\AgdaSymbol{→}\AgdaSpace{}%
\AgdaBound{X}\AgdaSpace{}%
\AgdaSymbol{→}\AgdaSpace{}%
\AgdaBound{X}\AgdaSymbol{)}\AgdaSpace{}%
\AgdaSymbol{→}\AgdaSpace{}%
\AgdaBound{X}\<%
\\
\>[0]\AgdaFunction{natrec'}\AgdaSpace{}%
\AgdaSymbol{=}\AgdaSpace{}%
\AgdaFunction{natind}\<%
\end{code}
We will defer the details of using induction and recursion principles for later
sections, when we actually give examples of pidgin proofs some of our grammars can
handle. For now, the keen reader should try using Agda.

%Question for conclusion: how do we teach agda proofs vs programs? i.e.
%how can it infer if its generating langauge for a computer scientist or a
%mathematician %Variables in mathematics are meant to be simple (like e₀) but in
%Agda, its generally advisable to use more expresive variables. %We can either
%name types in the definitional judgment, or the typing judgment, but it makes it
%more readible if they are only used when necessary (the minimilist perspective,
%only use dependent types when you have to)
