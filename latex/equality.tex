\subsection{What is Equality?}

% quote peter dybjer at oplss
% all out assault on equality

\begin{displayquote}
... the univalence axiom validates the common, but formally unjustified, practice
of identifying isomorphic objects. [cite hottbook]
\end{displayquote}


Mathematicians, and most people generally, have an intuition
for equality, that of an identification between two pieces of information
which intuitively must be the same thing, i.e. $2+2=4$. The philosophically
inclined might ask about identification generally. We showcase different
notions of identifying things in mathematics, logic, and type theory :

\begin{itemize}
\item Equivalence of propositions
\item Equality of sets
\item Equality of members of sets
\item Isomorphism of structures
\item Equality of terms
\item Equality of types
\end{itemize}

While there are notions of equality, sameness, or identification outside of
these formal domains, we don't dare take a philosophical stab at these notions here.
Earlier, we saw two notions of equality in type theory, the judgmental equality
in our introduction to MLTT, and the propositional equality which was used in
the twin prime conjecture. Judgmental equality is the means of computing, for
instance, that $2+2=4$, for there is no way of proving this other than appealing
to the definition of addition. Propositional equality, on the other hand, is
actually a type. It is defined as follows in Agda, with an accompanying natural
language definition from [cite hottbook] :

\begin{code}[hide]%
\>[0]\AgdaKeyword{module}\AgdaSpace{}%
\AgdaModule{equality}\AgdaSpace{}%
\AgdaKeyword{where}\<%
\end{code}
\begin{code}%
\>[0][@{}l@{\AgdaIndent{1}}]%
\>[2]\AgdaKeyword{data}\AgdaSpace{}%
\AgdaOperator{\AgdaDatatype{\AgdaUnderscore{}≡'\AgdaUnderscore{}}}\AgdaSpace{}%
\AgdaSymbol{\{}\AgdaBound{A}\AgdaSpace{}%
\AgdaSymbol{:}\AgdaSpace{}%
\AgdaPrimitive{Set}\AgdaSymbol{\}}\AgdaSpace{}%
\AgdaSymbol{:}\AgdaSpace{}%
\AgdaSymbol{(}\AgdaBound{a}\AgdaSpace{}%
\AgdaBound{b}\AgdaSpace{}%
\AgdaSymbol{:}\AgdaSpace{}%
\AgdaBound{A}\AgdaSymbol{)}\AgdaSpace{}%
\AgdaSymbol{→}\AgdaSpace{}%
\AgdaPrimitive{Set}\AgdaSpace{}%
\AgdaKeyword{where}\<%
\\
\>[2][@{}l@{\AgdaIndent{0}}]%
\>[4]\AgdaInductiveConstructor{r}\AgdaSpace{}%
\AgdaSymbol{:}\AgdaSpace{}%
\AgdaSymbol{(}\AgdaBound{a}\AgdaSpace{}%
\AgdaSymbol{:}\AgdaSpace{}%
\AgdaBound{A}\AgdaSymbol{)}\AgdaSpace{}%
\AgdaSymbol{→}\AgdaSpace{}%
\AgdaBound{a}\AgdaSpace{}%
\AgdaOperator{\AgdaDatatype{≡'}}\AgdaSpace{}%
\AgdaBound{a}\<%
\end{code}
\begin{definition}
  The formation rule says that given a type $A:\UU$ and two elements $a,b:A$, we can form the type $(\id[A]{a}{b}):\UU$ in the same universe.
  The basic way to construct an element of $\id{a}{b}$ is to know that $a$ and $b$ are the same.
  Thus, the introduction rule is a dependent function
  \[\refl{} : \prod_{a:A} (\id[A]{a}{a}) \]
  called \define{reflexivity},
  which says that every element of $A$ is equal to itself (in a specified way).  We regard $\refl{a}$ as being the
  constant path %path\indexdef{path!constant}\indexsee{loop!constant}{path, constant}
  at the point $a$.
\end{definition}


The astute might ask, what
does it mean to ``construct an element of $\id{a}{b}$''? For the mathematician
use to thinking in terms of sets $\{\id{a}{b} \mid a,b \in \mathbb{N} \}$ isn't
a well-defined notion. Due to its use of the axiom of extensionality, the set
theoretic notion of equality is, no suprise, extensional.  This means that sets
are identified when they have the same elements, and equality is therefore
external to the notion of set. To inhabit a type means to provide evidence for
that inhabitation. The reflexivity constructor is therefore a means of
providing evidence of an equality. This evidence approach is disctinctly
constructive, and a big reason why classical and constructive mathematics,
especially when treated in an intuitionistic type theory suitable for a
programming language implementation, are such different beasts.

In Martin-Löf Type Theory, there are two fundamental notions of equality,
propositional and definitional.  While propositional equality is inductively
defined (as above) as a type which may have possibly more than one inhabitant,
definitional equality, denoted $-\equiv -$ and perhaps more aptly named
computational equality, is familiarly what most people think of as equality.
Namely, two terms which compute to the same canonical form are computationally
equal. In intensional type theory, propositional equality is a weaker notion
than computational equality : all propositionally equal terms are
computationally equal. However, computational equality does not imply
propistional equality - if it does, then one enters into the space of
extensional type theory.

Prior to the homotopical interpretation of identity types, debates about
extensional and intensional type theories centred around two features or bugs :
extensional type theory sacrificed decideable type checking, while intensional
type theories required extra beauracracy when dealing with equality in proofs.
One approach in intensional type theories treated types as setoids, therefore
leading to so-called ``Setoid Hell''. These debates reflected Martin-Löf's
flip-flopping on the issue. His seminal 1979 Constructive Mathematics and
Computer Programming, which took an extensional view, was soon betrayed by
lectures he gave soon thereafter in Padova in 1980.  Martin-Löf was a born
again intensional type theorist.  These Padova lectures were later published in
the "Bibliopolis Book", and went on to inspire the European (and Gothenburg in
particular) approach to implementing proof assitants, whereas the
extensionalists were primarily eminating from Robert Constable's group at
Cornell.

This tension has now been at least partially resolved, or at the very least
clarified, by an insight Voevodsky was apparently most proud of : the
introduction of h-levels. We'll delegate these details to more advanced references, it
is mentioned here to indicate that extensional type theory was really ``set
theory'' in disguise, in that it collapses the higher path structure of
identity types. The work over the past 10 years has elucidated the intensional
and extensional positions. HoTT, by allowing higher paths, is unashamedly
intentional, and admits a collapse into the extensional universe if so desired.
We now the examine the structure induced by this propositional equality.
