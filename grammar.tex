
% \section{A Spectrum of GF Grammars for types}
\section{Prior GF Formalizations}

Prior to the grammars explored thesis, Ranta produced two main results
\cite{rantaLog} \cite{aarneHott}. These are incredibly important precedents in
this approach to proof translation, and serve as important comparative work for
which this work responds.

\subsection{CADE 2011}

In \cite{rantaLog}, Ranta designed a grammar which
allowed for predicate logic with a domain specific lexicon supporting mathematical
theories , say geometry or arithmetic, on top of the logic. The syntax was
both meant to be relatively complete, so that typical logical utterances of
interest could be accommodated, as well as support relatively non-trivial linguistic
nuance like lists of terms, predicates, and propositions, in-situ and
bounded quantification, like other ways of constructing more syntactically
nuanced predicates. The more interesting syntactic details captured in this work
was by means of an extended grammar on top of the core. The bidirectional
transformation between the core and extended grammars via a PGF also show the
viability and necessity of using more expressive programming languages (Haskell) when
doing thorough translations.

Lists are natural to humans - this is reflected in our language. The RGL supports listing the sentences, noun phrases, and other
grammatical categories. One can then use PGF to unroll the lists into binary
operators, or alternatively transform them in the opposite direction.
, we first mention that GF
natively supports list categories, the judgment \term{cat [C] {n}} can be
desugared to
\begin{verbatim}
  cat ListC ;
  fun BaseC : C -> ... -> C -> ListC ; -- n C ’s
  fun ConsC : C -> ListC -> ListC
\end{verbatim}

As a case study for this grammar, the proposition $\forall x (Nat(x) \supset
Even(x) \lor Odd(x))$ can be given a maximized and minimized version. The tree
representing the \emph{syntactically complete} phrase ``for all natural numbers
x, x is even or x is odd" would be minimized to a tree which linearizes to the
\emph{semantically adequate} phrase ``every natural number is even or odd".

% The trees below also revels how the semantically adequate tree is also
% simpler to understand as representing
% \begin{verbatim}
% PUnivs
%   (BaseVar X)
%   Nat
%   (PConj COr (PAtom (APred1 Even (IVar X))) (PAtom (APred1 Odd (IVar X))))
% |
% V
% PAtom (APred1 (ConjPred1 COr (BasePred1 Even Odd)) (IUniv Nat))
% \
% end{verbatim}

We see that our criteria of semantic adequacy and syntactic completeness can
both occur in the same grammar, with the different subsets related not by a
direct GF translation but a PGF level transformation. Problematically, this
syntactically complete phrase produces four ASTs, with the ``or" and ``forall"
competing for precedence. Where PGF may only give one translation to the
extended syntax, this doesn't give the user of the grammar confidence that her
phrase was correctly interpreted.

In the opposite direction, the desugaring of a logically ``informal"
statement into something less linguistically idiomatic is also accomplished.
Ranta claims ``Finding extended syntax equivalents for core syntax trees is
trickier than the opposite direction". While this may be true for this
particular grammar, we argue that this may not hold generally. Dealing with these ambiguities must be
solved first and foremost to satisfy the PL designer who only accepts
unambiguous parses. For instance, the gf shell shows ``the sum of the
sum of x and y and z is equal to the sum of x and the sum of y and z" giving 32
unique parses. Ranta also outlines the mapping, $\llbracket -
\rrbracket : Core \to Extended$, which should hypothetically return a set of extended
sentences for a more comprehensive grammar.

\begin{itemize}
\item Flattening a list
  $x\ and\ y\ and\ z\ \mapsto x,\ y\ and\ z$
\item Aggregation
  $x\ is\ even\ or\ x\ is\ odd\ \mapsto x\ is\ even\ or\ odd$
\item In-situ quantification \\
  $\forall\ n\ \in Nat,\ x\ is\ even\ or\ x\ is\ odd \mapsto every\ Nat\ is\ even\ or \odd$
\item Negation
  $it\ is\ not\ that\ case\ that\ x\ is\ even\ \mapsto \x is\ not\ even$
\item Reflexivitazion
  $x\ is\ equal\ to\ x\ \mapsto x\ is\ equal\ to\ itself$
\item Modification
  $x\ is\ a\ number\ and\ x\ is\ even\ \mapsto x\ is\ an\ even\ number$
\end{itemize}

Scaling this to cover more phenomena, such as those from [cite ganesalingam], will
pose challenges. Extending this work in general without very sophisticated
statistical methods is impossible because mathematicians will speak uniquely,
and so choosing how to extend a grammar that covers the multiplicity of ways of saying
``the same thing" will require many choices and a significant corpus of examples. Efficient
communication, is a pragmatic feature which this only begins to barely address. The most
interesting linguistic phenomena covered by this grammar, In-situ
quantification, has been at the heart of the Montague tradition.

In some sense, this grammar serves as a case study for what this thesis is
trying to do. However, we note that the core logic only supports propositions
without proofs - it is not a type theory with terms. Additionally, the domain of
arithmetic is an important case study, but scaling this grammar (or any other,
for that matter) to allow for \emph{semantic adequacy} of real mathematics is
still far away, or as Ranta concedes, ``it seems that text generation involves
undecidable optimization problems that have no ultimate automatic solution." It
would be interesting to further extend this grammar with both terms and an
Agda-like concrete syntax.

\subsubsection{An Additional PGF Grammar}

One of the difficulties encountered in this work was reverse engineering Ranta's
code - the large size of a grammar and declarative nature of the code makes it
incredibly difficult to isolate individual features one may wish to understand.
This is true for both GF and PGF, and therefore a lot of work went into
filtering the grammars to understand behaviors of individual components of
interest. Careful usage of the GF module system may allow one to look at
``subgrammars" for some circumstances, but there is not proper methodology to
extract a sub-grammar and therefore it was found that writing a grammar from
scratch was often the easiest way to do this. Grammars can be written
compositionally (adding new categories and functions, refactoring linearization
types, etc.) but decomposing them is not a compositional process.

We wrote a smaller version [cite mycode] of, just focused on propositional
logic, but with the added interest of not just translating between Trees, but
also allowing Haskell computation and evaluation of expressions. Although this
exercise was in some ways a digression from the language of proofs, it also
highlighted many interesting problems.

We begin with an example : the idea was to create a PGF layer for the evaluation
of propositional expressions to their Boolean values, and then create a question
answering system which gave different types of answers - the binary valued
answer, the most verbose possible answer, and the answer which was deemed the
most semantically adequate, \codeword{Simple}, \codeword{Verbose}, and
\codeword{Compressed}, respectively. The system is capable of the following :
\begin{verbatim}
is it the case that if the sum of 3 , 4 and 5 is prime , odd and even then 4
  is prime and even

  Simple : yes .
  Verbose : yes . if the sum of 3 and the sum of 4 and 5 is prime and the sum
    of 3 and the sum of 4 and 5 is odd and the sum of 3 and the sum of 4 and
    5 is even then 4 is prime and 4 is even .
  Compressed : yes . if the sum of 3 , 4 and 5 is prime , odd and even then
    4 is prime and even .
\end{verbatim}

The extended grammar in this case only had lists of propositions and predicates,
and so it was much simpler than [cite logic]. GF list categories are then
transformed into Haskell lists via PGF, so the syntactic sugar for a GF list is actually
functionally tied to its external behavior as well. The functions for our
discussion are:
\begin{verbatim}
  IsNumProp : NumPred -> Object -> Prop ;
  LstNumPred : Conj -> [NumPred] -> NumPred ;
  LstProp : Conj  -> [Prop] -> Prop ;
\end{verbatim}

Note that a numerical predicate, \codeword{NumPred}, represents, for instance,
primality. In order for our pipeline to answer the question, we had to not only
do transform trees, $\llbracket - \rrbracket : \{pgfAST\} \rightarrow
\{pgfAST\}$ , but also evaluate them in more classical domains $\llbracket -
\rrbracket : \{pgfAST\} \rightarrow \mathds{N}$ for the arithmetic objects and
$\llbracket - \rrbracket : \{pgfAST\} \rightarrow \mathds{B}$,
\codeword{evalProp}, for the propositions.

The extension adds more complex cases  to cover when
evaluating propistions, because a normal ``propositional evaluator" doesn't have to
deal with lists. For the most part, this evaluation is able to just apply
boolean semantics to the \emph{canonical} propositional constructors, like \codeword{GNot}. However, a
bug that was subtle and difficult to find appeared, thereby forcing us to dig
deep inside GIsNumProp, preventing an easy solution to what would otherwise be a
simple example of denotational semantics.
\begin{verbatim}
evalProp :: GProp -> Bool
evalProp p = case p of
  ...
  GNot p -> not (evalProp p)
  ...
  GIsNumProp (GLstNumProp c (GListNumPred (x : xs))) obj ->
    let xo = evalProp (GIsNumProp x obj)
        xso = evalProp (GIsNumProp (GLstNumProp c (GListNumPred (xs))) obj) in
    case c of
      GAnd -> (&&) xo xso
      GOr -> (||) xo xso
  ...
\end{verbatim}
While this case is still relatively simple, an even more expressive abstract syntax
may yield many more subtle obstacles, which is the reason it's so hard
to understand PGF helper functions by just trying to read the code. The more semantic content one
incorporates into the GF grammar, the larger the PGF GADT, which leads to many
more cases when evaluating these trees.

There were many obstructions in engineering this relatively simple
example, particularly when it came to writing test cases. For the naive
way to test with GF is to translate, and the linearization and parsing functions
don't give the programmer many degrees of freedom. ASTs are not objects amenable
to human intuition, which makes it problematic because understanding the
transformations of them constantly requires parsing and linearizing to see their
``behavior". While some work has been done to allow testing
of GF grammars for NL applications [cite inari], the specific domain of formal languages in
GF requires a more refined notion of testing because they should be testable
relative to some model with well behaved mathematical properties. Debugging
something in the pipeline $String \rightarrow GADT \rightarrow GADT \rightarrow
String$ for a large scale grammar without a testing methodology for each
intermediate state is surely to be avoided.

Unfortunately, there is no published work on using Quickcheck [cite hughes] with
PGF. The bugs in this grammar were discovered via the input and output
\emph{appearance} of strings. Often, no string would be returned after a small
change, and discovering the source (abstract, concrete, or PGF) was
excruciating. In one case, a bug was discovered that was presumed to be from the
PGF evaluator, but was then back-traced to Ranta's grammar from which the code
had been refactored. The sentence which broke our pipeline from core to
extended, "4 is prime , 5 is even and if 6 is odd then 7 is even", would be
easily generated (or at least its AST) by quickcheck.

An important observation that was made during this development : that theorems
should be the source of inspirations for deciding which PGF transformations
should take place. For instance, one could define $odd : \mathds{N}
\rightarrow Set$, $prime : \mathds{N} \rightarrow Set$ and prove that $\forall n
\in \mathds{N}.\; n > 2 \times prime\; n \implies odd\; n$. We can use this
theorem as a source of translation, and in fact encode a PGF rule that
transforms anything of the form ``n is prime and n is odd" to ``n is prime",
subject to the condition that $n \neq 2$. One could then take a whole set of
theorems from predicate calculus and encode them as Haskell functions which
simplify the expressions to a minified expression with the
same meaning, up to some notion of equivalence. The verbose ``if $a$ then $b$
and if $a$ then $c$, can be more canonically read as ``if $a$ then $b$ and $c$".
The application of these theorems as evaluation functions in Haskell could
help give our QA example more informative and direct answers.

We hope this intricate look at a fairly simple grammar highlights some very
serious considerations one should make when writing a PGF embedded grammar.
These include : how does the semantic space the grammar seeks to approximate
effects the PGF translation, how testing formal grammars is non-trivial but
necessary future work, and finally, how information (in this case theorems) from
the domain of approximation can shape and inspire the PGF transformations 
during the translation process.



\subsection{Stockholm Math Seminar 2014}

In 2014, Ranta gave an unpublished talk at the Stockholm Mathematics seminar
\cite{aarneHott}. Fortunately the code is available, although many of the design
choices aren't documented in the grammar. This project aimed to provide a
translation like the one desired in our current work, but it took a real piece
of mathematics text as the main influence on the design of the Abstract syntax.

This work took a page of text from Peter Aczel's book which more or less goes
over standard HoTT definitions and theorems, and allows the translation of the
latex to a pidgin logical language. The central motivation of this grammar was
to capture, entirely ``real" natural language mathematics, i.e. that which was
written for the mathematician. Therefore, it isn't reminiscent of the slender
abstract syntax the type theorist adores, and sacrificed ``syntactic
completeness" for ``semantic adequacy". This means that the abstract syntax is
much larger and very expressive, but it no longer becomes easy to reason about
and additionally quite ad-hoc. Another defect is that this grammar
overgenerates, so producing a unique parse from the PL side will become tricky.
Nonetheless, this means that it's presumably possible to carve a subset of the
GF HoTT abstract file to accommodate an Agda program, but one encounters rocks as soon
as one begins to dig. For example, in \autoref{fig:M1} is some rendered latex
taken verbatim from Ranta's test code.

With some of hours of tinkering on the pidgin logic concrete syntax and some
reverse engineering with help from the GF shell, one is able to get these
definitions in \autoref{fig:M2}, which are intended to share the same syntactic
space as cubicalTT. We note the first definition of ``contractability" actually
runs in cubicalTT up to renaming a lexical items, and it is clear that the
translation from that to Agda should be a benign task. However, the
\emph{equivalence} syntax is stuck with the artifact from the bloated abstract
syntax for the of the anaphoric use of ``it", which may presumably be fixed with
a few hours more of tinkering, but becomes even more complicated when not just
defining new types, but actually writing real mathematical proofs, or relatively
large terms. To extend this grammar to accommodate a chapter worth of material,
let alone a book, will not just require extending the lexicon, but encountering
other syntactic phenomena that will further be difficult to compress when
defining Agda's concrete syntax.


\begin{code}[hide]%
\>[0]\AgdaSymbol{\{-\#}\AgdaSpace{}%
\AgdaKeyword{OPTIONS}\AgdaSpace{}%
\AgdaPragma{--omega-in-omega}\AgdaSpace{}%
\AgdaPragma{--type-in-type}\AgdaSpace{}%
\AgdaSymbol{\#-\}}\<%
\\
%
\\[\AgdaEmptyExtraSkip]%
\>[0]\AgdaKeyword{module}\AgdaSpace{}%
\AgdaModule{ContrClean}\AgdaSpace{}%
\AgdaKeyword{where}\<%
\\
%
\\[\AgdaEmptyExtraSkip]%
\>[0]\AgdaKeyword{open}\AgdaSpace{}%
\AgdaKeyword{import}\AgdaSpace{}%
\AgdaModule{Agda.Builtin.Sigma}\AgdaSpace{}%
\AgdaKeyword{public}\<%
\\
%
\\[\AgdaEmptyExtraSkip]%
\>[0]\AgdaKeyword{variable}\<%
\\
\>[0][@{}l@{\AgdaIndent{0}}]%
\>[2]\AgdaGeneralizable{A}\AgdaSpace{}%
\AgdaGeneralizable{B}\AgdaSpace{}%
\AgdaSymbol{:}\AgdaSpace{}%
\AgdaPrimitive{Set}\<%
\\
%
\\[\AgdaEmptyExtraSkip]%
\>[0]\AgdaKeyword{data}\AgdaSpace{}%
\AgdaOperator{\AgdaDatatype{\AgdaUnderscore{}≡\AgdaUnderscore{}}}\AgdaSpace{}%
\AgdaSymbol{\{}\AgdaBound{A}\AgdaSpace{}%
\AgdaSymbol{:}\AgdaSpace{}%
\AgdaPrimitive{Set}\AgdaSymbol{\}}\AgdaSpace{}%
\AgdaSymbol{(}\AgdaBound{a}\AgdaSpace{}%
\AgdaSymbol{:}\AgdaSpace{}%
\AgdaBound{A}\AgdaSymbol{)}\AgdaSpace{}%
\AgdaSymbol{:}\AgdaSpace{}%
\AgdaBound{A}\AgdaSpace{}%
\AgdaSymbol{→}\AgdaSpace{}%
\AgdaPrimitive{Set}\AgdaSpace{}%
\AgdaKeyword{where}\<%
\\
\>[0][@{}l@{\AgdaIndent{0}}]%
\>[2]\AgdaInductiveConstructor{r}\AgdaSpace{}%
\AgdaSymbol{:}\AgdaSpace{}%
\AgdaBound{a}\AgdaSpace{}%
\AgdaOperator{\AgdaDatatype{≡}}\AgdaSpace{}%
\AgdaBound{a}\<%
\\
%
\\[\AgdaEmptyExtraSkip]%
\>[0]\AgdaKeyword{infix}\AgdaSpace{}%
\AgdaNumber{20}\AgdaSpace{}%
\AgdaOperator{\AgdaDatatype{\AgdaUnderscore{}≡\AgdaUnderscore{}}}\<%
\\
%
\\[\AgdaEmptyExtraSkip]%
\>[0]\AgdaFunction{id}\AgdaSpace{}%
\AgdaSymbol{:}\AgdaSpace{}%
\AgdaGeneralizable{A}\AgdaSpace{}%
\AgdaSymbol{→}\AgdaSpace{}%
\AgdaGeneralizable{A}\<%
\\
\>[0]\AgdaFunction{id}\AgdaSpace{}%
\AgdaSymbol{=}\AgdaSpace{}%
\AgdaSymbol{λ}\AgdaSpace{}%
\AgdaBound{z}\AgdaSpace{}%
\AgdaSymbol{→}\AgdaSpace{}%
\AgdaBound{z}\<%
\\
\>[0]\<%
\end{code}

\begin{figure}[H]
\textbf{Definition}:
A type $A$ is contractible, if there is $a : A$, called the center of contraction, such that for all $x : A$, $\equalH {a}{x}$.
\caption{Rendered Latex} \label{fig:R1}
\begin{verbatim}
isContr ( A : Set ) : Set = ( a : A ) ( * ) ( ( x : A ) -> Id ( a ) ( x ) )
\end{verbatim}
\begin{code}%
\>[0]\AgdaFunction{isContr}\AgdaSpace{}%
\AgdaSymbol{:}\AgdaSpace{}%
\AgdaSymbol{(}\AgdaBound{A}\AgdaSpace{}%
\AgdaSymbol{:}\AgdaSpace{}%
\AgdaPrimitive{Set}\AgdaSymbol{)}\AgdaSpace{}%
\AgdaSymbol{→}\AgdaSpace{}%
\AgdaPrimitive{Set}\<%
\\
\>[0]\AgdaFunction{isContr}\AgdaSpace{}%
\AgdaBound{A}\AgdaSpace{}%
\AgdaSymbol{=}%
\>[13]\AgdaRecord{Σ}\AgdaSpace{}%
\AgdaBound{A}\AgdaSpace{}%
\AgdaSymbol{λ}\AgdaSpace{}%
\AgdaBound{a}\AgdaSpace{}%
\AgdaSymbol{→}\AgdaSpace{}%
\AgdaSymbol{(}\AgdaBound{x}\AgdaSpace{}%
\AgdaSymbol{:}\AgdaSpace{}%
\AgdaBound{A}\AgdaSymbol{)}\AgdaSpace{}%
\AgdaSymbol{→}\AgdaSpace{}%
\AgdaSymbol{(}\AgdaBound{a}\AgdaSpace{}%
\AgdaOperator{\AgdaDatatype{≡}}\AgdaSpace{}%
\AgdaBound{x}\AgdaSymbol{)}\<%
\end{code}
\caption{Contractibility} \label{fig:R2}
\end{figure}

In \autoref{fig:R2}, we show the different syntax presentations of the
\emph{equivalence}, which is merely a bijection when restricted to sets. This is
of such fundamental idea in mathematics and HoTT in particular that it merits
its own chapter in [cite hott], but we only showcase one of its many equivalent
definitions. We see that the pidgin syntax is stuck with the anaphoric artifact
from the bloated abstract syntax, \codeword{fiber} has the type \codeword{it :
Set} instead of something like \codeword{(y : B) : Set}, and the y variable is
unbound in the \codeword{fiber} expression. This may presumably be fixed with a
few hours more of tinkering, but becomes even more complicated when not just
defining new types, but actually writing real mathematical proofs.

\begin{figure}[H]
\textbf{Definition}:
A map $f : \arrowH {A}{B}$ is an equivalence, if for all $y : B$, its fiber, $\comprehensionH {x}{A}{\equalH {\appH {f}{x}}{y}}$, is contractible.
We write $\equivalenceH {A}{B}$, if there is an equivalence $\arrowH {A}{B}$.
\begin{verbatim}
Equivalence ( f : A -> B ) : Set =
  ( y : B ) -> ( isContr ( fiber it ) ) ; ; ;
  fiber it : Set = ( x : A ) ( * ) ( Id ( f ( x ) ) ( y ) )
\end{verbatim}
\begin{code}%
\>[0]\AgdaFunction{Equivalence}\AgdaSpace{}%
\AgdaSymbol{:}\AgdaSpace{}%
\AgdaSymbol{(}\AgdaBound{A}\AgdaSpace{}%
\AgdaBound{B}\AgdaSpace{}%
\AgdaSymbol{:}\AgdaSpace{}%
\AgdaPrimitive{Set}\AgdaSymbol{)}\AgdaSpace{}%
\AgdaSymbol{→}\AgdaSpace{}%
\AgdaSymbol{(}\AgdaBound{f}\AgdaSpace{}%
\AgdaSymbol{:}\AgdaSpace{}%
\AgdaBound{A}\AgdaSpace{}%
\AgdaSymbol{→}\AgdaSpace{}%
\AgdaBound{B}\AgdaSymbol{)}\AgdaSpace{}%
\AgdaSymbol{→}\AgdaSpace{}%
\AgdaPrimitive{Set}\<%
\\
\>[0]\AgdaFunction{Equivalence}\AgdaSpace{}%
\AgdaBound{A}\AgdaSpace{}%
\AgdaBound{B}\AgdaSpace{}%
\AgdaBound{f}\AgdaSpace{}%
\AgdaSymbol{=}\AgdaSpace{}%
\AgdaSymbol{∀}\AgdaSpace{}%
\AgdaSymbol{(}\AgdaBound{y}\AgdaSpace{}%
\AgdaSymbol{:}\AgdaSpace{}%
\AgdaBound{B}\AgdaSymbol{)}\AgdaSpace{}%
\AgdaSymbol{→}\AgdaSpace{}%
\AgdaFunction{isContr}\AgdaSpace{}%
\AgdaSymbol{(}\AgdaFunction{fiber'}\AgdaSpace{}%
\AgdaBound{y}\AgdaSymbol{)}\<%
\\
\>[0][@{}l@{\AgdaIndent{0}}]%
\>[2]\AgdaKeyword{where}\<%
\\
\>[2][@{}l@{\AgdaIndent{0}}]%
\>[4]\AgdaFunction{fiber'}\AgdaSpace{}%
\AgdaSymbol{:}\AgdaSpace{}%
\AgdaSymbol{(}\AgdaBound{y}\AgdaSpace{}%
\AgdaSymbol{:}\AgdaSpace{}%
\AgdaBound{B}\AgdaSymbol{)}\AgdaSpace{}%
\AgdaSymbol{→}\AgdaSpace{}%
\AgdaPrimitive{Set}\<%
\\
%
\>[4]\AgdaFunction{fiber'}\AgdaSpace{}%
\AgdaBound{y}\AgdaSpace{}%
\AgdaSymbol{=}\AgdaSpace{}%
\AgdaRecord{Σ}\AgdaSpace{}%
\AgdaBound{A}\AgdaSpace{}%
\AgdaSymbol{(λ}\AgdaSpace{}%
\AgdaBound{x}\AgdaSpace{}%
\AgdaSymbol{→}\AgdaSpace{}%
\AgdaBound{y}\AgdaSpace{}%
\AgdaOperator{\AgdaDatatype{≡}}\AgdaSpace{}%
\AgdaBound{f}\AgdaSpace{}%
\AgdaBound{x}\AgdaSymbol{)}\<%
\end{code}
\caption{Contractibility} \label{fig:R3}
\end{figure}


Additionally, we give the Agda code in \autoref{fig:M3}, so-as to see what the
end result of such a program would be. The astute reader will also notice a
semantic in the pidgin rendering error relative to the Agda implementation.
\codeword{fiber} has the type \codeword{it : Set} instead of something like
\codeword{(y : B) : Set}, and the y variable is unbound in the \codeword{fiber}
expression. This demonstrates that to design a grammar prioritizing
\emph{semantic adequacy} and subsequently trying to incorporate \emph{syntactic
completeness} becomes a very difficult problem. Depending on the application of
the grammar, the emphasis on this axis is most assuredly a choice one should
consider up front.

While both these grammars have their strengths and weaknesses, one shall see
shortly that the approach in this thesis, taking an actual programming language
parser in Backus-Naur Form Converter (BNFC), GFifying it, and trying to use the
abstract syntax to model natural language, gives in some sense a dual challenge,
where the abstract syntax remains simple, but its linearizations become
must increase in complexity.

-- Proof using isPropIsContr. This is slow and the direct proof below is better

isPropIsEquiv' : (f : A → B) → isProp (isEquiv f)
equiv-proof (isPropIsEquiv' f u0 u1 i) y =
  isPropIsContr (u0 .equiv-proof y) (u1 .equiv-proof y) i

-- Direct proof that computes quite ok (can be optimized further if
-- necessary, see:
q-- HTTPSqqq://github.com/mortberg/cubicaltt/blob/pi4s3_dimclosures/examples/brunerie2.ctt#L562




% Example expressions the grammar can parse are seen below, which have been
% verified by hand to be isomorphic to the corresponding cubicaltt BNFC trees:

% \begin{verbatim}

% data bool : Set where true | false
% data nat : Set where zero | suc ( n : nat )
% caseBool ( x : Set ) ( y z : x ) : bool -> Set = split false -> y || true -> z
% indBool ( x : bool -> Set ) ( y : x false ) ( z : x true ) : ( b : bool ) -> x b = split false -> y || true  -> z
% funExt  ( a : Set )   ( b : a -> Set )   ( f g :  ( x : a )  -> b x )   ( p :  ( x : a )  -> ( b x )   ( f x ) == ( g x )  )  : (  ( y : a )  -> b y )  f == g = undefined
% foo ( b : bool ) : bool = b

% \end{verbatim}

% [Todo] add use cases
