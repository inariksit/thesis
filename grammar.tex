PGF 
% \section{A Spectrum of GF Grammars for types}
\section{Prior GF Formalizations}

\subsection{CADE 2011} \label{cade}

In \cite{rantaLog}, Ranta designed a grammar which allowed for predicate logic
with a domain specific lexicon supporting mathematical theories like geometry or
arithmetic. The syntax was both meant to be relatively complete so that typical
logical utterances of interest could be accommodated, as well as support
relatively non-trivial linguistic nuance. The nuances included lists of terms,
predicates, and propositions and in-situ and bounded quantification. The more
interesting syntactic details captured in this work was by means of an extended
grammar on top of the core. The bidirectional transformation between the core
and extended grammars via a Haskell transformation also show the viability and
necessity of using more expressive programming languages as intermediaries when
doing thorough translations.

As a simple example, the proposition $\forall x (Nat(x) \supset
Even(x) \lor Odd(x))$ can be given a maximized and minimized version. The tree
representing the syntactically complete phrase ``for all natural numbers
x, x is even or x is odd" would be minimized to a tree which linearizes to the
semantically adequate phrase ``every natural number is even or odd".

We see that our criteria of semantic adequacy and syntactic completeness can
both occur in the same grammar, with Haskell level transformation allowing one
to go between the them. Problematically, this syntactically complete phrase
produces four ASTs, with the ``or" and ``forall" competing for precedence. There
is therefore no assurance give the user of the grammar confidence that her
phrase was correctly interpreted without deciding which translation is best.

In the opposite direction, the desugaring of a logically ``informal" statement
into something less linguistically idiomatic is also accomplished. Ranta claims
``Finding extended syntax equivalents for core syntax trees is trickier than the
opposite direction". While this may be true for this particular grammar, we
argue that this may not hold generally.


The RGL supports listing the sentences, noun phrases, and other grammatical
categories. One can then use Haskell to unroll the lists into binary operators, or
alternatively transform them in the opposite direction. , we first mention that
GF natively supports list categories, the judgment \term{cat [C] {n}} can be
desugared to

\begin{verbatim}
  cat ListC ;
  fun BaseC : C -> ... -> C -> ListC ; -- n C ’s
  fun ConsC : C -> ListC -> ListC
\end{verbatim}

We could therefore transform the extended language phrase ``The sum of x , y ,
and z is equal to itself" in to core language phrase ``the sum of the sum of x
and y and z is equal to the sum of x and the sum of y and z". Parsing this core
string gives 32 unique trees, and dealing with ambiguities must be solved first
and foremost to satisfy the PL designer who only accepts unambiguous parses.

Ranta outlines the mapping, $\llbracket - \rrbracket : Core \to Extended$,
which should hypothetically return a set of extended sentences for a more
comprehensive grammar.

\begin{itemize}
\item Flattening a list
  $x\ and\ y\ and\ z\ \mapsto x,\ y\ and\ z$
\item Aggregation
  $x\ is\ even\ or\ x\ is\ odd\ \mapsto x\ is\ even\ or\ odd$
\item In-situ quantification \\
  $\forall\ n\ \in Nat,\ x\ is\ even\ or\ x\ is\ odd \mapsto every\ Nat\ is\ even\ or \odd$
\item Negation
  $it\ is\ not\ that\ case\ that\ x\ is\ even\ \mapsto \x is\ not\ even$
\item Reflexivitazion
  $x\ is\ equal\ to\ x\ \mapsto x\ is\ equal\ to\ itself$
\item Modification
  $x\ is\ a\ number\ and\ x\ is\ even\ \mapsto x\ is\ an\ even\ number$
\end{itemize}

Scaling this to cover more phenomena, such as those from Ganesalingam's analysis
will pose challenges. Extending this work in general without very sophisticated
statistical methods is impossible because mathematicians will speak uniquely,
and so choosing how to extend a grammar that covers the multiplicity of ways of
saying ``the same thing" will require many choices and a significant corpus of
examples. The most interesting linguistic phenomena covered by this grammar,
In-situ quantification, has been at the heart of the Montague tradition.

While this grammar serves as a precedent for this work generally, we note that
the core logic only supports propositions without proofs - it is not a type
theory with terms. Additionally, the domain of arithmetic is an important case
study, but scaling this grammar (or any other, for that matter) to allow for
\emph{semantic adequacy} of real mathematics is still far away, or as Ranta
concedes, ``it seems that text generation involves undecidable optimization
problems that have no ultimate automatic solution."

\subsubsection{A Question Answering Example}

One of the difficulties encountered in this work was reverse engineering Ranta's
code - the large size and declarative nature of a grammar makes it incredibly
difficult to isolate individual features one may wish to understand. This is
true for both GF and its Haskell embedding. Significant efforts went into
filtering the grammars to understand behaviors of individual components of
interest. Careful usage of the GF module system may allow one to look at
``subgrammars" in some circumstances, but there is not proper methodology to
extract a sub-grammar and therefore it was found that writing a grammar from
scratch was often the easiest way to do this. Grammars can be written
compositionally (adding new categories and functions, refactoring linearization
types, etc.) but decomposing them is not a compositional process.

We wrote a smaller version of the above grammar \cite{warrickCub} just focused on
propositional logic. It included an added component not just translating between
ASTs, but also allowing intermediary computation and of logical
propositions and numerical expressions. Although this exercise was in some ways
a digression from the language of proofs, it also highlighted many interesting
problems.

We begin with an example : the idea was to create a PGF layer for the evaluation
of propositional expressions to their Boolean values, and then create a question
answering system which gave different types of answers - the binary valued
answer, the most verbose possible answer, and the answer which was deemed the
most semantically adequate, \codeword{Simple}, \codeword{Verbose}, and
\codeword{Compressed}, respectively. The system is capable of the following :
\begin{verbatim}
is it the case that if the sum of 3 , 4 and 5 is prime , odd and even then 4
  is prime and even

  Simple : yes .
  Verbose : yes . if the sum of 3 and the sum of 4 and 5 is prime and the sum
    of 3 and the sum of 4 and 5 is odd and the sum of 3 and the sum of 4 and
    5 is even then 4 is prime and 4 is even .
  Compressed : yes . if the sum of 3 , 4 and 5 is prime , odd and even then
    4 is prime and even .
\end{verbatim}

The extended grammar in this case only had lists of propositions and predicates,
and so it was much simpler than [cite logic]. GF list categories are then
transformed into Haskell lists via PGF, so the syntactic sugar for a GF list is actually
functionally tied to its external behavior as well. The functions for our
discussion are:
\begin{verbatim}
  IsNumProp : NumPred -> Object -> Prop ;
  LstNumPred : Conj -> [NumPred] -> NumPred ;
  LstProp : Conj  -> [Prop] -> Prop ;
\end{verbatim}

Note that a numerical predicate, \codeword{NumPred}, represents, for instance,
primality. In order for our pipeline to answer the question, we had to not only
do transform trees, $\llbracket - \rrbracket : \{pgfAST\} \rightarrow
\{pgfAST\}$ , but also evaluate them in more classical domains $\llbracket -
\rrbracket : \{pgfAST\} \rightarrow \mathds{N}$ for the arithmetic objects and
$\llbracket - \rrbracket : \{pgfAST\} \rightarrow \mathds{B}$,
\codeword{evalProp}, for the propositions.

The extension adds more complex cases  to cover when
evaluating propistions, because a normal ``propositional evaluator" doesn't have to
deal with lists. For the most part, this evaluation is able to just apply
boolean semantics to the \emph{canonical} propositional constructors, like \codeword{GNot}. However, a
bug that was subtle and difficult to find appeared, thereby forcing us to dig
deep inside GIsNumProp, preventing an easy solution to what would otherwise be a
simple example of denotational semantics.
\begin{verbatim}
evalProp :: GProp -> Bool
evalProp p = case p of
  ...
  GNot p -> not (evalProp p)
  ...
  GIsNumProp (GLstNumProp c (GListNumPred (x : xs))) obj ->
    let xo = evalProp (GIsNumProp x obj)
        xso = evalProp (GIsNumProp (GLstNumProp c (GListNumPred (xs))) obj) in
    case c of
      GAnd -> (&&) xo xso
      GOr -> (||) xo xso
  ...
\end{verbatim}
While this case is still relatively simple, an even more expressive abstract syntax
may yield many more subtle obstacles, which is the reason it's so hard
to understand PGF helper functions by just trying to read the code. The more semantic content one
incorporates into the GF grammar, the larger the PGF GADT, which leads to many
more cases when evaluating these trees.

There were many obstructions in engineering this relatively simple
example, particularly when it came to writing test cases. For the naive
way to test with GF is to translate, and the linearization and parsing functions
don't give the programmer many degrees of freedom. ASTs are not objects amenable
to human intuition, which makes it problematic because understanding the
transformations of them constantly requires parsing and linearizing to see their
``behavior". While some work has been done to allow testing
of GF grammars for NL applications [cite inari], the specific domain of formal languages in
GF requires a more refined notion of testing because they should be testable
relative to some model with well behaved mathematical properties. Debugging
something in the pipeline $String \rightarrow GADT \rightarrow GADT \rightarrow
String$ for a large scale grammar without a testing methodology for each
intermediate state is surely to be avoided.

Unfortunately, there is no published work on using Quickcheck [cite hughes] with
PGF. The bugs in this grammar were discovered via the input and output
\emph{appearance} of strings. Often, no string would be returned after a small
change, and discovering the source (abstract, concrete, or PGF) was
excruciating. In one case, a bug was discovered that was presumed to be from the
PGF evaluator, but was then back-traced to Ranta's grammar from which the code
had been refactored. The sentence which broke our pipeline from core to
extended, "4 is prime , 5 is even and if 6 is odd then 7 is even", would be
easily generated (or at least its AST) by quickcheck.

An important observation that was made during this development : that theorems
should be the source of inspirations for deciding which PGF transformations
should take place. For instance, one could define $odd : \mathds{N}
\rightarrow Set$, $prime : \mathds{N} \rightarrow Set$ and prove that $\forall n
\in \mathds{N}.\; n > 2 \times prime\; n \implies odd\; n$. We can use this
theorem as a source of translation, and in fact encode a PGF rule that
transforms anything of the form ``n is prime and n is odd" to ``n is prime",
subject to the condition that $n \neq 2$. One could then take a whole set of
theorems from predicate calculus and encode them as Haskell functions which
simplify the expressions to a minified expression with the
same meaning, up to some notion of equivalence. The verbose ``if $a$ then $b$
and if $a$ then $c$, can be more canonically read as ``if $a$ then $b$ and $c$".
The application of these theorems as evaluation functions in Haskell could
help give our QA example more informative and direct answers.

We hope this intricate look at a fairly simple grammar highlights some very
serious considerations one should make when writing a PGF embedded grammar.
These include : how does the semantic space the grammar seeks to approximate
effects the PGF translation, how testing formal grammars is non-trivial but
necessary future work, and finally, how information (in this case theorems) from
the domain of approximation can shape and inspire the PGF transformations 
during the translation process.
