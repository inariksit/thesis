
\section{A Spectrum of GF Grammars for types}

\subsection{Prior GF Formalizations}

Prior to the grammars explored thin this thesis, Ranta produced two main results
\cite{rantaLog} \cite{aarneHott}. These are incredibly important precedents in
this approach to proof translation, and serve as important comparative work for
which this work responds. In \cite{rantaLog}, Ranta designed a grammar which
allowed for predicate logic with domain specific lexicon supporting mathematical
theories on top of the logic like geometry or arithmetic. The the syntax was
both meant to be relatively complete, so that typical logical utterances of
interest could be accommodated, as well as relatively non-trivial linguistic
nuance including lists of terms, predicates, and propositions, in-situ and
bounded quantification, and multiple forms of constructing more syntactically
nuanced predicates. The syntactic nuance captured in this work was by means of an
extended grammar, via a Portable Grammar Format (PGF), on top of the minimal,
core logical formalism.

One could translate from the core and extended via a denotational semantics
approach. The tree representing the \emph{syntactically complete} phrase ``for
all natural numbers x, x is even or x is odd" would be evaluated to a tree which
linearizes to the \emph{semantically adequate} phrase ``every natural number is
even or odd". In the opposite direction, the desugaring of a logically informal
statement into something linguistically idiomatic is also accomplished. In some
sense, this grammar serves as a case study for what this thesis is trying to do.
However, we note that the core logic only supports propositions without proofs -
it is not a type theory with terms. This means that we are being slightly
abusive to our terms, as the formal/informal translation is taking place is at
the PGF level. The GF translation between concrete syntaxes supports multiple
NLs, but the syntactic completeness has no mechanism of verification via Agda's
type checker. Additionally, the domain of arithmetic is an important case study,
but scaling this grammar (or any other, for that matter) to allow for
\emph{semantic adequacy} of real mathematics is still far away, or as Ranta
concedes, ``it seems that text generation involves undecidable optimization
problems that have no ultimate automatic solution." It would be interesting to
further extend this grammar with both terms and an Agda-like concrete syntax. 

In 2014, Ranta gave an unpublished talk at the Stockholm Mathematics seminar
\cite{aarneHott}. Fortunately the code is available, although many of the design
choices aren't documented in the grammar. This project aimed to provide a
translation like the one desired in our current work, but it took a real piece
of mathematics text as the main influence on the design of the Abstract syntax.

This work took a page of text from Peter Aczel's book which more or less goes
over standard HoTT definitions and theorems, and allows the translation of the
latex to a pidgin logical language. The central motivation of this grammar was
to capture, entirely ``real" natural language mathematics, i.e. that which was
written for the mathematician. Therefore, it isn't reminiscent of the slender
abstract syntax the type theorist adores, and sacrificed ``syntactic
completeness" for ``semantic adequacy". This means that the abstract syntax is
much larger and very expressive, but it no longer becomes easy to reason about
and additionally quite ad-hoc. Another defect is that this grammar
overgenerates, so producing a unique parse from the PL side will become tricky.
Nonetheless, this means that it's presumably possible to carve a subset of the
GF HoTT abstract file to accommodate an Agda program, but one encounters rocks as soon
as one begins to dig. For example, in \autoref{fig:M1} is some rendered latex
taken verbatim from Ranta's test code.

With some of hours of tinkering on the pidgin logic concrete syntax and some
reverse engineering with help from the GF shell, one is able to get these
definitions in \autoref{fig:M2}, which are intended to share the same syntactic
space as cubicalTT. We note the first definition of ``contractability" actually
runs in cubicalTT up to renaming a lexical items, and it is clear that the
translation from that to Agda should be a benign task. However, the
\emph{equivalence} syntax is stuck with the artifact from the bloated abstract
syntax for the of the anaphoric use of ``it", which may presumably be fixed with
a few hours more of tinkering, but becomes even more complicated when not just
defining new types, but actually writing real mathematical proofs, or relatively
large terms. To extend this grammar to accommodate a chapter worth of material,
let alone a book, will not just require extending the lexicon, but encountering
other syntactic phenomena that will further be difficult to compress when
defining Agda's concrete syntax.


\begin{code}[hide]%
\>[0]\AgdaSymbol{\{-\#}\AgdaSpace{}%
\AgdaKeyword{OPTIONS}\AgdaSpace{}%
\AgdaPragma{--omega-in-omega}\AgdaSpace{}%
\AgdaPragma{--type-in-type}\AgdaSpace{}%
\AgdaSymbol{\#-\}}\<%
\\
%
\\[\AgdaEmptyExtraSkip]%
\>[0]\AgdaKeyword{module}\AgdaSpace{}%
\AgdaModule{ContrClean}\AgdaSpace{}%
\AgdaKeyword{where}\<%
\\
%
\\[\AgdaEmptyExtraSkip]%
\>[0]\AgdaKeyword{open}\AgdaSpace{}%
\AgdaKeyword{import}\AgdaSpace{}%
\AgdaModule{Agda.Builtin.Sigma}\AgdaSpace{}%
\AgdaKeyword{public}\<%
\\
%
\\[\AgdaEmptyExtraSkip]%
\>[0]\AgdaKeyword{variable}\<%
\\
\>[0][@{}l@{\AgdaIndent{0}}]%
\>[2]\AgdaGeneralizable{A}\AgdaSpace{}%
\AgdaGeneralizable{B}\AgdaSpace{}%
\AgdaSymbol{:}\AgdaSpace{}%
\AgdaPrimitive{Set}\<%
\\
%
\\[\AgdaEmptyExtraSkip]%
\>[0]\AgdaKeyword{data}\AgdaSpace{}%
\AgdaOperator{\AgdaDatatype{\AgdaUnderscore{}≡\AgdaUnderscore{}}}\AgdaSpace{}%
\AgdaSymbol{\{}\AgdaBound{A}\AgdaSpace{}%
\AgdaSymbol{:}\AgdaSpace{}%
\AgdaPrimitive{Set}\AgdaSymbol{\}}\AgdaSpace{}%
\AgdaSymbol{(}\AgdaBound{a}\AgdaSpace{}%
\AgdaSymbol{:}\AgdaSpace{}%
\AgdaBound{A}\AgdaSymbol{)}\AgdaSpace{}%
\AgdaSymbol{:}\AgdaSpace{}%
\AgdaBound{A}\AgdaSpace{}%
\AgdaSymbol{→}\AgdaSpace{}%
\AgdaPrimitive{Set}\AgdaSpace{}%
\AgdaKeyword{where}\<%
\\
\>[0][@{}l@{\AgdaIndent{0}}]%
\>[2]\AgdaInductiveConstructor{r}\AgdaSpace{}%
\AgdaSymbol{:}\AgdaSpace{}%
\AgdaBound{a}\AgdaSpace{}%
\AgdaOperator{\AgdaDatatype{≡}}\AgdaSpace{}%
\AgdaBound{a}\<%
\\
%
\\[\AgdaEmptyExtraSkip]%
\>[0]\AgdaKeyword{infix}\AgdaSpace{}%
\AgdaNumber{20}\AgdaSpace{}%
\AgdaOperator{\AgdaDatatype{\AgdaUnderscore{}≡\AgdaUnderscore{}}}\<%
\\
%
\\[\AgdaEmptyExtraSkip]%
\>[0]\AgdaFunction{id}\AgdaSpace{}%
\AgdaSymbol{:}\AgdaSpace{}%
\AgdaGeneralizable{A}\AgdaSpace{}%
\AgdaSymbol{→}\AgdaSpace{}%
\AgdaGeneralizable{A}\<%
\\
\>[0]\AgdaFunction{id}\AgdaSpace{}%
\AgdaSymbol{=}\AgdaSpace{}%
\AgdaSymbol{λ}\AgdaSpace{}%
\AgdaBound{z}\AgdaSpace{}%
\AgdaSymbol{→}\AgdaSpace{}%
\AgdaBound{z}\<%
\\
\>[0]\<%
\end{code}

\begin{figure}[H]
\textbf{Definition}:
A type $A$ is contractible, if there is $a : A$, called the center of contraction, such that for all $x : A$, $\equalH {a}{x}$.
\caption{Rendered Latex} \label{fig:R1}
\begin{verbatim}
isContr ( A : Set ) : Set = ( a : A ) ( * ) ( ( x : A ) -> Id ( a ) ( x ) )
\end{verbatim}
\begin{code}%
\>[0]\AgdaFunction{isContr}\AgdaSpace{}%
\AgdaSymbol{:}\AgdaSpace{}%
\AgdaSymbol{(}\AgdaBound{A}\AgdaSpace{}%
\AgdaSymbol{:}\AgdaSpace{}%
\AgdaPrimitive{Set}\AgdaSymbol{)}\AgdaSpace{}%
\AgdaSymbol{→}\AgdaSpace{}%
\AgdaPrimitive{Set}\<%
\\
\>[0]\AgdaFunction{isContr}\AgdaSpace{}%
\AgdaBound{A}\AgdaSpace{}%
\AgdaSymbol{=}%
\>[13]\AgdaRecord{Σ}\AgdaSpace{}%
\AgdaBound{A}\AgdaSpace{}%
\AgdaSymbol{λ}\AgdaSpace{}%
\AgdaBound{a}\AgdaSpace{}%
\AgdaSymbol{→}\AgdaSpace{}%
\AgdaSymbol{(}\AgdaBound{x}\AgdaSpace{}%
\AgdaSymbol{:}\AgdaSpace{}%
\AgdaBound{A}\AgdaSymbol{)}\AgdaSpace{}%
\AgdaSymbol{→}\AgdaSpace{}%
\AgdaSymbol{(}\AgdaBound{a}\AgdaSpace{}%
\AgdaOperator{\AgdaDatatype{≡}}\AgdaSpace{}%
\AgdaBound{x}\AgdaSymbol{)}\<%
\end{code}
\caption{Contractibility} \label{fig:R2}
\end{figure}

In \autoref{fig:R2}, we show the different syntax presentations of the
\emph{equivalence}, which is merely a bijection when restricted to sets. This is
of such fundamental idea in mathematics and HoTT in particular that it merits
its own chapter in [cite hott], but we only showcase one of its many equivalent
definitions. We see that the pidgin syntax is stuck with the anaphoric artifact
from the bloated abstract syntax, \codeword{fiber} has the type \codeword{it :
Set} instead of something like \codeword{(y : B) : Set}, and the y variable is
unbound in the \codeword{fiber} expression. This may presumably be fixed with a
few hours more of tinkering, but becomes even more complicated when not just
defining new types, but actually writing real mathematical proofs.

\begin{figure}[H]
\textbf{Definition}:
A map $f : \arrowH {A}{B}$ is an equivalence, if for all $y : B$, its fiber, $\comprehensionH {x}{A}{\equalH {\appH {f}{x}}{y}}$, is contractible.
We write $\equivalenceH {A}{B}$, if there is an equivalence $\arrowH {A}{B}$.
\begin{verbatim}
Equivalence ( f : A -> B ) : Set =
  ( y : B ) -> ( isContr ( fiber it ) ) ; ; ;
  fiber it : Set = ( x : A ) ( * ) ( Id ( f ( x ) ) ( y ) )
\end{verbatim}
\begin{code}%
\>[0]\AgdaFunction{Equivalence}\AgdaSpace{}%
\AgdaSymbol{:}\AgdaSpace{}%
\AgdaSymbol{(}\AgdaBound{A}\AgdaSpace{}%
\AgdaBound{B}\AgdaSpace{}%
\AgdaSymbol{:}\AgdaSpace{}%
\AgdaPrimitive{Set}\AgdaSymbol{)}\AgdaSpace{}%
\AgdaSymbol{→}\AgdaSpace{}%
\AgdaSymbol{(}\AgdaBound{f}\AgdaSpace{}%
\AgdaSymbol{:}\AgdaSpace{}%
\AgdaBound{A}\AgdaSpace{}%
\AgdaSymbol{→}\AgdaSpace{}%
\AgdaBound{B}\AgdaSymbol{)}\AgdaSpace{}%
\AgdaSymbol{→}\AgdaSpace{}%
\AgdaPrimitive{Set}\<%
\\
\>[0]\AgdaFunction{Equivalence}\AgdaSpace{}%
\AgdaBound{A}\AgdaSpace{}%
\AgdaBound{B}\AgdaSpace{}%
\AgdaBound{f}\AgdaSpace{}%
\AgdaSymbol{=}\AgdaSpace{}%
\AgdaSymbol{∀}\AgdaSpace{}%
\AgdaSymbol{(}\AgdaBound{y}\AgdaSpace{}%
\AgdaSymbol{:}\AgdaSpace{}%
\AgdaBound{B}\AgdaSymbol{)}\AgdaSpace{}%
\AgdaSymbol{→}\AgdaSpace{}%
\AgdaFunction{isContr}\AgdaSpace{}%
\AgdaSymbol{(}\AgdaFunction{fiber'}\AgdaSpace{}%
\AgdaBound{y}\AgdaSymbol{)}\<%
\\
\>[0][@{}l@{\AgdaIndent{0}}]%
\>[2]\AgdaKeyword{where}\<%
\\
\>[2][@{}l@{\AgdaIndent{0}}]%
\>[4]\AgdaFunction{fiber'}\AgdaSpace{}%
\AgdaSymbol{:}\AgdaSpace{}%
\AgdaSymbol{(}\AgdaBound{y}\AgdaSpace{}%
\AgdaSymbol{:}\AgdaSpace{}%
\AgdaBound{B}\AgdaSymbol{)}\AgdaSpace{}%
\AgdaSymbol{→}\AgdaSpace{}%
\AgdaPrimitive{Set}\<%
\\
%
\>[4]\AgdaFunction{fiber'}\AgdaSpace{}%
\AgdaBound{y}\AgdaSpace{}%
\AgdaSymbol{=}\AgdaSpace{}%
\AgdaRecord{Σ}\AgdaSpace{}%
\AgdaBound{A}\AgdaSpace{}%
\AgdaSymbol{(λ}\AgdaSpace{}%
\AgdaBound{x}\AgdaSpace{}%
\AgdaSymbol{→}\AgdaSpace{}%
\AgdaBound{y}\AgdaSpace{}%
\AgdaOperator{\AgdaDatatype{≡}}\AgdaSpace{}%
\AgdaBound{f}\AgdaSpace{}%
\AgdaBound{x}\AgdaSymbol{)}\<%
\end{code}
\caption{Contractibility} \label{fig:R3}
\end{figure}


% \begin{figure}

%  \textbf{Definition}:
%  A type $A$ is contractible, if there is $a : A$, called the center of contraction, such that for all $x : A$, $\equalH {a}{x}$.

%  \textbf{Definition}:
%  A map $f : \arrowH {A}{B}$ is an equivalence, if for all $y : B$, its fiber, $\comprehensionH {x}{A}{\equalH {\appH {f}{x}}{y}}$, is contractible.
%  We write $\equivalenceH {A}{B}$, if there is an equivalence $\arrowH {A}{B}$.
% \caption{Rendered Latex} \label{fig:M1}


% \begin{verbatim}
% isContr ( A : Set ) : Set = ( a : A ) ( * ) ( ( x : A ) -> Id ( a ) ( x ) )

% Equivalence ( f : A -> B ) : Set = 
%   ( y : B ) -> ( isContr ( fiber it ) ) ; ; ; 
%   fiber it : Set = ( x : A ) ( * ) ( Id ( f ( x ) ) ( y ) )
% \end{verbatim}
% \caption{Pidgin cubicalTT} \label{fig:M2}

% 
\begin{code}[hide]%
\>[0]\AgdaSymbol{\{-\#}\AgdaSpace{}%
\AgdaKeyword{OPTIONS}\AgdaSpace{}%
\AgdaPragma{--omega-in-omega}\AgdaSpace{}%
\AgdaPragma{--type-in-type}\AgdaSpace{}%
\AgdaSymbol{\#-\}}\<%
\\
%
\\[\AgdaEmptyExtraSkip]%
\>[0]\AgdaKeyword{module}\AgdaSpace{}%
\AgdaModule{ContrClean}\AgdaSpace{}%
\AgdaKeyword{where}\<%
\\
%
\\[\AgdaEmptyExtraSkip]%
\>[0]\AgdaKeyword{open}\AgdaSpace{}%
\AgdaKeyword{import}\AgdaSpace{}%
\AgdaModule{Agda.Builtin.Sigma}\AgdaSpace{}%
\AgdaKeyword{public}\<%
\\
%
\\[\AgdaEmptyExtraSkip]%
\>[0]\AgdaKeyword{variable}\<%
\\
\>[0][@{}l@{\AgdaIndent{0}}]%
\>[2]\AgdaGeneralizable{A}\AgdaSpace{}%
\AgdaGeneralizable{B}\AgdaSpace{}%
\AgdaSymbol{:}\AgdaSpace{}%
\AgdaPrimitive{Set}\<%
\\
%
\\[\AgdaEmptyExtraSkip]%
\>[0]\AgdaKeyword{data}\AgdaSpace{}%
\AgdaOperator{\AgdaDatatype{\AgdaUnderscore{}≡\AgdaUnderscore{}}}\AgdaSpace{}%
\AgdaSymbol{\{}\AgdaBound{A}\AgdaSpace{}%
\AgdaSymbol{:}\AgdaSpace{}%
\AgdaPrimitive{Set}\AgdaSymbol{\}}\AgdaSpace{}%
\AgdaSymbol{(}\AgdaBound{a}\AgdaSpace{}%
\AgdaSymbol{:}\AgdaSpace{}%
\AgdaBound{A}\AgdaSymbol{)}\AgdaSpace{}%
\AgdaSymbol{:}\AgdaSpace{}%
\AgdaBound{A}\AgdaSpace{}%
\AgdaSymbol{→}\AgdaSpace{}%
\AgdaPrimitive{Set}\AgdaSpace{}%
\AgdaKeyword{where}\<%
\\
\>[0][@{}l@{\AgdaIndent{0}}]%
\>[2]\AgdaInductiveConstructor{r}\AgdaSpace{}%
\AgdaSymbol{:}\AgdaSpace{}%
\AgdaBound{a}\AgdaSpace{}%
\AgdaOperator{\AgdaDatatype{≡}}\AgdaSpace{}%
\AgdaBound{a}\<%
\\
%
\\[\AgdaEmptyExtraSkip]%
\>[0]\AgdaKeyword{infix}\AgdaSpace{}%
\AgdaNumber{20}\AgdaSpace{}%
\AgdaOperator{\AgdaDatatype{\AgdaUnderscore{}≡\AgdaUnderscore{}}}\<%
\\
%
\\[\AgdaEmptyExtraSkip]%
\>[0]\AgdaFunction{id}\AgdaSpace{}%
\AgdaSymbol{:}\AgdaSpace{}%
\AgdaGeneralizable{A}\AgdaSpace{}%
\AgdaSymbol{→}\AgdaSpace{}%
\AgdaGeneralizable{A}\<%
\\
\>[0]\AgdaFunction{id}\AgdaSpace{}%
\AgdaSymbol{=}\AgdaSpace{}%
\AgdaSymbol{λ}\AgdaSpace{}%
\AgdaBound{z}\AgdaSpace{}%
\AgdaSymbol{→}\AgdaSpace{}%
\AgdaBound{z}\<%
\\
\>[0]\<%
\end{code}

\begin{figure}[H]
\textbf{Definition}:
A type $A$ is contractible, if there is $a : A$, called the center of contraction, such that for all $x : A$, $\equalH {a}{x}$.
\caption{Rendered Latex} \label{fig:R1}
\begin{verbatim}
isContr ( A : Set ) : Set = ( a : A ) ( * ) ( ( x : A ) -> Id ( a ) ( x ) )
\end{verbatim}
\begin{code}%
\>[0]\AgdaFunction{isContr}\AgdaSpace{}%
\AgdaSymbol{:}\AgdaSpace{}%
\AgdaSymbol{(}\AgdaBound{A}\AgdaSpace{}%
\AgdaSymbol{:}\AgdaSpace{}%
\AgdaPrimitive{Set}\AgdaSymbol{)}\AgdaSpace{}%
\AgdaSymbol{→}\AgdaSpace{}%
\AgdaPrimitive{Set}\<%
\\
\>[0]\AgdaFunction{isContr}\AgdaSpace{}%
\AgdaBound{A}\AgdaSpace{}%
\AgdaSymbol{=}%
\>[13]\AgdaRecord{Σ}\AgdaSpace{}%
\AgdaBound{A}\AgdaSpace{}%
\AgdaSymbol{λ}\AgdaSpace{}%
\AgdaBound{a}\AgdaSpace{}%
\AgdaSymbol{→}\AgdaSpace{}%
\AgdaSymbol{(}\AgdaBound{x}\AgdaSpace{}%
\AgdaSymbol{:}\AgdaSpace{}%
\AgdaBound{A}\AgdaSymbol{)}\AgdaSpace{}%
\AgdaSymbol{→}\AgdaSpace{}%
\AgdaSymbol{(}\AgdaBound{a}\AgdaSpace{}%
\AgdaOperator{\AgdaDatatype{≡}}\AgdaSpace{}%
\AgdaBound{x}\AgdaSymbol{)}\<%
\end{code}
\caption{Contractibility} \label{fig:R2}
\end{figure}

In \autoref{fig:R2}, we show the different syntax presentations of the
\emph{equivalence}, which is merely a bijection when restricted to sets. This is
of such fundamental idea in mathematics and HoTT in particular that it merits
its own chapter in [cite hott], but we only showcase one of its many equivalent
definitions. We see that the pidgin syntax is stuck with the anaphoric artifact
from the bloated abstract syntax, \codeword{fiber} has the type \codeword{it :
Set} instead of something like \codeword{(y : B) : Set}, and the y variable is
unbound in the \codeword{fiber} expression. This may presumably be fixed with a
few hours more of tinkering, but becomes even more complicated when not just
defining new types, but actually writing real mathematical proofs.

\begin{figure}[H]
\textbf{Definition}:
A map $f : \arrowH {A}{B}$ is an equivalence, if for all $y : B$, its fiber, $\comprehensionH {x}{A}{\equalH {\appH {f}{x}}{y}}$, is contractible.
We write $\equivalenceH {A}{B}$, if there is an equivalence $\arrowH {A}{B}$.
\begin{verbatim}
Equivalence ( f : A -> B ) : Set =
  ( y : B ) -> ( isContr ( fiber it ) ) ; ; ;
  fiber it : Set = ( x : A ) ( * ) ( Id ( f ( x ) ) ( y ) )
\end{verbatim}
\begin{code}%
\>[0]\AgdaFunction{Equivalence}\AgdaSpace{}%
\AgdaSymbol{:}\AgdaSpace{}%
\AgdaSymbol{(}\AgdaBound{A}\AgdaSpace{}%
\AgdaBound{B}\AgdaSpace{}%
\AgdaSymbol{:}\AgdaSpace{}%
\AgdaPrimitive{Set}\AgdaSymbol{)}\AgdaSpace{}%
\AgdaSymbol{→}\AgdaSpace{}%
\AgdaSymbol{(}\AgdaBound{f}\AgdaSpace{}%
\AgdaSymbol{:}\AgdaSpace{}%
\AgdaBound{A}\AgdaSpace{}%
\AgdaSymbol{→}\AgdaSpace{}%
\AgdaBound{B}\AgdaSymbol{)}\AgdaSpace{}%
\AgdaSymbol{→}\AgdaSpace{}%
\AgdaPrimitive{Set}\<%
\\
\>[0]\AgdaFunction{Equivalence}\AgdaSpace{}%
\AgdaBound{A}\AgdaSpace{}%
\AgdaBound{B}\AgdaSpace{}%
\AgdaBound{f}\AgdaSpace{}%
\AgdaSymbol{=}\AgdaSpace{}%
\AgdaSymbol{∀}\AgdaSpace{}%
\AgdaSymbol{(}\AgdaBound{y}\AgdaSpace{}%
\AgdaSymbol{:}\AgdaSpace{}%
\AgdaBound{B}\AgdaSymbol{)}\AgdaSpace{}%
\AgdaSymbol{→}\AgdaSpace{}%
\AgdaFunction{isContr}\AgdaSpace{}%
\AgdaSymbol{(}\AgdaFunction{fiber'}\AgdaSpace{}%
\AgdaBound{y}\AgdaSymbol{)}\<%
\\
\>[0][@{}l@{\AgdaIndent{0}}]%
\>[2]\AgdaKeyword{where}\<%
\\
\>[2][@{}l@{\AgdaIndent{0}}]%
\>[4]\AgdaFunction{fiber'}\AgdaSpace{}%
\AgdaSymbol{:}\AgdaSpace{}%
\AgdaSymbol{(}\AgdaBound{y}\AgdaSpace{}%
\AgdaSymbol{:}\AgdaSpace{}%
\AgdaBound{B}\AgdaSymbol{)}\AgdaSpace{}%
\AgdaSymbol{→}\AgdaSpace{}%
\AgdaPrimitive{Set}\<%
\\
%
\>[4]\AgdaFunction{fiber'}\AgdaSpace{}%
\AgdaBound{y}\AgdaSpace{}%
\AgdaSymbol{=}\AgdaSpace{}%
\AgdaRecord{Σ}\AgdaSpace{}%
\AgdaBound{A}\AgdaSpace{}%
\AgdaSymbol{(λ}\AgdaSpace{}%
\AgdaBound{x}\AgdaSpace{}%
\AgdaSymbol{→}\AgdaSpace{}%
\AgdaBound{y}\AgdaSpace{}%
\AgdaOperator{\AgdaDatatype{≡}}\AgdaSpace{}%
\AgdaBound{f}\AgdaSpace{}%
\AgdaBound{x}\AgdaSymbol{)}\<%
\end{code}
\caption{Contractibility} \label{fig:R3}
\end{figure}

% % \begin{code}[hide]%
\>[0]\<%
\\
\>[0]\AgdaComment{-- \{-\# OPTIONS --omega-in-omega --type-in-type \#-\}}\<%
\\
%
\\[\AgdaEmptyExtraSkip]%
\>[0]\AgdaKeyword{module}\AgdaSpace{}%
\AgdaModule{ex}\AgdaSpace{}%
\AgdaKeyword{where}\<%
\\
%
\\[\AgdaEmptyExtraSkip]%
\>[0]\AgdaKeyword{data}\AgdaSpace{}%
\AgdaDatatype{aℕ}\AgdaSpace{}%
\AgdaSymbol{:}\AgdaSpace{}%
\AgdaPrimitive{Set}\AgdaSpace{}%
\AgdaKeyword{where}\<%
\\
\>[0][@{}l@{\AgdaIndent{0}}]%
\>[2]\AgdaInductiveConstructor{zero'}\AgdaSpace{}%
\AgdaSymbol{:}\AgdaSpace{}%
\AgdaDatatype{aℕ}\<%
\\
%
\\[\AgdaEmptyExtraSkip]%
\>[0]\AgdaKeyword{variable}\<%
\\
\>[0][@{}l@{\AgdaIndent{0}}]%
\>[2]\AgdaGeneralizable{A}\AgdaSpace{}%
\AgdaSymbol{:}\AgdaSpace{}%
\AgdaPrimitive{Set}\<%
\\
%
\>[2]\AgdaGeneralizable{D}\AgdaSpace{}%
\AgdaSymbol{:}\AgdaSpace{}%
\AgdaPrimitive{Set}\<%
\\
%
\>[2]\AgdaGeneralizable{stuff}\AgdaSpace{}%
\AgdaSymbol{:}\AgdaSpace{}%
\AgdaPrimitive{Set}\<%
\\
%
\\[\AgdaEmptyExtraSkip]%
\>[0]\AgdaFunction{definition-body}\AgdaSpace{}%
\AgdaSymbol{=}\AgdaSpace{}%
\AgdaDatatype{aℕ}\<%
\\
%
\\[\AgdaEmptyExtraSkip]%
\>[0]\AgdaFunction{T}\AgdaSpace{}%
\AgdaSymbol{=}\AgdaSpace{}%
\AgdaDatatype{aℕ}\AgdaSpace{}%
\AgdaSymbol{→}\AgdaSpace{}%
\AgdaDatatype{aℕ}\<%
\\
\>[0]\AgdaFunction{L}\AgdaSpace{}%
\AgdaSymbol{=}\AgdaSpace{}%
\AgdaDatatype{aℕ}\<%
\\
\>[0]\AgdaFunction{E}\AgdaSpace{}%
\AgdaSymbol{=}\AgdaSpace{}%
\AgdaDatatype{aℕ}\<%
\\
\>[0]\AgdaFunction{C}\AgdaSpace{}%
\AgdaSymbol{=}\AgdaSpace{}%
\AgdaDatatype{aℕ}\<%
\\
%
\\[\AgdaEmptyExtraSkip]%
\>[0]\AgdaFunction{proof}\AgdaSpace{}%
\AgdaSymbol{:}\AgdaSpace{}%
\AgdaFunction{L}\<%
\\
\>[0]\AgdaFunction{proof}\AgdaSpace{}%
\AgdaSymbol{=}\AgdaSpace{}%
\AgdaInductiveConstructor{zero'}\<%
\\
%
\\[\AgdaEmptyExtraSkip]%
\>[0]\AgdaFunction{corollaryStuff}\AgdaSpace{}%
\AgdaSymbol{=}\AgdaSpace{}%
\AgdaDatatype{aℕ}\<%
\\
%
\\[\AgdaEmptyExtraSkip]%
\>[0]\AgdaFunction{proofNeedingLemma}\AgdaSpace{}%
\AgdaSymbol{:}\AgdaSpace{}%
\AgdaDatatype{aℕ}\AgdaSpace{}%
\AgdaSymbol{→}\AgdaSpace{}%
\AgdaDatatype{aℕ}\AgdaSpace{}%
\AgdaSymbol{→}\AgdaSpace{}%
\AgdaDatatype{aℕ}\<%
\\
\>[0]\AgdaFunction{proofNeedingLemma}\AgdaSpace{}%
\AgdaBound{x}\AgdaSpace{}%
\AgdaSymbol{=}\AgdaSpace{}%
\AgdaSymbol{λ}\AgdaSpace{}%
\AgdaBound{x₁}\AgdaSpace{}%
\AgdaSymbol{→}\AgdaSpace{}%
\AgdaInductiveConstructor{zero'}\<%
\\
\>[0]\<%
\end{code}

\subsubsection{Agda Programming}

Listed is the syntax Agda uses for judgements: \term{T} : \term{Set} means
\term{T} is a type, \term{t} : \term{T} means a term \term{t} has type \term{T},
and \term{t} = \term{t'} means \term{t} is defined to be judgmentally equal to
\term{t'}. Once one has made this equality judgement, Agda can normalize the
definitionally equal terms to the same normal form. Let's compare these Agda
judgements to those keywords ubiquitous in mathematics:

\begin{figure}
\centering
\begin{minipage}[t]{.3\textwidth}
\vspace{2cm}
\begin{itemize}
\item Axiom
\item Definition
\item Lemma
\item Theorem
\item Proof
\item Corollary
\item Example
\end{itemize}
\end{minipage}%
\begin{minipage}[t]{.55\textwidth}
\begin{code}%
\>[0]\AgdaKeyword{postulate}%
\>[12]\AgdaComment{-- Axiom}\<%
\\
\>[0][@{}l@{\AgdaIndent{0}}]%
\>[2]\AgdaPostulate{axiom}\AgdaSpace{}%
\AgdaSymbol{:}\AgdaSpace{}%
\AgdaGeneralizable{A}\<%
\\
%
\\[\AgdaEmptyExtraSkip]%
\>[0]\AgdaFunction{definition}\AgdaSpace{}%
\AgdaSymbol{:}\AgdaSpace{}%
\AgdaGeneralizable{stuff}\AgdaSpace{}%
\AgdaSymbol{→}\AgdaSpace{}%
\AgdaPrimitive{Set}\AgdaSpace{}%
\AgdaComment{--Definition}\<%
\\
\>[0]\AgdaFunction{definition}\AgdaSpace{}%
\AgdaBound{s}\AgdaSpace{}%
\AgdaSymbol{=}\AgdaSpace{}%
\AgdaFunction{definition-body}\<%
\\
%
\\[\AgdaEmptyExtraSkip]%
\>[0]\AgdaFunction{theorem}\AgdaSpace{}%
\AgdaSymbol{:}\AgdaSpace{}%
\AgdaFunction{T}%
\>[16]\AgdaComment{-- Theorem Statement}\<%
\\
\>[0]\AgdaFunction{theorem}\AgdaSpace{}%
\AgdaSymbol{=}\AgdaSpace{}%
\AgdaFunction{proofNeedingLemma}\AgdaSpace{}%
\AgdaFunction{lemma}\AgdaSpace{}%
\AgdaComment{-- Proof}\<%
\\
\>[0][@{}l@{\AgdaIndent{0}}]%
\>[2]\AgdaKeyword{where}\<%
\\
\>[2][@{}l@{\AgdaIndent{0}}]%
\>[4]\AgdaFunction{lemma}\AgdaSpace{}%
\AgdaSymbol{:}\AgdaSpace{}%
\AgdaFunction{L}%
\>[18]\AgdaComment{-- Lemma Statement}\<%
\\
%
\>[4]\AgdaFunction{lemma}\AgdaSpace{}%
\AgdaSymbol{=}\AgdaSpace{}%
\AgdaFunction{proof}\<%
\\
%
\\[\AgdaEmptyExtraSkip]%
\>[0]\AgdaFunction{corollary}\AgdaSpace{}%
\AgdaSymbol{:}\AgdaSpace{}%
\AgdaFunction{corollaryStuff}\AgdaSpace{}%
\AgdaSymbol{→}\AgdaSpace{}%
\AgdaFunction{C}\<%
\\
\>[0]\AgdaFunction{corollary}\AgdaSpace{}%
\AgdaBound{coro-term}\AgdaSpace{}%
\AgdaSymbol{=}\AgdaSpace{}%
\AgdaFunction{theorem}\AgdaSpace{}%
\AgdaBound{coro-term}\<%
\\
%
\\[\AgdaEmptyExtraSkip]%
\>[0]\AgdaFunction{example}\AgdaSpace{}%
\AgdaSymbol{:}\AgdaSpace{}%
\AgdaFunction{E}%
\>[16]\AgdaComment{-- Example Statement}\<%
\\
\>[0]\AgdaFunction{example}\AgdaSpace{}%
\AgdaSymbol{=}\AgdaSpace{}%
\AgdaFunction{proof}\<%
\end{code}
\end{minipage}
\caption{Mathematical Assertions and Agda Judgements} \label{fig:O1}
\end{figure}

Formation rules are given by the first line of the data declaration, followed
by some number of constructors which correspond to the introduction forms of the
type being defined. Therefore, to define a type for Booleans, $𝔹$, we present
these rules both in the proof theoretic and Agda syntax. We note that the
context $\Gamma$ is not present in Agda.

\begin{minipage}[t]{.4\textwidth}
\vspace{3mm}
\[
  \begin{prooftree}
    \infer1[]{ \vdash 𝔹 : {\rm type}}
  \end{prooftree}
\]
\[
  \begin{prooftree}
    \infer1[]{ \Gamma \vdash true : 𝔹  }
  \end{prooftree}
  \quad \quad
  \begin{prooftree}
    \infer1[]{ \Gamma \vdash false : 𝔹  }
  \end{prooftree}
\]
\end{minipage}
\begin{minipage}[t]{.3\textwidth}
\begin{code}%
\>[0]\AgdaKeyword{data}\AgdaSpace{}%
\AgdaDatatype{𝔹}\AgdaSpace{}%
\AgdaSymbol{:}\AgdaSpace{}%
\AgdaPrimitive{Set}\AgdaSpace{}%
\AgdaKeyword{where}\AgdaSpace{}%
\AgdaComment{-- formation rule}\<%
\\
\>[0][@{}l@{\AgdaIndent{0}}]%
\>[2]\AgdaInductiveConstructor{true}%
\>[8]\AgdaSymbol{:}\AgdaSpace{}%
\AgdaDatatype{𝔹}\AgdaSpace{}%
\AgdaComment{-- introduction rule}\<%
\\
%
\>[2]\AgdaInductiveConstructor{false}\AgdaSpace{}%
\AgdaSymbol{:}\AgdaSpace{}%
\AgdaDatatype{𝔹}\<%
\end{code}
\end{minipage}

The elimination forms are deriveable from the introduction rules, and the
computation rules can then be extracted by via the harmonious relationship
between the introduction and elmination forms \cite{pfenningHar}. Agda's pattern
matching is equivalent to the deriveable dependently typed elimination forms
\cite{coqPat}, and one can simply pattern match on a boolean, producing multiple
lines for each constructor of the variable's type, to extract the classic
recursion principle for Booleans. The \term{if then else} statement shown below
is really just the boolean elimination form. It is not standard to include the
premises of the eqaulity rules.

\begin{minipage}[t]{.4\textwidth}
\[
  \begin{prooftree}
    \hypo{̌\Gamma \vdash A : {\rm type} }
    \hypo{\Gamma \vdash b : 𝔹 }
    \hypo{\Gamma \vdash a1 : A}
    \hypo{\Gamma \vdash a2 : A }
    \infer4[]{\Gamma \vdash boolrec\{a1;a2\}(b) : A }
  \end{prooftree}
\]
$$\Gamma \vdash boolrec\{a1;a2\}(true) \equiv a1 : A$$
$$\Gamma \vdash boolrec\{a1;a2\}(false) \equiv a2 : A$$
\end{minipage}
\hfill
\begin{minipage}[t]{.5\textwidth}
\begin{code}%
\>[0]\AgdaOperator{\AgdaFunction{if\AgdaUnderscore{}then\AgdaUnderscore{}else\AgdaUnderscore{}}}\AgdaSpace{}%
\AgdaSymbol{:}\<%
\\
\>[0][@{}l@{\AgdaIndent{0}}]%
\>[2]\AgdaSymbol{\{}\AgdaBound{A}\AgdaSpace{}%
\AgdaSymbol{:}\AgdaSpace{}%
\AgdaPrimitive{Set}\AgdaSymbol{\}}\AgdaSpace{}%
\AgdaSymbol{→}\AgdaSpace{}%
\AgdaDatatype{𝔹}\AgdaSpace{}%
\AgdaSymbol{→}\AgdaSpace{}%
\AgdaBound{A}\AgdaSpace{}%
\AgdaSymbol{→}\AgdaSpace{}%
\AgdaBound{A}\AgdaSpace{}%
\AgdaSymbol{→}\AgdaSpace{}%
\AgdaBound{A}\<%
\\
\>[0]\AgdaOperator{\AgdaFunction{if}}\AgdaSpace{}%
\AgdaInductiveConstructor{true}\AgdaSpace{}%
\AgdaOperator{\AgdaFunction{then}}\AgdaSpace{}%
\AgdaBound{a1}\AgdaSpace{}%
\AgdaOperator{\AgdaFunction{else}}\AgdaSpace{}%
\AgdaBound{a2}\AgdaSpace{}%
\AgdaSymbol{=}\AgdaSpace{}%
\AgdaBound{a1}\<%
\\
\>[0]\AgdaOperator{\AgdaFunction{if}}\AgdaSpace{}%
\AgdaInductiveConstructor{false}\AgdaSpace{}%
\AgdaOperator{\AgdaFunction{then}}\AgdaSpace{}%
\AgdaBound{a1}\AgdaSpace{}%
\AgdaOperator{\AgdaFunction{else}}\AgdaSpace{}%
\AgdaBound{a2}\AgdaSpace{}%
\AgdaSymbol{=}\AgdaSpace{}%
\AgdaBound{a2}\<%
\end{code}
\end{minipage}

When using Agda one is interactively building a proof via holes. There is an
Agda Emacs mode which enables this. Glossing over many details, we show sample
code in the proof development state prior to pattern matching on \codeword{b}.
We have a hole, \codeword{{ b }0}, and the proof state is displayed to the
right. It shows both the current context with \codeword{A, b, a1, a2}, the goal
which is something of type \codeword{A}, and what we have, \codeword{B},
represents the type of the variable in the hole.

\hfill
\begin{minipage}[t]{.4\textwidth}
\begin{verbatim}
if_then_else_ :
  {A : Set} → B → A → A → A
if b then a1 else a2 = { b }0
\end{verbatim}
\end{minipage}
\hfill
\begin{minipage}[t]{.5\textwidth}
\begin{verbatim}
Goal: A
Have: B
———————————————
a2 : A
a1 : A
b  : B
A  : Set   (not in scope)
\end{verbatim}
\end{minipage}

The interactivity is performed via emacs commands, and every time one updates
the hole with a new term, we can immediately view the next goal with an updated
context. The underscore in \term{if_then_else_} denotes the placement of the
arguements, as Agda allows mixfix operations. Agda allows for more nuanced
syntacic features like unicode. This is interesting from the \emph{concrete
syntax} perspective as the arguement placement and symbolic expressiveness makes
Agda's syntax feel more familiar to the mathematician. We also observe the use
of parametric polymorphism, namely, that we can extract a member of some
arbtitrary type \term{A} from a boolean value given two members of \term{A}.

This polymorphism allows one to implement simple programs like boolean negation,
\term{~}, and more interestingly, \term{functionalNegation}, where one can use
functions as arguements. \term{functionalNegation} is a functional, or higher
order functions, which take functions as arguements and return functions. We
also notice in \term{functionalNegation} that one can work directly with a
built-in $\lambda$ to ensure the correct return type.

\begin{code}%
\>[0]\AgdaFunction{\textasciitilde{}}\AgdaSpace{}%
\AgdaSymbol{:}\AgdaSpace{}%
\AgdaDatatype{𝔹}\AgdaSpace{}%
\AgdaSymbol{→}\AgdaSpace{}%
\AgdaDatatype{𝔹}\<%
\\
\>[0]\AgdaFunction{\textasciitilde{}}\AgdaSpace{}%
\AgdaBound{b}\AgdaSpace{}%
\AgdaSymbol{=}\AgdaSpace{}%
\AgdaOperator{\AgdaFunction{if}}\AgdaSpace{}%
\AgdaBound{b}\AgdaSpace{}%
\AgdaOperator{\AgdaFunction{then}}\AgdaSpace{}%
\AgdaInductiveConstructor{false}\AgdaSpace{}%
\AgdaOperator{\AgdaFunction{else}}\AgdaSpace{}%
\AgdaInductiveConstructor{true}\<%
\\
%
\\[\AgdaEmptyExtraSkip]%
\>[0]\AgdaFunction{functionalNegation}\AgdaSpace{}%
\AgdaSymbol{:}\AgdaSpace{}%
\AgdaDatatype{𝔹}\AgdaSpace{}%
\AgdaSymbol{→}\AgdaSpace{}%
\AgdaSymbol{(}\AgdaDatatype{𝔹}\AgdaSpace{}%
\AgdaSymbol{→}\AgdaSpace{}%
\AgdaDatatype{𝔹}\AgdaSymbol{)}\AgdaSpace{}%
\AgdaSymbol{→}\AgdaSpace{}%
\AgdaSymbol{(}\AgdaDatatype{𝔹}\AgdaSpace{}%
\AgdaSymbol{→}\AgdaSpace{}%
\AgdaDatatype{𝔹}\AgdaSymbol{)}\<%
\\
\>[0]\AgdaFunction{functionalNegation}\AgdaSpace{}%
\AgdaBound{b}\AgdaSpace{}%
\AgdaBound{f}\AgdaSpace{}%
\AgdaSymbol{=}\AgdaSpace{}%
\AgdaOperator{\AgdaFunction{if}}\AgdaSpace{}%
\AgdaBound{b}\AgdaSpace{}%
\AgdaOperator{\AgdaFunction{then}}\AgdaSpace{}%
\AgdaBound{f}\AgdaSpace{}%
\AgdaOperator{\AgdaFunction{else}}\AgdaSpace{}%
\AgdaSymbol{λ}\AgdaSpace{}%
\AgdaBound{b'}\AgdaSpace{}%
\AgdaSymbol{→}\AgdaSpace{}%
\AgdaBound{f}\AgdaSpace{}%
\AgdaSymbol{(}\AgdaFunction{\textasciitilde{}}\AgdaSpace{}%
\AgdaBound{b'}\AgdaSymbol{)}\<%
\end{code}

This simple example leads us to one of the domains our subsequent grammars will
describe, like arithmetic (see \ref{npf}). We show how to inductively define
natural numbers in Agda, with the formation and introduction rules included
beside for contrast.

\begin{minipage}[t]{.4\textwidth}
\vspace{3mm}
\[
  \begin{prooftree}
    \infer1[]{ \vdash ℕ : {\rm type}}
  \end{prooftree}
\]
\[
  \begin{prooftree}
    \infer1[]{ \Gamma \vdash 0 : ℕ  }
  \end{prooftree}
  \quad \quad
  \begin{prooftree}
    \hypo{\Gamma \vdash n : ℕ}
    \infer1[]{ \Gamma \vdash (suc\ n) : ℕ  }
  \end{prooftree}
\]
\end{minipage}
\begin{minipage}[t]{.3\textwidth}
\begin{code}%
\>[0]\AgdaKeyword{data}\AgdaSpace{}%
\AgdaDatatype{ℕ}\AgdaSpace{}%
\AgdaSymbol{:}\AgdaSpace{}%
\AgdaPrimitive{Set}\AgdaSpace{}%
\AgdaKeyword{where}\<%
\\
\>[0][@{}l@{\AgdaIndent{0}}]%
\>[2]\AgdaInductiveConstructor{zero}\AgdaSpace{}%
\AgdaSymbol{:}\AgdaSpace{}%
\AgdaDatatype{ℕ}\<%
\\
%
\>[2]\AgdaInductiveConstructor{suc}%
\>[7]\AgdaSymbol{:}\AgdaSpace{}%
\AgdaDatatype{ℕ}\AgdaSpace{}%
\AgdaSymbol{→}\AgdaSpace{}%
\AgdaDatatype{ℕ}\<%
\end{code}
\end{minipage}

This is a recursive type, whereby pattern matching over $ℕ$ allows one to use an
induction hypothesis over the subtree and gurantee termination when making
recurive calls on the function being defined. We can define a recursion
principle for $ℕ$, which gives one the power to build iterators.
Again, we include the elimination and equality rules for syntactic
juxtaposition.

\[
  \begin{prooftree}
    \hypo{̌\Gamma \vdash X : {\rm type} }
    \hypo{\Gamma \vdash n : ℕ }
    \hypo{\Gamma \vdash e₀ : X}
    \hypo{\Gamma, x : ℕ, y : X \vdash e₁ : X }
    \infer4[]{\Gamma \vdash natrec\{e\;x.y.e₁\}(n) : X }
  \end{prooftree}
\]
$$\Gamma \vdash natrec\{e₀;x.y.e₁\}(n) \equiv e₀ : X$$
$$\Gamma \vdash natrec\{e₀;x.y.e₁\}(suc\ n) \equiv e₁[x := n,y := natrec\{e₀;x.y.e₁\}(n)] : X$$
\begin{code}%
\>[0]\AgdaFunction{natrec}\AgdaSpace{}%
\AgdaSymbol{:}\AgdaSpace{}%
\AgdaSymbol{\{}\AgdaBound{X}\AgdaSpace{}%
\AgdaSymbol{:}\AgdaSpace{}%
\AgdaPrimitive{Set}\AgdaSymbol{\}}\AgdaSpace{}%
\AgdaSymbol{→}\AgdaSpace{}%
\AgdaDatatype{ℕ}\AgdaSpace{}%
\AgdaSymbol{→}\AgdaSpace{}%
\AgdaBound{X}\AgdaSpace{}%
\AgdaSymbol{→}\AgdaSpace{}%
\AgdaSymbol{(}\AgdaDatatype{ℕ}\AgdaSpace{}%
\AgdaSymbol{→}\AgdaSpace{}%
\AgdaBound{X}\AgdaSpace{}%
\AgdaSymbol{→}\AgdaSpace{}%
\AgdaBound{X}\AgdaSymbol{)}\AgdaSpace{}%
\AgdaSymbol{→}\AgdaSpace{}%
\AgdaBound{X}\<%
\\
\>[0]\AgdaFunction{natrec}\AgdaSpace{}%
\AgdaInductiveConstructor{zero}\AgdaSpace{}%
\AgdaBound{e₀}\AgdaSpace{}%
\AgdaBound{e₁}\AgdaSpace{}%
\AgdaSymbol{=}\AgdaSpace{}%
\AgdaBound{e₀}\<%
\\
\>[0]\AgdaFunction{natrec}\AgdaSpace{}%
\AgdaSymbol{(}\AgdaInductiveConstructor{suc}\AgdaSpace{}%
\AgdaBound{n}\AgdaSymbol{)}\AgdaSpace{}%
\AgdaBound{e₀}\AgdaSpace{}%
\AgdaBound{e₁}\AgdaSpace{}%
\AgdaSymbol{=}\AgdaSpace{}%
\AgdaBound{e₁}\AgdaSpace{}%
\AgdaBound{n}\AgdaSpace{}%
\AgdaSymbol{(}\AgdaFunction{natrec}\AgdaSpace{}%
\AgdaBound{n}\AgdaSpace{}%
\AgdaBound{e₀}\AgdaSpace{}%
\AgdaBound{e₁}\AgdaSymbol{)}\<%
\end{code}

Since we are in a dependently typed setting, however, we prove theorems as well
as write programs. Therefore, we can see this recursion principle as a special
case of the induction principle \term{natind}, which represents the by induction
for the natural numbers. One may notice that while the types are different, the
programs \term{natrec} and \term{natind} are actually the same, up to
α-equivalence. One can therefore, as a corollary, actually just include the type
infomation and Agda can infer the speciliazation for you, as seen in
\term{natrec'} below.

\[
  \begin{prooftree}
    \hypo{̌\Gamma, x : ℕ \vdash X : {\rm type} }
    \hypo{\Gamma \vdash n : ℕ }
    \hypo{\Gamma \vdash e₀ : X[x := 0] }
    \hypo{\Gamma, y : ℕ, z : X[x := y] \vdash e₁ : X[x := suc\ y]}
    \infer4[]{\Gamma \vdash natind\{e₀,\;x.y.e₁\}(n) : X[x := n]}
  \end{prooftree}
\]
$$\Gamma \vdash natind\{e₀;x.y.e₁\}(n) \equiv e₀ : X[x := 0]$$
$$\Gamma \vdash natind\{e₀;x.y.e₁\}(suc\ n) \equiv e₁[x := n,y := natind\{e₀;x.y.e₁\}(n)] : X[x := suc\ n]$$
\begin{code}%
\>[0]\AgdaFunction{natind}\AgdaSpace{}%
\AgdaSymbol{:}\AgdaSpace{}%
\AgdaSymbol{\{}\AgdaBound{X}\AgdaSpace{}%
\AgdaSymbol{:}\AgdaSpace{}%
\AgdaDatatype{ℕ}\AgdaSpace{}%
\AgdaSymbol{→}\AgdaSpace{}%
\AgdaPrimitive{Set}\AgdaSymbol{\}}\AgdaSpace{}%
\AgdaSymbol{→}\AgdaSpace{}%
\AgdaSymbol{(}\AgdaBound{n}\AgdaSpace{}%
\AgdaSymbol{:}\AgdaSpace{}%
\AgdaDatatype{ℕ}\AgdaSymbol{)}\AgdaSpace{}%
\AgdaSymbol{→}\AgdaSpace{}%
\AgdaBound{X}\AgdaSpace{}%
\AgdaInductiveConstructor{zero}\AgdaSpace{}%
\AgdaSymbol{→}\AgdaSpace{}%
\AgdaSymbol{((}\AgdaBound{n}\AgdaSpace{}%
\AgdaSymbol{:}\AgdaSpace{}%
\AgdaDatatype{ℕ}\AgdaSymbol{)}\AgdaSpace{}%
\AgdaSymbol{→}\AgdaSpace{}%
\AgdaBound{X}\AgdaSpace{}%
\AgdaBound{n}\AgdaSpace{}%
\AgdaSymbol{→}\AgdaSpace{}%
\AgdaBound{X}\AgdaSpace{}%
\AgdaSymbol{(}\AgdaInductiveConstructor{suc}\AgdaSpace{}%
\AgdaBound{n}\AgdaSymbol{))}\AgdaSpace{}%
\AgdaSymbol{→}\AgdaSpace{}%
\AgdaBound{X}\AgdaSpace{}%
\AgdaBound{n}\<%
\\
\>[0]\AgdaFunction{natind}\AgdaSpace{}%
\AgdaInductiveConstructor{zero}\AgdaSpace{}%
\AgdaBound{base}\AgdaSpace{}%
\AgdaBound{step}\AgdaSpace{}%
\AgdaSymbol{=}\AgdaSpace{}%
\AgdaBound{base}\<%
\\
\>[0]\AgdaFunction{natind}\AgdaSpace{}%
\AgdaSymbol{(}\AgdaInductiveConstructor{suc}\AgdaSpace{}%
\AgdaBound{n}\AgdaSymbol{)}\AgdaSpace{}%
\AgdaBound{base}\AgdaSpace{}%
\AgdaBound{step}\AgdaSpace{}%
\AgdaSymbol{=}\AgdaSpace{}%
\AgdaBound{step}\AgdaSpace{}%
\AgdaBound{n}\AgdaSpace{}%
\AgdaSymbol{(}\AgdaFunction{natind}\AgdaSpace{}%
\AgdaBound{n}\AgdaSpace{}%
\AgdaBound{base}\AgdaSpace{}%
\AgdaBound{step}\AgdaSymbol{)}\<%
\\
%
\\[\AgdaEmptyExtraSkip]%
\>[0]\AgdaFunction{natrec'}\AgdaSpace{}%
\AgdaSymbol{:}\AgdaSpace{}%
\AgdaSymbol{\{}\AgdaBound{X}\AgdaSpace{}%
\AgdaSymbol{:}\AgdaSpace{}%
\AgdaPrimitive{Set}\AgdaSymbol{\}}\AgdaSpace{}%
\AgdaSymbol{→}\AgdaSpace{}%
\AgdaDatatype{ℕ}\AgdaSpace{}%
\AgdaSymbol{→}\AgdaSpace{}%
\AgdaBound{X}\AgdaSpace{}%
\AgdaSymbol{→}\AgdaSpace{}%
\AgdaSymbol{(}\AgdaDatatype{ℕ}\AgdaSpace{}%
\AgdaSymbol{→}\AgdaSpace{}%
\AgdaBound{X}\AgdaSpace{}%
\AgdaSymbol{→}\AgdaSpace{}%
\AgdaBound{X}\AgdaSymbol{)}\AgdaSpace{}%
\AgdaSymbol{→}\AgdaSpace{}%
\AgdaBound{X}\<%
\\
\>[0]\AgdaFunction{natrec'}\AgdaSpace{}%
\AgdaSymbol{=}\AgdaSpace{}%
\AgdaFunction{natind}\<%
\end{code}
We will defer the details of using induction and recursion principles for later
when we actually give examples of pidgin proofs some of our grammars can
handle.

% % \begin{code}[hide]%
\>[0]\<%
\\
\>[0]\AgdaKeyword{module}\AgdaSpace{}%
\AgdaModule{primitives}\AgdaSpace{}%
\AgdaKeyword{where}\<%
\\
\>[0]\<%
\end{code}

Formation rules, are given by the data declaration, followed by some number of
constructors which correspond to the 


A proof the proof-theoretic this looks like the following


\begin{prooftree}
  \hypo{ \Gamma, A &\vdash B }
  \infer1[abs]{ \Gamma &\vdash A\to B }
  \hypo{ \Gamma \vdash A }
  \infer2[app]{ \Gamma \vdash B }
\end{prooftree}


\begin{code}%
\>[0]\<%
\\
\>[0]\AgdaKeyword{data}\AgdaSpace{}%
\AgdaDatatype{𝔹}\AgdaSpace{}%
\AgdaSymbol{:}\AgdaSpace{}%
\AgdaPrimitive{Set}\AgdaSpace{}%
\AgdaKeyword{where}\<%
\\
\>[0][@{}l@{\AgdaIndent{0}}]%
\>[2]\AgdaInductiveConstructor{true}\AgdaSpace{}%
\AgdaSymbol{:}\AgdaSpace{}%
\AgdaDatatype{𝔹}\<%
\\
%
\>[2]\AgdaInductiveConstructor{false}\AgdaSpace{}%
\AgdaSymbol{:}\AgdaSpace{}%
\AgdaDatatype{𝔹}\<%
\\
\>[0]\<%
\end{code}


-- $ \frac{\Gamma, x : A \vdash b : B} {\Gamma \vdash \lambda x. b : A \rightarrow
-- B} $

\begin{code}%
\>[0]\<%
\\
\>[0]\AgdaComment{-- if\AgdaUnderscore{}then\AgdaUnderscore{}else\AgdaUnderscore{} : \{A : Set\} → 𝔹 → A → A → A}\<%
\\
\>[0]\AgdaComment{-- if true then a1 else a2 = a1}\<%
\\
\>[0]\AgdaComment{-- if false then a1 else a2 = a2}\<%
\\
\>[0]\<%
\end{code}

-- data ℕ : Type where
--   zero : ℕ
--   suc  : ℕ → ℕ

-- data List (A : Type) : Type where
  

-- data Vector : 



-- \begin{code}%
\>[0]\<%
\\
\>[0]\AgdaComment{-- Type : Set₁}\<%
\\
\>[0]\AgdaComment{-- Type = Set}\<%
\\
%
\\[\AgdaEmptyExtraSkip]%
\>[0]\AgdaComment{-- \textbackslash{}end\{code\}}\<%


% \caption{Agda} \label{fig:M3}
% \end{figure}

Additionally, we give the Agda code in \autoref{fig:M3}, so-as to see what the
end result of such a program would be. The astute reader will also notice a
semantic in the pidgin rendering error relative to the Agda implementation.
\codeword{fiber} has the type \codeword{it : Set} instead of something like
\codeword{(y : B) : Set}, and the y variable is unbound in the \codeword{fiber}
expression. This demonstrates that to design a grammar prioritizing
\emph{semantic adequacy} and subsequently trying to incorporate \emph{syntactic
completeness} becomes a very difficult problem. Depending on the application of
the grammar, the emphasis on this axis is most assuredly a choice one should
consider up front.

While both these grammars have their strengths and weaknesses, one shall see
shortly that the approach in this thesis, taking an actual programming language
parser in Backus-Naur Form Converter (BNFC), GFifying it, and trying to use the
abstract syntax to model natural language, gives in some sense a dual challenge,
where the abstract syntax remains simple, but its linearizations become
must increase in complexity.


% below is prior text, probably discard

% We now discuss the various iterations of code which experimented with NL aspects

% We should again emphasize the role of, in particular, Rantas two grammars, one
% formalizing logic, and the other working with a case study of a real text\cite{aarneHott}



% We now discuss the GF implementation, capable of parsing both natural language
% and Agda syntax. The parser was appropriated from the cubicaltt BNFC parser,
% de-cubified and then gf-ified. The languages are tightly coupled, so the
% translation is actually quite simple. Some main differences are:

% \begin{itemize}[noitemsep]

% \item GF treats abstract and concrete syntax seperately. This allows GF to
% support many concrete syntax implementation of a given grammar

% \item Fixity is dealt with at the concrete syntax layer in GF.  This allows for
% more refined control of fixity, but also results in difficulties : during
% linearization their can be the insertion of extra parens.

% \item GF supports dependent hypes and higher order abstract syntax, which makes
% it suitable to typecheck at the parsing stage. It would very interesting to see
% if this is interoperable with the current version of this work in later
% iterations [Todo - add github link referncing work I've done in this direction]

% \item GF also is enhanced by a PGF back-end, allowing an embedding of grammars
% into, among other languages, Haskell.

% \end{itemize}

% While GF is targeted towards natural language translation, there's nothing
% stopping it from being used as a PL tool as well, like, for instance, the
% front-end of a compiler. The innovation of this thesis is to combine both uses,
% thereby allowing translation between Controlled Natural Languages and
% programming languages.

% Example expressions the grammar can parse are seen below, which have been
% verified by hand to be isomorphic to the corresponding cubicaltt BNFC trees:

% \begin{verbatim}

% data bool : Set where true | false 
% data nat : Set where zero | suc ( n : nat )  
% caseBool ( x : Set ) ( y z : x ) : bool -> Set = split false -> y || true -> z
% indBool ( x : bool -> Set ) ( y : x false ) ( z : x true ) : ( b : bool ) -> x b = split false -> y || true  -> z
% funExt  ( a : Set )   ( b : a -> Set )   ( f g :  ( x : a )  -> b x )   ( p :  ( x : a )  -> ( b x )   ( f x ) == ( g x )  )  : (  ( y : a )  -> b y )  f == g = undefined
% foo ( b : bool ) : bool = b

% \end{verbatim}

% [Todo] add use cases

