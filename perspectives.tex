\subsection{Philosophical Perspectives}

\begin{displayquote}

...when it comes to understanding the power of mathematical language to guide our
thought and help us reason well, formal mathematical languages like the ones
used by interactive proof assistants provide informative models of informal
mathematical language. The formal languages underlying foundational frameworks
such as set theory and type theory were designed to provide an account of the
correct rules of mathematical reasoning, and, as Gödel observed, they do a
remarkably good job. But correctness isn’t everything: we want our mathematical
languages to enable us to reason efficiently and effectively as well. To that
end, we need not just accounts as to what makes a mathematical argument correct,
but also accounts of the structural features of our theorizing that help us
manage mathematical complexity. \emph{Jeremy Avigad} \cite{avigad2015mathematics}

\end{displayquote}

\subsubsection{Linguistic and Programming Language Abstractions}

The key development of this thesis is to explore the formal and informal
distinction of presenting mathematics as understood by mathematicians and computer
scientists by means of rule-based, syntax oriented machine translation.

Computational linguistics, particularly in the tradition of type
theoretical semantics\cite{ranta1994type}, gives one a way of comparing natural
and programming languages. Type theoretical semantics is concerned with the
semantics of natural language in the logical tradition of Montague, who
synthesized work in the shadows of Chomsky \cite{Chomsky57} and Frege
\cite{frege79}. This work ended up inspiring the GF. Indeed, one such description
of GF is that it is a compiler tool applied to domain specific machine
translation. Comparing the ``compiler view" of PLs and the ``linguistic view"
of NLs, we may interpolate this comparison to other general phenomenon in the
respective domains.

We make this comparison via two abstraction ladders, visible in
\autoref{fig:N1}. Observe that the PL dimension on the left represents synthetic
processes, those which we design, make decisions about, and describe formally.
The NL abstractions on the right represent analytic observations, and are
subject to empirical observations .

This diagram only serves as an atlas for the different abstractions - it is
certainly subject to modifications depending on the mathematician, linguist, or
philosopher investigating such matters. The PL abstractions as serve as an
actual high altitude blueprint for the design of programming languages. While
this view may ignore interesting details and research, it is unlikely
to create angst in the computer science communities. The linguistic
abstractions, on the other hand, are at the intersection of many fascinating
debates. There is consensus among linguists which abstractions and models of
those abstractions are more practically useful, theoretically compelling, or
empirically testable.

\begin{figure}
\centering
\begin{tikzcd}
Strings \ar[d,"Lexical\ Analysis"] \ar[dd,bend right=+90.0, swap,"Front\ End"]
&[5m]
\\ Lexemes \ar[d,"Parsing"] &[5em]
\\ ASTs \ar[d,"Type\ Checker"] &[5m]
\\ Typed\ ASTs
  \ar[dd, bend left, "Code\ Generator"] 
  \ar[dd, bend right, swap, "Interpreter"] &[5m]
\\ ...
\\ Target\ Language
\end{tikzcd}
\hspace{1cm}
\begin{tikzcd}
  Phonemes \arrow[d, "Morhphophonological
  \\ Anaylsis" description]
  \\ Morphemes \arrow[d, "Parse"]
  \\ \{\ Syntactic\ Representation\ \} \arrow[d, "Montague"', bend right=49]
    \arrow[d, "Ranta", bend left=49] \arrow[d, "..." description]
  \\ {\{\ STLC,\ ...\ ,\ DTT\ \}} \arrow[d, "?" description]
  \\ {\{\ Nearal Encoding\ ,\ ...\ Internal\ Language\ \}} \arrow[d, "?" description]
  \\ Phonemes
\end{tikzcd}
\caption{PL (left) and NL (right) Abstraction Ladders} \label{fig:N1}
\end{figure}


There are also many relevant concerns not addressed in either abstraction chain.
We may consider intrinsic and extrensic abstractions that diverge from the
idealized picture. For PLs we can inquire about systems with multiple
programming languages, or what the intensional behaviors of programs are during
evaluation. Agda, for example, requires the evaluation of terms during
typechecking. It is implemented with 4.5 different stages between the syntax
written by the programmers and the ``fully reflected AST" \cite{andreasEmail}.
Agda's type-checker is so powerful that the design, implemenation, and use of
Agda revolves around it, (which, ironically, is already called during the
parsing phase). It is not anticipated that floating point computation, for
instance, would ever be considered when implementing new features of Agda, at
least not for the foreseeable future. Indeed, the ways Agda represents ASTs were
an obstacle encountered doing this work, because one must decide which stage one
should connect an Agda AST with GF's representation.

\begin{figure}
\centering
\begin{tikzcd}
Strings \ar[r,"Lexical\ Analysis"] \ar[rr,bend right,"GF\ Parser"'] &[10em] Lexemes
\ar[r,"Parsing"] &[10em] ASTs \ar[ll,bend right, "GF\ Linearization"] 
\end{tikzcd}
\caption{GF in a nutshell} \label{fig:N2}
\end{figure}

Let's zoom in a little and observe the so-called front-end part of the compiler.
Displayed in \autoref{fig:N2} is the ``kernel" of GF. What makes GF so
compelling is its ability to translate between inductively defined languages
that type theorists specify and relatively expressive fragments of natural
languages, via the composition of GF's parser and linearizer . The decision to
overlay the abstraction ladders at the syntactic and semantic level led to GF's
development.

For natural language, some intrinsic properties of interest could be viewed at
the neurological level, where one may contrast the internal language
(i-language) with the mechanism of externalization (generally speech) as
observed by Chomsky \cite{Chomsky1995}. Extrinsic to the linguistic abstractions
depicted, pragmatics is absent. There are stark differences between NLs and PLs
where our comparative abstractions break down. Classifying PLs as languages is
best read as an incomplete metaphor due to perceived similarities.

Nonetheless, the point of this thesis is to take a crack at that exact question
: how can one compare programming and natural languages, in the sense that a
natural language, when restricted to a small enough (and presumably
well-behaved) domain, behaves as a programming language. Simultaneously, we
probe the topic of Natural Language Generation (NLG). Given a logic or type
system with some embedded theory, say arithmetic over $\mathbb{N}$, we ask how
do we not just find a natural language representation which interprets our
expressions, but also does so in a way that a competent speaker understands?

The specific linguistic domain we focus on, that of mathematics, is a particular
sweet spot at the intersection of these natural and formal language spaces. It
should be noted that this problem, that of translating between \emph{formal} and
\emph{informal} mathematics as stated, is both vague and difficult. It is
practically difficult in the sense that it may be either infeasibly
complexity or undecidable. But it also poses questions of what it means to be a
correct specification of a theorem, which is not something we can capture in a
formal system.

Like all collective human endeavors, mathematics is a historical construction -
that is, its conventions, notations, understanding, methodologies, and means of
publication and distribution have been in a constant flux. There is no consensus
on what mathematics is, how it is to be done, and most relevant for this
treatise, how it is to be expressed on paper. Over time mathematics has been
filtered of natural language artifacts, culminating in some sense with a formal
proof, despite mathematicians not being intimately familiar this formality as it
is treated in type theory. This work emphasizes the need for a new foundational
mentality, whereby we try to bring natural language back into ``formal
mathematics" in a controlled way.

Mathematical constructions like numbers and shapes arose out of ad-hoc needs as
humans cultures grew and evolved over the millennia. Unfortunately, most of this
evolution remains undocumented. While mathematical intuitions precede
mathematical constructions (the observation of a spherical planet precedes the
human use of a ruler compass construction to generate a circle), we should
assume that mathematics arises naturally out of our linguistic capacity. This
may very well not be the case, but it is impossible to imagine humans developing
mathematical constructions elaborating anything particularly interesting without
linguistic faculties. Regardless of the empirical or philosophical dispute one
takes with this linguistic view of mathematical abilities, we seek to make a
first order approximation of our linguistic view for the sake of this work.

\subsubsection{Formalization and Informalization}

\emph{Formalization} is the process of taking a piece of natural language
mathematics, embedding it in into a theorem prover, constructing a model, and
working with types instead of sets. This often requires significant amounts of
work - Hales formalization of the Kepler Conjecture took an estimated 20 human
years of labor. We note some interesting artifacts about a piece of mathematics
being formalized:

\begin{itemize}
\item it may be formalized differently by two different people in many different ways
\item it may have to be modified, to include hidden lemmas, to correct an
  error, or other bureaucratic obstacles
\item it may not type-check, and only be presumed hypothetically to be a
  correct formalization with some ad-hoc evidence 
\end{itemize}

\emph{Informalization}, on the other hand is a process of taking a piece formal
syntax, and turning it into a natural language utterance, along with commentary
motivating and or relating it to other mathematics. It is a clarification of the
meaning of a piece of code, suppressing certain details and sometimes
redundantly reiterating other details. In figure \autoref{fig:N3} we offer a few
dimensions of comparison.

\begin{figure}
\centering
\begin{tabular}{|c|c|c|} \hline
  Category & Formal Proof & Informal Proof \\ \hline
  Audience & Agda (and Human) & Human \\ \hline
  Translation & Compiler & Human \\ \hline
  Objectivity & Objective & Subjective \\ \hline % not always true
  Historical & 20th Century & <= Euclid \\ \hline
  Orientation & Syntax & Semantics \\ \hline
  Inferability & Complete & Domain Expertise Necessary \\ \hline
  Verification & PL Designer & Human \\ \hline
  Ambiguity & Unambiguous & Ambiguous \\ \hline

\end{tabular}
\caption{Informal and Formal Proofs} \label{fig:N3}
\end{figure}

Mathematicians working in either direction know this is a respectable task,
often leading to new methods, abstractions, and research altogether. Just as any
machine translating a Virginia Woolf novel from English to Mandarin is
incomparable to a human translator's expertise, machines formalizing or
informalizing mathematics are destined to be futile relative to a mathematician
making such a translation. Despite this futility, it shouldn't deter one so 
inspired to try.

\subsubsection{Syntactic Completeness and Semantic Adequacy}

There are two fundamental criteria that must be assessed for one to judge the
success of an approach over both formalization and informalization. The first,
\emph{syntactic completeness}, means that a construction contains all the
syntactic information necessary to verify its correctness. It says that a term
type-checks in the PL case, or some natural language form can be
deterministically transformed to a term that does type-check. We may therefore
ask the following : given an utterance or natural language expression that a
mathematician might understand, does the GF grammar emit a well-formed syntactic
expression in the target logic or programming language?

This problem of creating a syntactically complete mathematical landscape is
certainly infeasible generally - a mathematician might not be able to
reconstruct the unstated syntactic details of a proof in an discipline outside
her expertise. Additionally, certain necessary syntactic details may also
detract from a natural language reading comprehension of a proof. Perhaps most
importantly, one does not know a priori that the generated expression in the
type theory has its \emph{intended meaning}. The saying ``grammars leak" can be
transposed to say ``natural language proofs leak" insofar as they are certain to
contain syntactically necessary omissions.

Conversely, given a well formed syntactic expression in, for instance, Agda, one
can ask if the resulting English expression generated by GF is
\emph{semantically adequate}. This general notion of semantic adequacy is
delicate, and certainly not formally definable. Mathematicians themselves may
dispute the proof of a given proposition or the correct definition of some
notion.

There are few working mathematicians who would not understand some standard
theorem in an arbitrary introductory analysis text, even if they may dispute
it's presentation, clarity, pedagogy, or take other issues. Whether one asks
that semantic adequacy means some kind of sociological consensus among those
with relevant expertise, or a more relaxed criterion that some expert herself
understands the argument, a dubious perspective in scientific circles, semantic
adequacy should appease at least one and potentially more mathematicians.

An example of a syntactically complete but semantically inadequate statement
which one of our grammars parses is ``is it the case that the sum of 3 and the
sum of 4 and 10 is prime and 9999 is odd". Alternatively, most mathematical
proofs in most mathematical papers are most likely syntactically incomplete -
anyone interested in formalizing a piece of mathematics from some
\emph{elementary} resource will learn this quickly.

\begin{figure}[H]
\centering
\begin{tikzcd}
Syntactically\ Complete \ar[r,"Informalization"] &[10em]
Semantically\ Adequate \ar[l,bend right, "Formalization"] 
\end{tikzcd}
\caption{Formal and Informal Mathematics} \label{fig:N4}
\end{figure}

We introduce these definitions, syntactic completeness and semantic adequacy, to
highlight perspectives and insights that seems to underlie the biggest
differences between informal and formal mathematics, as is show in
\autoref{fig:N4}. We claim that mathematics, as done via a theorem prover, is a
syntax oriented endeavor, whereas mathematics, as practiced by mathematicians,
prioritizes semantic understanding. Developing a system which is able to
formalize and informalize utterances which preserve syntactic completeness and
semantic adequacy is likely impossible. Even introducing objective criteria to
really judge these definitions in special will require a lot of work.

This perspective represents an observation, it is not intended to judge whether
the syntactic or semantic perspective in mathematics is better - there is a
perpetual dialogue unfolding, where dialectical adjustments are perpetually
taking place.

\paragraph{The Syntactic Nature of Agda}

The act of writing and reading an Agda proof are significantly different
endeavors, as the term shadows the application of typing rules which enable its
construction. When the Agda user builds her proof, she is outsourcing the
bookkeeping to the type-checker. This isn't purely a mechanical process, she
often does have to think how her definitions will interact with downstream
programs. Like all mathematicians, she must ask if assertions sensible to begin
with, and she must carefully craft a strategy of attack before hacking away.

For someone reading Agda code, if proofs were semantically coherent, where only
a few occasional comments about various intentions and conclusions
would be needed to understand the material. In reality, the human witness of a
large term may easily be confused why it fulfills the typing judgment.
The reader may have to reexamine parts of the proof by trying to rebuild it
interactively with Agda's help.

Yet, papers are often written exclusively in Latex, where Agda proofs have been
reverse engineered to appease the audience and only relevant semantic details
have been preserved. Even in cases where Agda code is included in a paper, it is
most often the types which are emphasized and produced. Complex proof terms are
seldom legible. The description and commentary is still largely necessary to
convey the \emph{important material}, regardless if the Agda code is
self-contained. And while literate Agda is a bridge, the commentary still
unfolds the code.

\paragraph{Coq & other ITPs}

In Coq programming language, proof terms are built using Ltac, a scripting
language for writing imperative syntactic meta-programs over the core language,
Gallina. The user rarely sees the internal proof tree that one becomes familiar
with in Agda. The tactics are not typed, often feel very adhoc, and tacticals,
sequences of tactics, may carry very little semantic value (or even possibly
muddy one's understanding when reading proofs with unknown tactics). Ltac often
descends into the sorrows of so-called untyped languages, although there are
recent attempts to change this \cite{mtac2} \cite{ltac2}. From one perspective,
the use of tactics is an additional syntactic obfuscation of what a proof should
look like from the mathematicians perspective - and remedies are a research
topic. Alecytron is one impressive development in giving Coq proofs more
readability through a interactive back-end which shows the proof state, and
offers other semantically appealing models like interactive graphics
\cite{coqAlec}. This has the advantage of making intermediate types visible,
something mathematicians do when writing their proofs (even though everything is
ultimately inferable by the type-checker).

From another perspective tactics sometimes enhance high level proof
understanding, as tactics like \emph{ring} or \emph{omega} often save the reader
overhead of parsing pedantic and uninformative details (what mathematicians will
often leave to the reader). For certain proofs, especially those involving many
hundreds of cases, the metaprogramming facilities actually give one exclusive
advantages not offered to the classical mathematician using pen and paper.

Other interactive proof assistants, like Lean, Isabelle, the HOL family, merit
attention we don't have the space to give them here. There have been surveys
comparing them \cite{10.5555/1121735}. It would be incredibly beneficial to
explicitly compare these ITPs, in terms of features they offer, how their syntax
and semantics differ, and how their syntax and semantics affects the ways proofs
are constructed and shown. Additionally, a larger corpus of proofs would
showcase these differences and give additional direction to the work we do here.

Mathematicians may indeed like some of the facilities theorem provers provide,
but ultimately, they may not see that as the "essence" of what they are doing.
What is this essence? We will try to shine a small light on perhaps the most
fundamental question in mathematics.

\subsubsection{What is a proof?}

\begin{displayquote}

A proof is what makes a judgment evident \cite{mlMeanings}.

\end{displayquote}

The proofs of Agda and any are \emph{formal proofs}. Formal proofs have no
holes, and while there may very well be bugs in the underlying technologies
supporting these proofs, formal proofs are seen as some kind of immutable form
of data. One could say they provide \emph{objective evidence} for judgments,
which themselves are \emph{objective} entities when encoded on a computer. What
we call formal proofs might better be considered proofs that may be communicable
to aliens. However, formal proofs certainly aren't the objects mathematicians
deal with daily.

Mathematics, and the act of proving theorems, according to Brouwer is a social
process. Suppose we have two humans, $h_1$ and $h_2$. If $h_1$ claims to have a
proof $p_1$, and elaborates it to $p_2$ who claims she can either verify $p_1$
or reproduce and re-articulate it via $p_1'$, such that $h_1$ and $h_2$ agree
that $p_1$ and $p_1'$ are equivalent, then they have discovered some
mathematics. In this guise mathematics may be viewed as a science,
because there is a notion of reproducibility of evidence.

\paragraph{The Architect and the Engineer}

An apt comparison is to see the mathematician is architect, whereas the computer
scientist responsible for formalizing the mathematics is an engineer. The
mathematics is the building which, like all human endeavors, is created via
resources and labor of many people. The role of the architect is to envision the
facade, the exterior layer directly perceived by others, giving a building its
character, purpose, and function. The engineer is on the other hand, tasked with
assuring the building gets built, doesn't collapse, and functions with many
implicit features which the user of the building may not notice : the running
water, insulation, and electricity. Whereas the architect is responsible for the
building's \emph{specification}, the engineer is tasked with its
\emph{implementation}.

Informal proofs are specifications and formal proofs are implementations. Two
different authors may informalize the same code differently - they may suppress
different details and choose to emphasize different details, leading to two
unique, but possibly similar proofs. Extrapolating our analogy, the same two
architects given the same engineering plans could produce two entirely different
looking and functioning buildings. It is the architect who has the vision, and
the engineers who end up implementing the architects art.

We also pose a different analogy, comparing the mathematician and the physicist.
The physicist will often say under her breath, ``don't tell anyone in the math
department I'm doing this" when swapping an integral and a sum or other loose
but effective tricks in her blackboard calculations. While there is an implicit
assumption that there are theorems in analysis which may justify these
calculations, it is not the physicist's objective to be as rigorous as the
mathematician. This is because the physicist is using the mathematics as a
syntactic mechanism to reflect the semantic domain of particles, energy, and
other physical processes which the mathematics in physics serves to describe.
In this case, the ``pen and paper" mathematician fills the roll of the
physicist, and the Agda user the mathematician. Formality when using mathematics
is a spectrum.

There isn't a natural notion of equivalence between informal and formal proofs,
but perhaps the categorical idea adjunction is more relevant. The fact that the
``acceptable" natural language utterances aren't inductively defined precludes
us from actually constructing a canonical mathematical model of formal/informal
relationship. However, if GF perspective of translation is used, there can at
least be an approximation of what a model may look like. The linguist interested
in the language of mathematics should perhaps be seen as a scientist, whose
purpose is to contribute basic ideas and insights about the natural world from
which the architects and engineers can use to inform their designs.

Mathematicians naturally seek model independence in their results (i.e., they
don't need a direct encoding of Fermat's last theorem in set theory in order to
trust its validity). The implementation of a result in Agda versus Coq may lead
to intentionally different objects which represent the same thing extensionally.
It's also noted a proof doesn't obey the same universality that it does when
it's on paper or verbalized - that reliance on Agda 2.6.2, and its current standard
library, when updated in the future, may ``break proofs".
We believe the GF approach offers at least a step in the direction of
foundational agnosticism which may appease some of these issues.

This thesis examines not just a practical problem, but touches many deep issues
in some space in the intersection of the foundations, of mathematics, logic,
computer science, and their relations studied via linguistic formalisms. These
subjects, and their various relations, are the subject of countless hours of
work and consideration by many great minds. We believe our work provides a
nontrivial perspective at many important issues in this canon of great thinkers.
We emphasize the following questions:

\begin{itemize}[noitemsep]
\item What are mathematical objects?
\item How do their encodings in different foundational formalisms affect their
  interpretations?
\item How does mathematics develop as a social process?
\item How does what mathematics is and how it is done rely on given technologies
  of a given historical era?
\end{itemize}

While various branches of linguistics have seen rapid evolution due to, in large
part, their adoption of mathematical tools, the dual application of linguistic
tools to mathematics is quite sparse and open terrain.

The view of what mathematics is in a philosophical and mathematical sense,
particularly with respect to foundations, requires deep consideration in its
relation to linguistics. And while this work is perhaps just a finer grain of
sandpaper on an incomplete and primordial marble sculpture, it is hoped that the
sculptor's own reflection is a little bit more clear after we polish it here.

\paragraph{A Reconsideration of Proof}

\begin{displayquote}
Though philosophical discussion of visual thinking in mathematics has
concentrated on its role in proof, visual thinking may be more valuable for
discovery than proof. \emph{Marcus Giaquinto} \cite{sep-epistemology-visual-thinking}
\end{displayquote}

We here touch upon additional non-syntactic phenomena - ``proofs without words"
\cite{proofWW} and other diagrammatic and visual reasoning tools in mathematics
and programming. The role of visualization in programming, logic, and
mathematics offers an abundance of contrast to syntactically oriented
alphanumeric alphabets, e.g. strings of symbols. Visualization are ubiquitous in
contemporary mathematics: plotting diagrams, knots, diagram chases in category
theory, and a myriad of other visual tools which both assist understanding and
inform our syntactic descriptions. We find these languages appealing because of
their focus on a different kind of internal semantic sensation. The diagrammatic
languages for monoidal categories, for example, also allow interpretations of
formal proofs via topological deformations, and they have given semantic
representations to various graphical languages like circuit diagrams and petri
nets \cite{fong2016algebra}.

Additionally, graphical programming languages which facilitating diagrammatic programming are
one instance of a nonlinear syntax. Globular, which allows one to carry out
higher categorical constructions via globular sets is an interesting case study
for a graphical programming language which is designed for theorem proving
\cite{Bar2016GlobularAO}. Alecytron supports basic data structure visualization,
like red-black trees which carry semantic content less easy in a string
based-setting \cite{coqAlec}. These languages prove tricky but possible to
implement grammars for in GF, because one would presumably have to map to an
internal AST.

There are also, famously, blind mathematicians who work in topology, geometry,
and analysis \cite{2002TheWO}. Bernard Morin, blinded at a young age, was a
topologist who discovered the first eversion of a sphere by using clay models
which were then diagrammatically transcribed by a colleague on the board. This
is a remarkable use of mathematical tools PL researchers cannot yet handle, and
warrants careful consideration of what the boundaries of proof assistants are
capable of in terms of giving mathematicians more tangible constructions.

This brief discussion of visualizations should blur the syntactic understanding
of mathematics which we emphasize in this work. We highlight the difference
between \emph{a proof} and the \emph{the understanding of a proof}. The
understanding of a proof, is not done by anything but a human. And this internal
understanding and processing of mathematical information, what I'll
tongue-and-cheek call \emph{i-mathematics}, with its externalization facilities
being our main concerns in this thesis, requires much more work by future
scholars.