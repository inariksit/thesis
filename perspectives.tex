\section{Perspectives}

\begin{displayquote}

...when it comes to understanding the power of mathematical language to guide our
thought and help us reason well, formal mathematical languages like the ones
used by interactive proof assistants provide informative models of informal
mathematical language. The formal languages underlying foundational frameworks
such as set theory and type theory were designed to provide an account of the
correct rules of mathematical reasoning, and, as Gödel observed, they do a
remarkably good job. But correctness isn’t everything: we want our mathematical
languages to enable us to reason efficiently and effectively as well. To that
end, we need not just accounts as to what makes a mathematical argument correct,
but also accounts of the structural features of our theorizing that help us
manage mathematical complexity.\cite{avigad2015mathematics}

\end{displayquote}

\subsection{Linguistic and Programming Language Abstractions}

The key development of this thesis it to explore the formal and informal
distinction of presenting mathematics as understood by mathematicians and computer
scientists by means of rule-based, syntax oriented machine translation.

Computational linguistics, particularly those in the tradition of type
theoretical semantics\cite{ranta1994type}, gives one a way of comparing natural
and programming languages. Type theoretical semantics it is concerned with the
semantics of natural language in the logical tradition of Montague, who
synthesized work in the shadows of Chomsky \cite{Chomsky57} and Frege
\cite{frege79}. This work ended up inspiring the GF system, a side effect of
which was to realize that machine translation was possible as a side effect of
this abstracted view of natural language semantics. Indeed, one such description
of GF is that it is a compiler tool applied to domain specific machine
translation. We may compare the ``compiler view" of PLs and the ``linguistics view"
of NLs, and interpolate this comparison to other general phenomenon in the
respective domains.

We will reference these programming language and linguistic abstraction ladders,
and after viewing \autoref{fig:M1}, the reader should examine this
comparison with her own knowledge and expertise in mind. These respective
ladders are perhaps the most important lens one should keep in mind while
reading this thesis. Importantly, we should observe that the PL dimension, the
left diagram, represents synthetic processes, those which we design, make
decisions about, and describe formally. Alternatively, the NL abstractions on
the right represent analytic observations. They are therefore are subject to
different, in some ways orthogonal, constraints.

The linguistic abstractions are subject to empirical observations and
constraints, and this diagram only serves as an atlas for the different
abstractions and relations between these abstractions, which may be subject to
modifications depending on the linguist or philosopher investigating such
matters. The PL abstractions as represented, while also an approximations,
serves as an actual high altitude blueprint for the design of programming
languages. While the devil is in the details and this view is greatly
simplified, the representation of PL design is unlikely to create angst in the
computer science communities. The linguistic abstractions are at the
intersection of many fascinating debates between linguists, and there is
certainly nothing close to any type of consensus among linguists which
linguistic abstractions, as well as their hierarchical arrangement, are more
practically useful, theoretically compelling, or empirically testable.


\begin{figure}
\centering
\begin{tikzcd}
Strings \ar[d,"Lexical\ Analysis"] \ar[dd,bend right=+90.0, swap,"Front\ End"]
&[5m]
\\ Lexemes \ar[d,"Parsing"] &[5em]
\\ ASTs \ar[d,"Type\ Checker"] &[5m]
\\ Typed\ ASTs
  \ar[dd, bend left, "Code\ Generator"] 
  \ar[dd, bend right, swap, "Interpreter"] &[5m]
\\ ...
\\ Target\ Language
\end{tikzcd}
\hspace{1cm}
\begin{tikzcd}
  Phonemes \arrow[d, "Morhphophonological
  \\ Anaylsis" description]
  \\ Morphemes \arrow[d, "Parse"]
  \\ \{\ Syntactic\ Representation\ \} \arrow[d, "Montague"', bend right=49]
    \arrow[d, "Ranta", bend left=49] \arrow[d, "..." description]
  \\ {\{\ STLC,\ ...\ ,\ DTT\ \}} \arrow[d, "?" description]
  \\ {\{\ Nearal Encoding\ ,\ ...\ Internal\ Language\ \}} \arrow[d, "?" description]
  \\ Phonemes
\end{tikzcd}
\caption{PL (left) and NL (right) Abstraction Ladders} \label{fig:M1}
\end{figure}


There are also many relevant concerns not addressed in either abstraction chain
that are necessary to give a more comprehsive snapshot. For instance, we may
consider intrinsic and extrensic abstractions that diverge from the idealized
picture. In PL extrensic domain, we can inquire about 

\begin{itemize}

\item systems with multiple interactive programming language 
\item how the programming languages behave with respect to given programs
\item embedding programming languages into one another

\end{itemize}

Alternatively, intrinsic to a given PL, there picture is also not so clear.
Agda, for example, requires the evaluation of terms during typechecking. It is
implemented with 4.5 different stages between the syntax written by the
programmers and the ``fully reflected Abstract Syntax Tree (AST)" \cite{andreasEmail}. But this
example is perhaps an outlier, because Agda's type-checker is so powerful that
the design, implemenation, and use of Agda revolves around it,
(which, ironically, is already called during the parsing phase). It is not
anticipated that floating point computation, for instance, would ever be
considered when implementing new features of Agda, at least not for the
foreseeable future. Indeed, the ways Agda represents ASTs were an obstacle
encountered doing this work, because deciding which parsing stage one should connect
to the Portable Grammar Format (PGF) embedding is nontrivial.

% \begin{displayquote}
% \begin{enumerate}
% \item unicode: before parsing
% \item Concrete: after happy parsing Parser/Parser.y that does a little desugaring already---[but ideally shouldn't].  Expressions are not parsed yet.
% \item Concrete.Definitions: the "nice" syntax after the nicifier, preparing for the scope checker
% \item Abstract: after scope checking, mostly desugared, expressions parsed
% \item Internal: after type checking, fully desugared
% \item Reflected: quoted from Internal
% \begin{end}
% \end{displayquote}


\begin{figure}
\centering
\begin{tikzcd}
Strings \ar[r,"Lexical\ Analysis"] \ar[rr,bend right,"GF\ Parser"'] &[10em] Lexemes
\ar[r,"Parsing"] &[10em] ASTs \ar[ll,bend right, "GF\ Linearization"] 
\end{tikzcd}
\caption{GF in a nutshell} \label{fig:M2}
\end{figure}

Let's zoom in a little and observe the so-called front-end part of the compiler.
Displayed in \autoref{fig:M2} is the highest possible overview of GF. This is a
deceptively simple depiction of such a powerful and intricate system. What makes
GF so compelling is its ability to translate between inductively defined
languages that type theorists specify and relatively expressive fragments of
natural languages, via the composition of GF's parsing and linearization
capabilities. It is in some sense the attempt to overlay the abstraction ladders
at the syntactic level and semantic led to this development.

For natural language, some intrinsic properties might take place, if one
chooses, at the neurological level, where one somehow can contrast the internal
language (i-language) with the mechanism of externalization (generally speech) as proposed by
Chomsky \cite{Chomsky1995}. Extrinsic to the linguistic abstractions depicted, pragmatics is
absent.
.
The point is to recognize their are stark differences between natural languages
and programming languages which are even more apparent when one gets to certain
abstractions. Classifying both programming languages as
languages is best read as an incomplete (and even sometimes contradictory)
metaphor, due to perceived similarities (of which their are ample).

Nonetheless, the point of this thesis is to take a crack at that exact question
: how can one compare programming and natural languages, in the sense that a
natural language, when restricted to a small enough (and presumably
well-behaved) domain, behaves as a programming language. Simultaneously, we
probe the topic of Natural Language Generation (NLG). Given a
logic or type system with some theory inside (say arithmetic over the naturals),
how do we not just find a natural language representation which interprets our
expressions, but also does so in a way that is linguistically coherent in a
sense that a competent speaker can make sense of it in a facile way.

The specific linguistic domain we focus on, that of mathematics, is a particular
sweet spot at the intersection of these natural and formal language spaces. It
should be noted that this problem, that of translating between \emph{formal} and
\emph{informal} mathematics as stated, is both vague and difficult. It is
difficult in both the practical sense, that it may be either of infeasible
complexity or even perhaps undecidable, but it is also difficult in the
philosophical sense. One may entertain the prospect of syntactically translated
mathematics may a priori may deflate its effectiveness or meaningfulness. Like
all collective human endeavors, mathematics is a historical construction - that
is, its conventions, notations, understanding, methodologies, and means of
publication and distribution have all been in a constant flux. There is no
consensus on what mathematics is, how it is to be done, and most relevant for
this treatise, how it is to be expressed.

Historically, mathematics has been filtered of natural language artifacts,
culminating in some sense with Frege's development of a formal proof. A
mathematician often never sees a formal proof as it is treated in Logic and Type
Theory. We hope this work helps with a new foundational mentality, whereby we
try to bring natural language back into mathematics in a controlled way, or at
least to bridge the gap between our technologies, specifically injecting ITPs
into a mathematicians toolbox.


We present a sketch of the difference of this so-called formal/informal
distinction. Mathematics, that is mathematical constructions like numbers and
geometrical figures, arose out of ad-hoc needs as humans cultures grew and
evolved over the millennia. Indeed, just like many of the most interesting human
developments of which there is a sparsely documented record until relatively
recently, it is likely to remain a mystery what the long historical arc of
mathematics could have looked like in the context of human evolution. And while
mathematical intuitions precede mathematical constructions (the spherical planet
precedes the human use of a ruler compass construction to generate a circle), we
should take it as a starting point that mathematics arises naturally out of our
linguistic capacity. This may very well not be the case, or at least not
universally so, but it is impossible to imagine humans developing mathematical
constructions elaborating anything particularly general without linguistic
faculties. Despite whatever empirical or philosophical dispute one takes with
this linguistic view of mathematical abilities, we seek to make a first order
approximation of our linguistic view for the sake of this work. The discussion around
mathematics relation to linguistics generally, regardless of the stance
one takes, should benefit from this work.

\subsection{Formalization and Informalization}

Formalization is the process of taking an informal piece of natural language
mathematics, embedding it in into a theorem prover, constructing a model,
and working with types instead of sets. This often requires significant amounts of
work. We note some interesting artifacts about a piece of mathematics
being formalized:

\begin{itemize}

\item it may be formalized differently by two different people in many different ways
\item it may have to be modified, to include hidden lemmas, to correct of an
  error, or other bureaucratic obstacles
\item it may not type check, and only be presumed hypothetically to be 'a
  correct formalization' given evidence 

\end{itemize}

Informalization, on the other hand is a process of taking a piece formal syntax, and turning it into a natural
language utterance, along with commentary motivating and or relating it to other
mathematics. It is a clarification of the meaning of a piece of
code, suppressing certain details and sometimes
redundantly reiterating other details. In figure \autoref{fig:M5} we offer a few
dimensions of comparison.

\begin{figure}
\centering
\begin{tabular}{|c|c|c|} \hline
  Category & Formal Proof & Informal Proof \\ \hline
  Audience & Agda (and Human) & Human \\ \hline
  Translation & Compiler & Human \\ \hline
  Objectivity & Objective & Subjective \\ \hline % not always true
  Historical & 20th Century & <= Euclid \\ \hline
  Orientation & Syntax & Semantics \\ \hline
  Inferability & Complete & Domain Expertise Necessary \\ \hline
  Verification & PL Designer & Human \\ \hline
  Ambiguity & Unambiguous & Ambiguous \\ \hline

\end{tabular}
\caption{Informal and Formal Proofs} \label{fig:M5}
\end{figure}

Mathematicians working in either direction know this is a respectable task,
often leading to new methods, abstractions, and research altogether. And just as
any type of machine translation, rule-based or statistical, on Virginia Woolf
novel or Emily Dickinson poem from English to Mandarin would be 
absurd, so-to would the pretense that the methods we explore here using GF
could actually match the competence of mathematicians translating work between a
computer a book. Despite the futility of surpassing a mathematician at proof
translation, it shouldn't deter those so inspired to try.

\subsection{Syntactic Completeness and Semantic Adequacy}

The GF pipeline, that of bidirectional translation through an intermediary
abstract syntax representation, has two fundamental criteria that must be
assessed for one to judge the success of an approach over both formalization and
informalization.

The first criterion mentioned above, which we'll call \emph{syntactic
  completeness}, says that a term either type-checks, or some natural language
form can be deterministically transformed to a term that does type-check.

It asks the following : given an utterance or natural language expression that a
mathematician might understand, does the GF grammar emit a well-formed syntactic
expression in the target logic or programming language? The saying ``grammars
leak", can be transposed to say (NL) ``proofs leak" in that they are certain to
contain omissions.

This problem of syntactically complete mathematics is certain to be infeasible
in many cases - a mathematician might not be able to reconstruct the unstated
syntactic details of a proof in an discipline outside her expertise, it is at
worthy pursuit to ask why it is so difficult! Additionally, certain inferable
details may also detract from the natural language reading rather than assist.
Perhaps most importantly, one does not know a priori that the generated
expression in the logic has its intended meaning, other than through some meta
verification procedure.

Conversely, given a well formed syntactic expression in, for instance, Agda, one
can ask if the resulting English expression generated by GF is
\emph{semantically adequate}.

This notion of semantic adequacy is also delicate, as mathematicians themselves
may dispute, for instance, the proof of a given proposition or the correct
definition of some notion. However, if it is doubtful that there would be many
mathematicians who would not understand some standard theorem statement and
proof in an arbitrary introductory analysis text, even if one may dispute it's
presentation, clarity, pedagogy, or other pedantic details. Whether one asks
that semantic adequacy means some kind of sociological consensus among those
with relevant expertise, or a more relaxed criterion that some expert herself
understands the argument, a dubious perspective in scientific circles, semantic
adequacy should appease at least one and potentially more mathematicians.


\begin{figure}[H]
\centering
\begin{tikzcd}
Syntactically\ Complete \ar[r,"Informalization"] &[10em]
Semantically\ Coherent \ar[l,bend right, "Formalization"] 
\end{tikzcd}
\caption{Formal and Informal Mathematics} \label{fig:M6}
\end{figure}

We introduce these terms, syntactic completeness and semantic adequacy to
highlight perspectives and insights that seems to underlie the biggest
differences between informal and formal mathematics, as is show in
\autoref{fig:M6}. We claim that mathematics, as done via a theorem prover, is a
syntax oriented endeavor, whereas mathematics, as practiced by mathematicians,
prioritizes semantic understanding. Developing a system which is able to
formalize and informalize utterances which preserve syntactic completeness and
semantic adequacy, respectively, is probably infeasible. Even introducing
objective criteria to really judge these definitions is likely to be infeasible.

This perspective represents an observation and is not intended to judge whether
the syntactic or semantic perspective on mathematics is better - there is a
dialectical phenomena between the two. Let's highlight some advantages both
provide, and try to distinguish more precisely what a syntactic and semantic
perspective may be. 

When the Agda user builds her proof, she is outsourcing much of the bookkeeping
to the type-checker. This isn't purely a mechanical process though, she often
does have to think, how her definitions will interact with downstream programs,
as well as whether they are even sensible to begin with (i.e. does this have a
proof). The syntactic side is expressly clear from the readers perspective as
well. If Agda proofs were semantically coherent, one would only need to look at
code, with perhaps a few occasional remarks about various intentions and
conclusions, to understand the mathematics being expressed. Yet, papers are
often written exclusively in Latex, where Agda proofs have to be reverse
engineered, preserving only semantic details and forsaking syntactic nuance.


Oftentimes the code is kept in the appendix so as to provide a complete
syntactic blueprint. But the act of writing an Agda proof and reading it is
often orthogonal, as the term shadows the application of typing rules
which enable its construction. The construction of the proof is
entirely engaged with the types, whereas the human witness of a large term is
either lost as to why it fulfills the typing judgment, she has to reexamine
parts of the proof reasoning in her head or perhaps, try to rebuild 
interactively with Agda's help.

Even in cases where Agda code is included in a paper, it is most often the types
which are emphasized and produced. Complex proof terms are seldom to be read on
their own terms. The natural language description and commentary is still
largely necessary to convey whatever results, regardless if the Agda code is
self-contained. And while literate Agda is some type of bridge, it is still the
commentary which in some sense unfolds the code and ultimately makes the Agda
code legible.

This is particularly pronounced in the Coq programming language, where proof
terms are built using Ltac, which can be seen as some kind of imperative
syntactic metaprogramming over the core language, Gallina. The user rarely sees
the internal proof tree that one becomes familiar with in Agda. The tactics are
not typed, often feel very adhoc, and tacticals, sequences of tactics, may carry
very little semantic value (or even possibly muddy one's understanding when
reading proofs with unknown tactics). Indeed, since Ltac isn't itself typed, it
often descends into the sorrows of so-called untyped languages (which are really
uni-typed), and there are recent attempts to change this \cite{mtac2}
\cite{ltac2}. From our perspective, the use of tactics is an additional
syntactic obfuscation of what a proof should look like from the mathematicians
perspective - and it is important to attempt to remedy this is. Alecytron is one
impressive development in giving Coq proofs more readability through a
interactive back-end which shows the proof state, and offers other semantically
appealing models like interactive graphics \cite{coqAlec}. This kind of system
could and should inspire other proof assistants to allow for experimentation
with syntactic alternative to linear code.

Tactics obviously have their uses, and sometimes enhance high level proof
understanding, as tactics like \emph{ring} or \emph{omega} often save the reader overhead
of parsing pedantic and uninformative details. For certain proofs,
especially those involving many hundreds of cases, the metaprogramming
facilities actually give one exclusive advantages not offered to the classical
mathematician using pen and paper. Nonetheless, the dependent type theorist's
dream that all mathematicians begin using theorem provers in their everyday work
is largely just a dream, and with relatively little mainstream adoption by
mathematicians, the future is all but clear.

Mathematicians may indeed like some of the facilities theorem provers provide,
but ultimately, they may not see that as the "essence" of what they are doing.
What is this essence? We will try to shine a small light on perhaps the most
fundamental question in mathematics.

\subsection{What is a proof?}

\begin{displayquote}

A proof is what makes a judgment evident \cite{mlMeanings}.

\end{displayquote}

The proofs of Agda, and any programming language supporting proof development,
are \emph{formal proofs}. Formal proofs have no holes, and while there may very
well be bugs in the underlying technologies supporting these proofs, formal
proofs are seen as some kind of immutable form of data. One could say they
provide \emph{objective evidence} for judgments, which themselves are objective
entities when encoded on a computer. What we call formal proofs might provide a
science fiction writer an interesting thought experiment as regards
communicating mathematics with an alien species incapable of understanding our
language otherwise. Formal proofs, however, certainly don't appease all
mathematicians writing for other mathematicians.

Mathematics, and the act of proving theorems, according to Brouwer is a social
process. And because social processes between humans involve our linguistic
faculties, a we hope to elucidates what a proof with a simplified description.
Suppose we have two humans, $h_1$ and $h_2$. If $h_1$ claims to have a proof
$p_1$, and elaborates it to $p_2$ who claims she can either verify $p_1$ or
reproduce and re-articulate it via $p_1'$, such that $h1$ and $h2$ agree that
$p1$ and $p_1'$ are equivalent, then they have discovered some mathematics. In
fact, in this guise mathematics, can be viewed as a science, even if in fact it
is constructed instead of discovered.

An apt comparison is to see the mathematician is architect, whereas the computer
scientist responsible for formalizing the mathematics is an engineer. The
mathematics is the building which, like all human endeavors, is created via
resources and labor of many people. The role of the architect is to envision the
facade, the exterior layer directly perceived by others, giving a building its
character, purpose, and function. The engineer is on the other hand, tasked with
assuring the building gets built, doesn't collapse, and functions with many
implicit features which the user of the building may not notice : the running
water, insulation, and electricity. Whereas the architect is responsible for the
building's \emph{specification}, the engineer is tasked with its
\emph{implementation}.

We claim informal proofs are specifications and formal proofs are
implementations. Additionally, via the propositions-as-types interpretation, one
may see a logic as a specification and a PL as an implementation of a given
logic, often with multiple ways of assigning terms to a given type. Therefore,
one may see the mathematician ambiently developing a theorem in classical first
order logic as providing a specification of a proposition in that language,
whereas a given implementation of that theorem in Agda could be viewed as a
model construction of some NL fragment, where truth in the model would
correspond to termination of type-checking. Alternatively, during the
informalization process, two different authors may suppress different details,
or phrase a given utterance entirely differently, possibly leading to two
different, but possibly similar proofs. Extrapolating our analogy, the same 
two architects given the same engineering plans could produce two entirely
different looking and functioning buildings.  Oftentimes though, it is the
architect who has the vision, and the engineers who end up implementing the
architects art.

There isn't a natural notion of equivalence between informal and formal proofs,
but rather, loosely, some kind of adjunction in the categorical sense between
these two sets. We note the fact that the ``acceptable" Natural language
utterances aren't inductively defined. This precludes us from actually
constructing a canonical mathematical model of formal/informal relationship, but
we most certainly believe that if the GF perspective of translation is used,
there can at least be an approximation of what a model may look like. It is our
contention that the linguist interested in the language of mathematics should
perhaps be seen as a scientist, whose point is to contribute basic ideas and
insights from which the architects and engineers can use to inform their
designs.


% Importantly, we are just now at a point historically where our
% technologies support the implementations, and modern theorem provers like Coq
% and Agda provide an incredible amount of help in realizing the them.

Mathematicians seek model independence in their results (i.e., they don't need a
direct encoding of Fermat's last theorem in set theory in order to trust its
validity). This is one possible reason why there is so much reluctance to adopt
proof assistant, because the implementation of a result in Coq, Agda, or HOL4
may lead to many permutations of the same result, each presumably representing
the same piece of knowledge. It's also noted a proof doesn't obey the same
universality that it does when it's on paper or verbalized - that Agda 2.6.2,
and its standard library, when updated in the future, may ``break proofs", as
was seen in the introduction. While this is a unanimous problem with all
software, we believe the GF approach offers at least a vision of not only
linguistic, but also foundation agnosticism with respect to mathematics.

This thesis examines not just a practical problem, but touches many deep issues
in some space in the intersection of the foundations, of mathematics, logic,
computer science, and their relations studied via linguistic formalisms. These
subjects, and their various relations, are the subject of countless hours of
work and consideration by many great minds. We barely scratches the surface of a
few of these developments, but it nonetheless, it is hoped, provides a
nontrivial perspective at many important issues.

Recapitulating much of what was said, we hope that the following questions may
have a new perspective :

\begin{itemize}

\item what are mathematical objects ?
\item how do their encodings in different foundational formalisms affect their
  interpretations ?
\item how does is mathematics develop as a social process ?
\item how does what mathematics is and how it is done rely on given technologies
  of a given historical era. ?
  
\end{itemize}

While various branches of linguistics have seen rapid evolution due to, in large
part, their adoption of mathematical tools, the dual application of linguistic
tools to mathematics is quite sparse and open terrain. We hope the reader can
walk away with an new appreciation to some of these questions and topics after
reading this. These nuances we will not explore here, but shall be further
elaborated in the future and and more importantly, hopefully inspire other
readers to respond accordingly.

Although not given in specific detail, the view of what mathematics is, in both
a philosophical and mathematical sense, as well as from the view of what a
foundational perspective, requires deep consideration in its relation to
linguistics. And while this work is perhaps just a finer grain of sandpaper on
an incomplete and primordial marble sculpture, it is hoped that the sculptor's
own reflection is a little bit more clear after we polish it here.

\subsection{What is a proof revisited}

\begin{displayquote}
Though philosophical discussion of visual thinking in mathematics has
concentrated on its role in proof, visual thinking may be more valuable for
discovery than proof \cite{sep-epistemology-visual-thinking}
\end{displayquote}

As an addendum to asking such a presumably simple question in the previous
section, we briefly address the one particular oversimplification which was
made. We briefly touch on what isn't just syntactic about mathematics, namely
so-called ``Proofs without Words" \cite{proofWW} and other diagrammatic and visual
reasoning tools generally. Because our work focuses on syntax, and is not
generalized to other mathematical tools, we hope one considers this as well when
pondering the language of mathematics.

The role of visualization in programming, logic, and mathematics generally offers
an abundance of contrast to syntactically oriented alphanumeric alphabets, i.e.
strings of symbols, which we discuss here. Although the trees in GF are visual,
they are of intermediary form between strings in different languages, and
therefore the type of syntax we're discussing here is strings, we hope a brief
exploration of alternatives for concrete syntax will be fruitful. Targeting
latex via GF for instance, is a small step in this direction.

Graphical Programming languages facilitating diagrammatic programming are one
instance of a nonlinear syntax which would prove tricky but possible to
implement via GF.  Additionally, Globular, which allows one to
carry out higher categorical constructions via globular sets is an interesting
case study for a graphical programming language which is designed for 
theorem proving \cite{Bar2016GlobularAO}.
Additionally, Alecytron  supports basic data structure
visualization, like red-black trees which carry semantic content less easy in a
string based-setting \cite{coqAlec}.

Visualization are ubiquitous in contemporary mathematics, whether it be
analytic functions, knots, diagram chases in category theory, and a myriad of
other visual tools which both assist understanding and inform our syntactic
descriptions. We find these languages appealing because of their focus on a different kind of
internal semantic sensation. The diagrammatic languages for monoidal categories, for example, also allow
interpretations of formal proofs via topological deformations, and they have
given semantic representations to various graphical languages like circuit
diagrams and petri nets \cite{fong2016algebra}.

We also note that, while programming languages whose visual syntax evaluates to
strings, means that all diagrams can in some sense be encoded in more
traditional syntax, this is only for the computers sake - the human may consume
the diagram as an abstract entity other than a string. There are often words to
describe, but not to give visual intuition to many of the mathematical ideas we
grasp. There are also, famously blind mathematicians who work in topology,
geometry, and analysis \cite{2002TheWO}. Bernard Morin, blinded at a young age,
was a topologist who discovered the first eversion of a sphere by using clay
models which were then diagrammatically transcribed by a colleague on the board.
This remarkable use of something PL researchers would have a hard time
imagining, in some sense, and warrants careful consideration of what the
boundaries of proof assistants are capable of in terms of giving mathematicians
more tangible constructions.

For if there is one message one should take away from this thesis, it is that
there needs to be a coming to terms in the mathematics and TT communities, of
the difference between \emph{a proof}, both formal and informal, and the
\emph{the understanding of a proof}. The first is a mathematical judgment where
one supplies evidence, via the form of a term that Agda can type-check and
verify. A NL proof can be reviewed by a human. The understanding of a proof,
however, is not done by anything but a human. And this internal understanding
and processing of mathematical information, what I'll tongue-and-cheek call
i-mathematics, with its externalization facilities being our main concerns in
this thesis, requires much more work by future scholars.
