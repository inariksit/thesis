\documentclass[11pt, a4paper]{article}

\usepackage{mlt-thesis-2015}

% With Xetex/Luatex this shouldn't be used
%\usepackage[utf8]{inputenc}



\usepackage{multicol}
% \usepackage{graphicx}

\usepackage{csquotes}
\usepackage{float}

\usepackage[english]{babel}
\usepackage{graphicx}
\usepackage{setspace}

\usepackage{tikz-cd}

% from here


\usepackage{dsfont}
\usepackage{fontspec}
\usepackage{fullpage}
\usepackage{hyperref}
\usepackage{agda}

\usepackage{unicode-math}

%\usepackage{amssymb,amsmath,amsthm,stmaryrd,mathrsfs,wasysym}
\usepackage{stmaryrd}

%\usepackage{enumitem,mathtools,xspace}
\usepackage{amsfonts}
\usepackage{mathtools}
\usepackage{xspace}


\usepackage{enumitem}
\setlist[itemize]{noitemsep, topsep=0pt}

\usepackage{multicol}

\setmainfont{DejaVu Serif}
\setsansfont{DejaVu Sans}
\setmonofont{DejaVu Sans Mono}

% \setmonofont{Fira Mono}
% \setsansfont{Noto Sans}

\usepackage{newunicodechar}

\usepackage{xcolor}
\usepackage{listings}
\usepackage{xparse}
\NewDocumentCommand{\codeword}{v}{%
\texttt{\textcolor{gray}{#1}}%
}


\NewDocumentCommand{\term}{v}{%
\texttt{\textcolor{blue}{#1}}%
}
\NewDocumentCommand{\keyword}{v}{%
\texttt{\textcolor{orange}{#1}}%
}

% \usepackage{bussproofs}
\usepackage{ebproof}

\newunicodechar{ℓ}{\ensuremath{\mathnormal\ell}}
\newunicodechar{→}{\ensuremath{\mathnormal\rightarrow}}

\newtheorem{definition}{Definition}
\newtheorem{lem}{Lemma}
\newtheorem{proof}{Proof}
\newtheorem{thm}{Theorem}

\newcommand{\jdeq}{\equiv}      % An equality judgment
\newcommand{\refl}[1]{\ensuremath{\mathsf{refl}_{#1}}\xspace}
\newcommand{\define}[1]{\textbf{#1}}
\newcommand{\defeq}{\vcentcolon\equiv}  % A judgmental equality currently being defined

\newcommand{\id}[3][]{\ensuremath{#2 =_{#1} #3}\xspace}


\newcommand{\UU}{\ensuremath{\mathcal{U}}\xspace}
\let\bbU\UU
\let\type\UU

\newcommand{\equalH}[2]{#1 = #2}
\newcommand{\typingH}[2]{#1 : #2}
\newcommand{\lambdaH}[3]{\lambda_{#1 : #2} #3}
\newcommand{\arrowH}[2]{#1 \rightarrow #2}
\newcommand{\equivalenceH}[2]{#1 \simeq #2}
\newcommand{\comprehensionH}[3]{\{ #1 : #2 \mid #3 \}}
\newcommand{\idMapH}[1]{1_{ #1 }}
\newcommand{\fiberH}[2]{\{ #1 \}_{ #2 }}
\newcommand{\appH}[2]{#1 #2}
\newcommand{\defineH}[2]{#1 := #2}
\newcommand{\pairH}[2]{( #1 , #2 )}
\newcommand{\reflexivityH}[2]{r_{ #1 } #2}
\newcommand{\barH}[1]{\bar{ #1 }}
\newcommand{\idPropH}[2]{( #1 = #2 )}
\newcommand{\equivalenceMapH}[2]{E_ { #1 , #2 }}



% \DeclareMathSymbol{\mlq}{\mathord}{operators}{``}
% \DeclareMathSymbol{\mrq}{\mathord}{operators}{`'}

% \newcommand\doubleplus{+\kern-1.3ex+\kern0.8ex}

% for section numbers, doesn't work
% \usepackage{titlesec}
% \titleformat{\section}[block]
%   {\fontsize{12}{15}\bfseries\sffamily\filcenter}
%   {\thesection}
%   {1em}
%   {\MakeUppercase}
% \titleformat{\subsection}[hang]
%   {\fontsize{12}{15}\bfseries\sffamily}
%   {\thesubsection}
%   {1em}
%   {}

%\newcommand{\ct}{%
  %\mathchoice{\mathbin{\raisebox{0.5ex}{$\displaystyle\centerdot$}}}%
             %{\mathbin{\raisebox{0.5ex}{$\centerdot$}}}%
             %{\mathbin{\raisebox{0.25ex}{$\scriptstyle\,\centerdot\,$}}}%
             %{\mathbin{\raisebox{0.1ex}{$\scriptscriptstyle\,\centerdot\,$}}}
%}

% til here

\title{On the Grammar of Proof}
% \subtitle{Subtitle} case study in formal & nl proof translation
\author{Warrick Macmillan}

\begin{document}

%% ============================================================================
%% Title page
%% ============================================================================
\begin{titlepage}

\maketitle

\vfill

\begingroup
\renewcommand*{\arraystretch}{1.2}
\begin{tabular}{l@{\hskip 20mm}l}
\hline
Master's Thesis: & 30 credits\\
Programme: & Master’s Programme in Language Technology\\
Level: & Advanced level \\
Semester and year: & Fall, 2021\\
Supervisor & Aarne Ranta\\
Examiner 1 & Staffan Larsson\\
Examiner 2 & Stergios Chatzikyriakidis \\
Report number & (number will be provided by the administrators) \\
Keywords & The Language of Mathematics, Type Theory, \\
            & Grammatical Framework
\end{tabular}
\endgroup

\thispagestyle{empty}
\end{titlepage}

%% ============================================================================
%% Abstract
%% ============================================================================
\newpage
\singlespacing
\section*{Abstract}

% Brief summary of research question, background, method, results\ldots

The notion of \emph{formal proof} is a modern development, beginning with
Frege's foundational studies in modern mathematical logic. Formal proofs have
manifested more recently as verifiable computer programs written in programming
languages like Agda, via the propositions-as-types interpretation of logical
formulas. The notion of mathematical proof more generally, developed at least as
far back as Euclid, may be viewed as a natural language argument which provides
evidence for a proposition. Comparing notions of formal and natural language
mathematics is of both significant practical and theoretical concern, and one means
of comparison is seeking systematic ways of \emph{translating} between them.

This thesis gives one possible mechanism of translation between mathematical
vernacular and code via Grammatical Framework (GF), as a GF grammar can parse
and linearize. It can therefore translate between natural and programming
language utterances via a shared Abstract Syntax Tree (AST). While grammars for
programming languages are generally meant to be compact so-as to produce unique
parses, natural language grammars must account for both a natural language's
ambiguity and expressiveness - the fact that there are uncountable ways of
saying ``the same thing" makes it so interesting. Rectifying these opposing
interests in a single grammar is therefore incredibly challenging.

I introduce dual notions of understanding and analyzing mathematical language.
\emph{Syntactic completeness} is a criteria for judging constructions which
contain no errors and entirely encode an argument's subtlest details.
\emph{Semantically adequate} proofs and constructions are those which validate a
claim to a fluent mathematician, but may require some implicit knowledge,
explicitly defer arguments to the reader, or even contain errors. The grammars
written for this thesis and prior to it are therefore able to compared on this
spectrum. We demonstrate a syntactically complete grammar which can parse real
Agda proofs but generates poor natural language, and compare it to a
semantically adequate grammar which parses actual mathematics text, but
generates ill-formed types and programs. A Haskell embedding of these grammars
with intermediary transformations allows for at least a partial resolution of
these competing interests.

To further elaborate this discord between syntactic completeness and semantic
adequacy, we give parallel examples of mathematics text and Agda code,
with an Agda formalization of parts of the Homotopy Type Theory (HoTT) book
given to emphasize the needed for parallel corpus of programming and natural
language data for future translation endeavors. Additionally, the differences
between type theoretic, set theoretic, and logical language are explored
throughout this work, because foundational attitudes create inherent frictions
in the translation process. The insights gleaned from this work suggest new ways of
analyzing and understanding the difference between formal and informal proofs.

\thispagestyle{empty}

%% ============================================================================
%% Preface
%% ============================================================================
\newpage
\section*{Preface}

Acknowledgements, etc.

\thispagestyle{empty}

%% ============================================================================
%% Contents
%% ============================================================================
\newpage

\begin{spacing}{0.0}
\tableofcontents
\end{spacing}

\thispagestyle{empty}

%% ============================================================================
%% Introduction
%% ============================================================================
\newpage
\setcounter{page}{1}

% for it to be lagda
\begin{code}[hide]%
\>[0]\AgdaSymbol{\{-\#}\AgdaSpace{}%
\AgdaKeyword{OPTIONS}\AgdaSpace{}%
\AgdaPragma{--cubical}\AgdaSpace{}%
\AgdaSymbol{\#-\}}\<%
\\
%
\\[\AgdaEmptyExtraSkip]%
\>[0]\AgdaKeyword{module}\AgdaSpace{}%
\AgdaModule{roadmap}\AgdaSpace{}%
\AgdaKeyword{where}\<%
\end{code}

2208

TODO
\begin{itemize}
\item  Fix figure titles
\item internal References
\item Citations
\item fix quotes to include author
\item per inari, add more subsections
\item fix codeword, term, etc. to more uniform way
\item uniform way of referencing the grammars
\end{itemize}

\section{Introduction}
\label{sec:intro}

The central concern of this thesis is the syntax of mathematics, programming
languages, and their respective mutual influence, as conceived and practiced by
mathematicians and computer scientists.  From one vantage point, the role of
syntax in mathematics may be regarded as a 2nd order concern, a topic for
discussion during a Fika, an artifact of ad hoc development by the working
mathematician whose real goals are producing genuine mathematical knowledge.
For the programmers and computer scientists, syntax may be regarding as a
matter of taste, with friendly debates recurring regarding the use of
semicolons, brackets, and white space.  Yet, when viewed through the lens of
the propositions-as-types paradigm, these discussions intersect in new and
interesting ways.  When one introduces a third paradigm through which to
analyze the use of syntax in mathematics and programming, namely linguistics, I
propose what some may regard as superficial detail, indeed becomes a central
paradigm raising many interesting and important questions. 


\subsection{Beyond Computational Trinitarianism}

\begin{displayquote}

The doctrine of computational trinitarianism holds that computation manifests
itself in three forms: proofs of propositions, programs of a type, and mappings
between structures. These three aspects give rise to three sects of worship:
Logic, which gives primacy to proofs and propositions; Languages, which gives
primacy to programs and types; Categories, which gives primacy to mappings and
structures.\cite{harperTrinity}
\end{displayquote}

We begin this discussion of the three relationships between three respective
fields, mathematics, computer science, and logic. The aptly named 
trinity, shown in \autoref{fig:M1}, are related via both \emph{formal} and \emph{informal}
methods. The propositions as types paradigm, for example, is a heuristic. Yet
it also offers many examples of successful ideas translating between the domains.
Alternatively, the interpretation of a Type Theory(TT) into a category theory is
incredibly \emph{formal}.


\begin{figure}[H]
\centering
\begin{tikzcd}
                                                                            &  &  & Logic \arrow[llldddd, "Denotational\ Semantics" description] \arrow[rrrdddd, "Include\ Terms" description] &  &  &                                                                                                       \\
                                                                            &  &  &                                                                                                            &  &  &                                                                                                       \\
                                                                            &  &  &                                                                                                            &  &  &                                                                                                       \\
                                                                            &  &  &                                                                                                            &  &  &                                                                                                       \\
Math \arrow[rrruuuu, "Embedded\ in\ FOL", bend left] \arrow[rrrrrr, "ITP"'] &  &  &                                                                                                            &  &  & CS \arrow[llllll, "Denotational\ Semantics", bend left] \arrow[llluuuu, "Remove\ Terms"', bend right]
\end{tikzcd}
\caption{The Holy Trinity} \label{fig:M1}
\end{figure}

We hope this thesis will help clarify another possible dimension in this
diagram, that of Linguistics, and call it the ``holy tetrahedron". The different
vertices also resemble
religions in their own right, with communities convinced that they have a
canonical perspective on foundations and the essence of mathematics. Questioning the holy trinity is an act of a heresy, and
it is the goal of this thesis to be a bit heretical by including a much less well understood 
perspective which provides additional challenges and
insights into the trinity.

\begin{figure}[H]
\centering
\begin{tikzcd}
     &  &  & Logic                                                                                                                     &  &  &            \\
     &  &  &                                                                                                                           &  &  &            \\
     &  &  & Linguistics \arrow[uu, "Montague\ Semantics"'] \arrow[llldd, "Distributional\ Semantics"'] \arrow[rrrdd, "TT\ Semantics"] &  &  &            \\
     &  &  &                                                                                                                           &  &  &            \\
Math &  &  &                                                                                                                           &  &  & CS\ (MLTT)
\end{tikzcd}
\caption{Formal Semantics} \label{fig:M2}
\end{figure}

One may see how the trinity give rise to \emph{formal} semantic interpretations
of natural language in \autoref{fig:M2}. Semantics is just one possible
linguistic phenomenon worth investigating in these domains, and could be
replaced by other linguistic paradigms. This thesis is alternatively concerned
with syntax.

Finally, as in \autoref{fig:M3}, we can ask : how does the trinity embed into
natural language? These are the most \emph{informal} arrows of tetrahedron, or
at least one reading of it. One can analyze mathematics using linguistic
methods, or try to give a natural language justification of Intuitionistic Type
Theory (ITT) using Martin-Löf's meaning explanations.

\begin{figure}[H]
\centering
\begin{tikzcd}
                                                &  &  & Logic \arrow[dd, "Embedding"] &  &  &                               \\
                                                &  &  &                               &  &  &                               \\
                                                &  &  & Linguistics                   &  &  &                               \\
                                                &  &  &                               &  &  &                               \\
Math \arrow[rrruu, "Language\ Of\ Mathematics"] &  &  &
&  &  & CS\ (MLTT) \arrow[llluu, "Meaning\ Explanations"]
\end{tikzcd}
\caption{Interpretations of Natural Language} \label{fig:M3}
\end{figure}

In this work, we will see that there are multiple GF grammars which model some
subset of each member of the trinity. Constructing these grammars, and asking
how they can be used in applications for mathematicians, logicians, and computer
scientists is an important practical and philosophical question. Therefore we
hope this attempt at giving the language of mathematics, in particular how
propositions and proofs are expressed and thought about in that language, a
stronger foundation.

\subsection{What is a Homomorophism?}

To get a feel for the syntactic paradigm we explore in this thesis, let us look at a basic mathematical
example: that of a group homomorphism as expressed in by a variety of somewhat
randomly sampled authors.  

% Wikipedia Defn:

\begin{definition}
In mathematics, given two groups, $(G, \ast)$ and $(H, \cdot)$, a group homomorphism from $(G, \ast)$ to $(H, \cdot)$ is a function $h : G \to H$ such that for all $u$ and $v$ in $G$ it holds that
  $$h(u \ast v) = h ( u ) \cdot h ( v )$$ 
\end{definition}

% http://math.mit.edu/~jwellens/Group%20Theory%20Forum.pdf

\begin{definition}
Let $G = (G,\cdot)$ and $G' = (G',\ast)$ be groups, and let $\phi : G \to G'$ be a map between them. We call $\phi$ a \textbf{homomorphism} if for every pair of elements $g, h \in G$, we have 
% \begin{center}
  $$\phi(g \ast h) = \phi ( g ) \cdot \phi ( h )$$ 
% \end{center}
\end{definition}

% http://www.maths.gla.ac.uk/~mwemyss/teaching/3alg1-7.pdf

\begin{definition}\label{def:def3}
Let $G$, $H$, be groups.  A map $\phi : G \to H$ is called a \emph{group homomorphism} if
  $$\phi(xy) = \phi ( x ) \phi ( y )$ for all $x, y \in G$$ 
(Note that $xy$ on the left is formed using the group operation in $G$, whilst the product $\phi ( x ) \phi ( y )$ is formed using the group operation $H$.)
\end{definition}

% NLab:

\begin{definition}\label{def:def4}
Classically, a group is a monoid in which every element has an inverse (necessarily unique).
\end{definition}

We inquire the reader to pay attention to nuance and difference in presentation
that is normally ignored or taken for granted by the fluent mathematician, ask
which definitions feel better, and how the reader herself might present the
definition differently.

If one want to distill the meaning of each of these presentations, there is a
significant amount of subliminal interpretation happening very much analogous to
our innate lingusitic ussage. The inverse and identity are discarded, even
though they are necessary data when defning a group. The order of presentation
of information is inconsistent, as well as the choice to use symbolic or natural
language information. In Definition~\ref{def:def3}, the group operation is used
implicitly, and its clarification a side remark.

Details aside, these all mean the same thing - don't they?  This thesis seeks to provide an
abstract framework to determine whether two lingusitically nuanced presenations
mean the same thing via their syntactic transformations. Obviously these
meanings  are not resolvable in any kind of absolute sense, but at least from a
translational sense. These syntactic transformations come in two flavors : parsing and
linearization, and are natively handled by a Logical Framework (LF) for
specifying grammars : Grammatical Framework (GF).

We now show yet another definition of a group homomorphism formalized in the
Agda programming language:

\begin{code}[hide]%
\>[0]\AgdaComment{--\{-\# OPTIONS --cubical \#-\}}\<%
\\
\>[0]\AgdaSymbol{\{-\#}\AgdaSpace{}%
\AgdaKeyword{OPTIONS}\AgdaSpace{}%
\AgdaPragma{--cubical}\AgdaSpace{}%
\AgdaPragma{--no-import-sorts}\AgdaSpace{}%
\AgdaPragma{--safe}\AgdaSpace{}%
\AgdaSymbol{\#-\}}\<%
\\
%
\\[\AgdaEmptyExtraSkip]%
\>[0]\AgdaKeyword{module}\AgdaSpace{}%
\AgdaModule{monoid}\AgdaSpace{}%
\AgdaKeyword{where}\<%
\\
%
\\[\AgdaEmptyExtraSkip]%
\>[0]\AgdaKeyword{module}\AgdaSpace{}%
\AgdaModule{Namespace1}\AgdaSpace{}%
\AgdaKeyword{where}\<%
\\
%
\\[\AgdaEmptyExtraSkip]%
\>[0][@{}l@{\AgdaIndent{0}}]%
\>[2]\AgdaKeyword{open}\AgdaSpace{}%
\AgdaKeyword{import}\AgdaSpace{}%
\AgdaModule{Cubical.Foundations.Prelude}\<%
\\
%
\>[2]\AgdaKeyword{open}\AgdaSpace{}%
\AgdaKeyword{import}\AgdaSpace{}%
\AgdaModule{Cubical.Foundations.Equiv}\<%
\\
%
\>[2]\AgdaKeyword{open}\AgdaSpace{}%
\AgdaKeyword{import}\AgdaSpace{}%
\AgdaModule{Cubical.Foundations.Structure}\<%
\\
%
\>[2]\AgdaKeyword{open}\AgdaSpace{}%
\AgdaKeyword{import}\AgdaSpace{}%
\AgdaModule{Cubical.Algebra.Group.Base}\<%
\\
%
\>[2]\AgdaKeyword{open}\AgdaSpace{}%
\AgdaKeyword{import}\AgdaSpace{}%
\AgdaModule{Cubical.Data.Sigma}\<%
\\
%
\\[\AgdaEmptyExtraSkip]%
%
\>[2]\AgdaKeyword{private}\<%
\\
\>[2][@{}l@{\AgdaIndent{0}}]%
\>[4]\AgdaKeyword{variable}\<%
\\
\>[4][@{}l@{\AgdaIndent{0}}]%
\>[6]\AgdaGeneralizable{ℓ}\AgdaSpace{}%
\AgdaGeneralizable{ℓ'}\AgdaSpace{}%
\AgdaGeneralizable{ℓ''}\AgdaSpace{}%
\AgdaGeneralizable{ℓ'''}\AgdaSpace{}%
\AgdaSymbol{:}\AgdaSpace{}%
\AgdaPostulate{Level}\<%
\end{code}
\begin{code}%
%
\>[2]\AgdaFunction{isGroupHom}\AgdaSpace{}%
\AgdaSymbol{:}\AgdaSpace{}%
\AgdaSymbol{(}\AgdaBound{G}\AgdaSpace{}%
\AgdaSymbol{:}\AgdaSpace{}%
\AgdaFunction{Group}\AgdaSpace{}%
\AgdaSymbol{\{}\AgdaGeneralizable{ℓ}\AgdaSymbol{\})}\AgdaSpace{}%
\AgdaSymbol{(}\AgdaBound{H}\AgdaSpace{}%
\AgdaSymbol{:}\AgdaSpace{}%
\AgdaFunction{Group}\AgdaSpace{}%
\AgdaSymbol{\{}\AgdaGeneralizable{ℓ'}\AgdaSymbol{\})}\AgdaSpace{}%
\AgdaSymbol{(}\AgdaBound{f}\AgdaSpace{}%
\AgdaSymbol{:}\AgdaSpace{}%
\AgdaOperator{\AgdaFunction{⟨}}\AgdaSpace{}%
\AgdaBound{G}\AgdaSpace{}%
\AgdaOperator{\AgdaFunction{⟩}}\AgdaSpace{}%
\AgdaSymbol{→}\AgdaSpace{}%
\AgdaOperator{\AgdaFunction{⟨}}\AgdaSpace{}%
\AgdaBound{H}\AgdaSpace{}%
\AgdaOperator{\AgdaFunction{⟩}}\AgdaSymbol{)}\AgdaSpace{}%
\AgdaSymbol{→}\AgdaSpace{}%
\AgdaPrimitive{Type}\AgdaSpace{}%
\AgdaSymbol{\AgdaUnderscore{}}\<%
\\
%
\>[2]\AgdaFunction{isGroupHom}\AgdaSpace{}%
\AgdaBound{G}\AgdaSpace{}%
\AgdaBound{H}\AgdaSpace{}%
\AgdaBound{f}\AgdaSpace{}%
\AgdaSymbol{=}\AgdaSpace{}%
\AgdaSymbol{(}\AgdaBound{x}\AgdaSpace{}%
\AgdaBound{y}\AgdaSpace{}%
\AgdaSymbol{:}\AgdaSpace{}%
\AgdaOperator{\AgdaFunction{⟨}}\AgdaSpace{}%
\AgdaBound{G}\AgdaSpace{}%
\AgdaOperator{\AgdaFunction{⟩}}\AgdaSymbol{)}\AgdaSpace{}%
\AgdaSymbol{→}\AgdaSpace{}%
\AgdaBound{f}\AgdaSpace{}%
\AgdaSymbol{(}\AgdaBound{x}\AgdaSpace{}%
\AgdaOperator{\AgdaFunction{G.+}}\AgdaSpace{}%
\AgdaBound{y}\AgdaSymbol{)}\AgdaSpace{}%
\AgdaOperator{\AgdaFunction{≡}}\AgdaSpace{}%
\AgdaSymbol{(}\AgdaBound{f}\AgdaSpace{}%
\AgdaBound{x}\AgdaSpace{}%
\AgdaOperator{\AgdaFunction{H.+}}\AgdaSpace{}%
\AgdaBound{f}\AgdaSpace{}%
\AgdaBound{y}\AgdaSymbol{)}\AgdaSpace{}%
\AgdaKeyword{where}\<%
\\
\>[2][@{}l@{\AgdaIndent{0}}]%
\>[4]\AgdaKeyword{module}\AgdaSpace{}%
\AgdaModule{G}\AgdaSpace{}%
\AgdaSymbol{=}\AgdaSpace{}%
\AgdaModule{GroupStr}\AgdaSpace{}%
\AgdaSymbol{(}\AgdaField{snd}\AgdaSpace{}%
\AgdaBound{G}\AgdaSymbol{)}\<%
\\
%
\>[4]\AgdaKeyword{module}\AgdaSpace{}%
\AgdaModule{H}\AgdaSpace{}%
\AgdaSymbol{=}\AgdaSpace{}%
\AgdaModule{GroupStr}\AgdaSpace{}%
\AgdaSymbol{(}\AgdaField{snd}\AgdaSpace{}%
\AgdaBound{H}\AgdaSymbol{)}\<%
\\
%
\\[\AgdaEmptyExtraSkip]%
%
\>[2]\AgdaKeyword{record}\AgdaSpace{}%
\AgdaRecord{GroupHom}\AgdaSpace{}%
\AgdaSymbol{(}\AgdaBound{G}\AgdaSpace{}%
\AgdaSymbol{:}\AgdaSpace{}%
\AgdaFunction{Group}\AgdaSpace{}%
\AgdaSymbol{\{}\AgdaGeneralizable{ℓ}\AgdaSymbol{\})}\AgdaSpace{}%
\AgdaSymbol{(}\AgdaBound{H}\AgdaSpace{}%
\AgdaSymbol{:}\AgdaSpace{}%
\AgdaFunction{Group}\AgdaSpace{}%
\AgdaSymbol{\{}\AgdaGeneralizable{ℓ'}\AgdaSymbol{\})}\AgdaSpace{}%
\AgdaSymbol{:}\AgdaSpace{}%
\AgdaPrimitive{Type}\AgdaSpace{}%
\AgdaSymbol{(}\AgdaPrimitive{ℓ-max}\AgdaSpace{}%
\AgdaBound{ℓ}\AgdaSpace{}%
\AgdaBound{ℓ'}\AgdaSymbol{)}\AgdaSpace{}%
\AgdaKeyword{where}\<%
\\
\>[2][@{}l@{\AgdaIndent{0}}]%
\>[4]\AgdaKeyword{constructor}\AgdaSpace{}%
\AgdaInductiveConstructor{grouphom}\<%
\\
%
\\[\AgdaEmptyExtraSkip]%
%
\>[4]\AgdaKeyword{field}\<%
\\
\>[4][@{}l@{\AgdaIndent{0}}]%
\>[6]\AgdaField{fun}\AgdaSpace{}%
\AgdaSymbol{:}\AgdaSpace{}%
\AgdaOperator{\AgdaFunction{⟨}}\AgdaSpace{}%
\AgdaBound{G}\AgdaSpace{}%
\AgdaOperator{\AgdaFunction{⟩}}\AgdaSpace{}%
\AgdaSymbol{→}\AgdaSpace{}%
\AgdaOperator{\AgdaFunction{⟨}}\AgdaSpace{}%
\AgdaBound{H}\AgdaSpace{}%
\AgdaOperator{\AgdaFunction{⟩}}\<%
\\
%
\>[6]\AgdaField{isHom}\AgdaSpace{}%
\AgdaSymbol{:}\AgdaSpace{}%
\AgdaFunction{isGroupHom}\AgdaSpace{}%
\AgdaBound{G}\AgdaSpace{}%
\AgdaBound{H}\AgdaSpace{}%
\AgdaField{fun}\<%
\end{code}
This actually \emph{was} the Cubical Agda implementation of a group homomorphism
sometime around the end of 2020. We see that, while a mathematician might be
able to infer the meaning of some of the syntax, the use of levels,
distinguising between isGroupHom and GroupHom itself, and many other details
might obscure what's going on.

We finally provide the current (May 2021) definition via Cubical Agda. One may
witness a significant number of differences from the previous version : concrete
syntax differences via changes in camel case, new uses of Group vs GroupStr, as
well as, most significantly, the identity and inverse preservation data not
appearing as corollaries, but part of the definition. Additionally, we had to
refactor the commented lines to those shown below to be compatible with our
outdated version of cubical. These changes reflect interesting syntactic
changes.

\begin{code}%
%
\>[2]\AgdaKeyword{record}\AgdaSpace{}%
\AgdaRecord{IsGroupHom}\AgdaSpace{}%
\AgdaSymbol{\{}\AgdaBound{A}\AgdaSpace{}%
\AgdaSymbol{:}\AgdaSpace{}%
\AgdaPrimitive{Type}\AgdaSpace{}%
\AgdaGeneralizable{ℓ}\AgdaSymbol{\}}\AgdaSpace{}%
\AgdaSymbol{\{}\AgdaBound{B}\AgdaSpace{}%
\AgdaSymbol{:}\AgdaSpace{}%
\AgdaPrimitive{Type}\AgdaSpace{}%
\AgdaGeneralizable{ℓ'}\AgdaSymbol{\}}\<%
\\
\>[2][@{}l@{\AgdaIndent{0}}]%
\>[4]\AgdaSymbol{(}\AgdaBound{M}\AgdaSpace{}%
\AgdaSymbol{:}\AgdaSpace{}%
\AgdaRecord{GroupStr}\AgdaSpace{}%
\AgdaBound{A}\AgdaSymbol{)}\AgdaSpace{}%
\AgdaSymbol{(}\AgdaBound{f}\AgdaSpace{}%
\AgdaSymbol{:}\AgdaSpace{}%
\AgdaBound{A}\AgdaSpace{}%
\AgdaSymbol{→}\AgdaSpace{}%
\AgdaBound{B}\AgdaSymbol{)}\AgdaSpace{}%
\AgdaSymbol{(}\AgdaBound{N}\AgdaSpace{}%
\AgdaSymbol{:}\AgdaSpace{}%
\AgdaRecord{GroupStr}\AgdaSpace{}%
\AgdaBound{B}\AgdaSymbol{)}\<%
\\
%
\>[4]\AgdaSymbol{:}\AgdaSpace{}%
\AgdaPrimitive{Type}\AgdaSpace{}%
\AgdaSymbol{(}\AgdaPrimitive{ℓ-max}\AgdaSpace{}%
\AgdaBound{ℓ}\AgdaSpace{}%
\AgdaBound{ℓ'}\AgdaSymbol{)}\<%
\\
%
\>[4]\AgdaKeyword{where}\<%
\\
%
\\[\AgdaEmptyExtraSkip]%
%
\>[4]\AgdaComment{-- Shorter qualified names}\<%
\\
%
\>[4]\AgdaKeyword{private}\<%
\\
\>[4][@{}l@{\AgdaIndent{0}}]%
\>[6]\AgdaKeyword{module}\AgdaSpace{}%
\AgdaModule{M}\AgdaSpace{}%
\AgdaSymbol{=}\AgdaSpace{}%
\AgdaModule{GroupStr}\AgdaSpace{}%
\AgdaBound{M}\<%
\\
%
\>[6]\AgdaKeyword{module}\AgdaSpace{}%
\AgdaModule{N}\AgdaSpace{}%
\AgdaSymbol{=}\AgdaSpace{}%
\AgdaModule{GroupStr}\AgdaSpace{}%
\AgdaBound{N}\<%
\\
%
\\[\AgdaEmptyExtraSkip]%
%
\>[4]\AgdaKeyword{field}\<%
\\
\>[4][@{}l@{\AgdaIndent{0}}]%
\>[6]\AgdaField{pres·}\AgdaSpace{}%
\AgdaSymbol{:}\AgdaSpace{}%
\AgdaSymbol{(}\AgdaBound{x}\AgdaSpace{}%
\AgdaBound{y}\AgdaSpace{}%
\AgdaSymbol{:}\AgdaSpace{}%
\AgdaBound{A}\AgdaSymbol{)}\AgdaSpace{}%
\AgdaSymbol{→}\AgdaSpace{}%
\AgdaBound{f}\AgdaSpace{}%
\AgdaSymbol{(}\AgdaOperator{\AgdaFunction{M.\AgdaUnderscore{}+\AgdaUnderscore{}}}\AgdaSpace{}%
\AgdaBound{x}\AgdaSpace{}%
\AgdaBound{y}\AgdaSymbol{)}\AgdaSpace{}%
\AgdaOperator{\AgdaFunction{≡}}\AgdaSpace{}%
\AgdaSymbol{(}\AgdaOperator{\AgdaFunction{N.\AgdaUnderscore{}+\AgdaUnderscore{}}}\AgdaSpace{}%
\AgdaSymbol{(}\AgdaBound{f}\AgdaSpace{}%
\AgdaBound{x}\AgdaSymbol{)}\AgdaSpace{}%
\AgdaSymbol{(}\AgdaBound{f}\AgdaSpace{}%
\AgdaBound{y}\AgdaSymbol{))}\<%
\\
%
\>[6]\AgdaField{pres1}\AgdaSpace{}%
\AgdaSymbol{:}\AgdaSpace{}%
\AgdaBound{f}\AgdaSpace{}%
\AgdaFunction{M.0g}\AgdaSpace{}%
\AgdaOperator{\AgdaFunction{≡}}\AgdaSpace{}%
\AgdaFunction{N.0g}\<%
\\
%
\>[6]\AgdaField{presinv}\AgdaSpace{}%
\AgdaSymbol{:}\AgdaSpace{}%
\AgdaSymbol{(}\AgdaBound{x}\AgdaSpace{}%
\AgdaSymbol{:}\AgdaSpace{}%
\AgdaBound{A}\AgdaSymbol{)}\AgdaSpace{}%
\AgdaSymbol{→}\AgdaSpace{}%
\AgdaBound{f}\AgdaSpace{}%
\AgdaSymbol{(}\AgdaOperator{\AgdaFunction{M.-\AgdaUnderscore{}}}\AgdaSpace{}%
\AgdaBound{x}\AgdaSymbol{)}\AgdaSpace{}%
\AgdaOperator{\AgdaFunction{≡}}\AgdaSpace{}%
\AgdaOperator{\AgdaFunction{N.-\AgdaUnderscore{}}}\AgdaSpace{}%
\AgdaSymbol{(}\AgdaBound{f}\AgdaSpace{}%
\AgdaBound{x}\AgdaSymbol{)}\<%
\\
%
\>[6]\AgdaComment{-- pres· : (x y : A) → f (x M.· y) ≡ f x N.· f y}\<%
\\
%
\>[6]\AgdaComment{-- pres1 : f M.1g ≡ N.1g}\<%
\\
%
\>[6]\AgdaComment{-- presinv : (x : A) → f (M.inv x) ≡ N.inv (f x)}\<%
\\
%
\\[\AgdaEmptyExtraSkip]%
%
\>[2]\AgdaFunction{GroupHom'}\AgdaSpace{}%
\AgdaSymbol{:}\AgdaSpace{}%
\AgdaSymbol{(}\AgdaBound{G}\AgdaSpace{}%
\AgdaSymbol{:}\AgdaSpace{}%
\AgdaFunction{Group}\AgdaSpace{}%
\AgdaSymbol{\{}\AgdaGeneralizable{ℓ}\AgdaSymbol{\})}\AgdaSpace{}%
\AgdaSymbol{(}\AgdaBound{H}\AgdaSpace{}%
\AgdaSymbol{:}\AgdaSpace{}%
\AgdaFunction{Group}\AgdaSpace{}%
\AgdaSymbol{\{}\AgdaGeneralizable{ℓ'}\AgdaSymbol{\})}\AgdaSpace{}%
\AgdaSymbol{→}\AgdaSpace{}%
\AgdaPrimitive{Type}\AgdaSpace{}%
\AgdaSymbol{(}\AgdaPrimitive{ℓ-max}\AgdaSpace{}%
\AgdaGeneralizable{ℓ}\AgdaSpace{}%
\AgdaGeneralizable{ℓ'}\AgdaSymbol{)}\<%
\\
%
\>[2]\AgdaComment{-- GroupHom' : (G : Group ℓ) (H : Group ℓ') → Type (ℓ-max ℓ ℓ')}\<%
\\
%
\>[2]\AgdaFunction{GroupHom'}\AgdaSpace{}%
\AgdaBound{G}\AgdaSpace{}%
\AgdaBound{H}\AgdaSpace{}%
\AgdaSymbol{=}\AgdaSpace{}%
\AgdaFunction{Σ[}\AgdaSpace{}%
\AgdaBound{f}\AgdaSpace{}%
\AgdaFunction{∈}\AgdaSpace{}%
\AgdaSymbol{(}\AgdaBound{G}\AgdaSpace{}%
\AgdaSymbol{.}\AgdaField{fst}\AgdaSpace{}%
\AgdaSymbol{→}\AgdaSpace{}%
\AgdaBound{H}\AgdaSpace{}%
\AgdaSymbol{.}\AgdaField{fst}\AgdaSymbol{)}\AgdaSpace{}%
\AgdaFunction{]}\AgdaSpace{}%
\AgdaRecord{IsGroupHom}\AgdaSpace{}%
\AgdaSymbol{(}\AgdaBound{G}\AgdaSpace{}%
\AgdaSymbol{.}\AgdaField{snd}\AgdaSymbol{)}\AgdaSpace{}%
\AgdaBound{f}\AgdaSpace{}%
\AgdaSymbol{(}\AgdaBound{H}\AgdaSpace{}%
\AgdaSymbol{.}\AgdaField{snd}\AgdaSymbol{)}\<%
\end{code}

While the last two definitions may carry degree of comprehension to a programmer
or mathematician not exposed to Agda, it is certainly comprehensible to a
computer : that is, it typechecks on a computer where Cubical Agda is installed.
While GF is designed for multilingual syntactic transformations and is targeted
for natural language translation, it's underlying theory is largely based on
ideas from the compiler communities. A cousin of the BNF Converter (BNFC), GF is
fully capable of parsing programming languages like Agda! And while the Agda
definitions are just another concrete syntactic presentation of a group
homomorphism, they are distinct from the natural language presentations above in
that the colors indicate it has indeed type checked.

While this example may not exemplify the power of Agda's type-checker, it is of
considerable interest to many. The type-checker has merely assured us that
\term{GroupHom(')} are well-formed types - not that we have a canonical representation
of a group homomorphism. The type-checker is much more useful than is
immediately evident: it delegates the work of verifying that a proof is correct,
that is, the work of judging whether a term has a type, to the computer. While
it's of practical concern is immediate to any exploited grad student grading
papers late on a Sunday night, its theoretical concern has led to many recent
developments in modern mathematics. Thomas Hales solution to the Kepler
Conjecture was seen as unverifiable by those reviewing it, and this led to Hales
outsourcing the verification to Interactive Theorem Provers (ITPs) HOL Light and
Isabelle. This computer delegated verification phase led to many minor
corrections in the original proof which were never spotted due to human
oversight.

Fields medalist Vladimir Voevodsky had the experience of being told one day
his proof of the Milnor conjecture was fatally flawed. Although the leak in the
proof was patched, this experience of temporarily believing much of his life's
work invalidated led him to investigate proof assintants as a tool for future
thought. Indeed, this proof verification error was a key event that led to the
Univalent Foundations
Project~\cite{theunivalentfoundationsprogram-homotopytypetheory-2013}.

While Agda and other programming languages are capable of encoding definitions,
theorems, and proofs, they have so far seen little adoption. In some cases they
have been treated with suspicion and scorn by many mathematicians. This isn't
entirely unfounded : it's a lot of work to learn how to use Agda or Coq,
software updates may cause proofs to break, and the inevitable imperfections we
humans are prone to instilled in these tools . Besides, Martin-Löf Type Theory,
the constructive foundational project which underlies these proof assistants, is
often misunderstood by those who dogmatically accept the law of the excluded
middle as the word of God.

It should be noted, the constructivist rejects neither the law of the excluded
middle, nor ZFC. She merely observes them, and admits their handiness in certain
citations. Excluded middle is indeed a helpful tool as many mathematicians
may attest. The contention is that it should be avoided whenever possible -
proofs which don't rely on it, or it's corallary of proof by contradction, are
much more ameanable to formalization in systems with decideable type checking.
And ZFC, while serving the mathematicians of the early 20th century, is 
lacking when it comes to the higher dimensional structure of n-categories and
infinity groupoids.

What these theorem provers give the mathematician is confidence that her work
is correct, and even more importantly, that the work which she takes for
granted and references in her work is also correct. The task before us is then
one of religious conversion. And one doesn't undertake a conversion by simply
by preaching. Foundational details aside, this thesis is meant to provide a
blueprint for the syntactic reformation that must take place.  

We don't insist a mathematician relinquish the beautiful language she has
come to love in expressing her ideas.  Rather, it asks her to make a
hypothetical compromise
for the time being, and use a Controlled Natural Language (CNL) to develop her
work. In exchange she'll get the confidence that Agda provides. Not only that,
she'll be able to search through a library, to see who else has possibly
already postulated and proved her conjecture. A version of this grandiose vision is 
explored in The Formal Abstracts Project \cite{halesCNL}, and it should
practically motivate work.  

Practicalities aside, this work also attempts to offer a nuanced philosophical
perspective on the matter by exploring why translation of mathematical language,
despite it's seemingly structured form, is difficult. We note that the natural
language definitions of monoid differ in form, but also in pragmatic content.
How one expresses formalities in natural language is incredibly diverse, and
Definition~\ref{def:def4} as compared with the prior homomorphism definitions is
particularly poignant in demonstrating this. These differ very much in nature to
the Agda definitions - especially pragmatically. The differences between the Cubical
Agda definitions may be loosely called pragmatic, in the sense that the choice
of definitions may have downstream effects on readability, maintainability, modularity, and other
considerations when trying to write good code, in a burgeoning area known as proof engineering.

A pragmatic treatment of the language of mathematics is the golden egg if one
wishes to articulate the nuance in how the notions proposition, proof, and
judgment are understood by humans. Nonetheless, this problem is just now seeing
attention. We hope that the treatment of syntax in this thesis, while a long
ways away from giving a pragmatic account of mathematics, will help pave the way
there.


\section{Preliminaries}

We give brief but relevant overviews of the background ideas and tools that went
into the generation of this thesis. 

% % from blog post
\subsection{Martin-Löf Type Theory}
\subsubsection{Judgments}

\begin{displayquote}

With Kant, something important happened, namely, that the term judgement, Ger.
Urteil, came to be used instead of proposition \cite{mlMeanings}.

\end{displayquote}

A central contribution of Per Martin-Löf in the development of type theory was
the recognition of the centrality of judgments in logic. Many mathematicians
aren't familiar with the spectrum of judgments available, and merely believe
they are concerned with \emph{the} notion of truth, namely \emph{the truth} of a
mathematical proposition or theorem. There are many judgments one can make which
most mathematicians aren't aware of or at least never mention. Examples of both familiar
and unfamiliar judgments include,

\begin{itemize}

\item $A$ is true
\item $A$ is a proposition
\item $A$ is possible
\item $A$ is necessarily true
\item $A$ is true at time $t$

\end{itemize}

These judgments are understood not in the object language in which we state our
propositions, possibilities, or probabilities, but as assertions in the
metalanguage which require evidence for us to know and believe them. Most
mathematicians may reach for their wallets if I come in and give a talk saying
it is possible that the Riemann Hypothesis is true, partially because they
already know that, and partially because it doesn't seem particularly
interesting to say that something is possible, in the same way that a physicist
may flinch if you say alchemy is possible. Most mathematicians, however, would
agree that $P = NP$ is a proposition, and it is also possible, but isn't true.

For the logician these judgments may well be interesting because their may be
logics in which the discussion of possibility or necessity is even more
interesting than the discussion of truth. And for the type theorist interested
in designing and building programming languages over many various logics, these
judgments become a prime focus. The role of the type-checker in a programming
language is to present evidence for, or decide the validity of the judgments.
The four main judgments of type theory are given in natural language on the left
and symbolically on the right.

\begin{multicols}{2}
\begin{itemize}
\item $T$ is a type
\item $T$ and $T'$ are equal types
\item $t$ is a term of type $T$
\item $t$ and $t'$ are equal terms of type $T$
\item $\vdash T \; {\rm type}$
\item $\vdash T = T'$
\item $\vdash t:T$
\item $\vdash t = t':T$
\end{itemize}
\end{multicols}

Frege's turnstile, $\vdash$, denotes a judgment.

These judgments become much more interesting when we add the ability for them to
be interpreted in a some context with judgment hypotheses. Given a series of
judgments $J_1,...,J_n$, denoted $\Gamma$, where $J_i$ can depend on previously
listed $J's$, we can make judgment $J$ under the hypotheses, e.g. $J_1,...,J_n
\vdash J$. Often these hypotheses $J_i$, alternatively called \emph{antecedents},
denote variables which may occur freely in the *consequent* judgment $J$. For
instance, the antecedent, $x : \mathbb{R}$ occurs freely in the syntactic
expression $\sin x$, a which is given meaning in the judgment $\vdash \sin x { :
} \mathbb{R}$. We write our hypothetical judgement as follows :

$$x : \mathbb{R} \vdash \sin x : \mathbb{R}$$



\subsubsection{Rules}

Martin-Löf systematically used the four fundamental judgments in the proof
theoretic style of Pragwitz. To this end, the intuitionistic formulation of the
logical connectives just gives rules which admit an immediate computational
interpretation. The main types of rules are type formation, introduction,
elimination, and computation rules. The introduction rules for a type admit an
induction principle derivable from that type's signature. Additionally, the
$\beta$ and $\eta$ computation rules are derivable via the composition of
introduction and elimination rules, which, if correctly formulated, should
satisfy a relation known as harmony.

The fundamental notion of the lambda calculus, the function, is 
abstracted over a variable and returns a term of some type when applied to an
argument which is subsequently reduced via the computational rules.
Dependent Type Theory (DTT) generalizes this to allow the return type be
parameterized by the variable being abstracted over. The dependent function
forms the basis of the LF which underlies Agda and GF. 

One reason why hypothetical judgments are so interesting is we can devise rules
which allow us to translate from the metalanguage to the object language using
lambda expressions. These play the role of a function in mathematics and
implication in logic. This comes out in the following introduction rule :

% $$ \frac{\Gamma, x : A \vdash b : B} {\Gamma \vdash \lambda x. b : A \rightarrow
% B} $$

Using this rule, we now see a typical judgment, typical in a field like from
real analysis,

$\vdash \lambda x. \sin x : \R \rightarrow \R$

Equality :

Mathematicians denote this judgement
\begin{align*} f {:} \mathbb{R} &\rightarrow \mathbb{R}\\ x &\mapsto \sin ( x )
\end{align*}

\subsection{Propositions, Sets, and Types}

While the rules of type theory have been well-articulated elsewhere, we provide
briefly compare the syntax of mathematical constructions in FOL, one possible
natural language use \cite{rantaLog}, and MLTT. From this vantage, these look
like simple symbolic manipulations, and in some sense, one doesn't need a the
expressive power of system like GF to parse these to the same form.

% \begin{figure}
% \centering
% \begin{tabular}{|c|c|c|c|} \hline
%   FOL & MLTT & NL FOL & NL MLTT \\ \hline
%   $\forall\ x\ P(x)$ & $\Pi x : \tau.\ P(x)$     & $for\ all\ x,\ p$  & $the product over x in\ p$ \\ hline
%   $\exists\ x\ P(x)$ & $\Sigma x : \tau.\ P(x)$  & $there\ exists\ an\ x\ such\ that\ p$ & $there\ exists\ an\ x\ \in \tau \such \that p$ \\ hline 
%   $p\ \supset\ q$    & $p\ \to\ q$               & $if\ p\ then\ q$   & $p to q$ \\ hline
%   $p\ \wedge\ q$     & $p\ \times\ q$            & $p\ and\ q$        & $the product of p and q$ \\ hline
%   $p\ \lor\ q$       & $p\ +\ q$                 & $p\ or\ q$         & $the coproduct of p and q$ \\ hline
%   $\neg\ p$          & $\neg\ p$                 & $it\ is\ not\ the\ case\ that\ p$ \\ hline  
%   $\top$             & $\top$                    & $true$             & $top$ \\ hline
%   $\bot$             & $\bot$                    & $false$            & $bottom$ \\ hline
%   $p\ =\ q$          & $p\ \equiv\ q$            & $p\ equals\ q$     & $definitionally equal$ \\ hline
% \end{tabular}
% \caption{FOL vs MLTT} \label{fig:M5}
% \end{figure}


% \begin{figure}
% \centering
% \begin{tabular}{|c|c|c|c|} \hline
%   FOL & MLTT & NL FOL & NL MLTT \\ \hline
%   $\forall\ x\ P(x)$ & $\Pi x : \tau.\ P(x)$     & $for\ all\ x,\ p$  & $the\  product\  over\  x\  in\ p$ \\ 
%   $\exists\ x\ P(x)$ & $\Sigma x : \tau.\ P(x)$  & $there\ exists\ an\ x\ such\ that\ p$ & $there\ exists\ an\ x\ in\ \tau such\ that\ p$ \\ 
%   $p\ \supset\ q$    & $p\ \to\ q$               & $if\ p\ then\ q$   & $p\  to\  q$ \\ 
%   $p\ \wedge\ q$     & $p\ \times\ q$            & $p\ and\ q$        & $the\  product\  of\  p\  and\  q$ \\ 
%   $p\ \lor\ q$       & $p\ +\ q$                 & $p\ or\ q$         & $the\  coproduct\  of\  p\  and\  q$ \\ 
%   $\neg\ p$          & $\neg\ p$                 & $it\ is\ not\ the\ case\ that\ p$ & $not\ p$ \\ 
%   $\top$             & $\top$                    & $true$             & $top$ \\ 
%   $\bot$             & $\bot$                    & $false$            & $bottom$ \\ 
%   $p\ =\ q$          & $p\ \equiv\ q$            & $p\ equals\ q$     & $definitionally\  equal$ \\ 
% \end{tabular}
% \caption{FOL vs MLTT} \label{fig:M5}
% \end{figure}


% \begin{multicols}{3}
%   \begin{itemize}
%     \item $\forall\ x\ P(x)$
%     \item $\exists\ x\ P(x)$
%     \item $p\ \supset\ q$
%     \item $p\ \wedge\ q$
%     \item $p\ \lor\ q$
%     \item $\neg\ p$
%     \item $\top$
%     \item $\bot$
%     \item $p\ =\ q$
%     \item $\Pi x : \tau.\ P(x)$
%     \item $\Sigma x : \tau.\ P(x)$
%     \item $p\ \to\ q$
%     \item $p\ \times\ q$
%     \item $p\ +\ q$
%     \item $\neg\ p$
%     \item $\top$
%     \item $\bot$
%     \item $p\ \equiv\ q$
%     \item $for\ all\ x,\ p$
%     \item $there\ exists\ an\ x\ such\ that\ p$
%     \item $if\ p\ then\ q$
%     \item $p\ and\ q$
%     \item $p\ or\ q$
%     \item $it\ is\ not\ the\ case\ that\ p$
%     \item $true$
%     \item $false$
%     \item $p\ equals\ q$
%   \end{itemize}
% \end{multicols}



Additionally, it is worth comparing the type theoretic and natural language
syntax with set theory. Now we bear witness to some deeper cracks than were
visible above. We note that the type theoretic syntax is \emph{the same} in both
tables, whereas the set theoretic and logical syntax shares no overlap. This is
because set theory and first order logic are treated as distinct domains,
whereas in vanilla MLTT, there is no distinguishing mathematical types from
logical types - everything is a type.

\begin{figure}[H]
\centering
\begin{tabular}{|c|c|c|c|} \hline
  FOL & MLTT & NL FOL & NL MLTT \\ \hline
  $\forall\ x\ P(x)$ & $\Pi x : \tau.\ P(x)$     & $for\ all\ x,\ p$  & $the\  product\  over\  x\  in\ p$ \\ 
  $\exists\ x\ P(x)$ & $\Sigma x : \tau.\ P(x)$  & $there\ exists\ an\ x\ such\ that\ p$ & $there\ exists\ an\ x\ in\ \tau such\ that\ p$ \\ 
  $p\ \supset\ q$    & $p\ \to\ q$               & $if\ p\ then\ q$   & $p\  to\  q$ \\ 
  $p\ \wedge\ q$     & $p\ \times\ q$            & $p\ and\ q$        & $the\  product\  of\  p\  and\  q$ \\ 
  $p\ \lor\ q$       & $p\ +\ q$                 & $p\ or\ q$         & $the\  coproduct\  of\  p\  and\  q$ \\ 
  $\neg\ p$          & $\neg\ p$                 & $it\ is\ not\ the\ case\ that\ p$ & $not\ p$ \\ 
  $\top$             & $\top$                    & $true$             & $top$ \\ 
  $\bot$             & $\bot$                    & $false$            & $bottom$ \\ 
  $p\ =\ q$          & $p\ \equiv\ q$            & $p\ equals\ q$     & $definitionally\  equal$ \\ \hline
\end{tabular}
\caption{FOL vs MLTT} \label{fig:M5}
\end{figure}


\begin{figure}[H]
\centering
\begin{tabular}{|c|c|c|c|} \hline
 Set Theory & MLTT & NL Set Theory & NL MLTT \\ \hline
 $S$          & $\tau$                 & $the\ set\ S$                     & $the\ type\ \tau$ \\ 
 $\mathbb{N}$ & $Nat$                  & $the\ set\ of\ natural\ numbers$  & $the\ type\ nat$ \\
 $S \times T$ & $S \times T$           & $the\ product\ of\ S\ and\ T$     & $the\  product\  of\  S\  and\  T$ \\
 $S \to T$    & $S \to T$              & $the\ function\ \from\ S\ to\ T$  & $p\  to\  q$ \\
 $\{x|P(x)\}$ & $\Sigma x : \tau.\ P(x)$ & $the\ set\ of\ x\ such\ that\ P$  & $there\ exists\ an\ x\ in\ \tau such\ that\ p$ \\
 $\emptyset$  & $\bot$                 & $the\ empty\ set$                 & $bottom$ \\
 $?$          & $\top$                 & $?$                             & $top$ \\
 $S \cup T$   & $?$                    & $the\ union\ of\ S\ and\ T$       & $?$ \\
 $S \subset T$ & $S <: T$              & $the\ subset\ S\ of\ T$          & $S\ is\ a\ subtype\ of\ T$ \\
 $?$          & $U_1$                  & $?$ & $the\ second\ Universe$        \\ \hline 
\end{tabular}
\caption{Sets vs MLTT} \label{fig:M6}
\end{figure}



% \begin{multicols}{3}
%   \begin{itemize}
%     \item $S$
%     \item $\mathbb{N}$
%     \item $S \times T$
%     \item $S \to T$
%     \item $\{x|P(x)\}$
%     \item $\emptyset$
%     \item $?$
%     \item $S \cup T$
%     \item $S \subset T$
%     \item $?$
%     \item $\tau$
%     \item $Nat$
%     \item $S \times T$
%     \item $S \to T$
%     \item $\Sigma x : \_ . P(x)$
%     \item $\bot$
%     \item $\top$
%     \item $?$
%     \item $S <: T$
%     \item $U_1$
%     \item $the\ set\ S$
%     \item $the\ set\ of\ natural\ numbers$
%     \item $the\ product\ of\ N\ and\ N$
%     \item $the\ function\ \from\ S\ to\ T$
%     \item $the\ set\ of\ x\ such\ that\ P$
%     \item $the\ empty\ set$
%     \item $top$
%     \item $the\ union\ of\ S\ and\ T$
%     \item $the\ subset\ S\ of\ T$
%     \item $the\ second\ Universe$
%   \end{itemize}
% \end{multicols}

The basic types are sometimes simpler to work with, because they are incredibly
expressive, but it also comes at a cost. The union of two sets simply gives a
predicate over the members of the sets, whereas union and intersection types are
often not considered ``core" to type theory, with multiple possible ways of
interpreting how to treat this set-theoretic concept. The behavior of subtypes
and subsets, while related in some ways, also represents a semantic departure
from sets and types. For example, while there can be a greatest type in some
sub-typing schema, there is no notion of a top set. This is why we use the type
theoretic NL syntax when there are question marks in the set theory column.

We also note that pragmatically, type theorists often interchange the logical,
set theoretic, and type theoretic lexicons when describing types. Because the
types were developed to overcome shortcomings of set theory and classical logic,
the lexicons of all three ended up being blended, and in some sense, the type
theorist can substitute certain words that a classical mathematician
wouldn't.  Whereas $p\ implies\ q$ and $function\ from\ X\ to\ Y$ are not to
be mixed, the type theorist may in some sense default to either.
Nontheless, pragmatically speaking, one would never catch a type theorist
saying $Nat implies Nat$ when expressing $Nat\ \to\ Nat$.


Continuing with sets, we compare elements with terms, this time, via examples.


\centering
\begin{tabular}{|c|c|c|c|} \hline
  Set Theory & MLTT & NL Set Theory & NL MLTT \\ \hline

\begin{multicols}{2}
  \begin{itemize}

  \end{itemize}
\end{multicols}

Mathemacians and T


% \begin{columns}

% \begin{column}{0.4 \textwidth}
% \begin{exampleblock}{Sets}
%   \begin{itemize}
%     \item $1$
%     \item $(1,0)$
%   \end{itemize}
% \end{exampleblock}
% \end{column}

% \begin{column}{0.4 \textwidth}
% \begin{block}{Programs}
%   \begin{itemize}
%     \item $suc\ zero$
%     \item $(suc\ zero, zero)$
%   \end{itemize}
% \end{block}

Nonetheless,
there are many nuances this side-by-side comparison doesn't offer. First  

While these differences may 



\subsection{Agda}

Agda is an attempt to faithfully formalize Martin-Löf's intensional type theory
\cite{ml1984}. It is a functionaly programming language which, through an
interactive environment, allows one to iteratively apply rules and develop
constructive mathematics. It's current incarnation, Agda2 (but just called
Agda), was preceded by ALF, Cayenne, and Alfa, and the Agda1. On top of the
basic MLTT, Agda incorporates dependent records, inductive definitions, pattern
matching, a versatile module system, and a myriad of other bells and whistles
which are of interest generally and in various states of development but not
relevant to this work.

For our purposes, we will only look at what can in some sense be seen as the
kernel of Agda. Developing a full-blown GF grammar to incorporate more
advanced Agda features would require efforts beyond the scope of this work.

Agda's purpose is to manifest the propositions-as-types paradigm in a practical
and useable programming language. And while there are still many reasons one may
wish to use other programming languages, or just pen and paper to do her work,
there is a sense of purity one gets when writing Agda code. There are many good
resources for learning Agda,
  


% \cite{dybjer}
% \cite{norrell}
% \cite{wadler}

so we'll only give a cursory overview of what is relevant for this thesis.


% \begin{code}[hide]%
\>[0]\<%
\\
\>[0]\AgdaKeyword{module}\AgdaSpace{}%
\AgdaModule{primitives}\AgdaSpace{}%
\AgdaKeyword{where}\<%
\\
\>[0]\<%
\end{code}

Formation rules, are given by the data declaration, followed by some number of
constructors which correspond to the 


A proof the proof-theoretic this looks like the following


\begin{prooftree}
  \hypo{ \Gamma, A &\vdash B }
  \infer1[abs]{ \Gamma &\vdash A\to B }
  \hypo{ \Gamma \vdash A }
  \infer2[app]{ \Gamma \vdash B }
\end{prooftree}


\begin{code}%
\>[0]\<%
\\
\>[0]\AgdaKeyword{data}\AgdaSpace{}%
\AgdaDatatype{𝔹}\AgdaSpace{}%
\AgdaSymbol{:}\AgdaSpace{}%
\AgdaPrimitive{Set}\AgdaSpace{}%
\AgdaKeyword{where}\<%
\\
\>[0][@{}l@{\AgdaIndent{0}}]%
\>[2]\AgdaInductiveConstructor{true}\AgdaSpace{}%
\AgdaSymbol{:}\AgdaSpace{}%
\AgdaDatatype{𝔹}\<%
\\
%
\>[2]\AgdaInductiveConstructor{false}\AgdaSpace{}%
\AgdaSymbol{:}\AgdaSpace{}%
\AgdaDatatype{𝔹}\<%
\\
\>[0]\<%
\end{code}


-- $ \frac{\Gamma, x : A \vdash b : B} {\Gamma \vdash \lambda x. b : A \rightarrow
-- B} $

\begin{code}%
\>[0]\<%
\\
\>[0]\AgdaComment{-- if\AgdaUnderscore{}then\AgdaUnderscore{}else\AgdaUnderscore{} : \{A : Set\} → 𝔹 → A → A → A}\<%
\\
\>[0]\AgdaComment{-- if true then a1 else a2 = a1}\<%
\\
\>[0]\AgdaComment{-- if false then a1 else a2 = a2}\<%
\\
\>[0]\<%
\end{code}

-- data ℕ : Type where
--   zero : ℕ
--   suc  : ℕ → ℕ

-- data List (A : Type) : Type where
  

-- data Vector : 



-- \begin{code}%
\>[0]\<%
\\
\>[0]\AgdaComment{-- Type : Set₁}\<%
\\
\>[0]\AgdaComment{-- Type = Set}\<%
\\
%
\\[\AgdaEmptyExtraSkip]%
\>[0]\AgdaComment{-- \textbackslash{}end\{code\}}\<%


Formation rules, are given by the data declaration, followed by some number of
constructors which correspond to the 
Introduction rules are 



% \section{Grammatical Framework}

\subsection{Thinking about GF}

A grammar specification in GF is actually just an abstract syntax. With an abstract
syntax specified, one can then define various linearization rules which
compositionally evaluate to strings. An Abstract Syntax Tree (AST) may then be
linearized to various strings admitted by different concrete syntaxes.
Conversely, given a string admitted by the language being defined, GF's powerful
parser will generate all the ASTs which linearize to that tree.

When defining a GF pipeline, one merely to construct an abstract syntax file and a
concrete syntax file such that they are coherent. In the abstract, one
specifies the \emph{semantics} of the domain one wants to translate over, which
is ironic, because we normally associate abstract syntax with \emph{just syntax}.
However, because GF was intended for implementing the natural language
phenomena, the types of semantic categories (or sorts) can grow much bigger than is
desirable in a programming language, where minimalism is generally favored.  The
\emph{foods grammar} is the \emph{hello world} of GF, and should be referred to
for those interested in example of how the abstract syntax serves as a semantic
space in non-formal NL applications \cite{ranta2011grammatical}.

Let us revisit the ``tetrahedral doctrine", now restricting our attention to the
subset of linguistics which GF occupies. We first examine how GF fits into the
trinity, as seen in  \autoref{fig:G1}. Immediately, GF abstract syntax with
dependent types can just be seen as an
implementation of MLTT with the added bonus of a parser.  Additionally, GF is a
relatively tame Type Theory, and therefore it would be easy to construct a model
in a general purpose programming language, like Agda.  Embeddings of GF already
exist in Coq [cite FraCoq], Haskell [cite pgf], and MMT [cite mmt],  These applications allow one to use GF's
parser so that a GF AST may be transformed into some kind of
inductively defined tree these languages all support. Future work could involve
modeling GF in Agda would allow one to prove things about GF
meta-theorems about soundness and termination, or perhaps statements about
specific grammars, such as one being unambiguous.

From the logical side, we note that GF's parser specification was done using
inference rules [cite krasimir]. Given the coupling of Context-Free Grammars
(CFGs) and operads (also known as multicategories) [cite lambek, etc], one could use much more
advanced mathematical machinery to articulate and understand GF.  We sketch this
briefly below [refer].

\begin{figure}[H]
\centering
\begin{tikzcd}
     &  &  & Logic                                                                                                                                             &  &  &            \\
     &  &  &                                                                                                                                                   &  &  &            \\
     &  &  & GF \arrow[uu, "GF\ Parser\ Specification"'] \arrow[llldd, "Theory\ of\ Operads"']
     \arrow[rrrdd, "Implementation\ of", bend left] \arrow[rrrdd, "Agda\ Embedding", bend right] &  &  &            \\
     &  &  &                                                                                                                                                   &  &  &            \\
Math &  &  &                                                                                                                                                   &  &  & CS\ (MLTT)
\end{tikzcd}
\caption{Models of GF} \label{fig:G1}
\end{figure}

One can additionally model these domains in GF, which is obviously the main
focus of this work. In \autoref{fig:G2}, we see that there are 3 grammars which
give allow one to translate in these domains. Ranta's grammar from CADE 2011,
built a propositional framework with a core grammar extended with other
categories to capture syntactic nuance. Ranta's grammar from the Stockholm University
mathematics seminar in 2014 took verbatim text from a publication of Peter Aczel
and sought to show that all the syntactic nuance by constructing a grammar
capable of NL translation. Finally, our work takes a BNFC grammar for a real
programming language cubicaltt [cite], GFifies it, producing an unambiguous
grammar.

\begin{figure}[H]
\centering
\begin{tikzcd}
                                              &  &  & Logic \arrow[dd, "Ranta\
                                              Logic\ (CADE\ 11)"] &  &  &                                       \\
                                              &  &  &                                          &  &  &                                       \\
                                              &  &  & GF                                       &  &  &                                       \\
                                              &  &  &                                          &  &  &                                       \\
Math \arrow[rrruu, "Ranta\ (HoTT\ 14)"] &  &  &                                          &  &  & CS\ (MLTT) \arrow[llluu, "cubicalTT"]
\end{tikzcd}
\caption{Trinitarian Grammars} \label{fig:G2}
\end{figure}

While these three grammars offer the most poignant points of comparison between
the computational, logical, and mathematical phenomena they attempt to capture,
we also note that there were many other smaller grammars developed during the
course of this work to supplement and experiment with various ideas presented.
Importantly, the ``Trinitarian Grammars" do not only model these different
domains, but they each do so in a unique way, making compromises and capturing
various linguistic and formal phenomena. The phenomena should be seen on a
spectrum of \emph{semantic adequacy} and \emph{syntactic completeness}, as in
autoref{fig:G3} .  


\begin{figure}[H]
\centering
\begin{tikzcd}
Lexicon\ Size                                                                                                                                          &  &  & Syntactic\ Completeness \\
                                                                                                                                                       &  &  & {}                      \\
                                                                                                                                                       &  &  &                         \\
Spectrum\ of\ GF \arrow[uuu, "Statistical\ Methods?"] \arrow[rrr, "Ranta\ HoTT\ '14"'] \arrow[rrruuu, "cubicalTT"] \arrow[rrruu, "Ranta\ Logic\ '14"'] &  &  & Semantic\ Adequacy
\end{tikzcd}
\caption{The Grammatical Dimension} \label{fig:G3}
\end{figure}

The cubicalTT grammar, seeking syntacitic completeness, only has a pidgin
English syntax, and therefore is only capable of parsing a programming language.
Ranta's HoTT grammar on the other hand, while capable of presenting a
quasi-logical form, would require extensive refactoring in order to transform
the ASTs to something that resembles the ASTs of a programming language. The
Logic grammar, which produces logically coherent and linguistically nuanced
expressions, does not yet cover proofs, and therefore would require additional extensions
to actually express anything a computer might understand, or, alternatively,
theorems capable of impressing a mathematician. Finally, we note that large-scale
coverage of linguistic phenomena for any of these grammars will additionally
need to incorporate statistical methods in some way. 

Before providing perspectives on the grammar design process, it is alo 
When designing grammars, the foremost question one should ask
A few remarks on designing GF grammars should be noted as well. 

The PMCFG class of languages is still quite tame when compared with, for
instance, Turing complete languages. Thus, the `abstract` and `concrete`
coupling tight, the evaluation is quite simple, and the programs tend to write
themselves once the correct types are chosen. This is not to say GF programming
is easier than in other languages, because often there are unforeseen
constraints that the programmer must get used to, limiting the palette available
when writing code. These constraints allow for fast parsing, but greatly limit
the types of programs one often thinks of writing.

\subsection{A Brief Introduction to GF}

GF is a very powerful yet simple system.  While learning the basics may not be
to difficult for the experienced programmer, GF requires the programmer to work
with, in some sense, an incredibly stiff set of constraints compared to general
purpose languages, and therefore its lack of expressiveness requires a different
way of thinking about programming.

The two functions displayed in \autoref{fig:N2}, $Parse : \{Strings\}
\rightarrow \{\{ASTs\}\}$ and $Linearize : \{ASTs\} \rightarrow \{Strings\}$, obey
the important property that :

 $$\forall s \in \{Strings\} \forall x \in (Parse(s)), Linearize(x) \equiv s$$

This seems somewhat natural from the programmers perspective. The limitation on
ASTs to linearize uniquely is actually a benefit, because it saves the user
having to make a choice about a translation (although, again, a statistical
mechanism could alleviate this constraint). We also want our translations to be
well-behaved mathematically, i.e. composing $Linearize$ and $Parse$ ad
infinitum should presumably not diverge.

GF captures languages more expressive than Chomsky's original 
CFG [cite] but is still remains decidable, with parsing in polynomial
time. Which polynomial depends on the grammar [cite krasimir]. 
It comes equipped with 6 basic judgments:

\begin{itemize}[noitemsep]
  \item Abstract : `cat, fun`
  \item Concrete : `lincat, lin, param`
  \item Auxiliary : `oper`
\end{itemize}

There are two judgments in an abstract file, for categories and named functions
defined over those categories, namely \term{cat} and \term{fun}. The categories
are just (succinct) names, and while GF allows dependent types, e.g. categories
which are parameterized over other categories and thereby allow for more
fine-grained semantic distinctions. We will leave these details aside, but do
note that GF's dependent types can be used to implement a programming language
which only parses well-typed terms (and can actually compute with them using
auxiliary declarations).

In a simply typed programming language we can choose categories, for
variables, types and expressions, or what might \term{Var}, \term{Typ}, and
\term{Exp} respectively. One can then define the functions for the simply typed
lambda calculus extended with natural numbers, known as Gödel's T.

\begin{verbatim} 
cat
  Typ ; Exp ; Var ;
fun
  Tarr : Typ -> Typ -> Typ ;
  Tnat : Typ ;

  Evar : Var -> Exp ;
  Elam : Var -> Typ -> Exp -> Exp ;
  Eapp : Exp -> Exp -> Exp ;

  Ezer : Exp ;
  Esuc : Exp -> Exp ;
  Enatrec : Exp -> Exp -> Exp ->  Exp ;

  X : Var ;
  Y : Var ;
  F : Var ;
  IntV : Int -> Var ;
\end{verbatim}

So far we have specified how to form expressions : types built out of
possibly higher order functions
between natural numbers, and expressions built out of lambda and
natural number terms. The variables are kept as a separate syntactic category,
and integers, \term{Int}, are predefined via GF's internals and simply allow one
to parse numeric expressions. One may then define a functional which takes a
function over the natural numbers and returns that function applied to $1$ - the
AST for this expression is :

\begin{verbatim} 
Elam
    F
    Tarr
        Tnat Tnat
      Eapp
        Evar
            F
        Evar
            IntV
                1
\end{verbatim} 

Dual to the abstract syntax there are parallel judgments when defining a concrete
syntax in GF, \term{lincat} and \term{lin} corresponding to \term{cat} and
\term{fun}, respectively. Wher the AST is the specification, the concrete
form is its implementation in a given lanaguage. The \term{lincat} serves to
give \emph{linearization types} which are quite simply either strings, records (or products
which support sub-typing and named fields), or tables (or coproducts) which can
make choices when computing with arbitrarily named parameters, which are
naturally isomorphic to the sets of some finite cardinality. The tables are
actually derivable from the records and their projections, which is how PGF is
defined internally, but they are so fundamental to GF programming and
expressiveness that they merit syntactic distincion.  The \term{lin}
is a term which matches the type signature of the \term{fun} with which it
shares a name. The \term{lincat} constrains the concrete types of the arguments,
and therefore subjects the GF user to how they are used. 

If we assume we are just working with strings, then we can simply define the
functions as recursively concatenating \term{++} strings. The lambda function
for pidgin English then has, as its linearization form as follows :

\begin{verbatim}
lin 
  Elam v t e = "function taking" ++ v ++ "in" ++ t ++ "to" ++ e ;
\end{verbatim}

Once all the relevant functions are giving correct linearizations, one can now
parse and linearize to the abstract syntax tree above the to string ``function
taking f in the natural numbers to the natural numbers to apply f to 1". This is
clearly unnatural for a variety of reasons, but it's an approximation of what
a computer scientist might say. Suppose instead, we choose to linearize this same
expression to a pidgin expression modeled off Haskell's syntax, ``\\ ( f
: nat -> nat ) -> f 1". We should notice the absence of parentheses for
application suggest something more subtle is happening with the linearization
process, for normally programming languages use fixity declarations to avoid
lispy looking code. Here are the linearization functions which allow for
linearization from the above AST :

\begin{verbatim}
lincat
  Typ = TermPrec ;
  Exp = TermPrec ;
lin
  Elam v t e = 
    mkPrec 0 ("\\" ++ parenth (v ++ ":" ++ usePrec 0 t) ++ "->" ++ usePrec 0 e) ;
  Eapp = infixl 2 "" ;
\end{verbatim}

Where did \term{TermPrec}, \term{infixl}, \term{parenth}, \term{mkPrec}, and
\term{usePrec} come from? These are all functions defined in the RGL. We show a
few of them below, thereby introducing the final, main GF judgments \term{param}
and \term{oper} for parameters and operators.

\begin{verbatim}
param 
  Bool = True | False ;
oper
  TermPrec : Type = {s : Str ; p : Prec} ;
  usePrec : Prec -> TermPrec -> Str = \p,x ->
    case lessPrec x.p p of {
      True => parenth x.s ;
      False => parenthOpt x.s
    } ;
  parenth : Str -> Str = \s -> "(" ++ s ++ ")" ;
  parenthOpt : Str -> Str = \s -> variants {s ; "(" ++ s ++ ")"} ;
\end{verbatim}

Parameters in GF, to a first approximation, are simply data types of unary
constructors with finite cardinality. Operators, on the other hand, encode the
logic of GF linearization rules. They are an unnecessary part of the language
because they don't introduce new logical content, but they do allow one to
abstract the function bodies of \term{lin}'s so that one may keep the actual
linearization rules looking clean. Since GF also support \term{oper}
overloading, one can often get away with often deceptively sleek looking
linearizations, and this is a key feature of the RGL. The variants is one of the
ways to encode multiple linearizations forms for a given tree, so here, for
example, we're breaking the nice property from above.

This more or less resembles a typical programming language, with very little
deviation from what when would expect specifying something in twelf.
Nonetheless, because this is both meant to somehow capture the logical form in
addition to the surface appearance of a language, the separation of concerns
leaves the user with an important decision to make regarding how one couples the
linear and abstract syntaxes. There are in some sense two extremes one can take
to get a well performing GF grammar.

Suppose you have a page of text from some random source of length $l$, and you
take it as an exercise to build a GF grammar which translates it. The first
extreme approach you could take would be to give each word in the text to a
unique category, a unique function for each category bearing the word's name,
along with a single really function with $l$ arguments for the whole sequence of
words in the text. One could then verbatim copy the words as just strings with
their corresponding names in the concrete syntax. This overfitted grammar would
fail : it wouldn't scale to other languages, wouldn't cover any texts other than
the one given to it, and wouldn't be at all informative. Alternatively, one
could create a grammar of a two categories $c$ and $s$ with two functions, $f_0
: c$ and $f_1 : c \rightarrow s$, whereby c would be given $n$ fields, each
strings, with the string given at position $i$ in $f_0$ matching $word_i$ from
the text. $f_1$ would merely concatenate it all. This grammar would be similarly
degenerate, despite also parsing the page of text.

This seemingly silly example highlights the most blatant tension the GF grammar
writer will face : how to balance syntactic and semantic content of the grammar
in between the concrete and the abstract syntax. It is also highly
relevant as concerns the domain of translation, for a programming language
with minimal syntax and the mathematicians language in expressing her ideas are
on vastly different sides of this issue.

We claim that syntactically complete grammars are much more easily dealt with
simple abstract syntax. However, to take allow a syntactically complete grammar
to capture semantic nuance and neutrality then humans requires immensely more
work on the concrete side. Semantically adequate grammars on the other hand,
require significantly more attention on the abstract side, because semantically
meaningful expressions often don't generalize - each part of an expressions
exhibits unique behaviors which can't be abstracted to apply to other parts of
the expression. Therefore, producing a syntactically complete expressions which
doesn't overgenerate parses also requires a lot work from the grammar writer.

We hope the subsequent examples will illuminate this tension. The problem with
treating a syntactically oriented domain like type theory with and a semantically
oriented one like mathematics with the same abstract syntax poses very serious
problems, but also highlights the power of other features of GF, like the RGL [cite]
and Haskell embedding PGF [cite].

The GF RGL is a very robust library for parsing grammatically coherent language.
It exists for many different natural languages with a core abstract syntax
shared by all of them. The API allows one to easily construct, sentence level
phrases once the lexicon has been defined, which are also greatly facilitated by
the API. 

PGF, is an embedding of a GF abstract syntax into Haskell, where the categories
are given ``shadow types", so that one can build turn an abstract syntax into (a
possibly massive) Generalized Algebraic Data Type (GADT) \term{Tree} with kind
\codeword{* -> *} where all the functions serve as constructors. If function
\codeword{h} returns category \codeword{c}, the Haskell constructor
\codeword{Gh} returns \codeword{Tree c}.

The PGF API also allows for the Haskell user to call the parse and linearization
functions, so that once the grammar is built, one can use Haskell as an
interface with the outside world. While GF originally was conceived as allowing
computation with ASTs, using a semantic computation judgment \term{def}, this
has approach has largely been overshadowed by Haskell. Once a grammar is
embedded in Haskell, one can use general recursion, monads, and all other types
of bells and whistles produced by the functional programming community to
compute with the embedded ASTs.

We note that this further muddies the water of
what syntax and semantics refer to in the GF lexicon. Although a GF
abstract syntax somehow represents the programmers idealized semantic domain,
once embedded in PGF the trees now may represent syntactic objects to be
evaluated or transformed to some other semantic domain which may or may not
eventually be linked back to a GF linearization.

These are all the main ingredients a GF user will hopefully need to understand
the grammars hereby elaborated, and hopefully these examples will showcase the
full potential of GF for the problem of mathematical translations.

% \subsection{Mathematical Model of GF}
% Note on the construction of free monoids

% Consider a language $L$ we want to represent, and we come up with a model that we
% build as a set of categories and functions over those categories.  Let $Cat(L)$,
% denote the categories.  Also suppose we define functions such that, given an
% ordered list $x_1,...,x_n;y \in Cat(L)$ we define a set of functions,
% $Fun_L(x_1,...,x_n;y)$ defined over the categories. In gf, a function can be
% denoted something like $\phi : x_1 \rightarrow ... \rightarrow x_n$. We may compose these based
% off their arities. So, given a function $\psi \in Fun_L(y_1,,...,y_n;z)$,
% functions $\phi_1,...\phi_n$ such that $\phi_i \in Fun_L(x_{i,1},...,x_{i,m};y_i)}$ 
%  we can plug these functions in together, or nest them such that
% $$\psi \circ (\phi_1,...,\phi_n) : \rightarrow (x_{i,j}) \rightarrow (y_{i})
% \rightarrow Z$$ 

% This is how abstract syntax trees are formed. It is also worth noting that they
% obey an associativity property, namely that 

% \begin{align*}
% &\theta \circ (\psi_1 \circ (\phi_{1,1},...,\phi_{1,k_1}),...,\psi_n \circ
% (\phi_{n,1},...,\phi_{n,k_n}))\\ = &(\theta \circ \psi_1,...\psi_n) \circ (\phi_{1,1},...,\phi_{1,k_1},...,\phi_{n,1},...,\phi_{n,k_n})
% \end{align*}

% This means that trees in GF are invariant as to how they are built - we
% can build a tree from the leaves to the root or vice versa.

% Example : consider the arithmetic grammar of exponentiation, multiplication, and
% addition defined over a single category of natural number expressions, whereby
% the function symbol is to be interpreted as a string and the tensor product,
% $\otimes$ as the concatenation during evaluation. 

% $$\_\^{}\_ : \mathds{N} \to \mathds{N} \to \mathds{N}$$
% $$\_*\_ : \mathds{N} \to \mathds{N} \to \mathds{N}$$
% $$\_+\_ : \mathds{N} \to \mathds{N} \to \mathds{N}$$

% We can think of constructing the trees by partial application, i.e., 

% $(\lambda x.\: 2 \otimes \^{} \otimes x) : \mathds{N} -> \mathds{N}$

% Lets try see the constructions yielding the string $(1 + 2) \^{} (3 * 4)$.

% We can either (i) construct this as the exponent of two fully formed expressions,
% namely a sum and a product applied to some numbers, or we can first apply the
% exponent to the two binary functions, yielding a quaternary function .

% $x ++ y$
% $x \doubleplus y$
% $``x \doubleplus y"$

% \begin{align*}
% &(\lambda x,y.\: x \otimes \^{} \otimes y)\\
% &\hspace{1cm} ((\lambda x,y.\:x \otimes + \otimes y)\; 1\; 2)\\
% &\hspace{1cm} ((\lambda x,y.\: x \otimes * \otimes y)\; 3\; 4) \\
% \mapsto\; &(\lambda x,y.\: x \otimes \^{} \otimes y)\\
% &\hspace{1cm} (1 + 2)\\
% &\hspace{1cm} (3 * 4))\\
% \mapsto\; &((1 + 2) \^{} (3 * 4))\\
% \end{align}

% \begin{align*}
% &((\lambda x,y.\: x \otimes \^{} \otimes y)\\
% &\hspace{1cm} (\lambda x,y.\:x \otimes + \otimes y)\\
% &\hspace{1cm} (\lambda x,y.\: x \otimes * \otimes y)) \\
% &\hspace{1cm} 1\; 2; 3; 4; \\
% \end{align}

% (1 + 2) \^{} (3 * 4)
  

% ((\lambda x,y. x \^{} y)
%   (\lambda x,y. x + y) 
%   (\lambda x,y. x * y))
%     1 2 3 4

% ((\lambda x,y. x + y) \^{} (\lambda x,y. x * y)) 1 2 3 4
% ((\lambda x,y. x + y) \^{} (\lambda x,y. x * y)) 1 2 3 4

% (1 + 2) \^{} (3 * 4)

% and then say
% (\lambda x. 2 \^{} x) (1 + 3) * (4 + 5)
% = 
% (\lambda x. 2 \^{} x) (1 + 3) * (4 + 5)

% $(\lambda x. 2 \wedge x) : \mathds{N} -> \mathds{N}$

% and then apply it to a complex arguement, say 
% (1 + 3) * (4 + 5)
% (\lambda x. 2 ^ x) : Nat -> Nat

% where 


% \lambda y : Pow y 1 : Nat -> Nat

% (times (plus 2 3) (plus 4 5))
% (Pow \circ (1,times)) : Nat -> Nat -> Nat

% (plus 2 3) (plus 4 5)

% can either be 

% 2^(1+3)*(4+5)


%   % \sin {:} \mathbb{R} &\rightarrow \mathbb{R}\\ x &\mapsto \sin ( x )
% % \circ (\phi_1,...,\phi_n) : \rightarrow (x_{i,j}) \rightarrow (y_{i})
% % \rightarrow Z$$ 

% The two functions displayed in, \autoref{fig:N2}.  If we can loosely call String
% the set of strings freely generated osome acan be 

% for now given a single linear presentation $C^{AST}$ , where

% AST_L String_L0 denote the sets GF ASTs and Strings in the languages generated
% by the rules of L's abstract syntax and L0s compositional representation.

% $$Parse : String -> {AST}$$
% $$Linearize : AST -> String$$

% with the important property that given a string s,


% And given an AST a, we can Parse . Linearize a belongs to {AST}

% Now we should explore why the linearizations are interesting. In part, this is
% because they have arisen from the role of grammars have played in the
% intersection and interaction between computer science and linguistics at least
% since Chomsky in the 50s, and they have different understandings and utilities
% in the respective disciplines. These two discplines converge in GF, which allows
% us to talk about natural languages (NLs) from programming languages (PLs)
% perspective.




\subsection{Natural Language and Mathematics}




\section{Previous Work}

The prior exploration of these interleaving subjects is vast, and we can only
sample the available literature here.  

\section{Grammatical Framework}

\subsection{Thinking about GF}

A grammar specification in GF is actually just an abstract syntax. With an abstract
syntax specified, one can then define various linearization rules which
compositionally evaluate to strings. An Abstract Syntax Tree (AST) may then be
linearized to various strings admitted by different concrete syntaxes.
Conversely, given a string admitted by the language being defined, GF's powerful
parser will generate all the ASTs which linearize to that tree.

When defining a GF pipeline, one merely to construct an abstract syntax file and a
concrete syntax file such that they are coherent. In the abstract, one
specifies the \emph{semantics} of the domain one wants to translate over, which
is ironic, because we normally associate abstract syntax with \emph{just syntax}.
However, because GF was intended for implementing the natural language
phenomena, the types of semantic categories (or sorts) can grow much bigger than is
desirable in a programming language, where minimalism is generally favored.  The
\emph{foods grammar} is the \emph{hello world} of GF, and should be referred to
for those interested in example of how the abstract syntax serves as a semantic
space in non-formal NL applications \cite{ranta2011grammatical}.

Let us revisit the ``tetrahedral doctrine", now restricting our attention to the
subset of linguistics which GF occupies. We first examine how GF fits into the
trinity, as seen in  \autoref{fig:G1}. Immediately, GF abstract syntax with
dependent types can just be seen as an
implementation of MLTT with the added bonus of a parser.  Additionally, GF is a
relatively tame Type Theory, and therefore it would be easy to construct a model
in a general purpose programming language, like Agda.  Embeddings of GF already
exist in Coq [cite FraCoq], Haskell [cite pgf], and MMT [cite mmt],  These applications allow one to use GF's
parser so that a GF AST may be transformed into some kind of
inductively defined tree these languages all support. Future work could involve
modeling GF in Agda would allow one to prove things about GF
meta-theorems about soundness and termination, or perhaps statements about
specific grammars, such as one being unambiguous.

From the logical side, we note that GF's parser specification was done using
inference rules [cite krasimir]. Given the coupling of Context-Free Grammars
(CFGs) and operads (also known as multicategories) [cite lambek, etc], one could use much more
advanced mathematical machinery to articulate and understand GF.  We sketch this
briefly below [refer].

\begin{figure}[H]
\centering
\begin{tikzcd}
     &  &  & Logic                                                                                                                                             &  &  &            \\
     &  &  &                                                                                                                                                   &  &  &            \\
     &  &  & GF \arrow[uu, "GF\ Parser\ Specification"'] \arrow[llldd, "Theory\ of\ Operads"']
     \arrow[rrrdd, "Implementation\ of", bend left] \arrow[rrrdd, "Agda\ Embedding", bend right] &  &  &            \\
     &  &  &                                                                                                                                                   &  &  &            \\
Math &  &  &                                                                                                                                                   &  &  & CS\ (MLTT)
\end{tikzcd}
\caption{Models of GF} \label{fig:G1}
\end{figure}

One can additionally model these domains in GF, which is obviously the main
focus of this work. In \autoref{fig:G2}, we see that there are 3 grammars which
give allow one to translate in these domains. Ranta's grammar from CADE 2011,
built a propositional framework with a core grammar extended with other
categories to capture syntactic nuance. Ranta's grammar from the Stockholm University
mathematics seminar in 2014 took verbatim text from a publication of Peter Aczel
and sought to show that all the syntactic nuance by constructing a grammar
capable of NL translation. Finally, our work takes a BNFC grammar for a real
programming language cubicaltt [cite], GFifies it, producing an unambiguous
grammar.

\begin{figure}[H]
\centering
\begin{tikzcd}
                                              &  &  & Logic \arrow[dd, "Ranta\
                                              Logic\ (CADE\ 11)"] &  &  &                                       \\
                                              &  &  &                                          &  &  &                                       \\
                                              &  &  & GF                                       &  &  &                                       \\
                                              &  &  &                                          &  &  &                                       \\
Math \arrow[rrruu, "Ranta\ (HoTT\ 14)"] &  &  &                                          &  &  & CS\ (MLTT) \arrow[llluu, "cubicalTT"]
\end{tikzcd}
\caption{Trinitarian Grammars} \label{fig:G2}
\end{figure}

While these three grammars offer the most poignant points of comparison between
the computational, logical, and mathematical phenomena they attempt to capture,
we also note that there were many other smaller grammars developed during the
course of this work to supplement and experiment with various ideas presented.
Importantly, the ``Trinitarian Grammars" do not only model these different
domains, but they each do so in a unique way, making compromises and capturing
various linguistic and formal phenomena. The phenomena should be seen on a
spectrum of \emph{semantic adequacy} and \emph{syntactic completeness}, as in
autoref{fig:G3} .  


\begin{figure}[H]
\centering
\begin{tikzcd}
Lexicon\ Size                                                                                                                                          &  &  & Syntactic\ Completeness \\
                                                                                                                                                       &  &  & {}                      \\
                                                                                                                                                       &  &  &                         \\
Spectrum\ of\ GF \arrow[uuu, "Statistical\ Methods?"] \arrow[rrr, "Ranta\ HoTT\ '14"'] \arrow[rrruuu, "cubicalTT"] \arrow[rrruu, "Ranta\ Logic\ '14"'] &  &  & Semantic\ Adequacy
\end{tikzcd}
\caption{The Grammatical Dimension} \label{fig:G3}
\end{figure}

The cubicalTT grammar, seeking syntacitic completeness, only has a pidgin
English syntax, and therefore is only capable of parsing a programming language.
Ranta's HoTT grammar on the other hand, while capable of presenting a
quasi-logical form, would require extensive refactoring in order to transform
the ASTs to something that resembles the ASTs of a programming language. The
Logic grammar, which produces logically coherent and linguistically nuanced
expressions, does not yet cover proofs, and therefore would require additional extensions
to actually express anything a computer might understand, or, alternatively,
theorems capable of impressing a mathematician. Finally, we note that large-scale
coverage of linguistic phenomena for any of these grammars will additionally
need to incorporate statistical methods in some way. 

Before providing perspectives on the grammar design process, it is alo 
When designing grammars, the foremost question one should ask
A few remarks on designing GF grammars should be noted as well. 

The PMCFG class of languages is still quite tame when compared with, for
instance, Turing complete languages. Thus, the `abstract` and `concrete`
coupling tight, the evaluation is quite simple, and the programs tend to write
themselves once the correct types are chosen. This is not to say GF programming
is easier than in other languages, because often there are unforeseen
constraints that the programmer must get used to, limiting the palette available
when writing code. These constraints allow for fast parsing, but greatly limit
the types of programs one often thinks of writing.

\subsection{A Brief Introduction to GF}

GF is a very powerful yet simple system.  While learning the basics may not be
to difficult for the experienced programmer, GF requires the programmer to work
with, in some sense, an incredibly stiff set of constraints compared to general
purpose languages, and therefore its lack of expressiveness requires a different
way of thinking about programming.

The two functions displayed in \autoref{fig:N2}, $Parse : \{Strings\}
\rightarrow \{\{ASTs\}\}$ and $Linearize : \{ASTs\} \rightarrow \{Strings\}$, obey
the important property that :

 $$\forall s \in \{Strings\} \forall x \in (Parse(s)), Linearize(x) \equiv s$$

This seems somewhat natural from the programmers perspective. The limitation on
ASTs to linearize uniquely is actually a benefit, because it saves the user
having to make a choice about a translation (although, again, a statistical
mechanism could alleviate this constraint). We also want our translations to be
well-behaved mathematically, i.e. composing $Linearize$ and $Parse$ ad
infinitum should presumably not diverge.

GF captures languages more expressive than Chomsky's original 
CFG [cite] but is still remains decidable, with parsing in polynomial
time. Which polynomial depends on the grammar [cite krasimir]. 
It comes equipped with 6 basic judgments:

\begin{itemize}[noitemsep]
  \item Abstract : `cat, fun`
  \item Concrete : `lincat, lin, param`
  \item Auxiliary : `oper`
\end{itemize}

There are two judgments in an abstract file, for categories and named functions
defined over those categories, namely \term{cat} and \term{fun}. The categories
are just (succinct) names, and while GF allows dependent types, e.g. categories
which are parameterized over other categories and thereby allow for more
fine-grained semantic distinctions. We will leave these details aside, but do
note that GF's dependent types can be used to implement a programming language
which only parses well-typed terms (and can actually compute with them using
auxiliary declarations).

In a simply typed programming language we can choose categories, for
variables, types and expressions, or what might \term{Var}, \term{Typ}, and
\term{Exp} respectively. One can then define the functions for the simply typed
lambda calculus extended with natural numbers, known as Gödel's T.

\begin{verbatim} 
cat
  Typ ; Exp ; Var ;
fun
  Tarr : Typ -> Typ -> Typ ;
  Tnat : Typ ;

  Evar : Var -> Exp ;
  Elam : Var -> Typ -> Exp -> Exp ;
  Eapp : Exp -> Exp -> Exp ;

  Ezer : Exp ;
  Esuc : Exp -> Exp ;
  Enatrec : Exp -> Exp -> Exp ->  Exp ;

  X : Var ;
  Y : Var ;
  F : Var ;
  IntV : Int -> Var ;
\end{verbatim}

So far we have specified how to form expressions : types built out of
possibly higher order functions
between natural numbers, and expressions built out of lambda and
natural number terms. The variables are kept as a separate syntactic category,
and integers, \term{Int}, are predefined via GF's internals and simply allow one
to parse numeric expressions. One may then define a functional which takes a
function over the natural numbers and returns that function applied to $1$ - the
AST for this expression is :

\begin{verbatim} 
Elam
    F
    Tarr
        Tnat Tnat
      Eapp
        Evar
            F
        Evar
            IntV
                1
\end{verbatim} 

Dual to the abstract syntax there are parallel judgments when defining a concrete
syntax in GF, \term{lincat} and \term{lin} corresponding to \term{cat} and
\term{fun}, respectively. Wher the AST is the specification, the concrete
form is its implementation in a given lanaguage. The \term{lincat} serves to
give \emph{linearization types} which are quite simply either strings, records (or products
which support sub-typing and named fields), or tables (or coproducts) which can
make choices when computing with arbitrarily named parameters, which are
naturally isomorphic to the sets of some finite cardinality. The tables are
actually derivable from the records and their projections, which is how PGF is
defined internally, but they are so fundamental to GF programming and
expressiveness that they merit syntactic distincion.  The \term{lin}
is a term which matches the type signature of the \term{fun} with which it
shares a name. The \term{lincat} constrains the concrete types of the arguments,
and therefore subjects the GF user to how they are used. 

If we assume we are just working with strings, then we can simply define the
functions as recursively concatenating \term{++} strings. The lambda function
for pidgin English then has, as its linearization form as follows :

\begin{verbatim}
lin 
  Elam v t e = "function taking" ++ v ++ "in" ++ t ++ "to" ++ e ;
\end{verbatim}

Once all the relevant functions are giving correct linearizations, one can now
parse and linearize to the abstract syntax tree above the to string ``function
taking f in the natural numbers to the natural numbers to apply f to 1". This is
clearly unnatural for a variety of reasons, but it's an approximation of what
a computer scientist might say. Suppose instead, we choose to linearize this same
expression to a pidgin expression modeled off Haskell's syntax, ``\\ ( f
: nat -> nat ) -> f 1". We should notice the absence of parentheses for
application suggest something more subtle is happening with the linearization
process, for normally programming languages use fixity declarations to avoid
lispy looking code. Here are the linearization functions which allow for
linearization from the above AST :

\begin{verbatim}
lincat
  Typ = TermPrec ;
  Exp = TermPrec ;
lin
  Elam v t e = 
    mkPrec 0 ("\\" ++ parenth (v ++ ":" ++ usePrec 0 t) ++ "->" ++ usePrec 0 e) ;
  Eapp = infixl 2 "" ;
\end{verbatim}

Where did \term{TermPrec}, \term{infixl}, \term{parenth}, \term{mkPrec}, and
\term{usePrec} come from? These are all functions defined in the RGL. We show a
few of them below, thereby introducing the final, main GF judgments \term{param}
and \term{oper} for parameters and operators.

\begin{verbatim}
param 
  Bool = True | False ;
oper
  TermPrec : Type = {s : Str ; p : Prec} ;
  usePrec : Prec -> TermPrec -> Str = \p,x ->
    case lessPrec x.p p of {
      True => parenth x.s ;
      False => parenthOpt x.s
    } ;
  parenth : Str -> Str = \s -> "(" ++ s ++ ")" ;
  parenthOpt : Str -> Str = \s -> variants {s ; "(" ++ s ++ ")"} ;
\end{verbatim}

Parameters in GF, to a first approximation, are simply data types of unary
constructors with finite cardinality. Operators, on the other hand, encode the
logic of GF linearization rules. They are an unnecessary part of the language
because they don't introduce new logical content, but they do allow one to
abstract the function bodies of \term{lin}'s so that one may keep the actual
linearization rules looking clean. Since GF also support \term{oper}
overloading, one can often get away with often deceptively sleek looking
linearizations, and this is a key feature of the RGL. The variants is one of the
ways to encode multiple linearizations forms for a given tree, so here, for
example, we're breaking the nice property from above.

This more or less resembles a typical programming language, with very little
deviation from what when would expect specifying something in twelf.
Nonetheless, because this is both meant to somehow capture the logical form in
addition to the surface appearance of a language, the separation of concerns
leaves the user with an important decision to make regarding how one couples the
linear and abstract syntaxes. There are in some sense two extremes one can take
to get a well performing GF grammar.

Suppose you have a page of text from some random source of length $l$, and you
take it as an exercise to build a GF grammar which translates it. The first
extreme approach you could take would be to give each word in the text to a
unique category, a unique function for each category bearing the word's name,
along with a single really function with $l$ arguments for the whole sequence of
words in the text. One could then verbatim copy the words as just strings with
their corresponding names in the concrete syntax. This overfitted grammar would
fail : it wouldn't scale to other languages, wouldn't cover any texts other than
the one given to it, and wouldn't be at all informative. Alternatively, one
could create a grammar of a two categories $c$ and $s$ with two functions, $f_0
: c$ and $f_1 : c \rightarrow s$, whereby c would be given $n$ fields, each
strings, with the string given at position $i$ in $f_0$ matching $word_i$ from
the text. $f_1$ would merely concatenate it all. This grammar would be similarly
degenerate, despite also parsing the page of text.

This seemingly silly example highlights the most blatant tension the GF grammar
writer will face : how to balance syntactic and semantic content of the grammar
in between the concrete and the abstract syntax. It is also highly
relevant as concerns the domain of translation, for a programming language
with minimal syntax and the mathematicians language in expressing her ideas are
on vastly different sides of this issue.

We claim that syntactically complete grammars are much more easily dealt with
simple abstract syntax. However, to take allow a syntactically complete grammar
to capture semantic nuance and neutrality then humans requires immensely more
work on the concrete side. Semantically adequate grammars on the other hand,
require significantly more attention on the abstract side, because semantically
meaningful expressions often don't generalize - each part of an expressions
exhibits unique behaviors which can't be abstracted to apply to other parts of
the expression. Therefore, producing a syntactically complete expressions which
doesn't overgenerate parses also requires a lot work from the grammar writer.

We hope the subsequent examples will illuminate this tension. The problem with
treating a syntactically oriented domain like type theory with and a semantically
oriented one like mathematics with the same abstract syntax poses very serious
problems, but also highlights the power of other features of GF, like the RGL [cite]
and Haskell embedding PGF [cite].

The GF RGL is a very robust library for parsing grammatically coherent language.
It exists for many different natural languages with a core abstract syntax
shared by all of them. The API allows one to easily construct, sentence level
phrases once the lexicon has been defined, which are also greatly facilitated by
the API. 

PGF, is an embedding of a GF abstract syntax into Haskell, where the categories
are given ``shadow types", so that one can build turn an abstract syntax into (a
possibly massive) Generalized Algebraic Data Type (GADT) \term{Tree} with kind
\codeword{* -> *} where all the functions serve as constructors. If function
\codeword{h} returns category \codeword{c}, the Haskell constructor
\codeword{Gh} returns \codeword{Tree c}.

The PGF API also allows for the Haskell user to call the parse and linearization
functions, so that once the grammar is built, one can use Haskell as an
interface with the outside world. While GF originally was conceived as allowing
computation with ASTs, using a semantic computation judgment \term{def}, this
has approach has largely been overshadowed by Haskell. Once a grammar is
embedded in Haskell, one can use general recursion, monads, and all other types
of bells and whistles produced by the functional programming community to
compute with the embedded ASTs.

We note that this further muddies the water of
what syntax and semantics refer to in the GF lexicon. Although a GF
abstract syntax somehow represents the programmers idealized semantic domain,
once embedded in PGF the trees now may represent syntactic objects to be
evaluated or transformed to some other semantic domain which may or may not
eventually be linked back to a GF linearization.

These are all the main ingredients a GF user will hopefully need to understand
the grammars hereby elaborated, and hopefully these examples will showcase the
full potential of GF for the problem of mathematical translations.

% \subsection{Mathematical Model of GF}
% Note on the construction of free monoids

% Consider a language $L$ we want to represent, and we come up with a model that we
% build as a set of categories and functions over those categories.  Let $Cat(L)$,
% denote the categories.  Also suppose we define functions such that, given an
% ordered list $x_1,...,x_n;y \in Cat(L)$ we define a set of functions,
% $Fun_L(x_1,...,x_n;y)$ defined over the categories. In gf, a function can be
% denoted something like $\phi : x_1 \rightarrow ... \rightarrow x_n$. We may compose these based
% off their arities. So, given a function $\psi \in Fun_L(y_1,,...,y_n;z)$,
% functions $\phi_1,...\phi_n$ such that $\phi_i \in Fun_L(x_{i,1},...,x_{i,m};y_i)}$ 
%  we can plug these functions in together, or nest them such that
% $$\psi \circ (\phi_1,...,\phi_n) : \rightarrow (x_{i,j}) \rightarrow (y_{i})
% \rightarrow Z$$ 

% This is how abstract syntax trees are formed. It is also worth noting that they
% obey an associativity property, namely that 

% \begin{align*}
% &\theta \circ (\psi_1 \circ (\phi_{1,1},...,\phi_{1,k_1}),...,\psi_n \circ
% (\phi_{n,1},...,\phi_{n,k_n}))\\ = &(\theta \circ \psi_1,...\psi_n) \circ (\phi_{1,1},...,\phi_{1,k_1},...,\phi_{n,1},...,\phi_{n,k_n})
% \end{align*}

% This means that trees in GF are invariant as to how they are built - we
% can build a tree from the leaves to the root or vice versa.

% Example : consider the arithmetic grammar of exponentiation, multiplication, and
% addition defined over a single category of natural number expressions, whereby
% the function symbol is to be interpreted as a string and the tensor product,
% $\otimes$ as the concatenation during evaluation. 

% $$\_\^{}\_ : \mathds{N} \to \mathds{N} \to \mathds{N}$$
% $$\_*\_ : \mathds{N} \to \mathds{N} \to \mathds{N}$$
% $$\_+\_ : \mathds{N} \to \mathds{N} \to \mathds{N}$$

% We can think of constructing the trees by partial application, i.e., 

% $(\lambda x.\: 2 \otimes \^{} \otimes x) : \mathds{N} -> \mathds{N}$

% Lets try see the constructions yielding the string $(1 + 2) \^{} (3 * 4)$.

% We can either (i) construct this as the exponent of two fully formed expressions,
% namely a sum and a product applied to some numbers, or we can first apply the
% exponent to the two binary functions, yielding a quaternary function .

% $x ++ y$
% $x \doubleplus y$
% $``x \doubleplus y"$

% \begin{align*}
% &(\lambda x,y.\: x \otimes \^{} \otimes y)\\
% &\hspace{1cm} ((\lambda x,y.\:x \otimes + \otimes y)\; 1\; 2)\\
% &\hspace{1cm} ((\lambda x,y.\: x \otimes * \otimes y)\; 3\; 4) \\
% \mapsto\; &(\lambda x,y.\: x \otimes \^{} \otimes y)\\
% &\hspace{1cm} (1 + 2)\\
% &\hspace{1cm} (3 * 4))\\
% \mapsto\; &((1 + 2) \^{} (3 * 4))\\
% \end{align}

% \begin{align*}
% &((\lambda x,y.\: x \otimes \^{} \otimes y)\\
% &\hspace{1cm} (\lambda x,y.\:x \otimes + \otimes y)\\
% &\hspace{1cm} (\lambda x,y.\: x \otimes * \otimes y)) \\
% &\hspace{1cm} 1\; 2; 3; 4; \\
% \end{align}

% (1 + 2) \^{} (3 * 4)
  

% ((\lambda x,y. x \^{} y)
%   (\lambda x,y. x + y) 
%   (\lambda x,y. x * y))
%     1 2 3 4

% ((\lambda x,y. x + y) \^{} (\lambda x,y. x * y)) 1 2 3 4
% ((\lambda x,y. x + y) \^{} (\lambda x,y. x * y)) 1 2 3 4

% (1 + 2) \^{} (3 * 4)

% and then say
% (\lambda x. 2 \^{} x) (1 + 3) * (4 + 5)
% = 
% (\lambda x. 2 \^{} x) (1 + 3) * (4 + 5)

% $(\lambda x. 2 \wedge x) : \mathds{N} -> \mathds{N}$

% and then apply it to a complex arguement, say 
% (1 + 3) * (4 + 5)
% (\lambda x. 2 ^ x) : Nat -> Nat

% where 


% \lambda y : Pow y 1 : Nat -> Nat

% (times (plus 2 3) (plus 4 5))
% (Pow \circ (1,times)) : Nat -> Nat -> Nat

% (plus 2 3) (plus 4 5)

% can either be 

% 2^(1+3)*(4+5)


%   % \sin {:} \mathbb{R} &\rightarrow \mathbb{R}\\ x &\mapsto \sin ( x )
% % \circ (\phi_1,...,\phi_n) : \rightarrow (x_{i,j}) \rightarrow (y_{i})
% % \rightarrow Z$$ 

% The two functions displayed in, \autoref{fig:N2}.  If we can loosely call String
% the set of strings freely generated osome acan be 

% for now given a single linear presentation $C^{AST}$ , where

% AST_L String_L0 denote the sets GF ASTs and Strings in the languages generated
% by the rules of L's abstract syntax and L0s compositional representation.

% $$Parse : String -> {AST}$$
% $$Linearize : AST -> String$$

% with the important property that given a string s,


% And given an AST a, we can Parse . Linearize a belongs to {AST}

% Now we should explore why the linearizations are interesting. In part, this is
% because they have arisen from the role of grammars have played in the
% intersection and interaction between computer science and linguistics at least
% since Chomsky in the 50s, and they have different understandings and utilities
% in the respective disciplines. These two discplines converge in GF, which allows
% us to talk about natural languages (NLs) from programming languages (PLs)
% perspective.




% \section{A Spectrum of GF Grammars for types}
\section{Prior GF Formalizations}

Prior to the grammars explored thesis, Ranta produced two main results
\cite{rantaLog} \cite{aarneHott}. These are incredibly important precedents in
this approach to proof translation, and serve as important comparative work for
which this work responds.

\subsection{CADE 2011}

In \cite{rantaLog}, Ranta designed a grammar which
allowed for predicate logic with a domain specific lexicon supporting mathematical
theories , say geometry or arithmetic, on top of the logic. The syntax was
both meant to be relatively complete, so that typical logical utterances of
interest could be accommodated, as well as support relatively non-trivial linguistic
nuance like lists of terms, predicates, and propositions, in-situ and
bounded quantification, like other ways of constructing more syntactically
nuanced predicates. The more interesting syntactic details captured in this work
was by means of an extended grammar on top of the core. The bidirectional
transformation between the core and extended grammars via a PGF also show the
viability and necessity of using more expressive programming languages (Haskell) when
doing thorough translations.

Lists are natural to humans - this is reflected in our language. The RGL supports listing the sentences, noun phrases, and other
grammatical categories. One can then use PGF to unroll the lists into binary
operators, or alternatively transform them in the opposite direction.
, we first mention that GF
natively supports list categories, the judgment \term{cat [C] {n}} can be
desugared to
\begin{verbatim}
  cat ListC ;
  fun BaseC : C -> ... -> C -> ListC ; -- n C ’s
  fun ConsC : C -> ListC -> ListC
\end{verbatim}

As a case study for this grammar, the proposition $\forall x (Nat(x) \supset
Even(x) \lor Odd(x))$ can be given a maximized and minimized version. The tree
representing the \emph{syntactically complete} phrase ``for all natural numbers
x, x is even or x is odd" would be minimized to a tree which linearizes to the
\emph{semantically adequate} phrase ``every natural number is even or odd".

% The trees below also revels how the semantically adequate tree is also
% simpler to understand as representing
% \begin{verbatim}
% PUnivs
%   (BaseVar X)
%   Nat
%   (PConj COr (PAtom (APred1 Even (IVar X))) (PAtom (APred1 Odd (IVar X))))
% |
% V
% PAtom (APred1 (ConjPred1 COr (BasePred1 Even Odd)) (IUniv Nat))
% \
% end{verbatim}

We see that our criteria of semantic adequacy and syntactic completeness can
both occur in the same grammar, with the different subsets related not by a
direct GF translation but a PGF level transformation. Problematically, this
syntactically complete phrase produces four ASTs, with the ``or" and ``forall"
competing for precedence. Where PGF may only give one translation to the
extended syntax, this doesn't give the user of the grammar confidence that her
phrase was correctly interpreted.

In the opposite direction, the desugaring of a logically ``informal"
statement into something less linguistically idiomatic is also accomplished.
Ranta claims ``Finding extended syntax equivalents for core syntax trees is
trickier than the opposite direction". While this may be true for this
particular grammar, we argue that this may not hold generally. Dealing with these ambiguities must be
solved first and foremost to satisfy the PL designer who only accepts
unambiguous parses. For instance, the gf shell shows ``the sum of the
sum of x and y and z is equal to the sum of x and the sum of y and z" giving 32
unique parses. Ranta also outlines the mapping, $\llbracket -
\rrbracket : Core \to Extended$, which should hypothetically return a set of extended
sentences for a more comprehensive grammar.

\begin{itemize}
\item Flattening a list
  $x\ and\ y\ and\ z\ \mapsto x,\ y\ and\ z$
\item Aggregation
  $x\ is\ even\ or\ x\ is\ odd\ \mapsto x\ is\ even\ or\ odd$
\item In-situ quantification \\
  $\forall\ n\ \in Nat,\ x\ is\ even\ or\ x\ is\ odd \mapsto every\ Nat\ is\ even\ or \odd$
\item Negation
  $it\ is\ not\ that\ case\ that\ x\ is\ even\ \mapsto \x is\ not\ even$
\item Reflexivitazion
  $x\ is\ equal\ to\ x\ \mapsto x\ is\ equal\ to\ itself$
\item Modification
  $x\ is\ a\ number\ and\ x\ is\ even\ \mapsto x\ is\ an\ even\ number$
\end{itemize}

Scaling this to cover more phenomena, such as those from [cite ganesalingam], will
pose challenges. Extending this work in general without very sophisticated
statistical methods is impossible because mathematicians will speak uniquely,
and so choosing how to extend a grammar that covers the multiplicity of ways of saying
``the same thing" will require many choices and a significant corpus of examples. Efficient
communication, is a pragmatic feature which this only begins to barely address. The most
interesting linguistic phenomena covered by this grammar, In-situ
quantification, has been at the heart of the Montague tradition.

In some sense, this grammar serves as a case study for what this thesis is
trying to do. However, we note that the core logic only supports propositions
without proofs - it is not a type theory with terms. Additionally, the domain of
arithmetic is an important case study, but scaling this grammar (or any other,
for that matter) to allow for \emph{semantic adequacy} of real mathematics is
still far away, or as Ranta concedes, ``it seems that text generation involves
undecidable optimization problems that have no ultimate automatic solution." It
would be interesting to further extend this grammar with both terms and an
Agda-like concrete syntax.

\subsubsection{An Additional PGF Grammar}

One of the difficulties encountered in this work was reverse engineering Ranta's
code - the large size of a grammar and declarative nature of the code makes it
incredibly difficult to isolate individual features one may wish to understand.
This is true for both GF and PGF, and therefore a lot of work went into
filtering the grammars to understand behaviors of individual components of
interest. Careful usage of the GF module system may allow one to look at
``subgrammars" for some circumstances, but there is not proper methodology to
extract a sub-grammar and therefore it was found that writing a grammar from
scratch was often the easiest way to do this. Grammars can be written
compositionally (adding new categories and functions, refactoring linearization
types, etc.) but decomposing them is not a compositional process.

We wrote a smaller version [cite mycode] of, just focused on propositional
logic, but with the added interest of not just translating between Trees, but
also allowing Haskell computation and evaluation of expressions. Although this
exercise was in some ways a digression from the language of proofs, it also
highlighted many interesting problems.

We begin with an example : the idea was to create a PGF layer for the evaluation
of propositional expressions to their Boolean values, and then create a question
answering system which gave different types of answers - the binary valued
answer, the most verbose possible answer, and the answer which was deemed the
most semantically adequate, \codeword{Simple}, \codeword{Verbose}, and
\codeword{Compressed}, respectively. The system is capable of the following :
\begin{verbatim}
is it the case that if the sum of 3 , 4 and 5 is prime , odd and even then 4
  is prime and even

  Simple : yes .
  Verbose : yes . if the sum of 3 and the sum of 4 and 5 is prime and the sum
    of 3 and the sum of 4 and 5 is odd and the sum of 3 and the sum of 4 and
    5 is even then 4 is prime and 4 is even .
  Compressed : yes . if the sum of 3 , 4 and 5 is prime , odd and even then
    4 is prime and even .
\end{verbatim}

The extended grammar in this case only had lists of propositions and predicates,
and so it was much simpler than [cite logic]. GF list categories are then
transformed into Haskell lists via PGF, so the syntactic sugar for a GF list is actually
functionally tied to its external behavior as well. The functions for our
discussion are:
\begin{verbatim}
  IsNumProp : NumPred -> Object -> Prop ;
  LstNumPred : Conj -> [NumPred] -> NumPred ;
  LstProp : Conj  -> [Prop] -> Prop ;
\end{verbatim}

Note that a numerical predicate, \codeword{NumPred}, represents, for instance,
primality. In order for our pipeline to answer the question, we had to not only
do transform trees, $\llbracket - \rrbracket : \{pgfAST\} \rightarrow
\{pgfAST\}$ , but also evaluate them in more classical domains $\llbracket -
\rrbracket : \{pgfAST\} \rightarrow \mathds{N}$ for the arithmetic objects and
$\llbracket - \rrbracket : \{pgfAST\} \rightarrow \mathds{B}$,
\codeword{evalProp}, for the propositions.

The extension adds more complex cases  to cover when
evaluating propistions, because a normal ``propositional evaluator" doesn't have to
deal with lists. For the most part, this evaluation is able to just apply
boolean semantics to the \emph{canonical} constructors, like \codeword{GNot}. However, a
bug that was subtle and difficult to find appeared, thereby forcing us to dig
deep inside GIsNumProp, preventing an easy solution to what would otherwise be a
simple example of denotational semantics.
\begin{verbatim}
evalProp :: GProp -> Bool
evalProp p = case p of
  ...
  GNot p -> not (evalProp p)
  ...
  GIsNumProp (GLstNumProp c (GListNumPred (x : xs))) obj ->
    let xo = evalProp (GIsNumProp x obj)
        xso = evalProp (GIsNumProp (GLstNumProp c (GListNumPred (xs))) obj) in
    case c of
      GAnd -> (&&) xo xso
      GOr -> (||) xo xso
  ...
\end{verbatim}
While still relatively tame to overcome, an even more expressive abstract syntax
may yield many more subtle obstacles like this, which is the reason its so hard
to understand by just trying to read the code. The more semantic content one
incorporates into the GF grammar, the larger the PGF GADT, which leads to many
more cases when evaluating these trees.

There were many obstructions in engineering this relatively simple
example, particularly when it came to writing test cases. For the naive
way to test with GF is to translate, and the linearization and parsing functions
don't give the programmer many degrees of freedom. ASTs are not objects amenable
to human intuition, which makes it problematic because understanding the
transformations of them constantly requires parsing and linearizing to see their
``behavior". While some work has been done [cite inari] in allowing for testing
of GF grammars for NL applications, the specific domain of formal languages in
GF requires a more refined notion of testing because they should be testable
relative to some model with well behaved mathematical properties. Debugging
something in the pipeline $String \rightarrow GADT \rightarrow GADT \rightarrow
String$ for a large scale grammar without a testing methodology for each
intermediate state is surely to be avoided.

Unfortunately, there is no published work on using Quickcheck [cite hughes] with
PGF. The bugs in this grammar were discovered via the input and output
\emph{appearance} of strings. Often, no string would be returned after a small
change, and discovering the source (abstract, concrete, or PGF) was
excruciating. In one case, a bug was discovered that was presumed to be from the
PGF evaluator, but was then back-traced to Ranta's grammar from which the code
had been refactored. The sentence which broke our pipeline from core to
extended, "4 is prime , 5 is even and if 6 is odd then 7 is even", would be
easily generated (or at least its AST) by quickcheck.

An important observation that was made during this development : that theorems
should be the source of inspirations for deciding which PGF transformations
should take place. For instance, one could define $odd : \mathds{N}
\rightarrow Set$, $prime : \mathds{N} \rightarrow Set$ and prove that $\forall n
\in \mathds{N}.\; n > 2 \times prime\; n \implies odd\; n$. We can use this
theorem as a source of translation, and in fact encode a PGF rule that
transforms anything of the form ``n is prime and n is odd" to ``n is prime",
subject to the condition that $n \neq 2$. One could then take a whole set of
theorems from predicate calculus and encode them as Haskell functions which
simplify the expressions to give some kind of kernel of the expression with the
same meaning, up to some notion of equivalence. The verbose ``if $a$ then $b$
and if $a$ then $c$, can be more canonically read as ``if $a$ then $b$ and $c$".
The application of these theorems as evaluation functions in Haskell could
help give our QA example more informative and direct answers.

We hope this intricate look at a fairly simple grammar highlights some very
serious considerations one should make when writing a PGF embedded grammar.
These include : how does the semantic space the grammar seeks to approximate
effects the PGF translation, how testing formal grammars is non-trivial but
necessary future work, and finally, how information (in this case theorems) from
the domain of approximation can shape and inspire the PGF transformations 
during the translation process.


% also show agda code which reflects difficulty with universes
% discuss mixture of logic language, tt, and sets in the example chosen, and the
% fact that despite its successes,
% contrast with hott book which is exclusively meant to be type-theoreticly formulatedwritten
% formal, personal, and sociological problems, for conclusion
% compare and contrast the categories alone, how they are similair and different
% after all three have been written
%IN ORDER TO USE GF TO MODEL FORMAL LANGUAGES, ONE HAS TO DEVELOP FORMAL MODELS
%OF GF ITSELF. this includes testing, extracting grammars, proving that GF
%grammar does or doesn't contain a string, how to control ambiguity with
%parsing. also, look at notions of subgrammars (and sublanguages), with the
%module systems, and with respect to
% can we come up with a model of computation for GF (i.e. lambdas, Strings, etc)?

\subsection{Stockholm Math Seminar 2014}

In 2014, Ranta gave an unpublished talk at the Stockholm Mathematics seminar
\cite{aarneHott}. Fortunately the code is available, although many of the design
choices aren't documented in the grammar. This project aimed to provide a
translation like the one desired in our current work, but it took a real piece
of mathematics text as the main influence on the design of the Abstract syntax.

This work took a page of text from Peter Aczel's book which more or less goes
over standard HoTT definitions and theorems, and allows the translation of the
latex to a pidgin logical language. The central motivation of this grammar was
to capture, entirely ``real" natural language mathematics, i.e. that which was
written for the mathematician. Therefore, it isn't reminiscent of the slender
abstract syntax the type theorist adores, and sacrificed ``syntactic
completeness" for ``semantic adequacy". This means that the abstract syntax is
much larger and very expressive, but it no longer becomes easy to reason about
and additionally quite ad-hoc. Another defect is that this grammar
overgenerates, so producing a unique parse from the PL side will become tricky.
Nonetheless, this means that it's presumably possible to carve a subset of the
GF HoTT abstract file to accommodate an Agda program, but one encounters rocks as soon
as one begins to dig. For example, in \autoref{fig:M1} is some rendered latex
taken verbatim from Ranta's test code.

With some of hours of tinkering on the pidgin logic concrete syntax and some
reverse engineering with help from the GF shell, one is able to get these
definitions in \autoref{fig:M2}, which are intended to share the same syntactic
space as cubicalTT. We note the first definition of ``contractability" actually
runs in cubicalTT up to renaming a lexical items, and it is clear that the
translation from that to Agda should be a benign task. However, the
\emph{equivalence} syntax is stuck with the artifact from the bloated abstract
syntax for the of the anaphoric use of ``it", which may presumably be fixed with
a few hours more of tinkering, but becomes even more complicated when not just
defining new types, but actually writing real mathematical proofs, or relatively
large terms. To extend this grammar to accommodate a chapter worth of material,
let alone a book, will not just require extending the lexicon, but encountering
other syntactic phenomena that will further be difficult to compress when
defining Agda's concrete syntax.


\begin{code}[hide]%
\>[0]\AgdaSymbol{\{-\#}\AgdaSpace{}%
\AgdaKeyword{OPTIONS}\AgdaSpace{}%
\AgdaPragma{--omega-in-omega}\AgdaSpace{}%
\AgdaPragma{--type-in-type}\AgdaSpace{}%
\AgdaSymbol{\#-\}}\<%
\\
%
\\[\AgdaEmptyExtraSkip]%
\>[0]\AgdaKeyword{module}\AgdaSpace{}%
\AgdaModule{ContrClean}\AgdaSpace{}%
\AgdaKeyword{where}\<%
\\
%
\\[\AgdaEmptyExtraSkip]%
\>[0]\AgdaKeyword{open}\AgdaSpace{}%
\AgdaKeyword{import}\AgdaSpace{}%
\AgdaModule{Agda.Builtin.Sigma}\AgdaSpace{}%
\AgdaKeyword{public}\<%
\\
%
\\[\AgdaEmptyExtraSkip]%
\>[0]\AgdaKeyword{variable}\<%
\\
\>[0][@{}l@{\AgdaIndent{0}}]%
\>[2]\AgdaGeneralizable{A}\AgdaSpace{}%
\AgdaGeneralizable{B}\AgdaSpace{}%
\AgdaSymbol{:}\AgdaSpace{}%
\AgdaPrimitive{Set}\<%
\\
%
\\[\AgdaEmptyExtraSkip]%
\>[0]\AgdaKeyword{data}\AgdaSpace{}%
\AgdaOperator{\AgdaDatatype{\AgdaUnderscore{}≡\AgdaUnderscore{}}}\AgdaSpace{}%
\AgdaSymbol{\{}\AgdaBound{A}\AgdaSpace{}%
\AgdaSymbol{:}\AgdaSpace{}%
\AgdaPrimitive{Set}\AgdaSymbol{\}}\AgdaSpace{}%
\AgdaSymbol{(}\AgdaBound{a}\AgdaSpace{}%
\AgdaSymbol{:}\AgdaSpace{}%
\AgdaBound{A}\AgdaSymbol{)}\AgdaSpace{}%
\AgdaSymbol{:}\AgdaSpace{}%
\AgdaBound{A}\AgdaSpace{}%
\AgdaSymbol{→}\AgdaSpace{}%
\AgdaPrimitive{Set}\AgdaSpace{}%
\AgdaKeyword{where}\<%
\\
\>[0][@{}l@{\AgdaIndent{0}}]%
\>[2]\AgdaInductiveConstructor{r}\AgdaSpace{}%
\AgdaSymbol{:}\AgdaSpace{}%
\AgdaBound{a}\AgdaSpace{}%
\AgdaOperator{\AgdaDatatype{≡}}\AgdaSpace{}%
\AgdaBound{a}\<%
\\
%
\\[\AgdaEmptyExtraSkip]%
\>[0]\AgdaKeyword{infix}\AgdaSpace{}%
\AgdaNumber{20}\AgdaSpace{}%
\AgdaOperator{\AgdaDatatype{\AgdaUnderscore{}≡\AgdaUnderscore{}}}\<%
\\
%
\\[\AgdaEmptyExtraSkip]%
\>[0]\AgdaFunction{id}\AgdaSpace{}%
\AgdaSymbol{:}\AgdaSpace{}%
\AgdaGeneralizable{A}\AgdaSpace{}%
\AgdaSymbol{→}\AgdaSpace{}%
\AgdaGeneralizable{A}\<%
\\
\>[0]\AgdaFunction{id}\AgdaSpace{}%
\AgdaSymbol{=}\AgdaSpace{}%
\AgdaSymbol{λ}\AgdaSpace{}%
\AgdaBound{z}\AgdaSpace{}%
\AgdaSymbol{→}\AgdaSpace{}%
\AgdaBound{z}\<%
\\
\>[0]\<%
\end{code}

\begin{figure}[H]
\textbf{Definition}:
A type $A$ is contractible, if there is $a : A$, called the center of contraction, such that for all $x : A$, $\equalH {a}{x}$.
\caption{Rendered Latex} \label{fig:R1}
\begin{verbatim}
isContr ( A : Set ) : Set = ( a : A ) ( * ) ( ( x : A ) -> Id ( a ) ( x ) )
\end{verbatim}
\begin{code}%
\>[0]\AgdaFunction{isContr}\AgdaSpace{}%
\AgdaSymbol{:}\AgdaSpace{}%
\AgdaSymbol{(}\AgdaBound{A}\AgdaSpace{}%
\AgdaSymbol{:}\AgdaSpace{}%
\AgdaPrimitive{Set}\AgdaSymbol{)}\AgdaSpace{}%
\AgdaSymbol{→}\AgdaSpace{}%
\AgdaPrimitive{Set}\<%
\\
\>[0]\AgdaFunction{isContr}\AgdaSpace{}%
\AgdaBound{A}\AgdaSpace{}%
\AgdaSymbol{=}%
\>[13]\AgdaRecord{Σ}\AgdaSpace{}%
\AgdaBound{A}\AgdaSpace{}%
\AgdaSymbol{λ}\AgdaSpace{}%
\AgdaBound{a}\AgdaSpace{}%
\AgdaSymbol{→}\AgdaSpace{}%
\AgdaSymbol{(}\AgdaBound{x}\AgdaSpace{}%
\AgdaSymbol{:}\AgdaSpace{}%
\AgdaBound{A}\AgdaSymbol{)}\AgdaSpace{}%
\AgdaSymbol{→}\AgdaSpace{}%
\AgdaSymbol{(}\AgdaBound{a}\AgdaSpace{}%
\AgdaOperator{\AgdaDatatype{≡}}\AgdaSpace{}%
\AgdaBound{x}\AgdaSymbol{)}\<%
\end{code}
\caption{Contractibility} \label{fig:R2}
\end{figure}

In \autoref{fig:R2}, we show the different syntax presentations of the
\emph{equivalence}, which is merely a bijection when restricted to sets. This is
of such fundamental idea in mathematics and HoTT in particular that it merits
its own chapter in [cite hott], but we only showcase one of its many equivalent
definitions. We see that the pidgin syntax is stuck with the anaphoric artifact
from the bloated abstract syntax, \codeword{fiber} has the type \codeword{it :
Set} instead of something like \codeword{(y : B) : Set}, and the y variable is
unbound in the \codeword{fiber} expression. This may presumably be fixed with a
few hours more of tinkering, but becomes even more complicated when not just
defining new types, but actually writing real mathematical proofs.

\begin{figure}[H]
\textbf{Definition}:
A map $f : \arrowH {A}{B}$ is an equivalence, if for all $y : B$, its fiber, $\comprehensionH {x}{A}{\equalH {\appH {f}{x}}{y}}$, is contractible.
We write $\equivalenceH {A}{B}$, if there is an equivalence $\arrowH {A}{B}$.
\begin{verbatim}
Equivalence ( f : A -> B ) : Set =
  ( y : B ) -> ( isContr ( fiber it ) ) ; ; ;
  fiber it : Set = ( x : A ) ( * ) ( Id ( f ( x ) ) ( y ) )
\end{verbatim}
\begin{code}%
\>[0]\AgdaFunction{Equivalence}\AgdaSpace{}%
\AgdaSymbol{:}\AgdaSpace{}%
\AgdaSymbol{(}\AgdaBound{A}\AgdaSpace{}%
\AgdaBound{B}\AgdaSpace{}%
\AgdaSymbol{:}\AgdaSpace{}%
\AgdaPrimitive{Set}\AgdaSymbol{)}\AgdaSpace{}%
\AgdaSymbol{→}\AgdaSpace{}%
\AgdaSymbol{(}\AgdaBound{f}\AgdaSpace{}%
\AgdaSymbol{:}\AgdaSpace{}%
\AgdaBound{A}\AgdaSpace{}%
\AgdaSymbol{→}\AgdaSpace{}%
\AgdaBound{B}\AgdaSymbol{)}\AgdaSpace{}%
\AgdaSymbol{→}\AgdaSpace{}%
\AgdaPrimitive{Set}\<%
\\
\>[0]\AgdaFunction{Equivalence}\AgdaSpace{}%
\AgdaBound{A}\AgdaSpace{}%
\AgdaBound{B}\AgdaSpace{}%
\AgdaBound{f}\AgdaSpace{}%
\AgdaSymbol{=}\AgdaSpace{}%
\AgdaSymbol{∀}\AgdaSpace{}%
\AgdaSymbol{(}\AgdaBound{y}\AgdaSpace{}%
\AgdaSymbol{:}\AgdaSpace{}%
\AgdaBound{B}\AgdaSymbol{)}\AgdaSpace{}%
\AgdaSymbol{→}\AgdaSpace{}%
\AgdaFunction{isContr}\AgdaSpace{}%
\AgdaSymbol{(}\AgdaFunction{fiber'}\AgdaSpace{}%
\AgdaBound{y}\AgdaSymbol{)}\<%
\\
\>[0][@{}l@{\AgdaIndent{0}}]%
\>[2]\AgdaKeyword{where}\<%
\\
\>[2][@{}l@{\AgdaIndent{0}}]%
\>[4]\AgdaFunction{fiber'}\AgdaSpace{}%
\AgdaSymbol{:}\AgdaSpace{}%
\AgdaSymbol{(}\AgdaBound{y}\AgdaSpace{}%
\AgdaSymbol{:}\AgdaSpace{}%
\AgdaBound{B}\AgdaSymbol{)}\AgdaSpace{}%
\AgdaSymbol{→}\AgdaSpace{}%
\AgdaPrimitive{Set}\<%
\\
%
\>[4]\AgdaFunction{fiber'}\AgdaSpace{}%
\AgdaBound{y}\AgdaSpace{}%
\AgdaSymbol{=}\AgdaSpace{}%
\AgdaRecord{Σ}\AgdaSpace{}%
\AgdaBound{A}\AgdaSpace{}%
\AgdaSymbol{(λ}\AgdaSpace{}%
\AgdaBound{x}\AgdaSpace{}%
\AgdaSymbol{→}\AgdaSpace{}%
\AgdaBound{y}\AgdaSpace{}%
\AgdaOperator{\AgdaDatatype{≡}}\AgdaSpace{}%
\AgdaBound{f}\AgdaSpace{}%
\AgdaBound{x}\AgdaSymbol{)}\<%
\end{code}
\caption{Contractibility} \label{fig:R3}
\end{figure}


% \begin{figure}

%  \textbf{Definition}:
%  A type $A$ is contractible, if there is $a : A$, called the center of contraction, such that for all $x : A$, $\equalH {a}{x}$.

%  \textbf{Definition}:
%  A map $f : \arrowH {A}{B}$ is an equivalence, if for all $y : B$, its fiber, $\comprehensionH {x}{A}{\equalH {\appH {f}{x}}{y}}$, is contractible.
%  We write $\equivalenceH {A}{B}$, if there is an equivalence $\arrowH {A}{B}$.
% \caption{Rendered Latex} \label{fig:M1}


% \begin{verbatim}
% isContr ( A : Set ) : Set = ( a : A ) ( * ) ( ( x : A ) -> Id ( a ) ( x ) )

% Equivalence ( f : A -> B ) : Set =
%   ( y : B ) -> ( isContr ( fiber it ) ) ; ; ;
%   fiber it : Set = ( x : A ) ( * ) ( Id ( f ( x ) ) ( y ) )
% \end{verbatim}
% \caption{Pidgin cubicalTT} \label{fig:M2}

% 
\begin{code}[hide]%
\>[0]\AgdaSymbol{\{-\#}\AgdaSpace{}%
\AgdaKeyword{OPTIONS}\AgdaSpace{}%
\AgdaPragma{--omega-in-omega}\AgdaSpace{}%
\AgdaPragma{--type-in-type}\AgdaSpace{}%
\AgdaSymbol{\#-\}}\<%
\\
%
\\[\AgdaEmptyExtraSkip]%
\>[0]\AgdaKeyword{module}\AgdaSpace{}%
\AgdaModule{ContrClean}\AgdaSpace{}%
\AgdaKeyword{where}\<%
\\
%
\\[\AgdaEmptyExtraSkip]%
\>[0]\AgdaKeyword{open}\AgdaSpace{}%
\AgdaKeyword{import}\AgdaSpace{}%
\AgdaModule{Agda.Builtin.Sigma}\AgdaSpace{}%
\AgdaKeyword{public}\<%
\\
%
\\[\AgdaEmptyExtraSkip]%
\>[0]\AgdaKeyword{variable}\<%
\\
\>[0][@{}l@{\AgdaIndent{0}}]%
\>[2]\AgdaGeneralizable{A}\AgdaSpace{}%
\AgdaGeneralizable{B}\AgdaSpace{}%
\AgdaSymbol{:}\AgdaSpace{}%
\AgdaPrimitive{Set}\<%
\\
%
\\[\AgdaEmptyExtraSkip]%
\>[0]\AgdaKeyword{data}\AgdaSpace{}%
\AgdaOperator{\AgdaDatatype{\AgdaUnderscore{}≡\AgdaUnderscore{}}}\AgdaSpace{}%
\AgdaSymbol{\{}\AgdaBound{A}\AgdaSpace{}%
\AgdaSymbol{:}\AgdaSpace{}%
\AgdaPrimitive{Set}\AgdaSymbol{\}}\AgdaSpace{}%
\AgdaSymbol{(}\AgdaBound{a}\AgdaSpace{}%
\AgdaSymbol{:}\AgdaSpace{}%
\AgdaBound{A}\AgdaSymbol{)}\AgdaSpace{}%
\AgdaSymbol{:}\AgdaSpace{}%
\AgdaBound{A}\AgdaSpace{}%
\AgdaSymbol{→}\AgdaSpace{}%
\AgdaPrimitive{Set}\AgdaSpace{}%
\AgdaKeyword{where}\<%
\\
\>[0][@{}l@{\AgdaIndent{0}}]%
\>[2]\AgdaInductiveConstructor{r}\AgdaSpace{}%
\AgdaSymbol{:}\AgdaSpace{}%
\AgdaBound{a}\AgdaSpace{}%
\AgdaOperator{\AgdaDatatype{≡}}\AgdaSpace{}%
\AgdaBound{a}\<%
\\
%
\\[\AgdaEmptyExtraSkip]%
\>[0]\AgdaKeyword{infix}\AgdaSpace{}%
\AgdaNumber{20}\AgdaSpace{}%
\AgdaOperator{\AgdaDatatype{\AgdaUnderscore{}≡\AgdaUnderscore{}}}\<%
\\
%
\\[\AgdaEmptyExtraSkip]%
\>[0]\AgdaFunction{id}\AgdaSpace{}%
\AgdaSymbol{:}\AgdaSpace{}%
\AgdaGeneralizable{A}\AgdaSpace{}%
\AgdaSymbol{→}\AgdaSpace{}%
\AgdaGeneralizable{A}\<%
\\
\>[0]\AgdaFunction{id}\AgdaSpace{}%
\AgdaSymbol{=}\AgdaSpace{}%
\AgdaSymbol{λ}\AgdaSpace{}%
\AgdaBound{z}\AgdaSpace{}%
\AgdaSymbol{→}\AgdaSpace{}%
\AgdaBound{z}\<%
\\
\>[0]\<%
\end{code}

\begin{figure}[H]
\textbf{Definition}:
A type $A$ is contractible, if there is $a : A$, called the center of contraction, such that for all $x : A$, $\equalH {a}{x}$.
\caption{Rendered Latex} \label{fig:R1}
\begin{verbatim}
isContr ( A : Set ) : Set = ( a : A ) ( * ) ( ( x : A ) -> Id ( a ) ( x ) )
\end{verbatim}
\begin{code}%
\>[0]\AgdaFunction{isContr}\AgdaSpace{}%
\AgdaSymbol{:}\AgdaSpace{}%
\AgdaSymbol{(}\AgdaBound{A}\AgdaSpace{}%
\AgdaSymbol{:}\AgdaSpace{}%
\AgdaPrimitive{Set}\AgdaSymbol{)}\AgdaSpace{}%
\AgdaSymbol{→}\AgdaSpace{}%
\AgdaPrimitive{Set}\<%
\\
\>[0]\AgdaFunction{isContr}\AgdaSpace{}%
\AgdaBound{A}\AgdaSpace{}%
\AgdaSymbol{=}%
\>[13]\AgdaRecord{Σ}\AgdaSpace{}%
\AgdaBound{A}\AgdaSpace{}%
\AgdaSymbol{λ}\AgdaSpace{}%
\AgdaBound{a}\AgdaSpace{}%
\AgdaSymbol{→}\AgdaSpace{}%
\AgdaSymbol{(}\AgdaBound{x}\AgdaSpace{}%
\AgdaSymbol{:}\AgdaSpace{}%
\AgdaBound{A}\AgdaSymbol{)}\AgdaSpace{}%
\AgdaSymbol{→}\AgdaSpace{}%
\AgdaSymbol{(}\AgdaBound{a}\AgdaSpace{}%
\AgdaOperator{\AgdaDatatype{≡}}\AgdaSpace{}%
\AgdaBound{x}\AgdaSymbol{)}\<%
\end{code}
\caption{Contractibility} \label{fig:R2}
\end{figure}

In \autoref{fig:R2}, we show the different syntax presentations of the
\emph{equivalence}, which is merely a bijection when restricted to sets. This is
of such fundamental idea in mathematics and HoTT in particular that it merits
its own chapter in [cite hott], but we only showcase one of its many equivalent
definitions. We see that the pidgin syntax is stuck with the anaphoric artifact
from the bloated abstract syntax, \codeword{fiber} has the type \codeword{it :
Set} instead of something like \codeword{(y : B) : Set}, and the y variable is
unbound in the \codeword{fiber} expression. This may presumably be fixed with a
few hours more of tinkering, but becomes even more complicated when not just
defining new types, but actually writing real mathematical proofs.

\begin{figure}[H]
\textbf{Definition}:
A map $f : \arrowH {A}{B}$ is an equivalence, if for all $y : B$, its fiber, $\comprehensionH {x}{A}{\equalH {\appH {f}{x}}{y}}$, is contractible.
We write $\equivalenceH {A}{B}$, if there is an equivalence $\arrowH {A}{B}$.
\begin{verbatim}
Equivalence ( f : A -> B ) : Set =
  ( y : B ) -> ( isContr ( fiber it ) ) ; ; ;
  fiber it : Set = ( x : A ) ( * ) ( Id ( f ( x ) ) ( y ) )
\end{verbatim}
\begin{code}%
\>[0]\AgdaFunction{Equivalence}\AgdaSpace{}%
\AgdaSymbol{:}\AgdaSpace{}%
\AgdaSymbol{(}\AgdaBound{A}\AgdaSpace{}%
\AgdaBound{B}\AgdaSpace{}%
\AgdaSymbol{:}\AgdaSpace{}%
\AgdaPrimitive{Set}\AgdaSymbol{)}\AgdaSpace{}%
\AgdaSymbol{→}\AgdaSpace{}%
\AgdaSymbol{(}\AgdaBound{f}\AgdaSpace{}%
\AgdaSymbol{:}\AgdaSpace{}%
\AgdaBound{A}\AgdaSpace{}%
\AgdaSymbol{→}\AgdaSpace{}%
\AgdaBound{B}\AgdaSymbol{)}\AgdaSpace{}%
\AgdaSymbol{→}\AgdaSpace{}%
\AgdaPrimitive{Set}\<%
\\
\>[0]\AgdaFunction{Equivalence}\AgdaSpace{}%
\AgdaBound{A}\AgdaSpace{}%
\AgdaBound{B}\AgdaSpace{}%
\AgdaBound{f}\AgdaSpace{}%
\AgdaSymbol{=}\AgdaSpace{}%
\AgdaSymbol{∀}\AgdaSpace{}%
\AgdaSymbol{(}\AgdaBound{y}\AgdaSpace{}%
\AgdaSymbol{:}\AgdaSpace{}%
\AgdaBound{B}\AgdaSymbol{)}\AgdaSpace{}%
\AgdaSymbol{→}\AgdaSpace{}%
\AgdaFunction{isContr}\AgdaSpace{}%
\AgdaSymbol{(}\AgdaFunction{fiber'}\AgdaSpace{}%
\AgdaBound{y}\AgdaSymbol{)}\<%
\\
\>[0][@{}l@{\AgdaIndent{0}}]%
\>[2]\AgdaKeyword{where}\<%
\\
\>[2][@{}l@{\AgdaIndent{0}}]%
\>[4]\AgdaFunction{fiber'}\AgdaSpace{}%
\AgdaSymbol{:}\AgdaSpace{}%
\AgdaSymbol{(}\AgdaBound{y}\AgdaSpace{}%
\AgdaSymbol{:}\AgdaSpace{}%
\AgdaBound{B}\AgdaSymbol{)}\AgdaSpace{}%
\AgdaSymbol{→}\AgdaSpace{}%
\AgdaPrimitive{Set}\<%
\\
%
\>[4]\AgdaFunction{fiber'}\AgdaSpace{}%
\AgdaBound{y}\AgdaSpace{}%
\AgdaSymbol{=}\AgdaSpace{}%
\AgdaRecord{Σ}\AgdaSpace{}%
\AgdaBound{A}\AgdaSpace{}%
\AgdaSymbol{(λ}\AgdaSpace{}%
\AgdaBound{x}\AgdaSpace{}%
\AgdaSymbol{→}\AgdaSpace{}%
\AgdaBound{y}\AgdaSpace{}%
\AgdaOperator{\AgdaDatatype{≡}}\AgdaSpace{}%
\AgdaBound{f}\AgdaSpace{}%
\AgdaBound{x}\AgdaSymbol{)}\<%
\end{code}
\caption{Contractibility} \label{fig:R3}
\end{figure}

% % \begin{code}[hide]%
\>[0]\<%
\\
\>[0]\AgdaComment{-- \{-\# OPTIONS --omega-in-omega --type-in-type \#-\}}\<%
\\
%
\\[\AgdaEmptyExtraSkip]%
\>[0]\AgdaKeyword{module}\AgdaSpace{}%
\AgdaModule{ex}\AgdaSpace{}%
\AgdaKeyword{where}\<%
\\
%
\\[\AgdaEmptyExtraSkip]%
\>[0]\AgdaKeyword{data}\AgdaSpace{}%
\AgdaDatatype{aℕ}\AgdaSpace{}%
\AgdaSymbol{:}\AgdaSpace{}%
\AgdaPrimitive{Set}\AgdaSpace{}%
\AgdaKeyword{where}\<%
\\
\>[0][@{}l@{\AgdaIndent{0}}]%
\>[2]\AgdaInductiveConstructor{zero'}\AgdaSpace{}%
\AgdaSymbol{:}\AgdaSpace{}%
\AgdaDatatype{aℕ}\<%
\\
%
\\[\AgdaEmptyExtraSkip]%
\>[0]\AgdaKeyword{variable}\<%
\\
\>[0][@{}l@{\AgdaIndent{0}}]%
\>[2]\AgdaGeneralizable{A}\AgdaSpace{}%
\AgdaSymbol{:}\AgdaSpace{}%
\AgdaPrimitive{Set}\<%
\\
%
\>[2]\AgdaGeneralizable{D}\AgdaSpace{}%
\AgdaSymbol{:}\AgdaSpace{}%
\AgdaPrimitive{Set}\<%
\\
%
\>[2]\AgdaGeneralizable{stuff}\AgdaSpace{}%
\AgdaSymbol{:}\AgdaSpace{}%
\AgdaPrimitive{Set}\<%
\\
%
\\[\AgdaEmptyExtraSkip]%
\>[0]\AgdaFunction{definition-body}\AgdaSpace{}%
\AgdaSymbol{=}\AgdaSpace{}%
\AgdaDatatype{aℕ}\<%
\\
%
\\[\AgdaEmptyExtraSkip]%
\>[0]\AgdaFunction{T}\AgdaSpace{}%
\AgdaSymbol{=}\AgdaSpace{}%
\AgdaDatatype{aℕ}\AgdaSpace{}%
\AgdaSymbol{→}\AgdaSpace{}%
\AgdaDatatype{aℕ}\<%
\\
\>[0]\AgdaFunction{L}\AgdaSpace{}%
\AgdaSymbol{=}\AgdaSpace{}%
\AgdaDatatype{aℕ}\<%
\\
\>[0]\AgdaFunction{E}\AgdaSpace{}%
\AgdaSymbol{=}\AgdaSpace{}%
\AgdaDatatype{aℕ}\<%
\\
\>[0]\AgdaFunction{C}\AgdaSpace{}%
\AgdaSymbol{=}\AgdaSpace{}%
\AgdaDatatype{aℕ}\<%
\\
%
\\[\AgdaEmptyExtraSkip]%
\>[0]\AgdaFunction{proof}\AgdaSpace{}%
\AgdaSymbol{:}\AgdaSpace{}%
\AgdaFunction{L}\<%
\\
\>[0]\AgdaFunction{proof}\AgdaSpace{}%
\AgdaSymbol{=}\AgdaSpace{}%
\AgdaInductiveConstructor{zero'}\<%
\\
%
\\[\AgdaEmptyExtraSkip]%
\>[0]\AgdaFunction{corollaryStuff}\AgdaSpace{}%
\AgdaSymbol{=}\AgdaSpace{}%
\AgdaDatatype{aℕ}\<%
\\
%
\\[\AgdaEmptyExtraSkip]%
\>[0]\AgdaFunction{proofNeedingLemma}\AgdaSpace{}%
\AgdaSymbol{:}\AgdaSpace{}%
\AgdaDatatype{aℕ}\AgdaSpace{}%
\AgdaSymbol{→}\AgdaSpace{}%
\AgdaDatatype{aℕ}\AgdaSpace{}%
\AgdaSymbol{→}\AgdaSpace{}%
\AgdaDatatype{aℕ}\<%
\\
\>[0]\AgdaFunction{proofNeedingLemma}\AgdaSpace{}%
\AgdaBound{x}\AgdaSpace{}%
\AgdaSymbol{=}\AgdaSpace{}%
\AgdaSymbol{λ}\AgdaSpace{}%
\AgdaBound{x₁}\AgdaSpace{}%
\AgdaSymbol{→}\AgdaSpace{}%
\AgdaInductiveConstructor{zero'}\<%
\\
\>[0]\<%
\end{code}

\subsubsection{Agda Programming}

Listed is the syntax Agda uses for judgements: \term{T} : \term{Set} means
\term{T} is a type, \term{t} : \term{T} means a term \term{t} has type \term{T},
and \term{t} = \term{t'} means \term{t} is defined to be judgmentally equal to
\term{t'}. Once one has made this equality judgement, Agda can normalize the
definitionally equal terms to the same normal form. Let's compare these Agda
judgements to those keywords ubiquitous in mathematics:

\begin{figure}
\centering
\begin{minipage}[t]{.3\textwidth}
\vspace{2cm}
\begin{itemize}
\item Axiom
\item Definition
\item Lemma
\item Theorem
\item Proof
\item Corollary
\item Example
\end{itemize}
\end{minipage}%
\begin{minipage}[t]{.55\textwidth}
\begin{code}%
\>[0]\AgdaKeyword{postulate}%
\>[12]\AgdaComment{-- Axiom}\<%
\\
\>[0][@{}l@{\AgdaIndent{0}}]%
\>[2]\AgdaPostulate{axiom}\AgdaSpace{}%
\AgdaSymbol{:}\AgdaSpace{}%
\AgdaGeneralizable{A}\<%
\\
%
\\[\AgdaEmptyExtraSkip]%
\>[0]\AgdaFunction{definition}\AgdaSpace{}%
\AgdaSymbol{:}\AgdaSpace{}%
\AgdaGeneralizable{stuff}\AgdaSpace{}%
\AgdaSymbol{→}\AgdaSpace{}%
\AgdaPrimitive{Set}\AgdaSpace{}%
\AgdaComment{--Definition}\<%
\\
\>[0]\AgdaFunction{definition}\AgdaSpace{}%
\AgdaBound{s}\AgdaSpace{}%
\AgdaSymbol{=}\AgdaSpace{}%
\AgdaFunction{definition-body}\<%
\\
%
\\[\AgdaEmptyExtraSkip]%
\>[0]\AgdaFunction{theorem}\AgdaSpace{}%
\AgdaSymbol{:}\AgdaSpace{}%
\AgdaFunction{T}%
\>[16]\AgdaComment{-- Theorem Statement}\<%
\\
\>[0]\AgdaFunction{theorem}\AgdaSpace{}%
\AgdaSymbol{=}\AgdaSpace{}%
\AgdaFunction{proofNeedingLemma}\AgdaSpace{}%
\AgdaFunction{lemma}\AgdaSpace{}%
\AgdaComment{-- Proof}\<%
\\
\>[0][@{}l@{\AgdaIndent{0}}]%
\>[2]\AgdaKeyword{where}\<%
\\
\>[2][@{}l@{\AgdaIndent{0}}]%
\>[4]\AgdaFunction{lemma}\AgdaSpace{}%
\AgdaSymbol{:}\AgdaSpace{}%
\AgdaFunction{L}%
\>[18]\AgdaComment{-- Lemma Statement}\<%
\\
%
\>[4]\AgdaFunction{lemma}\AgdaSpace{}%
\AgdaSymbol{=}\AgdaSpace{}%
\AgdaFunction{proof}\<%
\\
%
\\[\AgdaEmptyExtraSkip]%
\>[0]\AgdaFunction{corollary}\AgdaSpace{}%
\AgdaSymbol{:}\AgdaSpace{}%
\AgdaFunction{corollaryStuff}\AgdaSpace{}%
\AgdaSymbol{→}\AgdaSpace{}%
\AgdaFunction{C}\<%
\\
\>[0]\AgdaFunction{corollary}\AgdaSpace{}%
\AgdaBound{coro-term}\AgdaSpace{}%
\AgdaSymbol{=}\AgdaSpace{}%
\AgdaFunction{theorem}\AgdaSpace{}%
\AgdaBound{coro-term}\<%
\\
%
\\[\AgdaEmptyExtraSkip]%
\>[0]\AgdaFunction{example}\AgdaSpace{}%
\AgdaSymbol{:}\AgdaSpace{}%
\AgdaFunction{E}%
\>[16]\AgdaComment{-- Example Statement}\<%
\\
\>[0]\AgdaFunction{example}\AgdaSpace{}%
\AgdaSymbol{=}\AgdaSpace{}%
\AgdaFunction{proof}\<%
\end{code}
\end{minipage}
\caption{Mathematical Assertions and Agda Judgements} \label{fig:O1}
\end{figure}

Formation rules are given by the first line of the data declaration, followed
by some number of constructors which correspond to the introduction forms of the
type being defined. Therefore, to define a type for Booleans, $𝔹$, we present
these rules both in the proof theoretic and Agda syntax. We note that the
context $\Gamma$ is not present in Agda.

\begin{minipage}[t]{.4\textwidth}
\vspace{3mm}
\[
  \begin{prooftree}
    \infer1[]{ \vdash 𝔹 : {\rm type}}
  \end{prooftree}
\]
\[
  \begin{prooftree}
    \infer1[]{ \Gamma \vdash true : 𝔹  }
  \end{prooftree}
  \quad \quad
  \begin{prooftree}
    \infer1[]{ \Gamma \vdash false : 𝔹  }
  \end{prooftree}
\]
\end{minipage}
\begin{minipage}[t]{.3\textwidth}
\begin{code}%
\>[0]\AgdaKeyword{data}\AgdaSpace{}%
\AgdaDatatype{𝔹}\AgdaSpace{}%
\AgdaSymbol{:}\AgdaSpace{}%
\AgdaPrimitive{Set}\AgdaSpace{}%
\AgdaKeyword{where}\AgdaSpace{}%
\AgdaComment{-- formation rule}\<%
\\
\>[0][@{}l@{\AgdaIndent{0}}]%
\>[2]\AgdaInductiveConstructor{true}%
\>[8]\AgdaSymbol{:}\AgdaSpace{}%
\AgdaDatatype{𝔹}\AgdaSpace{}%
\AgdaComment{-- introduction rule}\<%
\\
%
\>[2]\AgdaInductiveConstructor{false}\AgdaSpace{}%
\AgdaSymbol{:}\AgdaSpace{}%
\AgdaDatatype{𝔹}\<%
\end{code}
\end{minipage}

The elimination forms are deriveable from the introduction rules, and the
computation rules can then be extracted by via the harmonious relationship
between the introduction and elmination forms \cite{pfenningHar}. Agda's pattern
matching is equivalent to the deriveable dependently typed elimination forms
\cite{coqPat}, and one can simply pattern match on a boolean, producing multiple
lines for each constructor of the variable's type, to extract the classic
recursion principle for Booleans. The \term{if then else} statement shown below
is really just the boolean elimination form. It is not standard to include the
premises of the eqaulity rules.

\begin{minipage}[t]{.4\textwidth}
\[
  \begin{prooftree}
    \hypo{̌\Gamma \vdash A : {\rm type} }
    \hypo{\Gamma \vdash b : 𝔹 }
    \hypo{\Gamma \vdash a1 : A}
    \hypo{\Gamma \vdash a2 : A }
    \infer4[]{\Gamma \vdash boolrec\{a1;a2\}(b) : A }
  \end{prooftree}
\]
$$\Gamma \vdash boolrec\{a1;a2\}(true) \equiv a1 : A$$
$$\Gamma \vdash boolrec\{a1;a2\}(false) \equiv a2 : A$$
\end{minipage}
\hfill
\begin{minipage}[t]{.5\textwidth}
\begin{code}%
\>[0]\AgdaOperator{\AgdaFunction{if\AgdaUnderscore{}then\AgdaUnderscore{}else\AgdaUnderscore{}}}\AgdaSpace{}%
\AgdaSymbol{:}\<%
\\
\>[0][@{}l@{\AgdaIndent{0}}]%
\>[2]\AgdaSymbol{\{}\AgdaBound{A}\AgdaSpace{}%
\AgdaSymbol{:}\AgdaSpace{}%
\AgdaPrimitive{Set}\AgdaSymbol{\}}\AgdaSpace{}%
\AgdaSymbol{→}\AgdaSpace{}%
\AgdaDatatype{𝔹}\AgdaSpace{}%
\AgdaSymbol{→}\AgdaSpace{}%
\AgdaBound{A}\AgdaSpace{}%
\AgdaSymbol{→}\AgdaSpace{}%
\AgdaBound{A}\AgdaSpace{}%
\AgdaSymbol{→}\AgdaSpace{}%
\AgdaBound{A}\<%
\\
\>[0]\AgdaOperator{\AgdaFunction{if}}\AgdaSpace{}%
\AgdaInductiveConstructor{true}\AgdaSpace{}%
\AgdaOperator{\AgdaFunction{then}}\AgdaSpace{}%
\AgdaBound{a1}\AgdaSpace{}%
\AgdaOperator{\AgdaFunction{else}}\AgdaSpace{}%
\AgdaBound{a2}\AgdaSpace{}%
\AgdaSymbol{=}\AgdaSpace{}%
\AgdaBound{a1}\<%
\\
\>[0]\AgdaOperator{\AgdaFunction{if}}\AgdaSpace{}%
\AgdaInductiveConstructor{false}\AgdaSpace{}%
\AgdaOperator{\AgdaFunction{then}}\AgdaSpace{}%
\AgdaBound{a1}\AgdaSpace{}%
\AgdaOperator{\AgdaFunction{else}}\AgdaSpace{}%
\AgdaBound{a2}\AgdaSpace{}%
\AgdaSymbol{=}\AgdaSpace{}%
\AgdaBound{a2}\<%
\end{code}
\end{minipage}

When using Agda one is interactively building a proof via holes. There is an
Agda Emacs mode which enables this. Glossing over many details, we show sample
code in the proof development state prior to pattern matching on \codeword{b}.
We have a hole, \codeword{{ b }0}, and the proof state is displayed to the
right. It shows both the current context with \codeword{A, b, a1, a2}, the goal
which is something of type \codeword{A}, and what we have, \codeword{B},
represents the type of the variable in the hole.

\hfill
\begin{minipage}[t]{.4\textwidth}
\begin{verbatim}
if_then_else_ :
  {A : Set} → B → A → A → A
if b then a1 else a2 = { b }0
\end{verbatim}
\end{minipage}
\hfill
\begin{minipage}[t]{.5\textwidth}
\begin{verbatim}
Goal: A
Have: B
———————————————
a2 : A
a1 : A
b  : B
A  : Set   (not in scope)
\end{verbatim}
\end{minipage}

The interactivity is performed via emacs commands, and every time one updates
the hole with a new term, we can immediately view the next goal with an updated
context. The underscore in \term{if_then_else_} denotes the placement of the
arguements, as Agda allows mixfix operations. Agda allows for more nuanced
syntacic features like unicode. This is interesting from the \emph{concrete
syntax} perspective as the arguement placement and symbolic expressiveness makes
Agda's syntax feel more familiar to the mathematician. We also observe the use
of parametric polymorphism, namely, that we can extract a member of some
arbtitrary type \term{A} from a boolean value given two members of \term{A}.

This polymorphism allows one to implement simple programs like boolean negation,
\term{~}, and more interestingly, \term{functionalNegation}, where one can use
functions as arguements. \term{functionalNegation} is a functional, or higher
order functions, which take functions as arguements and return functions. We
also notice in \term{functionalNegation} that one can work directly with a
built-in $\lambda$ to ensure the correct return type.

\begin{code}%
\>[0]\AgdaFunction{\textasciitilde{}}\AgdaSpace{}%
\AgdaSymbol{:}\AgdaSpace{}%
\AgdaDatatype{𝔹}\AgdaSpace{}%
\AgdaSymbol{→}\AgdaSpace{}%
\AgdaDatatype{𝔹}\<%
\\
\>[0]\AgdaFunction{\textasciitilde{}}\AgdaSpace{}%
\AgdaBound{b}\AgdaSpace{}%
\AgdaSymbol{=}\AgdaSpace{}%
\AgdaOperator{\AgdaFunction{if}}\AgdaSpace{}%
\AgdaBound{b}\AgdaSpace{}%
\AgdaOperator{\AgdaFunction{then}}\AgdaSpace{}%
\AgdaInductiveConstructor{false}\AgdaSpace{}%
\AgdaOperator{\AgdaFunction{else}}\AgdaSpace{}%
\AgdaInductiveConstructor{true}\<%
\\
%
\\[\AgdaEmptyExtraSkip]%
\>[0]\AgdaFunction{functionalNegation}\AgdaSpace{}%
\AgdaSymbol{:}\AgdaSpace{}%
\AgdaDatatype{𝔹}\AgdaSpace{}%
\AgdaSymbol{→}\AgdaSpace{}%
\AgdaSymbol{(}\AgdaDatatype{𝔹}\AgdaSpace{}%
\AgdaSymbol{→}\AgdaSpace{}%
\AgdaDatatype{𝔹}\AgdaSymbol{)}\AgdaSpace{}%
\AgdaSymbol{→}\AgdaSpace{}%
\AgdaSymbol{(}\AgdaDatatype{𝔹}\AgdaSpace{}%
\AgdaSymbol{→}\AgdaSpace{}%
\AgdaDatatype{𝔹}\AgdaSymbol{)}\<%
\\
\>[0]\AgdaFunction{functionalNegation}\AgdaSpace{}%
\AgdaBound{b}\AgdaSpace{}%
\AgdaBound{f}\AgdaSpace{}%
\AgdaSymbol{=}\AgdaSpace{}%
\AgdaOperator{\AgdaFunction{if}}\AgdaSpace{}%
\AgdaBound{b}\AgdaSpace{}%
\AgdaOperator{\AgdaFunction{then}}\AgdaSpace{}%
\AgdaBound{f}\AgdaSpace{}%
\AgdaOperator{\AgdaFunction{else}}\AgdaSpace{}%
\AgdaSymbol{λ}\AgdaSpace{}%
\AgdaBound{b'}\AgdaSpace{}%
\AgdaSymbol{→}\AgdaSpace{}%
\AgdaBound{f}\AgdaSpace{}%
\AgdaSymbol{(}\AgdaFunction{\textasciitilde{}}\AgdaSpace{}%
\AgdaBound{b'}\AgdaSymbol{)}\<%
\end{code}

This simple example leads us to one of the domains our subsequent grammars will
describe, like arithmetic (see \ref{npf}). We show how to inductively define
natural numbers in Agda, with the formation and introduction rules included
beside for contrast.

\begin{minipage}[t]{.4\textwidth}
\vspace{3mm}
\[
  \begin{prooftree}
    \infer1[]{ \vdash ℕ : {\rm type}}
  \end{prooftree}
\]
\[
  \begin{prooftree}
    \infer1[]{ \Gamma \vdash 0 : ℕ  }
  \end{prooftree}
  \quad \quad
  \begin{prooftree}
    \hypo{\Gamma \vdash n : ℕ}
    \infer1[]{ \Gamma \vdash (suc\ n) : ℕ  }
  \end{prooftree}
\]
\end{minipage}
\begin{minipage}[t]{.3\textwidth}
\begin{code}%
\>[0]\AgdaKeyword{data}\AgdaSpace{}%
\AgdaDatatype{ℕ}\AgdaSpace{}%
\AgdaSymbol{:}\AgdaSpace{}%
\AgdaPrimitive{Set}\AgdaSpace{}%
\AgdaKeyword{where}\<%
\\
\>[0][@{}l@{\AgdaIndent{0}}]%
\>[2]\AgdaInductiveConstructor{zero}\AgdaSpace{}%
\AgdaSymbol{:}\AgdaSpace{}%
\AgdaDatatype{ℕ}\<%
\\
%
\>[2]\AgdaInductiveConstructor{suc}%
\>[7]\AgdaSymbol{:}\AgdaSpace{}%
\AgdaDatatype{ℕ}\AgdaSpace{}%
\AgdaSymbol{→}\AgdaSpace{}%
\AgdaDatatype{ℕ}\<%
\end{code}
\end{minipage}

This is a recursive type, whereby pattern matching over $ℕ$ allows one to use an
induction hypothesis over the subtree and gurantee termination when making
recurive calls on the function being defined. We can define a recursion
principle for $ℕ$, which gives one the power to build iterators.
Again, we include the elimination and equality rules for syntactic
juxtaposition.

\[
  \begin{prooftree}
    \hypo{̌\Gamma \vdash X : {\rm type} }
    \hypo{\Gamma \vdash n : ℕ }
    \hypo{\Gamma \vdash e₀ : X}
    \hypo{\Gamma, x : ℕ, y : X \vdash e₁ : X }
    \infer4[]{\Gamma \vdash natrec\{e\;x.y.e₁\}(n) : X }
  \end{prooftree}
\]
$$\Gamma \vdash natrec\{e₀;x.y.e₁\}(n) \equiv e₀ : X$$
$$\Gamma \vdash natrec\{e₀;x.y.e₁\}(suc\ n) \equiv e₁[x := n,y := natrec\{e₀;x.y.e₁\}(n)] : X$$
\begin{code}%
\>[0]\AgdaFunction{natrec}\AgdaSpace{}%
\AgdaSymbol{:}\AgdaSpace{}%
\AgdaSymbol{\{}\AgdaBound{X}\AgdaSpace{}%
\AgdaSymbol{:}\AgdaSpace{}%
\AgdaPrimitive{Set}\AgdaSymbol{\}}\AgdaSpace{}%
\AgdaSymbol{→}\AgdaSpace{}%
\AgdaDatatype{ℕ}\AgdaSpace{}%
\AgdaSymbol{→}\AgdaSpace{}%
\AgdaBound{X}\AgdaSpace{}%
\AgdaSymbol{→}\AgdaSpace{}%
\AgdaSymbol{(}\AgdaDatatype{ℕ}\AgdaSpace{}%
\AgdaSymbol{→}\AgdaSpace{}%
\AgdaBound{X}\AgdaSpace{}%
\AgdaSymbol{→}\AgdaSpace{}%
\AgdaBound{X}\AgdaSymbol{)}\AgdaSpace{}%
\AgdaSymbol{→}\AgdaSpace{}%
\AgdaBound{X}\<%
\\
\>[0]\AgdaFunction{natrec}\AgdaSpace{}%
\AgdaInductiveConstructor{zero}\AgdaSpace{}%
\AgdaBound{e₀}\AgdaSpace{}%
\AgdaBound{e₁}\AgdaSpace{}%
\AgdaSymbol{=}\AgdaSpace{}%
\AgdaBound{e₀}\<%
\\
\>[0]\AgdaFunction{natrec}\AgdaSpace{}%
\AgdaSymbol{(}\AgdaInductiveConstructor{suc}\AgdaSpace{}%
\AgdaBound{n}\AgdaSymbol{)}\AgdaSpace{}%
\AgdaBound{e₀}\AgdaSpace{}%
\AgdaBound{e₁}\AgdaSpace{}%
\AgdaSymbol{=}\AgdaSpace{}%
\AgdaBound{e₁}\AgdaSpace{}%
\AgdaBound{n}\AgdaSpace{}%
\AgdaSymbol{(}\AgdaFunction{natrec}\AgdaSpace{}%
\AgdaBound{n}\AgdaSpace{}%
\AgdaBound{e₀}\AgdaSpace{}%
\AgdaBound{e₁}\AgdaSymbol{)}\<%
\end{code}

Since we are in a dependently typed setting, however, we prove theorems as well
as write programs. Therefore, we can see this recursion principle as a special
case of the induction principle \term{natind}, which represents the by induction
for the natural numbers. One may notice that while the types are different, the
programs \term{natrec} and \term{natind} are actually the same, up to
α-equivalence. One can therefore, as a corollary, actually just include the type
infomation and Agda can infer the speciliazation for you, as seen in
\term{natrec'} below.

\[
  \begin{prooftree}
    \hypo{̌\Gamma, x : ℕ \vdash X : {\rm type} }
    \hypo{\Gamma \vdash n : ℕ }
    \hypo{\Gamma \vdash e₀ : X[x := 0] }
    \hypo{\Gamma, y : ℕ, z : X[x := y] \vdash e₁ : X[x := suc\ y]}
    \infer4[]{\Gamma \vdash natind\{e₀,\;x.y.e₁\}(n) : X[x := n]}
  \end{prooftree}
\]
$$\Gamma \vdash natind\{e₀;x.y.e₁\}(n) \equiv e₀ : X[x := 0]$$
$$\Gamma \vdash natind\{e₀;x.y.e₁\}(suc\ n) \equiv e₁[x := n,y := natind\{e₀;x.y.e₁\}(n)] : X[x := suc\ n]$$
\begin{code}%
\>[0]\AgdaFunction{natind}\AgdaSpace{}%
\AgdaSymbol{:}\AgdaSpace{}%
\AgdaSymbol{\{}\AgdaBound{X}\AgdaSpace{}%
\AgdaSymbol{:}\AgdaSpace{}%
\AgdaDatatype{ℕ}\AgdaSpace{}%
\AgdaSymbol{→}\AgdaSpace{}%
\AgdaPrimitive{Set}\AgdaSymbol{\}}\AgdaSpace{}%
\AgdaSymbol{→}\AgdaSpace{}%
\AgdaSymbol{(}\AgdaBound{n}\AgdaSpace{}%
\AgdaSymbol{:}\AgdaSpace{}%
\AgdaDatatype{ℕ}\AgdaSymbol{)}\AgdaSpace{}%
\AgdaSymbol{→}\AgdaSpace{}%
\AgdaBound{X}\AgdaSpace{}%
\AgdaInductiveConstructor{zero}\AgdaSpace{}%
\AgdaSymbol{→}\AgdaSpace{}%
\AgdaSymbol{((}\AgdaBound{n}\AgdaSpace{}%
\AgdaSymbol{:}\AgdaSpace{}%
\AgdaDatatype{ℕ}\AgdaSymbol{)}\AgdaSpace{}%
\AgdaSymbol{→}\AgdaSpace{}%
\AgdaBound{X}\AgdaSpace{}%
\AgdaBound{n}\AgdaSpace{}%
\AgdaSymbol{→}\AgdaSpace{}%
\AgdaBound{X}\AgdaSpace{}%
\AgdaSymbol{(}\AgdaInductiveConstructor{suc}\AgdaSpace{}%
\AgdaBound{n}\AgdaSymbol{))}\AgdaSpace{}%
\AgdaSymbol{→}\AgdaSpace{}%
\AgdaBound{X}\AgdaSpace{}%
\AgdaBound{n}\<%
\\
\>[0]\AgdaFunction{natind}\AgdaSpace{}%
\AgdaInductiveConstructor{zero}\AgdaSpace{}%
\AgdaBound{base}\AgdaSpace{}%
\AgdaBound{step}\AgdaSpace{}%
\AgdaSymbol{=}\AgdaSpace{}%
\AgdaBound{base}\<%
\\
\>[0]\AgdaFunction{natind}\AgdaSpace{}%
\AgdaSymbol{(}\AgdaInductiveConstructor{suc}\AgdaSpace{}%
\AgdaBound{n}\AgdaSymbol{)}\AgdaSpace{}%
\AgdaBound{base}\AgdaSpace{}%
\AgdaBound{step}\AgdaSpace{}%
\AgdaSymbol{=}\AgdaSpace{}%
\AgdaBound{step}\AgdaSpace{}%
\AgdaBound{n}\AgdaSpace{}%
\AgdaSymbol{(}\AgdaFunction{natind}\AgdaSpace{}%
\AgdaBound{n}\AgdaSpace{}%
\AgdaBound{base}\AgdaSpace{}%
\AgdaBound{step}\AgdaSymbol{)}\<%
\\
%
\\[\AgdaEmptyExtraSkip]%
\>[0]\AgdaFunction{natrec'}\AgdaSpace{}%
\AgdaSymbol{:}\AgdaSpace{}%
\AgdaSymbol{\{}\AgdaBound{X}\AgdaSpace{}%
\AgdaSymbol{:}\AgdaSpace{}%
\AgdaPrimitive{Set}\AgdaSymbol{\}}\AgdaSpace{}%
\AgdaSymbol{→}\AgdaSpace{}%
\AgdaDatatype{ℕ}\AgdaSpace{}%
\AgdaSymbol{→}\AgdaSpace{}%
\AgdaBound{X}\AgdaSpace{}%
\AgdaSymbol{→}\AgdaSpace{}%
\AgdaSymbol{(}\AgdaDatatype{ℕ}\AgdaSpace{}%
\AgdaSymbol{→}\AgdaSpace{}%
\AgdaBound{X}\AgdaSpace{}%
\AgdaSymbol{→}\AgdaSpace{}%
\AgdaBound{X}\AgdaSymbol{)}\AgdaSpace{}%
\AgdaSymbol{→}\AgdaSpace{}%
\AgdaBound{X}\<%
\\
\>[0]\AgdaFunction{natrec'}\AgdaSpace{}%
\AgdaSymbol{=}\AgdaSpace{}%
\AgdaFunction{natind}\<%
\end{code}
We will defer the details of using induction and recursion principles for later
when we actually give examples of pidgin proofs some of our grammars can
handle.

% % \begin{code}[hide]%
\>[0]\<%
\\
\>[0]\AgdaKeyword{module}\AgdaSpace{}%
\AgdaModule{primitives}\AgdaSpace{}%
\AgdaKeyword{where}\<%
\\
\>[0]\<%
\end{code}

Formation rules, are given by the data declaration, followed by some number of
constructors which correspond to the 


A proof the proof-theoretic this looks like the following


\begin{prooftree}
  \hypo{ \Gamma, A &\vdash B }
  \infer1[abs]{ \Gamma &\vdash A\to B }
  \hypo{ \Gamma \vdash A }
  \infer2[app]{ \Gamma \vdash B }
\end{prooftree}


\begin{code}%
\>[0]\<%
\\
\>[0]\AgdaKeyword{data}\AgdaSpace{}%
\AgdaDatatype{𝔹}\AgdaSpace{}%
\AgdaSymbol{:}\AgdaSpace{}%
\AgdaPrimitive{Set}\AgdaSpace{}%
\AgdaKeyword{where}\<%
\\
\>[0][@{}l@{\AgdaIndent{0}}]%
\>[2]\AgdaInductiveConstructor{true}\AgdaSpace{}%
\AgdaSymbol{:}\AgdaSpace{}%
\AgdaDatatype{𝔹}\<%
\\
%
\>[2]\AgdaInductiveConstructor{false}\AgdaSpace{}%
\AgdaSymbol{:}\AgdaSpace{}%
\AgdaDatatype{𝔹}\<%
\\
\>[0]\<%
\end{code}


-- $ \frac{\Gamma, x : A \vdash b : B} {\Gamma \vdash \lambda x. b : A \rightarrow
-- B} $

\begin{code}%
\>[0]\<%
\\
\>[0]\AgdaComment{-- if\AgdaUnderscore{}then\AgdaUnderscore{}else\AgdaUnderscore{} : \{A : Set\} → 𝔹 → A → A → A}\<%
\\
\>[0]\AgdaComment{-- if true then a1 else a2 = a1}\<%
\\
\>[0]\AgdaComment{-- if false then a1 else a2 = a2}\<%
\\
\>[0]\<%
\end{code}

-- data ℕ : Type where
--   zero : ℕ
--   suc  : ℕ → ℕ

-- data List (A : Type) : Type where
  

-- data Vector : 



-- \begin{code}%
\>[0]\<%
\\
\>[0]\AgdaComment{-- Type : Set₁}\<%
\\
\>[0]\AgdaComment{-- Type = Set}\<%
\\
%
\\[\AgdaEmptyExtraSkip]%
\>[0]\AgdaComment{-- \textbackslash{}end\{code\}}\<%


% \caption{Agda} \label{fig:M3}
% \end{figure}

Additionally, we give the Agda code in \autoref{fig:M3}, so-as to see what the
end result of such a program would be. The astute reader will also notice a
semantic in the pidgin rendering error relative to the Agda implementation.
\codeword{fiber} has the type \codeword{it : Set} instead of something like
\codeword{(y : B) : Set}, and the y variable is unbound in the \codeword{fiber}
expression. This demonstrates that to design a grammar prioritizing
\emph{semantic adequacy} and subsequently trying to incorporate \emph{syntactic
completeness} becomes a very difficult problem. Depending on the application of
the grammar, the emphasis on this axis is most assuredly a choice one should
consider up front.

While both these grammars have their strengths and weaknesses, one shall see
shortly that the approach in this thesis, taking an actual programming language
parser in Backus-Naur Form Converter (BNFC), GFifying it, and trying to use the
abstract syntax to model natural language, gives in some sense a dual challenge,
where the abstract syntax remains simple, but its linearizations become
must increase in complexity.


% below is prior text, probably discard

% We now discuss the various iterations of code which experimented with NL aspects

% We should again emphasize the role of, in particular, Rantas two grammars, one
% formalizing logic, and the other working with a case study of a real text\cite{aarneHott}



% We now discuss the GF implementation, capable of parsing both natural language
% and Agda syntax. The parser was appropriated from the cubicaltt BNFC parser,
% de-cubified and then gf-ified. The languages are tightly coupled, so the
% translation is actually quite simple. Some main differences are:

% \begin{itemize}[noitemsep]

% \item GF treats abstract and concrete syntax seperately. This allows GF to
% support many concrete syntax implementation of a given grammar

% \item Fixity is dealt with at the concrete syntax layer in GF.  This allows for
% more refined control of fixity, but also results in difficulties : during
% linearization their can be the insertion of extra parens.

% \item GF supports dependent hypes and higher order abstract syntax, which makes
% it suitable to typecheck at the parsing stage. It would very interesting to see
% if this is interoperable with the current version of this work in later
% iterations [Todo - add github link referncing work I've done in this direction]

% \item GF also is enhanced by a PGF back-end, allowing an embedding of grammars
% into, among other languages, Haskell.

% \end{itemize}

% While GF is targeted towards natural language translation, there's nothing
% stopping it from being used as a PL tool as well, like, for instance, the
% front-end of a compiler. The innovation of this thesis is to combine both uses,
% thereby allowing translation between Controlled Natural Languages and
% programming languages.

% Example expressions the grammar can parse are seen below, which have been
% verified by hand to be isomorphic to the corresponding cubicaltt BNFC trees:

% \begin{verbatim}

% data bool : Set where true | false
% data nat : Set where zero | suc ( n : nat )
% caseBool ( x : Set ) ( y z : x ) : bool -> Set = split false -> y || true -> z
% indBool ( x : bool -> Set ) ( y : x false ) ( z : x true ) : ( b : bool ) -> x b = split false -> y || true  -> z
% funExt  ( a : Set )   ( b : a -> Set )   ( f g :  ( x : a )  -> b x )   ( p :  ( x : a )  -> ( b x )   ( f x ) == ( g x )  )  : (  ( y : a )  -> b y )  f == g = undefined
% foo ( b : bool ) : bool = b

% \end{verbatim}

% [Todo] add use cases


\section{Natural Number Proofs}

We now explore the ``main goal" of this work : proofs in GF. We commence with 
perhaps the most natural kind of proof one would expect, those over the
inductively defined natural numbers. As a proposed foundational alternative to
mathematics, dependent type theories allow types to depend on terms and
therefore allow propositions which include terms to be encoded as types.

In the simple type theory example, we included \emph{types} and
\emph{expressions} as distinct syntactic categories, whereby the linearization
of a type can't possibly call the linearization of a term. We now experiment
with a small dependently typed programming language with only $\Pi$-types. The
big difference for such a simple fragment like natural numbers in the dependent
setting is the fact that the recursion principle becomes an induction principle.
The types of a sub-expression being evaluated with a recursive call may depend
on the values being computing. Extra work is required in implementing
type-checkers for dependent language because they have to deal with a much more
sensitive and computationally expensive notion of type.

A dependent type theorist will assert that every time mathematicians use a
notion like $\mathbb{R}^n$, they are implicitly quantifying over the natural
numbers, namely $n$, and therefore are referring to a parameterized type, not a
\emph{set}. There are many more elaborate examples of dependency in mathematics,
but because this notation is ubiquitous, we note that the type theorist would
not be satisfied with many expressions from real analysis, because they assert
things about $\mathbb{R}^n$ all the time without ever proving anything by
induction over the numbers. Perhaps this seems pedantic, but it highlights a
large gap between the type-theorists syntactic approach to mathematics and the
mathematicians focus on the domain semantics of her field of interest.

Delaying a more in depth discussion of equality [ref later section], we here
assert that one proves equality in Agda by finding something that is \emph{irrefutably
equal} to itself, where the notion of irrefutably is in some sense gave birth to
subject matter of higher type theory.  Taking this for granted, we begin by
looking at one of the simplest natural numbers proof's : that addition is
associative.

\subsection{Associativity of Natural Numbers}

We define addition in Agda by recursion on the first argument, and notice that
the sum of two natural numbers is always a natural number. Therefore, agda has
the capacity to always compute the sum of two given natural numbers, via the
defining equations, and indeed $2+2=4$ is irrefutably true. Additionally, we
know that $0$ added to a number is always that number.

\begin{code}[hide]%
\>[0]\<%
\\
\>[0]\AgdaComment{-- \{-\# OPTIONS --omega-in-omega --type-in-type \#-\}}\<%
\\
%
\\[\AgdaEmptyExtraSkip]%
\>[0]\AgdaKeyword{module}\AgdaSpace{}%
\AgdaModule{nproof}\AgdaSpace{}%
\AgdaKeyword{where}\<%
\\
%
\\[\AgdaEmptyExtraSkip]%
\>[0]\AgdaKeyword{open}\AgdaSpace{}%
\AgdaKeyword{import}\AgdaSpace{}%
\AgdaModule{Agda.Builtin.Nat}\AgdaSpace{}%
\AgdaKeyword{renaming}\AgdaSpace{}%
\AgdaSymbol{(}\AgdaDatatype{Nat}\AgdaSpace{}%
\AgdaSymbol{to}\AgdaSpace{}%
\AgdaDatatype{ℕ}\AgdaSymbol{)}\AgdaSpace{}%
\AgdaKeyword{hiding}\AgdaSpace{}%
\AgdaSymbol{(}\AgdaOperator{\AgdaPrimitive{\AgdaUnderscore{}+\AgdaUnderscore{}}}\AgdaSymbol{)}\AgdaSpace{}%
\AgdaKeyword{public}\<%
\\
\>[0]\AgdaKeyword{import}\AgdaSpace{}%
\AgdaModule{Relation.Binary.PropositionalEquality}\AgdaSpace{}%
\AgdaSymbol{as}\AgdaSpace{}%
\AgdaModule{Eq}\<%
\\
\>[0]\AgdaKeyword{open}\AgdaSpace{}%
\AgdaModule{Eq}\AgdaSpace{}%
\AgdaKeyword{using}\AgdaSpace{}%
\AgdaSymbol{(}\AgdaOperator{\AgdaDatatype{\AgdaUnderscore{}≡\AgdaUnderscore{}}}\AgdaSymbol{;}\AgdaSpace{}%
\AgdaInductiveConstructor{refl}\AgdaSymbol{;}\AgdaSpace{}%
\AgdaFunction{trans}\AgdaSymbol{;}\AgdaSpace{}%
\AgdaFunction{sym}\AgdaSymbol{;}\AgdaSpace{}%
\AgdaFunction{cong}\AgdaSymbol{;}\AgdaSpace{}%
\AgdaFunction{cong-app}\AgdaSymbol{;}\AgdaSpace{}%
\AgdaFunction{subst}\AgdaSymbol{)}\<%
\\
\>[0]\AgdaKeyword{open}\AgdaSpace{}%
\AgdaModule{Eq.≡-Reasoning}\AgdaSpace{}%
\AgdaKeyword{using}\AgdaSpace{}%
\AgdaSymbol{(}\AgdaOperator{\AgdaFunction{begin\AgdaUnderscore{}}}\AgdaSymbol{;}\AgdaSpace{}%
\AgdaOperator{\AgdaFunction{\AgdaUnderscore{}≡⟨⟩\AgdaUnderscore{}}}\AgdaSymbol{;}\AgdaSpace{}%
\AgdaFunction{step-≡}\AgdaSymbol{;}\AgdaSpace{}%
\AgdaOperator{\AgdaFunction{\AgdaUnderscore{}∎}}\AgdaSymbol{)}\<%
\\
\>[0]\<%
\end{code}

\begin{code}[hide]%
\>[0]\AgdaFunction{ℕrec}\AgdaSpace{}%
\AgdaSymbol{:}\AgdaSpace{}%
\AgdaSymbol{\{}\AgdaBound{X}\AgdaSpace{}%
\AgdaSymbol{:}\AgdaSpace{}%
\AgdaPrimitive{Set}\AgdaSymbol{\}}\AgdaSpace{}%
\AgdaSymbol{->}\AgdaSpace{}%
\AgdaSymbol{(}\AgdaDatatype{ℕ}\AgdaSpace{}%
\AgdaSymbol{->}\AgdaSpace{}%
\AgdaBound{X}\AgdaSpace{}%
\AgdaSymbol{->}\AgdaSpace{}%
\AgdaBound{X}\AgdaSymbol{)}\AgdaSpace{}%
\AgdaSymbol{->}\AgdaSpace{}%
\AgdaBound{X}\AgdaSpace{}%
\AgdaSymbol{->}\AgdaSpace{}%
\AgdaDatatype{ℕ}\AgdaSpace{}%
\AgdaSymbol{->}\AgdaSpace{}%
\AgdaBound{X}\<%
\\
\>[0]\AgdaFunction{ℕrec}\AgdaSpace{}%
\AgdaBound{f}\AgdaSpace{}%
\AgdaBound{x}\AgdaSpace{}%
\AgdaInductiveConstructor{zero}\AgdaSpace{}%
\AgdaSymbol{=}\AgdaSpace{}%
\AgdaBound{x}\<%
\\
\>[0]\AgdaFunction{ℕrec}\AgdaSpace{}%
\AgdaBound{f}\AgdaSpace{}%
\AgdaBound{x}\AgdaSpace{}%
\AgdaSymbol{(}\AgdaInductiveConstructor{suc}\AgdaSpace{}%
\AgdaBound{n}\AgdaSymbol{)}\AgdaSpace{}%
\AgdaSymbol{=}\AgdaSpace{}%
\AgdaBound{f}\AgdaSpace{}%
\AgdaBound{n}\AgdaSpace{}%
\AgdaSymbol{(}\AgdaFunction{ℕrec}\AgdaSpace{}%
\AgdaBound{f}\AgdaSpace{}%
\AgdaBound{x}\AgdaSpace{}%
\AgdaBound{n}\AgdaSymbol{)}\<%
\\
%
\\[\AgdaEmptyExtraSkip]%
\>[0]\AgdaFunction{natrec}\AgdaSpace{}%
\AgdaSymbol{:}\AgdaSpace{}%
\AgdaSymbol{\{}\AgdaBound{X}\AgdaSpace{}%
\AgdaSymbol{:}\AgdaSpace{}%
\AgdaPrimitive{Set}\AgdaSymbol{\}}\AgdaSpace{}%
\AgdaSymbol{→}\AgdaSpace{}%
\AgdaDatatype{ℕ}\AgdaSpace{}%
\AgdaSymbol{→}\AgdaSpace{}%
\AgdaBound{X}\AgdaSpace{}%
\AgdaSymbol{→}\AgdaSpace{}%
\AgdaSymbol{(}\AgdaDatatype{ℕ}\AgdaSpace{}%
\AgdaSymbol{→}\AgdaSpace{}%
\AgdaBound{X}\AgdaSpace{}%
\AgdaSymbol{→}\AgdaSpace{}%
\AgdaBound{X}\AgdaSymbol{)}\AgdaSpace{}%
\AgdaSymbol{→}\AgdaSpace{}%
\AgdaBound{X}\<%
\\
\>[0]\AgdaFunction{natrec}\AgdaSpace{}%
\AgdaInductiveConstructor{zero}\AgdaSpace{}%
\AgdaBound{e₀}\AgdaSpace{}%
\AgdaBound{e₁}\AgdaSpace{}%
\AgdaSymbol{=}\AgdaSpace{}%
\AgdaBound{e₀}\<%
\\
\>[0]\AgdaFunction{natrec}\AgdaSpace{}%
\AgdaSymbol{(}\AgdaInductiveConstructor{suc}\AgdaSpace{}%
\AgdaBound{n}\AgdaSymbol{)}\AgdaSpace{}%
\AgdaBound{e₀}\AgdaSpace{}%
\AgdaBound{e₁}\AgdaSpace{}%
\AgdaSymbol{=}\AgdaSpace{}%
\AgdaBound{e₁}\AgdaSpace{}%
\AgdaBound{n}\AgdaSpace{}%
\AgdaSymbol{(}\AgdaFunction{natrec}\AgdaSpace{}%
\AgdaBound{n}\AgdaSpace{}%
\AgdaBound{e₀}\AgdaSpace{}%
\AgdaBound{e₁}\AgdaSymbol{)}\<%
\\
%
\\[\AgdaEmptyExtraSkip]%
\>[0]\AgdaFunction{natind}\AgdaSpace{}%
\AgdaSymbol{:}\AgdaSpace{}%
\AgdaSymbol{\{}\AgdaBound{X}\AgdaSpace{}%
\AgdaSymbol{:}\AgdaSpace{}%
\AgdaDatatype{ℕ}\AgdaSpace{}%
\AgdaSymbol{→}\AgdaSpace{}%
\AgdaPrimitive{Set}\AgdaSymbol{\}}\AgdaSpace{}%
\AgdaSymbol{→}\AgdaSpace{}%
\AgdaSymbol{(}\AgdaBound{n}\AgdaSpace{}%
\AgdaSymbol{:}\AgdaSpace{}%
\AgdaDatatype{ℕ}\AgdaSymbol{)}\AgdaSpace{}%
\AgdaSymbol{→}\AgdaSpace{}%
\AgdaBound{X}\AgdaSpace{}%
\AgdaInductiveConstructor{zero}\AgdaSpace{}%
\AgdaSymbol{→}\AgdaSpace{}%
\AgdaSymbol{((}\AgdaBound{n}\AgdaSpace{}%
\AgdaSymbol{:}\AgdaSpace{}%
\AgdaDatatype{ℕ}\AgdaSymbol{)}\AgdaSpace{}%
\AgdaSymbol{→}\AgdaSpace{}%
\AgdaBound{X}\AgdaSpace{}%
\AgdaBound{n}\AgdaSpace{}%
\AgdaSymbol{→}\AgdaSpace{}%
\AgdaBound{X}\AgdaSpace{}%
\AgdaSymbol{(}\AgdaInductiveConstructor{suc}\AgdaSpace{}%
\AgdaBound{n}\AgdaSymbol{))}\AgdaSpace{}%
\AgdaSymbol{→}\AgdaSpace{}%
\AgdaBound{X}\AgdaSpace{}%
\AgdaBound{n}\<%
\\
\>[0]\AgdaFunction{natind}\AgdaSpace{}%
\AgdaInductiveConstructor{zero}\AgdaSpace{}%
\AgdaBound{base}\AgdaSpace{}%
\AgdaBound{step}\AgdaSpace{}%
\AgdaSymbol{=}\AgdaSpace{}%
\AgdaBound{base}\<%
\\
\>[0]\AgdaFunction{natind}\AgdaSpace{}%
\AgdaSymbol{(}\AgdaInductiveConstructor{suc}\AgdaSpace{}%
\AgdaBound{n}\AgdaSymbol{)}\AgdaSpace{}%
\AgdaBound{base}\AgdaSpace{}%
\AgdaBound{step}\AgdaSpace{}%
\AgdaSymbol{=}\AgdaSpace{}%
\AgdaBound{step}\AgdaSpace{}%
\AgdaBound{n}\AgdaSpace{}%
\AgdaSymbol{(}\AgdaFunction{natind}\AgdaSpace{}%
\AgdaBound{n}\AgdaSpace{}%
\AgdaBound{base}\AgdaSpace{}%
\AgdaBound{step}\AgdaSymbol{)}\<%
\\
%
\\[\AgdaEmptyExtraSkip]%
\>[0]\<%
\end{code}


\begin{code}%
\>[0]\AgdaOperator{\AgdaFunction{\AgdaUnderscore{}+\AgdaUnderscore{}}}\AgdaSpace{}%
\AgdaSymbol{:}\AgdaSpace{}%
\AgdaDatatype{ℕ}\AgdaSpace{}%
\AgdaSymbol{→}\AgdaSpace{}%
\AgdaDatatype{ℕ}\AgdaSpace{}%
\AgdaSymbol{→}\AgdaSpace{}%
\AgdaDatatype{ℕ}\<%
\\
\>[0]\AgdaInductiveConstructor{zero}\AgdaSpace{}%
\AgdaOperator{\AgdaFunction{+}}\AgdaSpace{}%
\AgdaBound{n}\AgdaSpace{}%
\AgdaSymbol{=}\AgdaSpace{}%
\AgdaBound{n}\<%
\\
\>[0]\AgdaInductiveConstructor{suc}\AgdaSpace{}%
\AgdaBound{x}\AgdaSpace{}%
\AgdaOperator{\AgdaFunction{+}}\AgdaSpace{}%
\AgdaBound{n}\AgdaSpace{}%
\AgdaSymbol{=}\AgdaSpace{}%
\AgdaInductiveConstructor{suc}\AgdaSpace{}%
\AgdaSymbol{(}\AgdaBound{x}\AgdaSpace{}%
\AgdaOperator{\AgdaFunction{+}}\AgdaSpace{}%
\AgdaBound{n}\AgdaSymbol{)}\<%
\\
%
\\[\AgdaEmptyExtraSkip]%
\>[0]\AgdaFunction{2+2=4}\AgdaSpace{}%
\AgdaSymbol{:}\AgdaSpace{}%
\AgdaNumber{2}\AgdaSpace{}%
\AgdaOperator{\AgdaFunction{+}}\AgdaSpace{}%
\AgdaNumber{2}\AgdaSpace{}%
\AgdaOperator{\AgdaDatatype{≡}}\AgdaSpace{}%
\AgdaNumber{4}\<%
\\
\>[0]\AgdaFunction{2+2=4}\AgdaSpace{}%
\AgdaSymbol{=}\AgdaSpace{}%
\AgdaInductiveConstructor{refl}\<%
\\
%
\\[\AgdaEmptyExtraSkip]%
\>[0]\AgdaFunction{0+n=n}\AgdaSpace{}%
\AgdaSymbol{:}\AgdaSpace{}%
\AgdaSymbol{∀}\AgdaSpace{}%
\AgdaSymbol{(}\AgdaBound{n}\AgdaSpace{}%
\AgdaSymbol{:}\AgdaSpace{}%
\AgdaDatatype{ℕ}\AgdaSymbol{)}\AgdaSpace{}%
\AgdaSymbol{→}\AgdaSpace{}%
\AgdaNumber{0}\AgdaSpace{}%
\AgdaOperator{\AgdaFunction{+}}\AgdaSpace{}%
\AgdaBound{n}\AgdaSpace{}%
\AgdaOperator{\AgdaDatatype{≡}}\AgdaSpace{}%
\AgdaBound{n}\<%
\\
\>[0]\AgdaFunction{0+n=n}\AgdaSpace{}%
\AgdaBound{n}\AgdaSpace{}%
\AgdaSymbol{=}\AgdaSpace{}%
\AgdaInductiveConstructor{refl}\<%
\end{code}

\begin{code}[hide]%
\>[0]\AgdaKeyword{postulate}\<%
\\
\>[0][@{}l@{\AgdaIndent{0}}]%
\>[2]\AgdaPostulate{roadblockn}\AgdaSpace{}%
\AgdaSymbol{:}\AgdaSpace{}%
\AgdaSymbol{∀}\AgdaSpace{}%
\AgdaSymbol{(}\AgdaBound{m}\AgdaSpace{}%
\AgdaSymbol{:}\AgdaSpace{}%
\AgdaDatatype{ℕ}\AgdaSymbol{)}\AgdaSpace{}%
\AgdaSymbol{→}\AgdaSpace{}%
\AgdaBound{m}\AgdaSpace{}%
\AgdaOperator{\AgdaFunction{+}}\AgdaSpace{}%
\AgdaInductiveConstructor{zero}\AgdaSpace{}%
\AgdaOperator{\AgdaDatatype{≡}}\AgdaSpace{}%
\AgdaBound{m}\AgdaSpace{}%
\AgdaComment{-- identity cancels on the left}\<%
\\
%
\\[\AgdaEmptyExtraSkip]%
\>[0]\AgdaFunction{roadblock}\AgdaSpace{}%
\AgdaSymbol{=}\AgdaSpace{}%
\AgdaSymbol{λ}\AgdaSpace{}%
\AgdaSymbol{(}\AgdaBound{n}\AgdaSpace{}%
\AgdaSymbol{:}\AgdaSpace{}%
\AgdaDatatype{ℕ}\AgdaSymbol{)}\AgdaSpace{}%
\AgdaSymbol{→}\AgdaSpace{}%
\AgdaPostulate{roadblockn}\AgdaSpace{}%
\AgdaBound{n}\<%
\end{code}

\begin{code}[hide]%
\>[0]\AgdaFunction{n+0=n}\AgdaSpace{}%
\AgdaSymbol{:}\AgdaSpace{}%
\AgdaSymbol{∀}\AgdaSpace{}%
\AgdaSymbol{(}\AgdaBound{n}\AgdaSpace{}%
\AgdaSymbol{:}\AgdaSpace{}%
\AgdaDatatype{ℕ}\AgdaSymbol{)}\AgdaSpace{}%
\AgdaSymbol{→}\AgdaSpace{}%
\AgdaBound{n}\AgdaSpace{}%
\AgdaOperator{\AgdaFunction{+}}\AgdaSpace{}%
\AgdaNumber{0}\AgdaSpace{}%
\AgdaOperator{\AgdaDatatype{≡}}\AgdaSpace{}%
\AgdaBound{n}\<%
\\
\>[0]\AgdaFunction{n+0=n}\AgdaSpace{}%
\AgdaSymbol{=}\AgdaSpace{}%
\AgdaFunction{roadblock}\<%
\end{code}



\subsection{What is Equality?}

% quote peter dybjer at oplss
% all out assault on equality

\begin{displayquote}
... the univalence axiom validates the common, but formally unjustified, practice
of identifying isomorphic objects. [cite hottbook]
\end{displayquote}


Mathematicians, and most people generally, have an intuition
for equality, that of an identification between two pieces of information
which intuitively must be the same thing, i.e. $2+2=4$. The philosophically
inclined might ask about identification generally. We showcase different
notions of identifying things in mathematics, logic, and type theory :

\begin{itemize}
\item Equivalence of propositions
\item Equality of sets
\item Equality of members of sets
\item Isomorphism of structures
\item Equality of terms
\item Equality of types
\end{itemize}

While there are notions of equality, sameness, or identification outside of
these formal domains, we don't dare take a philosophical stab at these notions here.
Earlier, we saw two notions of equality in type theory, the judgmental equality
in our introduction to MLTT, and the propositional equality which was used in
the twin prime conjecture. Judgmental equality is the means of computing, for
instance, that $2+2=4$, for there is no way of proving this other than appealing
to the definition of addition. Propositional equality, on the other hand, is
actually a type. It is defined as follows in Agda, with an accompanying natural
language definition from [cite hottbook] :

\begin{code}[hide]%
\>[0]\AgdaKeyword{module}\AgdaSpace{}%
\AgdaModule{equality}\AgdaSpace{}%
\AgdaKeyword{where}\<%
\end{code}
\begin{code}%
\>[0][@{}l@{\AgdaIndent{1}}]%
\>[2]\AgdaKeyword{data}\AgdaSpace{}%
\AgdaOperator{\AgdaDatatype{\AgdaUnderscore{}≡'\AgdaUnderscore{}}}\AgdaSpace{}%
\AgdaSymbol{\{}\AgdaBound{A}\AgdaSpace{}%
\AgdaSymbol{:}\AgdaSpace{}%
\AgdaPrimitive{Set}\AgdaSymbol{\}}\AgdaSpace{}%
\AgdaSymbol{:}\AgdaSpace{}%
\AgdaSymbol{(}\AgdaBound{a}\AgdaSpace{}%
\AgdaBound{b}\AgdaSpace{}%
\AgdaSymbol{:}\AgdaSpace{}%
\AgdaBound{A}\AgdaSymbol{)}\AgdaSpace{}%
\AgdaSymbol{→}\AgdaSpace{}%
\AgdaPrimitive{Set}\AgdaSpace{}%
\AgdaKeyword{where}\<%
\\
\>[2][@{}l@{\AgdaIndent{0}}]%
\>[4]\AgdaInductiveConstructor{r}\AgdaSpace{}%
\AgdaSymbol{:}\AgdaSpace{}%
\AgdaSymbol{(}\AgdaBound{a}\AgdaSpace{}%
\AgdaSymbol{:}\AgdaSpace{}%
\AgdaBound{A}\AgdaSymbol{)}\AgdaSpace{}%
\AgdaSymbol{→}\AgdaSpace{}%
\AgdaBound{a}\AgdaSpace{}%
\AgdaOperator{\AgdaDatatype{≡'}}\AgdaSpace{}%
\AgdaBound{a}\<%
\end{code}
\begin{definition}
  The formation rule says that given a type $A:\UU$ and two elements $a,b:A$, we can form the type $(\id[A]{a}{b}):\UU$ in the same universe.
  The basic way to construct an element of $\id{a}{b}$ is to know that $a$ and $b$ are the same.
  Thus, the introduction rule is a dependent function
  \[\refl{} : \prod_{a:A} (\id[A]{a}{a}) \]
  called \define{reflexivity},
  which says that every element of $A$ is equal to itself (in a specified way).  We regard $\refl{a}$ as being the
  constant path %path\indexdef{path!constant}\indexsee{loop!constant}{path, constant}
  at the point $a$.
\end{definition}


The astute might ask, what
does it mean to ``construct an element of $\id{a}{b}$''? For the mathematician
use to thinking in terms of sets $\{\id{a}{b} \mid a,b \in \mathbb{N} \}$ isn't
a well-defined notion. Due to its use of the axiom of extensionality, the set
theoretic notion of equality is, no suprise, extensional.  This means that sets
are identified when they have the same elements, and equality is therefore
external to the notion of set. To inhabit a type means to provide evidence for
that inhabitation. The reflexivity constructor is therefore a means of
providing evidence of an equality. This evidence approach is disctinctly
constructive, and a big reason why classical and constructive mathematics,
especially when treated in an intuitionistic type theory suitable for a
programming language implementation, are such different beasts.

In Martin-Löf Type Theory, there are two fundamental notions of equality,
propositional and definitional.  While propositional equality is inductively
defined (as above) as a type which may have possibly more than one inhabitant,
definitional equality, denoted $-\equiv -$ and perhaps more aptly named
computational equality, is familiarly what most people think of as equality.
Namely, two terms which compute to the same canonical form are computationally
equal. In intensional type theory, propositional equality is a weaker notion
than computational equality : all propositionally equal terms are
computationally equal. However, computational equality does not imply
propistional equality - if it does, then one enters into the space of
extensional type theory.

Prior to the homotopical interpretation of identity types, debates about
extensional and intensional type theories centred around two features or bugs :
extensional type theory sacrificed decideable type checking, while intensional
type theories required extra beauracracy when dealing with equality in proofs.
One approach in intensional type theories treated types as setoids, therefore
leading to so-called ``Setoid Hell''. These debates reflected Martin-Löf's
flip-flopping on the issue. His seminal 1979 Constructive Mathematics and
Computer Programming, which took an extensional view, was soon betrayed by
lectures he gave soon thereafter in Padova in 1980.  Martin-Löf was a born
again intensional type theorist.  These Padova lectures were later published in
the "Bibliopolis Book", and went on to inspire the European (and Gothenburg in
particular) approach to implementing proof assitants, whereas the
extensionalists were primarily eminating from Robert Constable's group at
Cornell.

This tension has now been at least partially resolved, or at the very least
clarified, by an insight Voevodsky was apparently most proud of : the
introduction of h-levels. We'll delegate these details to more advanced references, it
is mentioned here to indicate that extensional type theory was really ``set
theory'' in disguise, in that it collapses the higher path structure of
identity types. The work over the past 10 years has elucidated the intensional
and extensional positions. HoTT, by allowing higher paths, is unashamedly
intentional, and admits a collapse into the extensional universe if so desired.
We now the examine the structure induced by this propositional equality.


\subsection{Ranta's HoTT Grammar}

In 2014, Ranta gave an unpublished talk at the Stockholm Mathematics Seminar
\cite{aarneHott}. Fortunately the code is available, although many of the design
choices aren't documented in the grammar. This project aimed to provide a
translation like the one desired in our current work, but it took a real piece
of mathematics text as the main influence on the design of the abstract syntax.

This work took a page of text from Peter Aczel's book which more or less goes
over standard HoTT definitions and theorems. The grammar allows the translation
of the latex document in English to the same document in French, and to a pidgin
logical language. The central motivation of this grammar was to capture entirely
``real" natural language mathematics, i.e. that which was written for the
mathematician. Therefore, it isn't reminiscent of the slender abstract syntax
the type theorist adores, and sacrificed ``syntactic completeness" for
``semantic adequacy". This means that the abstract syntax is much larger and
very expressive, but it no longer becomes easy to reason about and additionally
quite ad-hoc. Another defect is that this grammar overgenerates, so producing a
unique parse from the PL side would require a significant amount of refactoring.
Nonetheless, it is presumably possible to carve a subset of the
GF HoTT abstract file to accommodate an Agda program, but one encounters rocks
as soon as one begins to dig.

In \autoref{fig:R1} one can see different syntactic presentations of a notion of
\emph{contractability}, that a space is deformable into a single point, or that a Type
is actually inhabited by a unique term. Some rendered latex taken verbatim from
Ranta's test code, compared with the translated pidgin logic code (after
refactoring of Ranta's linearization scheme) and an Agda program. We see that it was
fairly easy to get the notation for our cubicalTT grammar [ref cubicaltt].
When parsing the logical form, unfortunately, the grammar is incredibly ambiguous.


\begin{code}[hide]%
\>[0]\AgdaSymbol{\{-\#}\AgdaSpace{}%
\AgdaKeyword{OPTIONS}\AgdaSpace{}%
\AgdaPragma{--omega-in-omega}\AgdaSpace{}%
\AgdaPragma{--type-in-type}\AgdaSpace{}%
\AgdaSymbol{\#-\}}\<%
\\
%
\\[\AgdaEmptyExtraSkip]%
\>[0]\AgdaKeyword{module}\AgdaSpace{}%
\AgdaModule{ContrClean}\AgdaSpace{}%
\AgdaKeyword{where}\<%
\\
%
\\[\AgdaEmptyExtraSkip]%
\>[0]\AgdaKeyword{open}\AgdaSpace{}%
\AgdaKeyword{import}\AgdaSpace{}%
\AgdaModule{Agda.Builtin.Sigma}\AgdaSpace{}%
\AgdaKeyword{public}\<%
\\
%
\\[\AgdaEmptyExtraSkip]%
\>[0]\AgdaKeyword{variable}\<%
\\
\>[0][@{}l@{\AgdaIndent{0}}]%
\>[2]\AgdaGeneralizable{A}\AgdaSpace{}%
\AgdaGeneralizable{B}\AgdaSpace{}%
\AgdaSymbol{:}\AgdaSpace{}%
\AgdaPrimitive{Set}\<%
\\
%
\\[\AgdaEmptyExtraSkip]%
\>[0]\AgdaKeyword{data}\AgdaSpace{}%
\AgdaOperator{\AgdaDatatype{\AgdaUnderscore{}≡\AgdaUnderscore{}}}\AgdaSpace{}%
\AgdaSymbol{\{}\AgdaBound{A}\AgdaSpace{}%
\AgdaSymbol{:}\AgdaSpace{}%
\AgdaPrimitive{Set}\AgdaSymbol{\}}\AgdaSpace{}%
\AgdaSymbol{(}\AgdaBound{a}\AgdaSpace{}%
\AgdaSymbol{:}\AgdaSpace{}%
\AgdaBound{A}\AgdaSymbol{)}\AgdaSpace{}%
\AgdaSymbol{:}\AgdaSpace{}%
\AgdaBound{A}\AgdaSpace{}%
\AgdaSymbol{→}\AgdaSpace{}%
\AgdaPrimitive{Set}\AgdaSpace{}%
\AgdaKeyword{where}\<%
\\
\>[0][@{}l@{\AgdaIndent{0}}]%
\>[2]\AgdaInductiveConstructor{r}\AgdaSpace{}%
\AgdaSymbol{:}\AgdaSpace{}%
\AgdaBound{a}\AgdaSpace{}%
\AgdaOperator{\AgdaDatatype{≡}}\AgdaSpace{}%
\AgdaBound{a}\<%
\\
%
\\[\AgdaEmptyExtraSkip]%
\>[0]\AgdaKeyword{infix}\AgdaSpace{}%
\AgdaNumber{20}\AgdaSpace{}%
\AgdaOperator{\AgdaDatatype{\AgdaUnderscore{}≡\AgdaUnderscore{}}}\<%
\\
%
\\[\AgdaEmptyExtraSkip]%
\>[0]\AgdaFunction{id}\AgdaSpace{}%
\AgdaSymbol{:}\AgdaSpace{}%
\AgdaGeneralizable{A}\AgdaSpace{}%
\AgdaSymbol{→}\AgdaSpace{}%
\AgdaGeneralizable{A}\<%
\\
\>[0]\AgdaFunction{id}\AgdaSpace{}%
\AgdaSymbol{=}\AgdaSpace{}%
\AgdaSymbol{λ}\AgdaSpace{}%
\AgdaBound{z}\AgdaSpace{}%
\AgdaSymbol{→}\AgdaSpace{}%
\AgdaBound{z}\<%
\\
\>[0]\<%
\end{code}

\begin{figure}[H]
\textbf{Definition}:
A type $A$ is contractible, if there is $a : A$, called the center of contraction, such that for all $x : A$, $\equalH {a}{x}$.
\caption{Rendered Latex} \label{fig:R1}
\begin{verbatim}
isContr ( A : Set ) : Set = ( a : A ) ( * ) ( ( x : A ) -> Id ( a ) ( x ) )
\end{verbatim}
\begin{code}%
\>[0]\AgdaFunction{isContr}\AgdaSpace{}%
\AgdaSymbol{:}\AgdaSpace{}%
\AgdaSymbol{(}\AgdaBound{A}\AgdaSpace{}%
\AgdaSymbol{:}\AgdaSpace{}%
\AgdaPrimitive{Set}\AgdaSymbol{)}\AgdaSpace{}%
\AgdaSymbol{→}\AgdaSpace{}%
\AgdaPrimitive{Set}\<%
\\
\>[0]\AgdaFunction{isContr}\AgdaSpace{}%
\AgdaBound{A}\AgdaSpace{}%
\AgdaSymbol{=}%
\>[13]\AgdaRecord{Σ}\AgdaSpace{}%
\AgdaBound{A}\AgdaSpace{}%
\AgdaSymbol{λ}\AgdaSpace{}%
\AgdaBound{a}\AgdaSpace{}%
\AgdaSymbol{→}\AgdaSpace{}%
\AgdaSymbol{(}\AgdaBound{x}\AgdaSpace{}%
\AgdaSymbol{:}\AgdaSpace{}%
\AgdaBound{A}\AgdaSymbol{)}\AgdaSpace{}%
\AgdaSymbol{→}\AgdaSpace{}%
\AgdaSymbol{(}\AgdaBound{a}\AgdaSpace{}%
\AgdaOperator{\AgdaDatatype{≡}}\AgdaSpace{}%
\AgdaBound{x}\AgdaSymbol{)}\<%
\end{code}
\caption{Contractibility} \label{fig:R2}
\end{figure}

In \autoref{fig:R2}, we show the different syntax presentations of the
\emph{equivalence}, which is merely a bijection when restricted to sets. This is
of such fundamental idea in mathematics and HoTT in particular that it merits
its own chapter in [cite hott], but we only showcase one of its many equivalent
definitions. We see that the pidgin syntax is stuck with the anaphoric artifact
from the bloated abstract syntax, \codeword{fiber} has the type \codeword{it :
Set} instead of something like \codeword{(y : B) : Set}, and the y variable is
unbound in the \codeword{fiber} expression. This may presumably be fixed with a
few hours more of tinkering, but becomes even more complicated when not just
defining new types, but actually writing real mathematical proofs.

\begin{figure}[H]
\textbf{Definition}:
A map $f : \arrowH {A}{B}$ is an equivalence, if for all $y : B$, its fiber, $\comprehensionH {x}{A}{\equalH {\appH {f}{x}}{y}}$, is contractible.
We write $\equivalenceH {A}{B}$, if there is an equivalence $\arrowH {A}{B}$.
\begin{verbatim}
Equivalence ( f : A -> B ) : Set =
  ( y : B ) -> ( isContr ( fiber it ) ) ; ; ;
  fiber it : Set = ( x : A ) ( * ) ( Id ( f ( x ) ) ( y ) )
\end{verbatim}
\begin{code}%
\>[0]\AgdaFunction{Equivalence}\AgdaSpace{}%
\AgdaSymbol{:}\AgdaSpace{}%
\AgdaSymbol{(}\AgdaBound{A}\AgdaSpace{}%
\AgdaBound{B}\AgdaSpace{}%
\AgdaSymbol{:}\AgdaSpace{}%
\AgdaPrimitive{Set}\AgdaSymbol{)}\AgdaSpace{}%
\AgdaSymbol{→}\AgdaSpace{}%
\AgdaSymbol{(}\AgdaBound{f}\AgdaSpace{}%
\AgdaSymbol{:}\AgdaSpace{}%
\AgdaBound{A}\AgdaSpace{}%
\AgdaSymbol{→}\AgdaSpace{}%
\AgdaBound{B}\AgdaSymbol{)}\AgdaSpace{}%
\AgdaSymbol{→}\AgdaSpace{}%
\AgdaPrimitive{Set}\<%
\\
\>[0]\AgdaFunction{Equivalence}\AgdaSpace{}%
\AgdaBound{A}\AgdaSpace{}%
\AgdaBound{B}\AgdaSpace{}%
\AgdaBound{f}\AgdaSpace{}%
\AgdaSymbol{=}\AgdaSpace{}%
\AgdaSymbol{∀}\AgdaSpace{}%
\AgdaSymbol{(}\AgdaBound{y}\AgdaSpace{}%
\AgdaSymbol{:}\AgdaSpace{}%
\AgdaBound{B}\AgdaSymbol{)}\AgdaSpace{}%
\AgdaSymbol{→}\AgdaSpace{}%
\AgdaFunction{isContr}\AgdaSpace{}%
\AgdaSymbol{(}\AgdaFunction{fiber'}\AgdaSpace{}%
\AgdaBound{y}\AgdaSymbol{)}\<%
\\
\>[0][@{}l@{\AgdaIndent{0}}]%
\>[2]\AgdaKeyword{where}\<%
\\
\>[2][@{}l@{\AgdaIndent{0}}]%
\>[4]\AgdaFunction{fiber'}\AgdaSpace{}%
\AgdaSymbol{:}\AgdaSpace{}%
\AgdaSymbol{(}\AgdaBound{y}\AgdaSpace{}%
\AgdaSymbol{:}\AgdaSpace{}%
\AgdaBound{B}\AgdaSymbol{)}\AgdaSpace{}%
\AgdaSymbol{→}\AgdaSpace{}%
\AgdaPrimitive{Set}\<%
\\
%
\>[4]\AgdaFunction{fiber'}\AgdaSpace{}%
\AgdaBound{y}\AgdaSpace{}%
\AgdaSymbol{=}\AgdaSpace{}%
\AgdaRecord{Σ}\AgdaSpace{}%
\AgdaBound{A}\AgdaSpace{}%
\AgdaSymbol{(λ}\AgdaSpace{}%
\AgdaBound{x}\AgdaSpace{}%
\AgdaSymbol{→}\AgdaSpace{}%
\AgdaBound{y}\AgdaSpace{}%
\AgdaOperator{\AgdaDatatype{≡}}\AgdaSpace{}%
\AgdaBound{f}\AgdaSpace{}%
\AgdaBound{x}\AgdaSymbol{)}\<%
\end{code}
\caption{Contractibility} \label{fig:R3}
\end{figure}


To extend this grammar to accommodate a chapter worth of material, let alone a
book, will not just require extending the lexicon, but encountering other
syntactic phenomena that will further be difficult to compress when defining
Agda's concrete syntax. This demonstrates that to design a grammar prioritizing
\emph{semantic adequacy} and subsequently trying to incorporate \emph{syntactic
completeness} becomes a very difficult problem. Depending on the application of
the grammar, the emphasis on this axis is most assuredly a choice one should
consider up front.

The next grammar, taking an actual programming language
parser in Backus-Naur Form Converter (BNFC), GFifying it, and trying to use the
abstract syntax to model natural language, gives in some sense a dual challenge,
where the abstract syntax remains simple as in our dependently typed grammar,
but its linearizations become increasingly complex, especially when generating
natural language.

\subsection{cubicalTT Grammar}

Cubical type theories arose out of the desire to give a complete computational
interpretation to HoTT, whereby nonviolence would become a theorem rather than
an axiom \cite{cohen:hal-01378906}. The utility of this is that canonicity, the
property of an expression having a irreducible normal form, is satisfied for all
expressions. Univalence, by introducing a type without computational behavior,
means that the constructivist using Agda will be able to define terms which
don't normalize.

The origin of cubical, looking beyond simplicial models of type theory to
cubical categories instead \cite{bezem2017univalence}, gave a blueprint for a
totally new type theory which natively supports proving functional
extensionality, which is a especially important for mathematicians. The ideas of
cubical became the origin for a new series of proof assistants, cubical [cite
https://github.com/simhu/cubical] and cubicaltt [cite
https://github.com/mortberg/cubicaltt], and Cubical Agda \cite{cubicalAgda}, as
well as other in originating from Robert Constables disciples in the NuPrl
tradition [cite redprl, redtt, jonprl]. cubicalTT, which was relatively
complete, had an unambiguous BNFC grammar, more or less represents a kernel of
Agda with cubical primitives. This final grammar, which we don't as cubicalTT,
took the actual cubicalTT grammar and GFified the subset which is in the
intersection with vanilla Agda. Extending our GF version to include cubical
primitives would facilitate the extension of the work to Cubical Agda, and we
hope future endeavors will go in this direction. Cubical Agda
supports Higher Inductive Types natively and is capable of all types of new
constructions [cite stuff] not mentioned in the HoTT book, but is also
incredibly experimental, with large changes to the standard library constantly
underway as in [refer intro].

Our grammar for vanilla dependent $\Pi$-types [refer earlier section] was actually
a subset of the current cubicalTT abstract syntax. We give a brief sketch of the
algorithm to go between a BNFC grammar and a GF grammar. BNFC essentially
combines the abstract and concrete syntax, enabling a hierarchy of numbered
expressions \term{ExpN} to minimize use of parentheses. So, given m names and
choosing $Name_i$, with the accompanying rule :

$$Name_i.\; ReturnCat_{i_n} ::= s^0_{i}\;C^0_{i_0}\;...\;C^{n-1}_{i_{n-1}}\;s^n_{i}\;;$$

where string $s^i_j$ may be empty and the $k$ in the $i^{th}_k$ subscript represents the 
precedence number of a category. These precedences are indicated with a
\term{Coercions N} keyword in BNFC. We can produce the following in GF.

$$cat\; Name_i\; \bigcap\{ReturnCat_i,C^0,..., C^{n-1}\}\;;$$
$$fun\; Name_i\:{:} C^0 \rightarrow ... \rightarrow C^{n-1} \rightarrow ReturnCat_i $$
$$lincat \: \bigcap\{ReturnCat_i,C^0,..., C^{n-1}\}\;; = TermPrec$$
$$lin \; Name_i\;c^0\;... \;c^n = mkPrec(i_n,(s^0_{i}\texttt{++}usePrec(i_0+1,c^0)\texttt{++}...\texttt{++}usePrec(i_{n-1}+1,c^{n-1})\texttt{++}s^n_{i})) ;$$

where $c^j \in C^j \; \forall i,j$, and \term{usePrec} and \term{mkPrec} come
from the RGL, as seen earlier. We also note that some \term{lincat} might
actually just be strings (or something else), for it is only when a precedence
is observed that the \codeword{TermPrec} is applicable. The use of
\term{usePrec} is only applicable when $i_k$ isn't empty. Additionally, this
doesn't account for the fact that already some categories may have been
witnessed in which case we want to intersect over the whole set of rules at
once. We reiterate the examples from the simply typed lambda calculus. The BNFC
code results in the GF code immediately below.

\begin{verbatim}
--BNFC
Lam. Exp  ::= "\\" [PTele] "->" Exp ;
Fun. Exp1 ::= Exp2 "->" Exp1 ;
-- GF
cat Exp ; PTele ;
fun
  Lam : [PTele] -> Exp -> Exp ;
  Fun : Exp -> Exp -> Exp ;
lincat Exp = TermPrec ; [PTele] = Str ;
lin 
  Lam pt e = mkPrec 0 ("\\" ++ pt ++ "->" ++ usePrec 0 e) ;
  Fun = mkPrec 1 (usePrec 2 x ++ "->" ++ usePrec 1 y) ;
\end{verbatim}

This more or less elaborates exactly how to implement a programming language
with unambiguous parsing in GF. There is also a simple means of translating
lists, including BNFC's \term{separator} and \term{terminator} keywords during
the linearization process. Finally, there is a custom \term{token} keyword, and
this is perhaps the most important feature absent in GF.
Because BNFC generates Haskell code reminiscent of the PGF embedding, it would
also be possible to translate the trees directly, if parsing complexity with GF
was found to be slower than BNFC.

Most interestingly is to look at what is absent in BNFC, namely, the ability to
add records and paremeters into the linearization types generally, although
these GF features are implicitly used to encode precedence. For one could add
unique categories in GF $Exp_1,...,Exp_n$, but this would clutter the abstract
syntax with information which isn't \emph{semantically} relevant. And while the
Haskell code generated by BNFC for cubicalTT is sent through a resolver to the
\emph{actual} abstract syntax used by the type-checker and evaluator, the fact
that it parses the concrete syntax into an appropriate intermediary form is
enough for our purposes.

We give the full grammar, including examples, in the appendix \ref{cubicaltt}.

\subsubsection{Difficulties}

While our grammar certainly supports a real programming language syntax, modulo
a few quarks, linearizing to a CNL for mathematics was not implemented due to
time constraints, and the difficulties already encountered for an even simpler
programming language \ref{assoc}, namely that types and terms in dependent type
theory can be of just about any grammatical category, where we list a few :

\begin{itemize}
\item nouns, ``zero"
\item adjectives, ``prime"
\item verbs, ``add"
\item verb phrase, ``apply the function to the subset of..."
\item sentence, ``if x is odd, then y is even"
\item paragraph or more, ``suppose x. then by y we know z. hence, w. but the v
  gives additionally gives us..."
\end{itemize}

In \cite{rantaZ}, the authors, generating human readable natural language from
specifications, used a word type with many different fields for different
grammatical categories (with the same grammatical categories sometimes
accounting for multiple fields), in addition to symbolic fields. While deemed
successful by the client, it would be interesting to apply this methodology to 
cubicalTT the grammar, and see how it scales once one begins to add more
of Agda's capabilities. Their system also involved other components, like
haskell transformations, and it is uncertain how these specific approaches would
also allow for the generation of more \emph{semantically adequate} language.

Other issues encountered in this grammar were Agda's pattern matching, whereby
arguments are arranged in a matrix, as opposed to explicit cases, or \emph{splits}.
While cubicalTT allows syntax like

\begin{verbatim}
equalNat : nat -> nat -> bool = split
    zero -> split@ ( nat -> bool ) with
      zero  -> true
      suc n -> false
    suc m -> split@ ( nat -> bool ) with
      zero  -> false
      suc n -> equalNat m n
\end{verbatim}

The problem is that when linearizing a split, one cannot know how many further
splits will take place, and so going from this form to the more ``readable" Agda
code below is outside of GF's linearization capabilities - although a proof of
this fact would require advanced mathematical capabilities.

\begin{verbatim}
equalNat : nat → nat → bool
equalNat zero zero = true ; 
equalNat zero (suc n2) = false ;
equalNat (suc n1) zero = false ;
equalNat (suc n1) (suc n2) = equalNat n1 n2
\end{verbatim}

One could instead just introduce a new form of declarations in the abstract
syntax so-as to allow for \term{equalNat}, but this would require more Haskell
overhead to allow for the correct AST transformations. 

The way lists are dealt with in natural language vs. programming
languages present obstacles, because the RGL's support for lists require certain
numbers of categories in the end node, e.g. \codeword{cat[2]}, whereas our Agda
grammar may instead have \codeword{cat[1]} or \codeword{cat[0]} for the same
category, thereby require overloading of categories for the two linearization
spaces, or alternatively adding more complexity to the linearization categories.

While presented succinctly here, these obstacles were legitimate difficulties
which obliged us to test them on smaller grammars to isolate the phenomena
trying to be overcome.

\subsubsection{More advanced Agda features}

Our grammar realistically covers just a small kernel of Agda's features and
syntax. Agda supports much more, both in terms of syntactic sugar and
semantically interesting. Aside from telescopes, other syntactic sugar features
of Agda include unicode support, do notation, idiom brackets, generalized
variable declarations, and more. While require significant work to extend the 
cubicalTT grammar with these, it is doubtful
these kinds of features offer significant theoretical challenges in terms of
translation to natural language.

From the semantic side, however, Agda offers many features which extend just the kernel
of the $\Pi$, $\Sigma$, and recursive data type definitions which form the basis
of any dependent type theory. These include universes, sized types, modules,
overloading for more ad-hoc polymorphism, proof by reflection, a sort system,
higher inductive types (thanks to cubical), and many more things visible in the Agda documentation [cite agda docs].
Additionally, it has more traditional PL features, like the ability to perform
side effects or call Haskell functions. Adding any one of these not only adds
overhead to the parser, but would require lots of thought in terms of how to
these features manifest in natural language for mathematicians (and programmers).
Additionally, these features make the metatheory of Agda much more expensive to
understand, in addition to the practical implications of introducing bugs in its
implementation.

Mathematics on the other hand, doesn't often introduce more advanced ``semantic
machinery" like those listed, at least not in a way that is explicitly designed.
Perhaps idioms and conventions change, as well as generalizations, i.e. Category
Theory, offering ways of presenting ideas more succinctly, but these are merely
reflected in the presentation, not in the underlying logical formalism. The
linguistic evolution of mathematics additionally reflects some kind of
meta-changes, but not in a coherent way that is yet understood. For many
mathematicians are largely interested in proving theorems and solving problems
specific to some domain, and many mathematicians are unfamiliar with logic as a
discipline as a whole, let alone type theory.

The resolution of these meta-ideas from both the type theoretic and mathematical
perspectives is what makes this problem of translation so philosophically
intriguing, as well as intractable. We hope these observations might offer some
light when trying to examine any one of these deep and undeveloped problems.

\subsection{Comparing the Grammars}

To conclude this section, we compare the Ranta's HoTT grammar and our cubicalTT
grammar. We hope that doing so offers some final insights into how to approach
the problem of syntactic
completeness and semantic adequacy. 

cubicalTT is in some sense takes expressions as its epicenter, whereby
declarations, branches, telescopes, where expressions, and lists offer syntactic
sugar so that it becomes a minimally readable programming language. It is a
synthetic approach to writing a grammar, whereby one has an a priori idea of
what an expression syntactically should be, with the most important feature
being that it is inductively generated. It is not really concerned with
semantics per-se, because this is the job of the type-checker and evaluator.

HoTT, on the other hand, analyses real text, and decisions about the grammar are
made posterior to observing phenomena.  The grammar makes distinction between
\codeword{Formulas}, namely expressions with symbolic support for latex, 
\codeword{Framework} which allows one to construct natural language sentences,
and a \codeword{HottLexicon}. This grammar, while having some inductive notion
of what an expression is, puts the bulk of work in producing valid sentences in \codeword{Framework}.

\begin{verbatim}
cat
  Paragraph ;        -- definition, theorem, etc
  Definition ;       -- definition of a new concept
  Assumption ;       -- assumption in a proof  -- let ...
  [Assumption]{1} ;  -- list of assumptions in one sentence -- let ... and ...
  Conclusion ;       -- conclusion in a proof -- thus P
  Prop ;             -- proposition (sentence or formula) -- A is contractible
  Sort ;             -- set, type, etc corresponding to a common noun
  Ind ;              -- individual, element, corresponding to a singular term
  Fun ;              -- function with individual value
  Pred ;             -- predicate: function with proposition value -- contractible
  [Ind] ;            -- list of individual expressions   -- 1, 2 and 3
  UnivPhrase ;       -- universal noun phrase          -- for all x,y : A
  ConclusionPhrase ; -- conclusion word                -- hence
  Label ;            -- name/number of definition, theorem, etc -- Id-induction
  Title ;            -- title for theorem, definition, etc
\end{verbatim}

The distinction between individuals, propositions,
sorts, functions, and predicates also allows more nuance, but delegates the work
of deciding what category a term represents much more difficult, which makes the
possibility of having some algorithm infer the right category much more
difficult.  Additionally, 
We see that the universal phrase, the notion of a $\Pi$-type, merits semantic 
distinction in this grammar, with unique functions being assigned for all the
(observed) ways of saying it.

\begin{verbatim}
  plainUnivPhrase   : [Var] -> Sort -> UnivPhrase ;  -- for x, y : A
  eachUnivPhrase    : [Var] -> Sort -> UnivPhrase ;  -- for each x,y : A
  allUnivPhrase     : [Var] -> Sort -> UnivPhrase ;  -- for all x,y : A
  ifUnivPhrase      : [Var] -> Sort -> UnivPhrase ;  -- if x,y : A
  if_thenUnivPhrase : [Var] -> Sort -> UnivPhrase ;  -- if x,y : A then
\end{verbatim}

This includes document structure categories, \codeword{Title}, \codeword{Label},
\codeword{Paragraph}, \codeword{Definition}, \codeword{Conclusion}, etc. While
these may resembling a module system in ways, they also reflect a different
semantic sense than Agda's module system, which gives the programmer greater
control of handling software complexity. \codeword{ConclusionPhrase} reflects
what Agda's typechecker infers and is displayed to the user, and is therefore
redundant from the programmers perspective.

We have implemented Agda representation, which we compare with the rendered
latex, as well as the cubicalTT syntax in theappendix \ref{comparison}

\subsubsection{Ideas for resolution}





% -- Proof using isPropIsContr. This is slow and the direct proof below is better
% -- Direct proof that computes quite ok (can be optimized further if
% -- necessary, see:
% q-- HTTPSqqq://github.com/mortberg/cubicaltt/blob/pi4s3_dimclosures/examples/brunerie2.ctt#L562




% \section{HoTT Proofs}

\subsection{Why HoTT for natural language?}

We note that all natural language definitions, theorems, and proofs are copied
here verbatim from the HoTT book.  This decision is admittedly arbitrary, but
does have some benefits.  We list some here : 

\begin{itemize}[noitemsep]

\item As the HoTT book was a collaborative effort, it mixes the language of
many individuals and editors, and can be seen as more ``linguistically
neutral''

\item By its very nature HoTT is interdiscplinary, conceived and constructed by
mathematicians, logicians, and computer scientists. It therefore is meant to
interface with all these discplines, and much of the book was indeed formalized
before it was written

\item It has become canonical reference in the field, and therefore benefits
from wide familiarity

\item It is open source, with publically available Latex files free for
modification and distribution

\end{itemize}

The genisis of higher type theory is a somewhat elementary observation : that
the identity type, parameterized by an arbitrary type $A$ and indexed by
elements of $A$, can actually be built iteratively from previous identities.
That is, $A$ may actually already be an identity defined over another type
$A'$, namely $A \defeq x=_{A'} y$ where $x,y:A'$. The key idea is that this
iterating identities admits a homotpical interpretation : 

\begin{itemize}[noitemsep]

\item Types are topological spaces
\item Terms are points in these space

\item Equality types $x=_{A} y$ are paths in $A$ with endpoints $x$ and $y$ in
$A$

\item Iterated equality types are paths between paths, or continuous path
deformations in some higher path space. This is, intuitively, what
mathematicians call a homotopy.

\end{itemize}

To be explicit, given a type $A$, we can form the homotopy $p=_{x=_{A} y}q$
with endpoints $p$ and $q$ inhabiting the path space $x=_{A} y$.

Let's start out by examining the inductive definition of the identity type.  We
present this definition as it appears in section 1.12 of the HoTT book.

\begin{definition}
  The formation rule says that given a type $A:\UU$ and two elements $a,b:A$, we can form the type $(\id[A]{a}{b}):\UU$ in the same universe.
  The basic way to construct an element of $\id{a}{b}$ is to know that $a$ and $b$ are the same.
  Thus, the introduction rule is a dependent function
  \[\refl{} : \prod_{a:A} (\id[A]{a}{a}) \]
  called \define{reflexivity},
  which says that every element of $A$ is equal to itself (in a specified way).  We regard $\refl{a}$ as being the
  constant path %path\indexdef{path!constant}\indexsee{loop!constant}{path, constant}
  at the point $a$.
\end{definition}

We recapitulate this definition in Agda, and treat : 

\begin{code}[hide]%
\>[0]\AgdaSymbol{\{-\#}\AgdaSpace{}%
\AgdaKeyword{OPTIONS}\AgdaSpace{}%
\AgdaPragma{--cubical}\AgdaSpace{}%
\AgdaSymbol{\#-\}}\<%
\\
%
\\[\AgdaEmptyExtraSkip]%
\>[0]\AgdaKeyword{module}\AgdaSpace{}%
\AgdaModule{hott}\AgdaSpace{}%
\AgdaKeyword{where}\<%
\end{code}

\begin{code}[hide]%
\>[0]\<%
\\
\>[0]\AgdaKeyword{module}\AgdaSpace{}%
\AgdaModule{Id}\AgdaSpace{}%
\AgdaKeyword{where}\<%
\\
\>[0]\<%
\end{code}
\begin{code}%
\>[0]\<%
\\
\>[0][@{}l@{\AgdaIndent{1}}]%
\>[2]\AgdaKeyword{data}\AgdaSpace{}%
\AgdaOperator{\AgdaDatatype{\AgdaUnderscore{}≡'\AgdaUnderscore{}}}\AgdaSpace{}%
\AgdaSymbol{\{}\AgdaBound{A}\AgdaSpace{}%
\AgdaSymbol{:}\AgdaSpace{}%
\AgdaPrimitive{Set}\AgdaSymbol{\}}\AgdaSpace{}%
\AgdaSymbol{:}\AgdaSpace{}%
\AgdaSymbol{(}\AgdaBound{a}\AgdaSpace{}%
\AgdaBound{b}\AgdaSpace{}%
\AgdaSymbol{:}\AgdaSpace{}%
\AgdaBound{A}\AgdaSymbol{)}\AgdaSpace{}%
\AgdaSymbol{→}\AgdaSpace{}%
\AgdaPrimitive{Set}\AgdaSpace{}%
\AgdaKeyword{where}\<%
\\
\>[2][@{}l@{\AgdaIndent{0}}]%
\>[4]\AgdaInductiveConstructor{r}\AgdaSpace{}%
\AgdaSymbol{:}\AgdaSpace{}%
\AgdaSymbol{(}\AgdaBound{a}\AgdaSpace{}%
\AgdaSymbol{:}\AgdaSpace{}%
\AgdaBound{A}\AgdaSymbol{)}\AgdaSpace{}%
\AgdaSymbol{→}\AgdaSpace{}%
\AgdaBound{a}\AgdaSpace{}%
\AgdaOperator{\AgdaDatatype{≡'}}\AgdaSpace{}%
\AgdaBound{a}\<%
\\
\>[0]\<%
\end{code}

\subsection{An introduction to equality}

There is already some tension brewing : most mathematicains have an intuition
for equality, that of an identitfication between two pieces of information
which intuitively must be the same thing, i.e. $2+2=4$. They might ask, what
does it mean to ``construct an element of $\id{a}{b}$''? For the mathematician
use to thinking in terms of sets $\{\id{a}{b} \mid a,b \in \mathbb{N} \}$ isn't
a well-defined notion. Due to its use of the axiom of extensionality, the set
theoretic notion of equality is, no suprise, extensional.  This means that sets
are identified when they have the same elements, and equality is therefore
external to the notion of set. To inhabit a type means to provide evidence for
that inhabitation. The reflexivity constructor is therefore a means of
providing evidence of an equality. This evidence approach is disctinctly
constructive, and a big reason why classical and constructive mathematics,
especially when treated in an intuitionistic type theory suitable for a
programming language implementation, are such different beasts.

In Martin-Löf Type Theory, there are two fundamental notions of equality,
propositional and definitional.  While propositional equality is inductively
defined (as above) as a type which may have possibly more than one inhabitant,
definitional equality, denoted $-\equiv -$ and perhaps more aptly named
computational equality, is familiarly what most people think of as equality.
Namely, two terms which compute to the same canonical form are computationally
equal. In intensional type theory, propositional equality is a weaker notion
than computational equality : all propositionally equal terms are
computationally equal. However, computational equality does not imply
propistional equality - if it does, then one enters into the space of
extensional type theory. 

Prior to the homotopical interpretation of identity types, debates about
extensional and intensional type theories centred around two features or bugs :
extensional type theory sacrificed decideable type checking, while intensional
type theories required extra beauracracy when dealing with equality in proofs.
One approach in intensional type theories treated types as setoids, therefore
leading to so-called ``Setoid Hell''. These debates reflected Martin-Löf's
flip-flopping on the issue. His seminal 1979 Constructive Mathematics and
Computer Programming, which took an extensional view, was soon betrayed by
lectures he gave soon thereafter in Padova in 1980.  Martin-Löf was a born
again intensional type theorist.  These Padova lectures were later published in
the "Bibliopolis Book", and went on to inspire the European (and Gothenburg in
particular) approach to implementing proof assitants, whereas the
extensionalists were primarily eminating from Robert Constable's group at
Cornell. 

This tension has now been at least partially resolved, or at the very least
clarified, by an insight Voevodsky was apparently most proud of : the
introduction of h-levels. We'll delegate these details for a later section, it
is mentioned here to indicate that extensional type theory was really ``set
theory'' in disguise, in that it collapses the higher path structure of
identity types. The work over the past 10 years has elucidated the intensional
and extensional positions. HoTT, by allowing higher paths, is unashamedly
intentional, and admits a collapse into the extensional universe if so desired.
We now the examine the structure induced by this propositional equality.

\subsection{All about Identity}

We start with a slight reformulation of the identity type, where the element
determining the equality is treated as a parameter rather than an index. This
is a matter of convenience more than taste, as it delegates work for Agda's
typechecker that the programmer may find a distraction. The reflexivity terms
can generally have their endpoints inferred, and therefore cuts down on the
beauracry which often obscures code. 

\begin{code}%
\>[0]\<%
\\
\>[0][@{}l@{\AgdaIndent{1}}]%
\>[2]\AgdaKeyword{data}\AgdaSpace{}%
\AgdaOperator{\AgdaDatatype{\AgdaUnderscore{}≡\AgdaUnderscore{}}}\AgdaSpace{}%
\AgdaSymbol{\{}\AgdaBound{A}\AgdaSpace{}%
\AgdaSymbol{:}\AgdaSpace{}%
\AgdaPrimitive{Set}\AgdaSymbol{\}}\AgdaSpace{}%
\AgdaSymbol{(}\AgdaBound{a}\AgdaSpace{}%
\AgdaSymbol{:}\AgdaSpace{}%
\AgdaBound{A}\AgdaSymbol{)}\AgdaSpace{}%
\AgdaSymbol{:}\AgdaSpace{}%
\AgdaBound{A}\AgdaSpace{}%
\AgdaSymbol{→}\AgdaSpace{}%
\AgdaPrimitive{Set}\AgdaSpace{}%
\AgdaKeyword{where}\<%
\\
\>[2][@{}l@{\AgdaIndent{0}}]%
\>[4]\AgdaInductiveConstructor{r}\AgdaSpace{}%
\AgdaSymbol{:}\AgdaSpace{}%
\AgdaBound{a}\AgdaSpace{}%
\AgdaOperator{\AgdaDatatype{≡}}\AgdaSpace{}%
\AgdaBound{a}\<%
\\
%
\\[\AgdaEmptyExtraSkip]%
%
\>[2]\AgdaKeyword{infix}\AgdaSpace{}%
\AgdaNumber{20}\AgdaSpace{}%
\AgdaOperator{\AgdaDatatype{\AgdaUnderscore{}≡\AgdaUnderscore{}}}\<%
\\
\>[0]\<%
\end{code}

It is of particular concern in this thesis, because it hightlights a
fundamental difference between the lingusitic and the formal approach to proof
presentation.  While the mathematician can whimsically choose to include the
reflexivity arguement or ignore it if she believes it can be inferred, the
programmer can't afford such a laxidasical attitude. Once the type has been
defined, the arguement strcuture is fixed, all future references to the
definition carefully adhere to its specification. The advantage that the
programmer does gain however, that of Agda's powerful inferential abilities,
allows for the insides to be seen via interaction windown. 

Perhaps not of much interest up front, this is incredibly important detail
which the mathematician never has to deal with explicity, but can easily make
type and term translation infeasible due to the fast and loose nature of the
mathematician's writing. Conversely, it may make Natural Language Generation
(NLG) incredibly clunky, adhering to strict rules when created sentences out of
programs. 

[ToDo, give a GF example]

A prime source of beauty in constructive mathematics arises from Gentzen's
recognition of a natural duality in the rules for introducing and using logical
connectives. The mutually coherence between introduction and elmination rules
form the basis of what has since been labeled harmony in a deductive system.
This harmony isn't just an artifact of beauty, it forms the basis for cuts in
proof normalization, and correspondingly, evaluation of terms in a programming
langauge. 

The idea is simple, each new connective, or type former, needs a means of
constructing its terms from its constiutuent parts, yielding introduction
rules. This however, isn't enough - we need a way of dissecting and using the
terms we construct. This yields an elimantion rule which can be uniquely
derived from an inductively defined type. These elimination forms yield
induction principles, or a general notion of proof by induction, when given an
interpration in mathematics. In the non-depedent case, this is known as a
recursion principle, and corresponds to recursion known by programmers far and
wide.  The proof by induction over natural numbers familiar to mathematicians
is just one special case of this induction principle at work--the power of
induction has been recognized and brought to the fore by computer scientists.

We now elaborate the most important induction principle in HoTT, namely, the
induction of an identity type.

\begin{definition}[Version 1]

Moreover, one of the amazing things about homotopy type theory is that all of the basic constructions and axioms---all of the
higher groupoid structure---arises automatically from the induction
principle for identity types.
Recall from [section 1.12]  that this says that if % \cref{sec:identity-types}
  \begin{itemize}[noitemsep]
    \item for every $x,y:A$ and every $p:\id[A]xy$ we have a type $D(x,y,p)$, and
    \item for every $a:A$ we have an element $d(a):D(a,a,\refl a)$,
  \end{itemize}
then
  \begin{itemize}[noitemsep]
    \item there exists an element $\indid{A}(D,d,x,y,p):D(x,y,p)$ for \emph{every}
    two elements $x,y:A$ and $p:\id[A]xy$, such that $\indid{A}(D,d,a,a,\refl a)
    \jdeq d(a)$.
  \end{itemize}
\end{definition}
The book then reiterates this definition, with basically no natural language,
essentially in the raw logical framework devoid of anything but dependent
function types.
\begin{definition}[Version 2]
In other words, given dependent functions
\begin{align*}
  D & :\prod_{(x,y:A)}(x= y) \; \to \; \type\\
  d & :\prod_{a:A} D(a,a,\refl{a})
\end{align*}
there is a dependent function
\[\indid{A}(D,d):\prod_{(x,y:A)}\prod_{(p:\id{x}{y})} D(x,y,p)\]
such that
\begin{equation}\label{eq:Jconv}
\indid{A}(D,d,a,a,\refl{a})\jdeq d(a)
\end{equation}
for every $a:A$.
Usually, every time we apply this induction rule we will either not care about the specific function being defined, or we will immediately give it a different name.

\end{definition}
Again, we define this, in Agda, staying as true to the syntax as possible.
\begin{code}%
\>[0]\<%
\\
\>[0][@{}l@{\AgdaIndent{1}}]%
\>[2]\AgdaFunction{J}\AgdaSpace{}%
\AgdaSymbol{:}%
\>[46I]\AgdaSymbol{\{}\AgdaBound{A}\AgdaSpace{}%
\AgdaSymbol{:}\AgdaSpace{}%
\AgdaPrimitive{Set}\AgdaSymbol{\}}\<%
\\
\>[.][@{}l@{}]\<[46I]%
\>[6]\AgdaSymbol{→}\AgdaSpace{}%
\AgdaSymbol{(}\AgdaBound{D}\AgdaSpace{}%
\AgdaSymbol{:}\AgdaSpace{}%
\AgdaSymbol{(}\AgdaBound{x}\AgdaSpace{}%
\AgdaBound{y}\AgdaSpace{}%
\AgdaSymbol{:}\AgdaSpace{}%
\AgdaBound{A}\AgdaSymbol{)}\AgdaSpace{}%
\AgdaSymbol{→}\AgdaSpace{}%
\AgdaSymbol{(}\AgdaBound{x}\AgdaSpace{}%
\AgdaOperator{\AgdaDatatype{≡}}\AgdaSpace{}%
\AgdaBound{y}\AgdaSymbol{)}\AgdaSpace{}%
\AgdaSymbol{→}%
\>[36]\AgdaPrimitive{Set}\AgdaSymbol{)}\<%
\\
%
\>[6]\AgdaSymbol{→}\AgdaSpace{}%
\AgdaSymbol{((}\AgdaBound{a}\AgdaSpace{}%
\AgdaSymbol{:}\AgdaSpace{}%
\AgdaBound{A}\AgdaSymbol{)}\AgdaSpace{}%
\AgdaSymbol{→}\AgdaSpace{}%
\AgdaSymbol{(}\AgdaBound{D}\AgdaSpace{}%
\AgdaBound{a}\AgdaSpace{}%
\AgdaBound{a}\AgdaSpace{}%
\AgdaInductiveConstructor{r}\AgdaSpace{}%
\AgdaSymbol{))}\AgdaSpace{}%
\AgdaComment{-- → (d : (a : A) → (D a a r ))}\<%
\\
%
\>[6]\AgdaSymbol{→}\AgdaSpace{}%
\AgdaSymbol{(}\AgdaBound{x}\AgdaSpace{}%
\AgdaBound{y}\AgdaSpace{}%
\AgdaSymbol{:}\AgdaSpace{}%
\AgdaBound{A}\AgdaSymbol{)}\<%
\\
%
\>[6]\AgdaSymbol{→}\AgdaSpace{}%
\AgdaSymbol{(}\AgdaBound{p}\AgdaSpace{}%
\AgdaSymbol{:}\AgdaSpace{}%
\AgdaBound{x}\AgdaSpace{}%
\AgdaOperator{\AgdaDatatype{≡}}\AgdaSpace{}%
\AgdaBound{y}\AgdaSymbol{)}\<%
\\
%
\>[6]\AgdaComment{------------------------------------}\<%
\\
%
\>[6]\AgdaSymbol{→}\AgdaSpace{}%
\AgdaBound{D}\AgdaSpace{}%
\AgdaBound{x}\AgdaSpace{}%
\AgdaBound{y}\AgdaSpace{}%
\AgdaBound{p}\<%
\\
%
\>[2]\AgdaFunction{J}\AgdaSpace{}%
\AgdaBound{D}\AgdaSpace{}%
\AgdaBound{d}\AgdaSpace{}%
\AgdaBound{x}\AgdaSpace{}%
\AgdaDottedPattern{\AgdaSymbol{.}}\AgdaDottedPattern{x}\AgdaSpace{}%
\AgdaInductiveConstructor{r}\AgdaSpace{}%
\AgdaSymbol{=}\AgdaSpace{}%
\AgdaBound{d}\AgdaSpace{}%
\AgdaBound{x}\<%
\\
\>[0]\<%
\end{code}

It should be noted that, for instance, we can choose to leave out the $d$ label
on the third line. Indeed minimizing the amount of dependent typing and using
vanilla function types when dependency is not necessary, is generally
considered ``best practice'' Agda, because it will get desugared by the time it
typechecks anyways. For the writer of the text; however, it was convenient to
define $d$ once, as there are not the same constraints on a mathematician
writing in latex. It will again, serve as a nontrivial exercise to deal with
when specifying the grammar, and will be dealt with later [ToDo add section].
It is also of note that we choose to include Martin-Löf's original name $J$, as
this is more common in the computer science literature.

Once the identity type has been defined, it is natural to develop an ``equality
calculus'',  so that we can actually use it in proof's, as well as develop the
higher groupoid structure of types. The first fact, that propositional equality
is an equivalence relation, is well motivated by needs of practical theorem
proving in Agda and the more homotopically minded mathematician. First, we show the symmetry of equality--that paths are reversible.

\begin{lem}\label{lem:opp}
  For every type $A$ and every $x,y:A$ there is a function
  \begin{equation*}
    (x= y)\to(y= x)
  \end{equation*}
  denoted $p\mapsto \opp{p}$, such that $\opp{\refl{x}}\jdeq\refl{x}$ for each $x:A$.
  We call $\opp{p}$ the \define{inverse} of $p$.
  %\indexdef{path!inverse}%
  %\indexdef{inverse!of path}%
  %\index{equality!symmetry of}%a
  %\index{symmetry!of equality}%
\end{lem}

\begin{proof}[First proof]
  Assume given $A:\UU$, and
  let $D:{\textstyle\prod_{(x,y:A)}}(x= y) \; \to \; \type$ be the type family defined by $D(x,y,p)\defeq (y= x)$.
  %$\prod_{(x:A)} \prod_{y:B}$
  In other words, $D$ is a function assigning to any $x,y:A$ and $p:x=y$ a type, namely the type $y=x$.
  Then we have an element
  \begin{equation*}
    d\defeq \lambda x.\ \refl{x}:\prod_{x:A} D(x,x,\refl{x}).
  \end{equation*}
  Thus, the induction principle for identity types gives us an element
  $\indid{A}(D,d,x,y,p): (y= x)$
  for each $p:(x= y)$.
  We can now define the desired function $\opp{(\blank)}$ to be 
  $\lambda p.\ \indid{A}(D,d,x,y,p)$, 
  i.e.\ we set 
  $\opp{p} \defeq \indid{A}(D,d,x,y,p)$.
  The conversion rule [missing reference] %rule~\eqref{eq:Jconv} 
  gives $\opp{\refl{x}}\jdeq \refl{x}$, as required.
\end{proof}
The Agda code is certainly more brief: 
\begin{code}%
\>[0]\<%
\\
\>[0][@{}l@{\AgdaIndent{1}}]%
\>[2]\AgdaOperator{\AgdaFunction{\AgdaUnderscore{}⁻¹}}\AgdaSpace{}%
\AgdaSymbol{:}\AgdaSpace{}%
\AgdaSymbol{\{}\AgdaBound{A}\AgdaSpace{}%
\AgdaSymbol{:}\AgdaSpace{}%
\AgdaPrimitive{Set}\AgdaSymbol{\}}\AgdaSpace{}%
\AgdaSymbol{\{}\AgdaBound{x}\AgdaSpace{}%
\AgdaBound{y}\AgdaSpace{}%
\AgdaSymbol{:}\AgdaSpace{}%
\AgdaBound{A}\AgdaSymbol{\}}\AgdaSpace{}%
\AgdaSymbol{→}\AgdaSpace{}%
\AgdaBound{x}\AgdaSpace{}%
\AgdaOperator{\AgdaDatatype{≡}}\AgdaSpace{}%
\AgdaBound{y}\AgdaSpace{}%
\AgdaSymbol{→}\AgdaSpace{}%
\AgdaBound{y}\AgdaSpace{}%
\AgdaOperator{\AgdaDatatype{≡}}\AgdaSpace{}%
\AgdaBound{x}\<%
\\
%
\>[2]\AgdaOperator{\AgdaFunction{\AgdaUnderscore{}⁻¹}}\AgdaSpace{}%
\AgdaSymbol{\{}\AgdaBound{A}\AgdaSymbol{\}}\AgdaSpace{}%
\AgdaSymbol{\{}\AgdaBound{x}\AgdaSymbol{\}}\AgdaSpace{}%
\AgdaSymbol{\{}\AgdaBound{y}\AgdaSymbol{\}}\AgdaSpace{}%
\AgdaBound{p}\AgdaSpace{}%
\AgdaSymbol{=}\AgdaSpace{}%
\AgdaFunction{J}\AgdaSpace{}%
\AgdaFunction{D}\AgdaSpace{}%
\AgdaFunction{d}\AgdaSpace{}%
\AgdaBound{x}\AgdaSpace{}%
\AgdaBound{y}\AgdaSpace{}%
\AgdaBound{p}\<%
\\
\>[2][@{}l@{\AgdaIndent{0}}]%
\>[4]\AgdaKeyword{where}\<%
\\
\>[4][@{}l@{\AgdaIndent{0}}]%
\>[6]\AgdaFunction{D}\AgdaSpace{}%
\AgdaSymbol{:}\AgdaSpace{}%
\AgdaSymbol{(}\AgdaBound{x}\AgdaSpace{}%
\AgdaBound{y}\AgdaSpace{}%
\AgdaSymbol{:}\AgdaSpace{}%
\AgdaBound{A}\AgdaSymbol{)}\AgdaSpace{}%
\AgdaSymbol{→}\AgdaSpace{}%
\AgdaBound{x}\AgdaSpace{}%
\AgdaOperator{\AgdaDatatype{≡}}\AgdaSpace{}%
\AgdaBound{y}\AgdaSpace{}%
\AgdaSymbol{→}\AgdaSpace{}%
\AgdaPrimitive{Set}\<%
\\
%
\>[6]\AgdaFunction{D}\AgdaSpace{}%
\AgdaBound{x}\AgdaSpace{}%
\AgdaBound{y}\AgdaSpace{}%
\AgdaBound{p}\AgdaSpace{}%
\AgdaSymbol{=}\AgdaSpace{}%
\AgdaBound{y}\AgdaSpace{}%
\AgdaOperator{\AgdaDatatype{≡}}\AgdaSpace{}%
\AgdaBound{x}\<%
\\
%
\>[6]\AgdaFunction{d}\AgdaSpace{}%
\AgdaSymbol{:}\AgdaSpace{}%
\AgdaSymbol{(}\AgdaBound{a}\AgdaSpace{}%
\AgdaSymbol{:}\AgdaSpace{}%
\AgdaBound{A}\AgdaSymbol{)}\AgdaSpace{}%
\AgdaSymbol{→}\AgdaSpace{}%
\AgdaFunction{D}\AgdaSpace{}%
\AgdaBound{a}\AgdaSpace{}%
\AgdaBound{a}\AgdaSpace{}%
\AgdaInductiveConstructor{r}\<%
\\
%
\>[6]\AgdaFunction{d}\AgdaSpace{}%
\AgdaBound{a}\AgdaSpace{}%
\AgdaSymbol{=}\AgdaSpace{}%
\AgdaInductiveConstructor{r}\<%
\\
%
\\[\AgdaEmptyExtraSkip]%
%
\>[2]\AgdaKeyword{infixr}\AgdaSpace{}%
\AgdaNumber{50}\AgdaSpace{}%
\AgdaOperator{\AgdaFunction{\AgdaUnderscore{}⁻¹}}\<%
\\
\>[0]\<%
\end{code}

While first encountering induction principles can be scary, they are actually
more mechanical than one may think. This is due to the the fact that they
uniquely compliment the introduction rules of an an inductive type, and are
simply a means of showing one can ``map out'', or derive an arbitrary type
dependent on the type which has been inductively defined. The mechanical nature
is what allows for Coq's induction tactic, and perhaps even more elegantly,
Agda's pattern matching capabilities. It is always easier to use pattern
matching for the novice Agda programmer, which almost feels like magic.
Nonetheless, for completeness sake, the book uses the induction principle for
much of Chapter 2. And pattern matching is unique to programming languages,
its elegance isn't matched in the mathematicians' lexicon.

Here is the same proof via ``natural language pattern matching'' and Agda
pattern matching:

\begin{proof}[Second proof]
  We want to construct, for each $x,y:A$ and $p:x=y$, an element $\opp{p}:y=x$.
  By induction, it suffices to do this in the case when $y$ is $x$ and $p$ is $\refl{x}$.
  But in this case, the type $x=y$ of $p$ and the type $y=x$ in which we are trying to construct $\opp{p}$ are both simply $x=x$.
  Thus, in the ``reflexivity case'', we can define $\opp{\refl{x}}$ to be simply $\refl{x}$.
  The general case then follows by the induction principle, and the conversion rule $\opp{\refl{x}}\jdeq\refl{x}$ is precisely the proof in the reflexivity case that we gave.
\end{proof}

\begin{code}%
\>[0]\<%
\\
\>[0][@{}l@{\AgdaIndent{1}}]%
\>[2]\AgdaOperator{\AgdaFunction{\AgdaUnderscore{}⁻¹'}}\AgdaSpace{}%
\AgdaSymbol{:}\AgdaSpace{}%
\AgdaSymbol{\{}\AgdaBound{A}\AgdaSpace{}%
\AgdaSymbol{:}\AgdaSpace{}%
\AgdaPrimitive{Set}\AgdaSymbol{\}}\AgdaSpace{}%
\AgdaSymbol{\{}\AgdaBound{x}\AgdaSpace{}%
\AgdaBound{y}\AgdaSpace{}%
\AgdaSymbol{:}\AgdaSpace{}%
\AgdaBound{A}\AgdaSymbol{\}}\AgdaSpace{}%
\AgdaSymbol{→}\AgdaSpace{}%
\AgdaBound{x}\AgdaSpace{}%
\AgdaOperator{\AgdaDatatype{≡}}\AgdaSpace{}%
\AgdaBound{y}\AgdaSpace{}%
\AgdaSymbol{→}\AgdaSpace{}%
\AgdaBound{y}\AgdaSpace{}%
\AgdaOperator{\AgdaDatatype{≡}}\AgdaSpace{}%
\AgdaBound{x}\<%
\\
%
\>[2]\AgdaOperator{\AgdaFunction{\AgdaUnderscore{}⁻¹'}}\AgdaSpace{}%
\AgdaSymbol{\{}\AgdaBound{A}\AgdaSymbol{\}}\AgdaSpace{}%
\AgdaSymbol{\{}\AgdaBound{x}\AgdaSymbol{\}}\AgdaSpace{}%
\AgdaSymbol{\{}\AgdaBound{y}\AgdaSymbol{\}}\AgdaSpace{}%
\AgdaInductiveConstructor{r}\AgdaSpace{}%
\AgdaSymbol{=}\AgdaSpace{}%
\AgdaInductiveConstructor{r}\<%
\\
\>[0]\<%
\end{code}

Next is trasitivity--concatenation of paths--and we omit the natural language
presentation, which is a slightly more sophisticated arguement than for
symmetry.  


\begin{code}%
\>[0][@{}l@{\AgdaIndent{1}}]%
\>[2]\AgdaOperator{\AgdaFunction{\AgdaUnderscore{}∙\AgdaUnderscore{}}}\AgdaSpace{}%
\AgdaSymbol{:}\AgdaSpace{}%
\AgdaSymbol{\{}\AgdaBound{A}\AgdaSpace{}%
\AgdaSymbol{:}\AgdaSpace{}%
\AgdaPrimitive{Set}\AgdaSymbol{\}}\AgdaSpace{}%
\AgdaSymbol{→}\AgdaSpace{}%
\AgdaSymbol{\{}\AgdaBound{x}\AgdaSpace{}%
\AgdaBound{y}\AgdaSpace{}%
\AgdaSymbol{:}\AgdaSpace{}%
\AgdaBound{A}\AgdaSymbol{\}}\AgdaSpace{}%
\AgdaSymbol{→}\AgdaSpace{}%
\AgdaSymbol{(}\AgdaBound{p}\AgdaSpace{}%
\AgdaSymbol{:}\AgdaSpace{}%
\AgdaBound{x}\AgdaSpace{}%
\AgdaOperator{\AgdaDatatype{≡}}\AgdaSpace{}%
\AgdaBound{y}\AgdaSymbol{)}\AgdaSpace{}%
\AgdaSymbol{→}\AgdaSpace{}%
\AgdaSymbol{\{}\AgdaBound{z}\AgdaSpace{}%
\AgdaSymbol{:}\AgdaSpace{}%
\AgdaBound{A}\AgdaSymbol{\}}\AgdaSpace{}%
\AgdaSymbol{→}\AgdaSpace{}%
\AgdaSymbol{(}\AgdaBound{q}\AgdaSpace{}%
\AgdaSymbol{:}\AgdaSpace{}%
\AgdaBound{y}\AgdaSpace{}%
\AgdaOperator{\AgdaDatatype{≡}}\AgdaSpace{}%
\AgdaBound{z}\AgdaSymbol{)}\AgdaSpace{}%
\AgdaSymbol{→}\AgdaSpace{}%
\AgdaBound{x}\AgdaSpace{}%
\AgdaOperator{\AgdaDatatype{≡}}\AgdaSpace{}%
\AgdaBound{z}\<%
\\
%
\>[2]\AgdaOperator{\AgdaFunction{\AgdaUnderscore{}∙\AgdaUnderscore{}}}%
\>[201I]\AgdaSymbol{\{}\AgdaBound{A}\AgdaSymbol{\}}\AgdaSpace{}%
\AgdaSymbol{\{}\AgdaBound{x}\AgdaSymbol{\}}\AgdaSpace{}%
\AgdaSymbol{\{}\AgdaBound{y}\AgdaSymbol{\}}\AgdaSpace{}%
\AgdaBound{p}\AgdaSpace{}%
\AgdaSymbol{\{}\AgdaBound{z}\AgdaSymbol{\}}\AgdaSpace{}%
\AgdaBound{q}\AgdaSpace{}%
\AgdaSymbol{=}\AgdaSpace{}%
\AgdaFunction{J}\AgdaSpace{}%
\AgdaFunction{D}\AgdaSpace{}%
\AgdaFunction{d}\AgdaSpace{}%
\AgdaBound{x}\AgdaSpace{}%
\AgdaBound{y}\AgdaSpace{}%
\AgdaBound{p}\AgdaSpace{}%
\AgdaBound{z}\AgdaSpace{}%
\AgdaBound{q}\<%
\\
\>[.][@{}l@{}]\<[201I]%
\>[6]\AgdaKeyword{where}\<%
\\
%
\>[6]\AgdaFunction{D}\AgdaSpace{}%
\AgdaSymbol{:}\AgdaSpace{}%
\AgdaSymbol{(}\AgdaBound{x₁}\AgdaSpace{}%
\AgdaBound{y₁}\AgdaSpace{}%
\AgdaSymbol{:}\AgdaSpace{}%
\AgdaBound{A}\AgdaSymbol{)}\AgdaSpace{}%
\AgdaSymbol{→}\AgdaSpace{}%
\AgdaBound{x₁}\AgdaSpace{}%
\AgdaOperator{\AgdaDatatype{≡}}\AgdaSpace{}%
\AgdaBound{y₁}\AgdaSpace{}%
\AgdaSymbol{→}\AgdaSpace{}%
\AgdaPrimitive{Set}\<%
\\
%
\>[6]\AgdaFunction{D}\AgdaSpace{}%
\AgdaBound{x}\AgdaSpace{}%
\AgdaBound{y}\AgdaSpace{}%
\AgdaBound{p}\AgdaSpace{}%
\AgdaSymbol{=}\AgdaSpace{}%
\AgdaSymbol{(}\AgdaBound{z}\AgdaSpace{}%
\AgdaSymbol{:}\AgdaSpace{}%
\AgdaBound{A}\AgdaSymbol{)}\AgdaSpace{}%
\AgdaSymbol{→}\AgdaSpace{}%
\AgdaSymbol{(}\AgdaBound{q}\AgdaSpace{}%
\AgdaSymbol{:}\AgdaSpace{}%
\AgdaBound{y}\AgdaSpace{}%
\AgdaOperator{\AgdaDatatype{≡}}\AgdaSpace{}%
\AgdaBound{z}\AgdaSymbol{)}\AgdaSpace{}%
\AgdaSymbol{→}\AgdaSpace{}%
\AgdaBound{x}\AgdaSpace{}%
\AgdaOperator{\AgdaDatatype{≡}}\AgdaSpace{}%
\AgdaBound{z}\<%
\\
%
\>[6]\AgdaFunction{d}\AgdaSpace{}%
\AgdaSymbol{:}\AgdaSpace{}%
\AgdaSymbol{(}\AgdaBound{z₁}\AgdaSpace{}%
\AgdaSymbol{:}\AgdaSpace{}%
\AgdaBound{A}\AgdaSymbol{)}\AgdaSpace{}%
\AgdaSymbol{→}\AgdaSpace{}%
\AgdaFunction{D}\AgdaSpace{}%
\AgdaBound{z₁}\AgdaSpace{}%
\AgdaBound{z₁}\AgdaSpace{}%
\AgdaInductiveConstructor{r}\<%
\\
%
\>[6]\AgdaFunction{d}\AgdaSpace{}%
\AgdaSymbol{=}\AgdaSpace{}%
\AgdaSymbol{λ}\AgdaSpace{}%
\AgdaBound{v}\AgdaSpace{}%
\AgdaBound{z}\AgdaSpace{}%
\AgdaBound{q}\AgdaSpace{}%
\AgdaSymbol{→}\AgdaSpace{}%
\AgdaBound{q}\<%
\\
%
\\[\AgdaEmptyExtraSkip]%
%
\>[2]\AgdaKeyword{infixl}\AgdaSpace{}%
\AgdaNumber{40}\AgdaSpace{}%
\AgdaOperator{\AgdaFunction{\AgdaUnderscore{}∙\AgdaUnderscore{}}}\<%
\end{code}

Putting on our spectacles, the reflexivity term serves as evidence of a
constant path for any given point of any given type. To the category theorist,
this makes up the data of an identity map. Likewise, conctanation of paths
starts to look like function composition. This, along with the identity laws
and associativity as proven below, gives us the data of a category. And we have
not only have a category, but the symmetry allows us to prove all paths are
isomorphisms, giving us a groupoid. This isn't a coincedence, it's a very deep
and fascinating articulation of power of the machinery we've so far built. The
fact the path space over a type naturally must satisfies coherence laws in an
even higher path space gives leads to this notion of types as higher groupoids.  

As regards the natural language--at this point, the bookkeeping starts to get hairy.  Paths between paths, and paths between paths between paths, these ideas start to lose geometric inutiotion. And the mathematician often fails to express, when writing $p= q$, that she is already reasoning in a path space. While clever, our brains aren't wired to do too much book-keeping.  Fortunately Agda does this for us, and we can use implicit arguements to avoid our code getting too messy.  [ToDo, add example]

We now proceed to show that we have a groupoid, where the objects are points,
the morphisms are paths. The isomorphisms arise from the path reversal.  Many
of the proofs beyond this point are either routinely made via the induction
principle, or even more routinely by just pattern matching on equality paths,
we omit the details which can be found in the HoTT book, but it is expected
that the GF parser will soon cover such examples.

\begin{code}%
%
\>[2]\AgdaFunction{iₗ}\AgdaSpace{}%
\AgdaSymbol{:}\AgdaSpace{}%
\AgdaSymbol{\{}\AgdaBound{A}\AgdaSpace{}%
\AgdaSymbol{:}\AgdaSpace{}%
\AgdaPrimitive{Set}\AgdaSymbol{\}}\AgdaSpace{}%
\AgdaSymbol{\{}\AgdaBound{x}\AgdaSpace{}%
\AgdaBound{y}\AgdaSpace{}%
\AgdaSymbol{:}\AgdaSpace{}%
\AgdaBound{A}\AgdaSymbol{\}}\AgdaSpace{}%
\AgdaSymbol{(}\AgdaBound{p}\AgdaSpace{}%
\AgdaSymbol{:}\AgdaSpace{}%
\AgdaBound{x}\AgdaSpace{}%
\AgdaOperator{\AgdaDatatype{≡}}\AgdaSpace{}%
\AgdaBound{y}\AgdaSymbol{)}\AgdaSpace{}%
\AgdaSymbol{→}\AgdaSpace{}%
\AgdaBound{p}\AgdaSpace{}%
\AgdaOperator{\AgdaDatatype{≡}}\AgdaSpace{}%
\AgdaInductiveConstructor{r}\AgdaSpace{}%
\AgdaOperator{\AgdaFunction{∙}}\AgdaSpace{}%
\AgdaBound{p}\<%
\\
%
\>[2]\AgdaFunction{iₗ}\AgdaSpace{}%
\AgdaSymbol{\{}\AgdaBound{A}\AgdaSymbol{\}}\AgdaSpace{}%
\AgdaSymbol{\{}\AgdaBound{x}\AgdaSymbol{\}}\AgdaSpace{}%
\AgdaSymbol{\{}\AgdaBound{y}\AgdaSymbol{\}}\AgdaSpace{}%
\AgdaBound{p}\AgdaSpace{}%
\AgdaSymbol{=}\AgdaSpace{}%
\AgdaFunction{J}\AgdaSpace{}%
\AgdaFunction{D}\AgdaSpace{}%
\AgdaFunction{d}\AgdaSpace{}%
\AgdaBound{x}\AgdaSpace{}%
\AgdaBound{y}\AgdaSpace{}%
\AgdaBound{p}\<%
\\
\>[2][@{}l@{\AgdaIndent{0}}]%
\>[4]\AgdaKeyword{where}\<%
\\
\>[4][@{}l@{\AgdaIndent{0}}]%
\>[6]\AgdaFunction{D}\AgdaSpace{}%
\AgdaSymbol{:}\AgdaSpace{}%
\AgdaSymbol{(}\AgdaBound{x}\AgdaSpace{}%
\AgdaBound{y}\AgdaSpace{}%
\AgdaSymbol{:}\AgdaSpace{}%
\AgdaBound{A}\AgdaSymbol{)}\AgdaSpace{}%
\AgdaSymbol{→}\AgdaSpace{}%
\AgdaBound{x}\AgdaSpace{}%
\AgdaOperator{\AgdaDatatype{≡}}\AgdaSpace{}%
\AgdaBound{y}\AgdaSpace{}%
\AgdaSymbol{→}\AgdaSpace{}%
\AgdaPrimitive{Set}\<%
\\
%
\>[6]\AgdaFunction{D}\AgdaSpace{}%
\AgdaBound{x}\AgdaSpace{}%
\AgdaBound{y}\AgdaSpace{}%
\AgdaBound{p}\AgdaSpace{}%
\AgdaSymbol{=}\AgdaSpace{}%
\AgdaBound{p}\AgdaSpace{}%
\AgdaOperator{\AgdaDatatype{≡}}\AgdaSpace{}%
\AgdaInductiveConstructor{r}\AgdaSpace{}%
\AgdaOperator{\AgdaFunction{∙}}\AgdaSpace{}%
\AgdaBound{p}\<%
\\
%
\>[6]\AgdaFunction{d}\AgdaSpace{}%
\AgdaSymbol{:}\AgdaSpace{}%
\AgdaSymbol{(}\AgdaBound{a}\AgdaSpace{}%
\AgdaSymbol{:}\AgdaSpace{}%
\AgdaBound{A}\AgdaSymbol{)}\AgdaSpace{}%
\AgdaSymbol{→}\AgdaSpace{}%
\AgdaFunction{D}\AgdaSpace{}%
\AgdaBound{a}\AgdaSpace{}%
\AgdaBound{a}\AgdaSpace{}%
\AgdaInductiveConstructor{r}\<%
\\
%
\>[6]\AgdaFunction{d}\AgdaSpace{}%
\AgdaBound{a}\AgdaSpace{}%
\AgdaSymbol{=}\AgdaSpace{}%
\AgdaInductiveConstructor{r}\<%
\\
%
\\[\AgdaEmptyExtraSkip]%
%
\>[2]\AgdaFunction{iᵣ}\AgdaSpace{}%
\AgdaSymbol{:}\AgdaSpace{}%
\AgdaSymbol{\{}\AgdaBound{A}\AgdaSpace{}%
\AgdaSymbol{:}\AgdaSpace{}%
\AgdaPrimitive{Set}\AgdaSymbol{\}}\AgdaSpace{}%
\AgdaSymbol{\{}\AgdaBound{x}\AgdaSpace{}%
\AgdaBound{y}\AgdaSpace{}%
\AgdaSymbol{:}\AgdaSpace{}%
\AgdaBound{A}\AgdaSymbol{\}}\AgdaSpace{}%
\AgdaSymbol{(}\AgdaBound{p}\AgdaSpace{}%
\AgdaSymbol{:}\AgdaSpace{}%
\AgdaBound{x}\AgdaSpace{}%
\AgdaOperator{\AgdaDatatype{≡}}\AgdaSpace{}%
\AgdaBound{y}\AgdaSymbol{)}\AgdaSpace{}%
\AgdaSymbol{→}\AgdaSpace{}%
\AgdaBound{p}\AgdaSpace{}%
\AgdaOperator{\AgdaDatatype{≡}}\AgdaSpace{}%
\AgdaBound{p}\AgdaSpace{}%
\AgdaOperator{\AgdaFunction{∙}}\AgdaSpace{}%
\AgdaInductiveConstructor{r}\<%
\\
%
\>[2]\AgdaFunction{iᵣ}\AgdaSpace{}%
\AgdaSymbol{\{}\AgdaBound{A}\AgdaSymbol{\}}\AgdaSpace{}%
\AgdaSymbol{\{}\AgdaBound{x}\AgdaSymbol{\}}\AgdaSpace{}%
\AgdaSymbol{\{}\AgdaBound{y}\AgdaSymbol{\}}\AgdaSpace{}%
\AgdaBound{p}\AgdaSpace{}%
\AgdaSymbol{=}\AgdaSpace{}%
\AgdaFunction{J}\AgdaSpace{}%
\AgdaFunction{D}\AgdaSpace{}%
\AgdaFunction{d}\AgdaSpace{}%
\AgdaBound{x}\AgdaSpace{}%
\AgdaBound{y}\AgdaSpace{}%
\AgdaBound{p}\<%
\\
\>[2][@{}l@{\AgdaIndent{0}}]%
\>[4]\AgdaKeyword{where}\<%
\\
\>[4][@{}l@{\AgdaIndent{0}}]%
\>[6]\AgdaFunction{D}\AgdaSpace{}%
\AgdaSymbol{:}\AgdaSpace{}%
\AgdaSymbol{(}\AgdaBound{x}\AgdaSpace{}%
\AgdaBound{y}\AgdaSpace{}%
\AgdaSymbol{:}\AgdaSpace{}%
\AgdaBound{A}\AgdaSymbol{)}\AgdaSpace{}%
\AgdaSymbol{→}\AgdaSpace{}%
\AgdaBound{x}\AgdaSpace{}%
\AgdaOperator{\AgdaDatatype{≡}}\AgdaSpace{}%
\AgdaBound{y}\AgdaSpace{}%
\AgdaSymbol{→}\AgdaSpace{}%
\AgdaPrimitive{Set}\<%
\\
%
\>[6]\AgdaFunction{D}\AgdaSpace{}%
\AgdaBound{x}\AgdaSpace{}%
\AgdaBound{y}\AgdaSpace{}%
\AgdaBound{p}\AgdaSpace{}%
\AgdaSymbol{=}\AgdaSpace{}%
\AgdaBound{p}\AgdaSpace{}%
\AgdaOperator{\AgdaDatatype{≡}}\AgdaSpace{}%
\AgdaBound{p}\AgdaSpace{}%
\AgdaOperator{\AgdaFunction{∙}}\AgdaSpace{}%
\AgdaInductiveConstructor{r}\<%
\\
%
\>[6]\AgdaFunction{d}\AgdaSpace{}%
\AgdaSymbol{:}\AgdaSpace{}%
\AgdaSymbol{(}\AgdaBound{a}\AgdaSpace{}%
\AgdaSymbol{:}\AgdaSpace{}%
\AgdaBound{A}\AgdaSymbol{)}\AgdaSpace{}%
\AgdaSymbol{→}\AgdaSpace{}%
\AgdaFunction{D}\AgdaSpace{}%
\AgdaBound{a}\AgdaSpace{}%
\AgdaBound{a}\AgdaSpace{}%
\AgdaInductiveConstructor{r}\<%
\\
%
\>[6]\AgdaFunction{d}\AgdaSpace{}%
\AgdaBound{a}\AgdaSpace{}%
\AgdaSymbol{=}\AgdaSpace{}%
\AgdaInductiveConstructor{r}\<%
\\
%
\\[\AgdaEmptyExtraSkip]%
%
\>[2]\AgdaFunction{leftInverse}\AgdaSpace{}%
\AgdaSymbol{:}\AgdaSpace{}%
\AgdaSymbol{\{}\AgdaBound{A}\AgdaSpace{}%
\AgdaSymbol{:}\AgdaSpace{}%
\AgdaPrimitive{Set}\AgdaSymbol{\}}\AgdaSpace{}%
\AgdaSymbol{\{}\AgdaBound{x}\AgdaSpace{}%
\AgdaBound{y}\AgdaSpace{}%
\AgdaSymbol{:}\AgdaSpace{}%
\AgdaBound{A}\AgdaSymbol{\}}\AgdaSpace{}%
\AgdaSymbol{(}\AgdaBound{p}\AgdaSpace{}%
\AgdaSymbol{:}\AgdaSpace{}%
\AgdaBound{x}\AgdaSpace{}%
\AgdaOperator{\AgdaDatatype{≡}}\AgdaSpace{}%
\AgdaBound{y}\AgdaSymbol{)}\AgdaSpace{}%
\AgdaSymbol{→}\AgdaSpace{}%
\AgdaBound{p}\AgdaSpace{}%
\AgdaOperator{\AgdaFunction{⁻¹}}\AgdaSpace{}%
\AgdaOperator{\AgdaFunction{∙}}\AgdaSpace{}%
\AgdaBound{p}\AgdaSpace{}%
\AgdaOperator{\AgdaDatatype{≡}}\AgdaSpace{}%
\AgdaInductiveConstructor{r}\<%
\\
%
\>[2]\AgdaFunction{leftInverse}\AgdaSpace{}%
\AgdaSymbol{\{}\AgdaBound{A}\AgdaSymbol{\}}\AgdaSpace{}%
\AgdaSymbol{\{}\AgdaBound{x}\AgdaSymbol{\}}\AgdaSpace{}%
\AgdaSymbol{\{}\AgdaBound{y}\AgdaSymbol{\}}\AgdaSpace{}%
\AgdaBound{p}\AgdaSpace{}%
\AgdaSymbol{=}\AgdaSpace{}%
\AgdaFunction{J}\AgdaSpace{}%
\AgdaFunction{D}\AgdaSpace{}%
\AgdaFunction{d}\AgdaSpace{}%
\AgdaBound{x}\AgdaSpace{}%
\AgdaBound{y}\AgdaSpace{}%
\AgdaBound{p}\<%
\\
\>[2][@{}l@{\AgdaIndent{0}}]%
\>[4]\AgdaKeyword{where}\<%
\\
\>[4][@{}l@{\AgdaIndent{0}}]%
\>[6]\AgdaFunction{D}\AgdaSpace{}%
\AgdaSymbol{:}\AgdaSpace{}%
\AgdaSymbol{(}\AgdaBound{x}\AgdaSpace{}%
\AgdaBound{y}\AgdaSpace{}%
\AgdaSymbol{:}\AgdaSpace{}%
\AgdaBound{A}\AgdaSymbol{)}\AgdaSpace{}%
\AgdaSymbol{→}\AgdaSpace{}%
\AgdaBound{x}\AgdaSpace{}%
\AgdaOperator{\AgdaDatatype{≡}}\AgdaSpace{}%
\AgdaBound{y}\AgdaSpace{}%
\AgdaSymbol{→}\AgdaSpace{}%
\AgdaPrimitive{Set}\<%
\\
%
\>[6]\AgdaFunction{D}\AgdaSpace{}%
\AgdaBound{x}\AgdaSpace{}%
\AgdaBound{y}\AgdaSpace{}%
\AgdaBound{p}\AgdaSpace{}%
\AgdaSymbol{=}\AgdaSpace{}%
\AgdaBound{p}\AgdaSpace{}%
\AgdaOperator{\AgdaFunction{⁻¹}}\AgdaSpace{}%
\AgdaOperator{\AgdaFunction{∙}}\AgdaSpace{}%
\AgdaBound{p}\AgdaSpace{}%
\AgdaOperator{\AgdaDatatype{≡}}\AgdaSpace{}%
\AgdaInductiveConstructor{r}\<%
\\
%
\>[6]\AgdaFunction{d}\AgdaSpace{}%
\AgdaSymbol{:}\AgdaSpace{}%
\AgdaSymbol{(}\AgdaBound{x}\AgdaSpace{}%
\AgdaSymbol{:}\AgdaSpace{}%
\AgdaBound{A}\AgdaSymbol{)}\AgdaSpace{}%
\AgdaSymbol{→}\AgdaSpace{}%
\AgdaFunction{D}\AgdaSpace{}%
\AgdaBound{x}\AgdaSpace{}%
\AgdaBound{x}\AgdaSpace{}%
\AgdaInductiveConstructor{r}\<%
\\
%
\>[6]\AgdaFunction{d}\AgdaSpace{}%
\AgdaBound{x}\AgdaSpace{}%
\AgdaSymbol{=}\AgdaSpace{}%
\AgdaInductiveConstructor{r}\<%
\\
%
\\[\AgdaEmptyExtraSkip]%
%
\>[2]\AgdaFunction{rightInverse}\AgdaSpace{}%
\AgdaSymbol{:}\AgdaSpace{}%
\AgdaSymbol{\{}\AgdaBound{A}\AgdaSpace{}%
\AgdaSymbol{:}\AgdaSpace{}%
\AgdaPrimitive{Set}\AgdaSymbol{\}}\AgdaSpace{}%
\AgdaSymbol{\{}\AgdaBound{x}\AgdaSpace{}%
\AgdaBound{y}\AgdaSpace{}%
\AgdaSymbol{:}\AgdaSpace{}%
\AgdaBound{A}\AgdaSymbol{\}}\AgdaSpace{}%
\AgdaSymbol{(}\AgdaBound{p}\AgdaSpace{}%
\AgdaSymbol{:}\AgdaSpace{}%
\AgdaBound{x}\AgdaSpace{}%
\AgdaOperator{\AgdaDatatype{≡}}\AgdaSpace{}%
\AgdaBound{y}\AgdaSymbol{)}\AgdaSpace{}%
\AgdaSymbol{→}\AgdaSpace{}%
\AgdaBound{p}\AgdaSpace{}%
\AgdaOperator{\AgdaFunction{∙}}\AgdaSpace{}%
\AgdaBound{p}\AgdaSpace{}%
\AgdaOperator{\AgdaFunction{⁻¹}}\AgdaSpace{}%
\AgdaOperator{\AgdaDatatype{≡}}\AgdaSpace{}%
\AgdaInductiveConstructor{r}\<%
\\
%
\>[2]\AgdaFunction{rightInverse}\AgdaSpace{}%
\AgdaSymbol{\{}\AgdaBound{A}\AgdaSymbol{\}}\AgdaSpace{}%
\AgdaSymbol{\{}\AgdaBound{x}\AgdaSymbol{\}}\AgdaSpace{}%
\AgdaSymbol{\{}\AgdaBound{y}\AgdaSymbol{\}}\AgdaSpace{}%
\AgdaBound{p}\AgdaSpace{}%
\AgdaSymbol{=}\AgdaSpace{}%
\AgdaFunction{J}\AgdaSpace{}%
\AgdaFunction{D}\AgdaSpace{}%
\AgdaFunction{d}\AgdaSpace{}%
\AgdaBound{x}\AgdaSpace{}%
\AgdaBound{y}\AgdaSpace{}%
\AgdaBound{p}\<%
\\
\>[2][@{}l@{\AgdaIndent{0}}]%
\>[4]\AgdaKeyword{where}\<%
\\
\>[4][@{}l@{\AgdaIndent{0}}]%
\>[6]\AgdaFunction{D}\AgdaSpace{}%
\AgdaSymbol{:}\AgdaSpace{}%
\AgdaSymbol{(}\AgdaBound{x}\AgdaSpace{}%
\AgdaBound{y}\AgdaSpace{}%
\AgdaSymbol{:}\AgdaSpace{}%
\AgdaBound{A}\AgdaSymbol{)}\AgdaSpace{}%
\AgdaSymbol{→}\AgdaSpace{}%
\AgdaBound{x}\AgdaSpace{}%
\AgdaOperator{\AgdaDatatype{≡}}\AgdaSpace{}%
\AgdaBound{y}\AgdaSpace{}%
\AgdaSymbol{→}\AgdaSpace{}%
\AgdaPrimitive{Set}\<%
\\
%
\>[6]\AgdaFunction{D}\AgdaSpace{}%
\AgdaBound{x}\AgdaSpace{}%
\AgdaBound{y}\AgdaSpace{}%
\AgdaBound{p}\AgdaSpace{}%
\AgdaSymbol{=}\AgdaSpace{}%
\AgdaBound{p}\AgdaSpace{}%
\AgdaOperator{\AgdaFunction{∙}}\AgdaSpace{}%
\AgdaBound{p}\AgdaSpace{}%
\AgdaOperator{\AgdaFunction{⁻¹}}\AgdaSpace{}%
\AgdaOperator{\AgdaDatatype{≡}}\AgdaSpace{}%
\AgdaInductiveConstructor{r}\<%
\\
%
\>[6]\AgdaFunction{d}\AgdaSpace{}%
\AgdaSymbol{:}\AgdaSpace{}%
\AgdaSymbol{(}\AgdaBound{a}\AgdaSpace{}%
\AgdaSymbol{:}\AgdaSpace{}%
\AgdaBound{A}\AgdaSymbol{)}\AgdaSpace{}%
\AgdaSymbol{→}\AgdaSpace{}%
\AgdaFunction{D}\AgdaSpace{}%
\AgdaBound{a}\AgdaSpace{}%
\AgdaBound{a}\AgdaSpace{}%
\AgdaInductiveConstructor{r}\<%
\\
%
\>[6]\AgdaFunction{d}\AgdaSpace{}%
\AgdaBound{a}\AgdaSpace{}%
\AgdaSymbol{=}\AgdaSpace{}%
\AgdaInductiveConstructor{r}\<%
\\
%
\\[\AgdaEmptyExtraSkip]%
%
\>[2]\AgdaFunction{doubleInv}\AgdaSpace{}%
\AgdaSymbol{:}\AgdaSpace{}%
\AgdaSymbol{\{}\AgdaBound{A}\AgdaSpace{}%
\AgdaSymbol{:}\AgdaSpace{}%
\AgdaPrimitive{Set}\AgdaSymbol{\}}\AgdaSpace{}%
\AgdaSymbol{\{}\AgdaBound{x}\AgdaSpace{}%
\AgdaBound{y}\AgdaSpace{}%
\AgdaSymbol{:}\AgdaSpace{}%
\AgdaBound{A}\AgdaSymbol{\}}\AgdaSpace{}%
\AgdaSymbol{(}\AgdaBound{p}\AgdaSpace{}%
\AgdaSymbol{:}\AgdaSpace{}%
\AgdaBound{x}\AgdaSpace{}%
\AgdaOperator{\AgdaDatatype{≡}}\AgdaSpace{}%
\AgdaBound{y}\AgdaSymbol{)}\AgdaSpace{}%
\AgdaSymbol{→}\AgdaSpace{}%
\AgdaBound{p}\AgdaSpace{}%
\AgdaOperator{\AgdaFunction{⁻¹}}\AgdaSpace{}%
\AgdaOperator{\AgdaFunction{⁻¹}}\AgdaSpace{}%
\AgdaOperator{\AgdaDatatype{≡}}\AgdaSpace{}%
\AgdaBound{p}\<%
\\
%
\>[2]\AgdaFunction{doubleInv}\AgdaSpace{}%
\AgdaSymbol{\{}\AgdaBound{A}\AgdaSymbol{\}}\AgdaSpace{}%
\AgdaSymbol{\{}\AgdaBound{x}\AgdaSymbol{\}}\AgdaSpace{}%
\AgdaSymbol{\{}\AgdaBound{y}\AgdaSymbol{\}}\AgdaSpace{}%
\AgdaBound{p}\AgdaSpace{}%
\AgdaSymbol{=}\AgdaSpace{}%
\AgdaFunction{J}\AgdaSpace{}%
\AgdaFunction{D}\AgdaSpace{}%
\AgdaFunction{d}\AgdaSpace{}%
\AgdaBound{x}\AgdaSpace{}%
\AgdaBound{y}\AgdaSpace{}%
\AgdaBound{p}\<%
\\
\>[2][@{}l@{\AgdaIndent{0}}]%
\>[4]\AgdaKeyword{where}\<%
\\
\>[4][@{}l@{\AgdaIndent{0}}]%
\>[6]\AgdaFunction{D}\AgdaSpace{}%
\AgdaSymbol{:}\AgdaSpace{}%
\AgdaSymbol{(}\AgdaBound{x}\AgdaSpace{}%
\AgdaBound{y}\AgdaSpace{}%
\AgdaSymbol{:}\AgdaSpace{}%
\AgdaBound{A}\AgdaSymbol{)}\AgdaSpace{}%
\AgdaSymbol{→}\AgdaSpace{}%
\AgdaBound{x}\AgdaSpace{}%
\AgdaOperator{\AgdaDatatype{≡}}\AgdaSpace{}%
\AgdaBound{y}\AgdaSpace{}%
\AgdaSymbol{→}\AgdaSpace{}%
\AgdaPrimitive{Set}\<%
\\
%
\>[6]\AgdaFunction{D}\AgdaSpace{}%
\AgdaBound{x}\AgdaSpace{}%
\AgdaBound{y}\AgdaSpace{}%
\AgdaBound{p}\AgdaSpace{}%
\AgdaSymbol{=}\AgdaSpace{}%
\AgdaBound{p}\AgdaSpace{}%
\AgdaOperator{\AgdaFunction{⁻¹}}\AgdaSpace{}%
\AgdaOperator{\AgdaFunction{⁻¹}}\AgdaSpace{}%
\AgdaOperator{\AgdaDatatype{≡}}\AgdaSpace{}%
\AgdaBound{p}\<%
\\
%
\>[6]\AgdaFunction{d}\AgdaSpace{}%
\AgdaSymbol{:}\AgdaSpace{}%
\AgdaSymbol{(}\AgdaBound{a}\AgdaSpace{}%
\AgdaSymbol{:}\AgdaSpace{}%
\AgdaBound{A}\AgdaSymbol{)}\AgdaSpace{}%
\AgdaSymbol{→}\AgdaSpace{}%
\AgdaFunction{D}\AgdaSpace{}%
\AgdaBound{a}\AgdaSpace{}%
\AgdaBound{a}\AgdaSpace{}%
\AgdaInductiveConstructor{r}\<%
\\
%
\>[6]\AgdaFunction{d}\AgdaSpace{}%
\AgdaBound{a}\AgdaSpace{}%
\AgdaSymbol{=}\AgdaSpace{}%
\AgdaInductiveConstructor{r}\<%
\\
%
\\[\AgdaEmptyExtraSkip]%
%
\>[2]\AgdaFunction{associativity}\AgdaSpace{}%
\AgdaSymbol{:\{}\AgdaBound{A}\AgdaSpace{}%
\AgdaSymbol{:}\AgdaSpace{}%
\AgdaPrimitive{Set}\AgdaSymbol{\}}\AgdaSpace{}%
\AgdaSymbol{\{}\AgdaBound{x}\AgdaSpace{}%
\AgdaBound{y}\AgdaSpace{}%
\AgdaBound{z}\AgdaSpace{}%
\AgdaBound{w}\AgdaSpace{}%
\AgdaSymbol{:}\AgdaSpace{}%
\AgdaBound{A}\AgdaSymbol{\}}\AgdaSpace{}%
\AgdaSymbol{(}\AgdaBound{p}\AgdaSpace{}%
\AgdaSymbol{:}\AgdaSpace{}%
\AgdaBound{x}\AgdaSpace{}%
\AgdaOperator{\AgdaDatatype{≡}}\AgdaSpace{}%
\AgdaBound{y}\AgdaSymbol{)}\AgdaSpace{}%
\AgdaSymbol{(}\AgdaBound{q}\AgdaSpace{}%
\AgdaSymbol{:}\AgdaSpace{}%
\AgdaBound{y}\AgdaSpace{}%
\AgdaOperator{\AgdaDatatype{≡}}\AgdaSpace{}%
\AgdaBound{z}\AgdaSymbol{)}\AgdaSpace{}%
\AgdaSymbol{(}\AgdaBound{r'}\AgdaSpace{}%
\AgdaSymbol{:}\AgdaSpace{}%
\AgdaBound{z}\AgdaSpace{}%
\AgdaOperator{\AgdaDatatype{≡}}\AgdaSpace{}%
\AgdaBound{w}\AgdaSpace{}%
\AgdaSymbol{)}\AgdaSpace{}%
\AgdaSymbol{→}\AgdaSpace{}%
\AgdaBound{p}\AgdaSpace{}%
\AgdaOperator{\AgdaFunction{∙}}\AgdaSpace{}%
\AgdaSymbol{(}\AgdaBound{q}\AgdaSpace{}%
\AgdaOperator{\AgdaFunction{∙}}\AgdaSpace{}%
\AgdaBound{r'}\AgdaSymbol{)}\AgdaSpace{}%
\AgdaOperator{\AgdaDatatype{≡}}\AgdaSpace{}%
\AgdaBound{p}\AgdaSpace{}%
\AgdaOperator{\AgdaFunction{∙}}\AgdaSpace{}%
\AgdaBound{q}\AgdaSpace{}%
\AgdaOperator{\AgdaFunction{∙}}\AgdaSpace{}%
\AgdaBound{r'}\<%
\\
%
\>[2]\AgdaFunction{associativity}\AgdaSpace{}%
\AgdaSymbol{\{}\AgdaBound{A}\AgdaSymbol{\}}\AgdaSpace{}%
\AgdaSymbol{\{}\AgdaBound{x}\AgdaSymbol{\}}\AgdaSpace{}%
\AgdaSymbol{\{}\AgdaBound{y}\AgdaSymbol{\}}\AgdaSpace{}%
\AgdaSymbol{\{}\AgdaBound{z}\AgdaSymbol{\}}\AgdaSpace{}%
\AgdaSymbol{\{}\AgdaBound{w}\AgdaSymbol{\}}\AgdaSpace{}%
\AgdaBound{p}\AgdaSpace{}%
\AgdaBound{q}\AgdaSpace{}%
\AgdaBound{r'}\AgdaSpace{}%
\AgdaSymbol{=}\AgdaSpace{}%
\AgdaFunction{J}\AgdaSpace{}%
\AgdaFunction{D₁}\AgdaSpace{}%
\AgdaFunction{d₁}\AgdaSpace{}%
\AgdaBound{x}\AgdaSpace{}%
\AgdaBound{y}\AgdaSpace{}%
\AgdaBound{p}\AgdaSpace{}%
\AgdaBound{z}\AgdaSpace{}%
\AgdaBound{w}\AgdaSpace{}%
\AgdaBound{q}\AgdaSpace{}%
\AgdaBound{r'}\<%
\\
\>[2][@{}l@{\AgdaIndent{0}}]%
\>[4]\AgdaKeyword{where}\<%
\\
\>[4][@{}l@{\AgdaIndent{0}}]%
\>[6]\AgdaFunction{D₁}\AgdaSpace{}%
\AgdaSymbol{:}\AgdaSpace{}%
\AgdaSymbol{(}\AgdaBound{x}\AgdaSpace{}%
\AgdaBound{y}\AgdaSpace{}%
\AgdaSymbol{:}\AgdaSpace{}%
\AgdaBound{A}\AgdaSymbol{)}\AgdaSpace{}%
\AgdaSymbol{→}\AgdaSpace{}%
\AgdaBound{x}\AgdaSpace{}%
\AgdaOperator{\AgdaDatatype{≡}}\AgdaSpace{}%
\AgdaBound{y}\AgdaSpace{}%
\AgdaSymbol{→}\AgdaSpace{}%
\AgdaPrimitive{Set}\<%
\\
%
\>[6]\AgdaFunction{D₁}\AgdaSpace{}%
\AgdaBound{x}\AgdaSpace{}%
\AgdaBound{y}\AgdaSpace{}%
\AgdaBound{p}\AgdaSpace{}%
\AgdaSymbol{=}\AgdaSpace{}%
\AgdaSymbol{(}\AgdaBound{z}\AgdaSpace{}%
\AgdaBound{w}\AgdaSpace{}%
\AgdaSymbol{:}\AgdaSpace{}%
\AgdaBound{A}\AgdaSymbol{)}\AgdaSpace{}%
\AgdaSymbol{(}\AgdaBound{q}\AgdaSpace{}%
\AgdaSymbol{:}\AgdaSpace{}%
\AgdaBound{y}\AgdaSpace{}%
\AgdaOperator{\AgdaDatatype{≡}}\AgdaSpace{}%
\AgdaBound{z}\AgdaSymbol{)}\AgdaSpace{}%
\AgdaSymbol{(}\AgdaBound{r'}\AgdaSpace{}%
\AgdaSymbol{:}\AgdaSpace{}%
\AgdaBound{z}\AgdaSpace{}%
\AgdaOperator{\AgdaDatatype{≡}}\AgdaSpace{}%
\AgdaBound{w}\AgdaSpace{}%
\AgdaSymbol{)}\AgdaSpace{}%
\AgdaSymbol{→}\AgdaSpace{}%
\AgdaBound{p}\AgdaSpace{}%
\AgdaOperator{\AgdaFunction{∙}}\AgdaSpace{}%
\AgdaSymbol{(}\AgdaBound{q}\AgdaSpace{}%
\AgdaOperator{\AgdaFunction{∙}}\AgdaSpace{}%
\AgdaBound{r'}\AgdaSymbol{)}\AgdaSpace{}%
\AgdaOperator{\AgdaDatatype{≡}}\AgdaSpace{}%
\AgdaBound{p}\AgdaSpace{}%
\AgdaOperator{\AgdaFunction{∙}}\AgdaSpace{}%
\AgdaBound{q}\AgdaSpace{}%
\AgdaOperator{\AgdaFunction{∙}}\AgdaSpace{}%
\AgdaBound{r'}\<%
\\
%
\>[6]\AgdaComment{-- d₁ : (x : A) → D₁ x x r }\<%
\\
%
\>[6]\AgdaComment{-- d₁ x z w q r' = r -- why can it infer this }\<%
\\
%
\>[6]\AgdaFunction{D₂}\AgdaSpace{}%
\AgdaSymbol{:}\AgdaSpace{}%
\AgdaSymbol{(}\AgdaBound{x}\AgdaSpace{}%
\AgdaBound{z}\AgdaSpace{}%
\AgdaSymbol{:}\AgdaSpace{}%
\AgdaBound{A}\AgdaSymbol{)}\AgdaSpace{}%
\AgdaSymbol{→}\AgdaSpace{}%
\AgdaBound{x}\AgdaSpace{}%
\AgdaOperator{\AgdaDatatype{≡}}\AgdaSpace{}%
\AgdaBound{z}\AgdaSpace{}%
\AgdaSymbol{→}\AgdaSpace{}%
\AgdaPrimitive{Set}\<%
\\
%
\>[6]\AgdaFunction{D₂}\AgdaSpace{}%
\AgdaBound{x}\AgdaSpace{}%
\AgdaBound{z}\AgdaSpace{}%
\AgdaBound{q}\AgdaSpace{}%
\AgdaSymbol{=}\AgdaSpace{}%
\AgdaSymbol{(}\AgdaBound{w}\AgdaSpace{}%
\AgdaSymbol{:}\AgdaSpace{}%
\AgdaBound{A}\AgdaSymbol{)}\AgdaSpace{}%
\AgdaSymbol{(}\AgdaBound{r'}\AgdaSpace{}%
\AgdaSymbol{:}\AgdaSpace{}%
\AgdaBound{z}\AgdaSpace{}%
\AgdaOperator{\AgdaDatatype{≡}}\AgdaSpace{}%
\AgdaBound{w}\AgdaSpace{}%
\AgdaSymbol{)}\AgdaSpace{}%
\AgdaSymbol{→}\AgdaSpace{}%
\AgdaInductiveConstructor{r}\AgdaSpace{}%
\AgdaOperator{\AgdaFunction{∙}}\AgdaSpace{}%
\AgdaSymbol{(}\AgdaBound{q}\AgdaSpace{}%
\AgdaOperator{\AgdaFunction{∙}}\AgdaSpace{}%
\AgdaBound{r'}\AgdaSymbol{)}\AgdaSpace{}%
\AgdaOperator{\AgdaDatatype{≡}}\AgdaSpace{}%
\AgdaInductiveConstructor{r}\AgdaSpace{}%
\AgdaOperator{\AgdaFunction{∙}}\AgdaSpace{}%
\AgdaBound{q}\AgdaSpace{}%
\AgdaOperator{\AgdaFunction{∙}}\AgdaSpace{}%
\AgdaBound{r'}\<%
\\
%
\>[6]\AgdaFunction{D₃}\AgdaSpace{}%
\AgdaSymbol{:}\AgdaSpace{}%
\AgdaSymbol{(}\AgdaBound{x}\AgdaSpace{}%
\AgdaBound{w}\AgdaSpace{}%
\AgdaSymbol{:}\AgdaSpace{}%
\AgdaBound{A}\AgdaSymbol{)}\AgdaSpace{}%
\AgdaSymbol{→}\AgdaSpace{}%
\AgdaBound{x}\AgdaSpace{}%
\AgdaOperator{\AgdaDatatype{≡}}\AgdaSpace{}%
\AgdaBound{w}\AgdaSpace{}%
\AgdaSymbol{→}\AgdaSpace{}%
\AgdaPrimitive{Set}\<%
\\
%
\>[6]\AgdaFunction{D₃}\AgdaSpace{}%
\AgdaBound{x}\AgdaSpace{}%
\AgdaBound{w}\AgdaSpace{}%
\AgdaBound{r'}\AgdaSpace{}%
\AgdaSymbol{=}%
\>[19]\AgdaInductiveConstructor{r}\AgdaSpace{}%
\AgdaOperator{\AgdaFunction{∙}}\AgdaSpace{}%
\AgdaSymbol{(}\AgdaInductiveConstructor{r}\AgdaSpace{}%
\AgdaOperator{\AgdaFunction{∙}}\AgdaSpace{}%
\AgdaBound{r'}\AgdaSymbol{)}\AgdaSpace{}%
\AgdaOperator{\AgdaDatatype{≡}}\AgdaSpace{}%
\AgdaInductiveConstructor{r}\AgdaSpace{}%
\AgdaOperator{\AgdaFunction{∙}}\AgdaSpace{}%
\AgdaInductiveConstructor{r}\AgdaSpace{}%
\AgdaOperator{\AgdaFunction{∙}}\AgdaSpace{}%
\AgdaBound{r'}\<%
\\
%
\>[6]\AgdaFunction{d₃}\AgdaSpace{}%
\AgdaSymbol{:}\AgdaSpace{}%
\AgdaSymbol{(}\AgdaBound{x}\AgdaSpace{}%
\AgdaSymbol{:}\AgdaSpace{}%
\AgdaBound{A}\AgdaSymbol{)}\AgdaSpace{}%
\AgdaSymbol{→}\AgdaSpace{}%
\AgdaFunction{D₃}\AgdaSpace{}%
\AgdaBound{x}\AgdaSpace{}%
\AgdaBound{x}\AgdaSpace{}%
\AgdaInductiveConstructor{r}\<%
\\
%
\>[6]\AgdaFunction{d₃}\AgdaSpace{}%
\AgdaBound{x}\AgdaSpace{}%
\AgdaSymbol{=}\AgdaSpace{}%
\AgdaInductiveConstructor{r}\<%
\\
%
\>[6]\AgdaFunction{d₂}\AgdaSpace{}%
\AgdaSymbol{:}\AgdaSpace{}%
\AgdaSymbol{(}\AgdaBound{x}\AgdaSpace{}%
\AgdaSymbol{:}\AgdaSpace{}%
\AgdaBound{A}\AgdaSymbol{)}\AgdaSpace{}%
\AgdaSymbol{→}\AgdaSpace{}%
\AgdaFunction{D₂}\AgdaSpace{}%
\AgdaBound{x}\AgdaSpace{}%
\AgdaBound{x}\AgdaSpace{}%
\AgdaInductiveConstructor{r}\<%
\\
%
\>[6]\AgdaFunction{d₂}\AgdaSpace{}%
\AgdaBound{x}\AgdaSpace{}%
\AgdaBound{w}\AgdaSpace{}%
\AgdaBound{r'}\AgdaSpace{}%
\AgdaSymbol{=}\AgdaSpace{}%
\AgdaFunction{J}\AgdaSpace{}%
\AgdaFunction{D₃}\AgdaSpace{}%
\AgdaFunction{d₃}\AgdaSpace{}%
\AgdaBound{x}\AgdaSpace{}%
\AgdaBound{w}\AgdaSpace{}%
\AgdaBound{r'}\<%
\\
%
\>[6]\AgdaFunction{d₁}\AgdaSpace{}%
\AgdaSymbol{:}\AgdaSpace{}%
\AgdaSymbol{(}\AgdaBound{x}\AgdaSpace{}%
\AgdaSymbol{:}\AgdaSpace{}%
\AgdaBound{A}\AgdaSymbol{)}\AgdaSpace{}%
\AgdaSymbol{→}\AgdaSpace{}%
\AgdaFunction{D₁}\AgdaSpace{}%
\AgdaBound{x}\AgdaSpace{}%
\AgdaBound{x}\AgdaSpace{}%
\AgdaInductiveConstructor{r}\<%
\\
%
\>[6]\AgdaFunction{d₁}\AgdaSpace{}%
\AgdaBound{x}\AgdaSpace{}%
\AgdaBound{z}\AgdaSpace{}%
\AgdaBound{w}\AgdaSpace{}%
\AgdaBound{q}\AgdaSpace{}%
\AgdaBound{r'}\AgdaSpace{}%
\AgdaSymbol{=}\AgdaSpace{}%
\AgdaFunction{J}\AgdaSpace{}%
\AgdaFunction{D₂}\AgdaSpace{}%
\AgdaFunction{d₂}\AgdaSpace{}%
\AgdaBound{x}\AgdaSpace{}%
\AgdaBound{z}\AgdaSpace{}%
\AgdaBound{q}\AgdaSpace{}%
\AgdaBound{w}\AgdaSpace{}%
\AgdaBound{r'}\<%
\\
\>[0]\<%
\end{code}

When one starts to look at structure above the groupoid level, i.e., the paths between paths between paths level, some interesting and nonintuitive results emerge. If one defines a path space that is seemingly trivial, namely, taking the same starting and end points, the higherdimensional structure yields non-trivial structure. 
We now arrive at the first ``interesting'' result in this book, the Eckmann-Hilton Arguement. It says that composition on the loop space of a loop space, the second loop space, is commutitive.



\begin{definition}

Thus, given a type $A$ with a point $a:A$, we define its \define{loop space}
\index{loop space}%
$\Omega(A,a)$ to be the type $\id[A]{a}{a}$.
We may sometimes write simply $\Omega A$ if the point $a$ is understood from context.

\end {definition}


\begin{definition}
It can also be useful to consider the loop space\index{loop space!iterated}\index{iterated loop space} \emph{of} the loop space of $A$, which is the space of 2-dimensional loops on the identity loop at $a$.
This is written $\Omega^2(A,a)$ and represented in type theory by the type $\id[({\id[A]{a}{a}})]{\refl{a}}{\refl{a}}$.
\end {definition}

\begin{thm}[Eckmann--Hilton]%\label{thm:EckmannHilton}
  The composition operation on the second loop space
  %
  \begin{equation*}
    \Omega^2(A)\times \Omega^2(A)\to \Omega^2(A)
  \end{equation*}
  is commutative: $\alpha\cdot\beta = \beta\cdot\alpha$, for any $\alpha, \beta:\Omega^2(A)$.
  %\index{Eckmann--Hilton argument}%
\end{thm}

\begin{proof}
First, observe that the composition of $1$-loops $\Omega A\times \Omega A\to \Omega A$ induces an operation
\[
\star : \Omega^2(A)\times \Omega^2(A)\to \Omega^2(A)
\]
as follows: consider elements $a, b, c : A$ and 1- and 2-paths,
%
\begin{align*}
 p &: a = b,       &       r &: b = c \\
 q &: a = b,       &       s &: b = c \\
 \alpha &: p = q,  &   \beta &: r = s
\end{align*}
%
as depicted in the following diagram (with paths drawn as arrows).

[TODO Finish Eckmann Hilton Arguement]
%\[
 %\xymatrix@+5em{
   %{a} \rtwocell<10>^p_q{\alpha}
   %&
   %{b} \rtwocell<10>^r_s{\beta}
   %&
   %{c}
 %}
%\]
%Composing the upper and lower 1-paths, respectively, we get two paths $p\ct r,\ q\ct s : a = c$, and there is then a ``horizontal composition''
%%
%\begin{equation*}
  %\alpha\hct\beta : p\ct r = q\ct s
%\end{equation*}
%%
%between them, defined as follows.
%First, we define $\alpha \rightwhisker r : p\ct r = q\ct r$ by path induction on $r$, so that
%\[ \alpha \rightwhisker \refl{b} \jdeq \opp{\mathsf{ru}_p} \ct \alpha \ct \mathsf{ru}_q \]
%where $\mathsf{ru}_p : p = p \ct \refl{b}$ is the right unit law from \cref{thm:omg}\ref{item:omg1}.
%We could similarly define $\rightwhisker$ by induction on $\alpha$, or on all paths in sight, resulting in different judgmental equalities, but for present purposes the definition by induction on $r$ will make things simpler.
%Similarly, we define $q\leftwhisker \beta : q\ct r = q\ct s$ by induction on $q$, so that
%\[ \refl{b} \leftwhisker \beta \jdeq \opp{\mathsf{lu}_r} \ct \beta \ct \mathsf{lu}_s \]
%where $\mathsf{lu}_r$ denotes the left unit law.
%The operations $\leftwhisker$ and $\rightwhisker$ are called \define{whiskering}\indexdef{whiskering}.
%Next, since $\alpha \rightwhisker r$ and $q\leftwhisker \beta$ are composable 2-paths, we can define the \define{horizontal composition}
%\indexdef{horizontal composition!of paths}%
%\indexdef{composition!of paths!horizontal}%
%by:
%\[
%\alpha\hct\beta\ \defeq\ (\alpha\rightwhisker r) \ct (q\leftwhisker \beta).
%\]
%Now suppose that $a \jdeq  b \jdeq  c$, so that all the 1-paths $p$, $q$, $r$, and $s$ are elements of $\Omega(A,a)$, and assume moreover that $p\jdeq q \jdeq r \jdeq s\jdeq \refl{a}$, so that $\alpha:\refl{a} = \refl{a}$ and $\beta:\refl{a} = \refl{a}$ are composable in both orders.
%In that case, we have
%\begin{align*}
  %\alpha\hct\beta
  %&\jdeq (\alpha\rightwhisker\refl{a}) \ct (\refl{a}\leftwhisker \beta)\\
  %&= \opp{\mathsf{ru}_{\refl{a}}} \ct \alpha \ct \mathsf{ru}_{\refl{a}} \ct \opp{\mathsf{lu}_{\refl a}} \ct \beta \ct \mathsf{lu}_{\refl{a}}\\
  %&\jdeq \opp{\refl{\refl{a}}} \ct \alpha \ct \refl{\refl{a}} \ct \opp{\refl{\refl a}} \ct \beta \ct \refl{\refl{a}}\\
  %&= \alpha \ct \beta.
%\end{align*}
%(Recall that $\mathsf{ru}_{\refl{a}} \jdeq \mathsf{lu}_{\refl{a}} \jdeq \refl{\refl{a}}$, by the computation rule for path induction.)
%On the other hand, we can define another horizontal composition analogously by
%\[
%\alpha\hct'\beta\ \defeq\ (p\leftwhisker \beta)\ct (\alpha\rightwhisker s)
%\]
%and we similarly learn that
%\[
%\alpha\hct'\beta = \beta\ct\alpha.
%\]
%\index{interchange law}%
%But, in general, the two ways of defining horizontal composition agree, $\alpha\hct\beta = \alpha\hct'\beta$, as we can see by induction on $\alpha$ and $\beta$ and then on the two remaining 1-paths, to reduce everything to reflexivity.
%Thus we have
%\[\alpha \ct \beta = \alpha\hct\beta = \alpha\hct'\beta = \beta\ct\alpha.
%\qedhere
%\]
\end{proof}


[Todo, clean up code so that its more tightly correspondent to the book proof]
The corresponding agda code is below :

\begin{code}%
\>[0]\<%
\\
\>[0][@{}l@{\AgdaIndent{1}}]%
\>[2]\AgdaComment{-- whiskering}\<%
\\
%
\>[2]\AgdaOperator{\AgdaFunction{\AgdaUnderscore{}∙ᵣ\AgdaUnderscore{}}}\AgdaSpace{}%
\AgdaSymbol{:}\AgdaSpace{}%
\AgdaSymbol{\{}\AgdaBound{A}\AgdaSpace{}%
\AgdaSymbol{:}\AgdaSpace{}%
\AgdaPrimitive{Set}\AgdaSymbol{\}}\AgdaSpace{}%
\AgdaSymbol{→}\AgdaSpace{}%
\AgdaSymbol{\{}\AgdaBound{b}\AgdaSpace{}%
\AgdaBound{c}\AgdaSpace{}%
\AgdaSymbol{:}\AgdaSpace{}%
\AgdaBound{A}\AgdaSymbol{\}}\AgdaSpace{}%
\AgdaSymbol{\{}\AgdaBound{a}\AgdaSpace{}%
\AgdaSymbol{:}\AgdaSpace{}%
\AgdaBound{A}\AgdaSymbol{\}}\AgdaSpace{}%
\AgdaSymbol{\{}\AgdaBound{p}\AgdaSpace{}%
\AgdaBound{q}\AgdaSpace{}%
\AgdaSymbol{:}\AgdaSpace{}%
\AgdaBound{a}\AgdaSpace{}%
\AgdaOperator{\AgdaDatatype{≡}}\AgdaSpace{}%
\AgdaBound{b}\AgdaSymbol{\}}\AgdaSpace{}%
\AgdaSymbol{(}\AgdaBound{α}\AgdaSpace{}%
\AgdaSymbol{:}\AgdaSpace{}%
\AgdaBound{p}\AgdaSpace{}%
\AgdaOperator{\AgdaDatatype{≡}}\AgdaSpace{}%
\AgdaBound{q}\AgdaSymbol{)}\AgdaSpace{}%
\AgdaSymbol{(}\AgdaBound{r'}\AgdaSpace{}%
\AgdaSymbol{:}\AgdaSpace{}%
\AgdaBound{b}\AgdaSpace{}%
\AgdaOperator{\AgdaDatatype{≡}}\AgdaSpace{}%
\AgdaBound{c}\AgdaSymbol{)}\AgdaSpace{}%
\AgdaSymbol{→}\AgdaSpace{}%
\AgdaBound{p}\AgdaSpace{}%
\AgdaOperator{\AgdaFunction{∙}}\AgdaSpace{}%
\AgdaBound{r'}\AgdaSpace{}%
\AgdaOperator{\AgdaDatatype{≡}}\AgdaSpace{}%
\AgdaBound{q}\AgdaSpace{}%
\AgdaOperator{\AgdaFunction{∙}}\AgdaSpace{}%
\AgdaBound{r'}\<%
\\
%
\>[2]\AgdaOperator{\AgdaFunction{\AgdaUnderscore{}∙ᵣ\AgdaUnderscore{}}}\AgdaSpace{}%
\AgdaSymbol{\{}\AgdaBound{A}\AgdaSymbol{\}}\AgdaSpace{}%
\AgdaSymbol{\{}\AgdaBound{b}\AgdaSymbol{\}}\AgdaSpace{}%
\AgdaSymbol{\{}\AgdaBound{c}\AgdaSymbol{\}}\AgdaSpace{}%
\AgdaSymbol{\{}\AgdaBound{a}\AgdaSymbol{\}}\AgdaSpace{}%
\AgdaSymbol{\{}\AgdaBound{p}\AgdaSymbol{\}}\AgdaSpace{}%
\AgdaSymbol{\{}\AgdaBound{q}\AgdaSymbol{\}}\AgdaSpace{}%
\AgdaBound{α}\AgdaSpace{}%
\AgdaBound{r'}\AgdaSpace{}%
\AgdaSymbol{=}\AgdaSpace{}%
\AgdaFunction{J}\AgdaSpace{}%
\AgdaFunction{D}\AgdaSpace{}%
\AgdaFunction{d}\AgdaSpace{}%
\AgdaBound{b}\AgdaSpace{}%
\AgdaBound{c}\AgdaSpace{}%
\AgdaBound{r'}\AgdaSpace{}%
\AgdaBound{a}\AgdaSpace{}%
\AgdaBound{α}\<%
\\
\>[2][@{}l@{\AgdaIndent{0}}]%
\>[4]\AgdaKeyword{where}\<%
\\
\>[4][@{}l@{\AgdaIndent{0}}]%
\>[6]\AgdaFunction{D}\AgdaSpace{}%
\AgdaSymbol{:}\AgdaSpace{}%
\AgdaSymbol{(}\AgdaBound{b}\AgdaSpace{}%
\AgdaBound{c}\AgdaSpace{}%
\AgdaSymbol{:}\AgdaSpace{}%
\AgdaBound{A}\AgdaSymbol{)}\AgdaSpace{}%
\AgdaSymbol{→}\AgdaSpace{}%
\AgdaBound{b}\AgdaSpace{}%
\AgdaOperator{\AgdaDatatype{≡}}\AgdaSpace{}%
\AgdaBound{c}\AgdaSpace{}%
\AgdaSymbol{→}\AgdaSpace{}%
\AgdaPrimitive{Set}\<%
\\
%
\>[6]\AgdaFunction{D}\AgdaSpace{}%
\AgdaBound{b}\AgdaSpace{}%
\AgdaBound{c}\AgdaSpace{}%
\AgdaBound{r'}\AgdaSpace{}%
\AgdaSymbol{=}\AgdaSpace{}%
\AgdaSymbol{(}\AgdaBound{a}\AgdaSpace{}%
\AgdaSymbol{:}\AgdaSpace{}%
\AgdaBound{A}\AgdaSymbol{)}\AgdaSpace{}%
\AgdaSymbol{\{}\AgdaBound{p}\AgdaSpace{}%
\AgdaBound{q}\AgdaSpace{}%
\AgdaSymbol{:}\AgdaSpace{}%
\AgdaBound{a}\AgdaSpace{}%
\AgdaOperator{\AgdaDatatype{≡}}\AgdaSpace{}%
\AgdaBound{b}\AgdaSymbol{\}}\AgdaSpace{}%
\AgdaSymbol{(}\AgdaBound{α}\AgdaSpace{}%
\AgdaSymbol{:}\AgdaSpace{}%
\AgdaBound{p}\AgdaSpace{}%
\AgdaOperator{\AgdaDatatype{≡}}\AgdaSpace{}%
\AgdaBound{q}\AgdaSymbol{)}\AgdaSpace{}%
\AgdaSymbol{→}\AgdaSpace{}%
\AgdaBound{p}\AgdaSpace{}%
\AgdaOperator{\AgdaFunction{∙}}\AgdaSpace{}%
\AgdaBound{r'}\AgdaSpace{}%
\AgdaOperator{\AgdaDatatype{≡}}\AgdaSpace{}%
\AgdaBound{q}\AgdaSpace{}%
\AgdaOperator{\AgdaFunction{∙}}\AgdaSpace{}%
\AgdaBound{r'}\<%
\\
%
\>[6]\AgdaFunction{d}\AgdaSpace{}%
\AgdaSymbol{:}\AgdaSpace{}%
\AgdaSymbol{(}\AgdaBound{a}\AgdaSpace{}%
\AgdaSymbol{:}\AgdaSpace{}%
\AgdaBound{A}\AgdaSymbol{)}\AgdaSpace{}%
\AgdaSymbol{→}\AgdaSpace{}%
\AgdaFunction{D}\AgdaSpace{}%
\AgdaBound{a}\AgdaSpace{}%
\AgdaBound{a}\AgdaSpace{}%
\AgdaInductiveConstructor{r}\<%
\\
%
\>[6]\AgdaFunction{d}\AgdaSpace{}%
\AgdaBound{a}\AgdaSpace{}%
\AgdaBound{a'}\AgdaSpace{}%
\AgdaSymbol{\{}\AgdaBound{p}\AgdaSymbol{\}}\AgdaSpace{}%
\AgdaSymbol{\{}\AgdaBound{q}\AgdaSymbol{\}}\AgdaSpace{}%
\AgdaBound{α}\AgdaSpace{}%
\AgdaSymbol{=}\AgdaSpace{}%
\AgdaFunction{iᵣ}\AgdaSpace{}%
\AgdaBound{p}\AgdaSpace{}%
\AgdaOperator{\AgdaFunction{⁻¹}}\AgdaSpace{}%
\AgdaOperator{\AgdaFunction{∙}}\AgdaSpace{}%
\AgdaBound{α}\AgdaSpace{}%
\AgdaOperator{\AgdaFunction{∙}}\AgdaSpace{}%
\AgdaFunction{iᵣ}\AgdaSpace{}%
\AgdaBound{q}\<%
\\
%
\\[\AgdaEmptyExtraSkip]%
%
\>[2]\AgdaComment{-- iᵣ == ruₚ}\<%
\\
%
\\[\AgdaEmptyExtraSkip]%
%
\>[2]\AgdaOperator{\AgdaFunction{\AgdaUnderscore{}∙ₗ\AgdaUnderscore{}}}\AgdaSpace{}%
\AgdaSymbol{:}\AgdaSpace{}%
\AgdaSymbol{\{}\AgdaBound{A}\AgdaSpace{}%
\AgdaSymbol{:}\AgdaSpace{}%
\AgdaPrimitive{Set}\AgdaSymbol{\}}\AgdaSpace{}%
\AgdaSymbol{→}\AgdaSpace{}%
\AgdaSymbol{\{}\AgdaBound{a}\AgdaSpace{}%
\AgdaBound{b}\AgdaSpace{}%
\AgdaSymbol{:}\AgdaSpace{}%
\AgdaBound{A}\AgdaSymbol{\}}\AgdaSpace{}%
\AgdaSymbol{(}\AgdaBound{q}\AgdaSpace{}%
\AgdaSymbol{:}\AgdaSpace{}%
\AgdaBound{a}\AgdaSpace{}%
\AgdaOperator{\AgdaDatatype{≡}}\AgdaSpace{}%
\AgdaBound{b}\AgdaSymbol{)}\AgdaSpace{}%
\AgdaSymbol{\{}\AgdaBound{c}\AgdaSpace{}%
\AgdaSymbol{:}\AgdaSpace{}%
\AgdaBound{A}\AgdaSymbol{\}}\AgdaSpace{}%
\AgdaSymbol{\{}\AgdaBound{r'}\AgdaSpace{}%
\AgdaBound{s}\AgdaSpace{}%
\AgdaSymbol{:}\AgdaSpace{}%
\AgdaBound{b}\AgdaSpace{}%
\AgdaOperator{\AgdaDatatype{≡}}\AgdaSpace{}%
\AgdaBound{c}\AgdaSymbol{\}}\AgdaSpace{}%
\AgdaSymbol{(}\AgdaBound{β}\AgdaSpace{}%
\AgdaSymbol{:}\AgdaSpace{}%
\AgdaBound{r'}\AgdaSpace{}%
\AgdaOperator{\AgdaDatatype{≡}}\AgdaSpace{}%
\AgdaBound{s}\AgdaSymbol{)}\AgdaSpace{}%
\AgdaSymbol{→}\AgdaSpace{}%
\AgdaBound{q}\AgdaSpace{}%
\AgdaOperator{\AgdaFunction{∙}}\AgdaSpace{}%
\AgdaBound{r'}\AgdaSpace{}%
\AgdaOperator{\AgdaDatatype{≡}}\AgdaSpace{}%
\AgdaBound{q}\AgdaSpace{}%
\AgdaOperator{\AgdaFunction{∙}}\AgdaSpace{}%
\AgdaBound{s}\<%
\\
%
\>[2]\AgdaOperator{\AgdaFunction{\AgdaUnderscore{}∙ₗ\AgdaUnderscore{}}}\AgdaSpace{}%
\AgdaSymbol{\{}\AgdaBound{A}\AgdaSymbol{\}}\AgdaSpace{}%
\AgdaSymbol{\{}\AgdaBound{a}\AgdaSymbol{\}}\AgdaSpace{}%
\AgdaSymbol{\{}\AgdaBound{b}\AgdaSymbol{\}}\AgdaSpace{}%
\AgdaBound{q}\AgdaSpace{}%
\AgdaSymbol{\{}\AgdaBound{c}\AgdaSymbol{\}}\AgdaSpace{}%
\AgdaSymbol{\{}\AgdaBound{r'}\AgdaSymbol{\}}\AgdaSpace{}%
\AgdaSymbol{\{}\AgdaBound{s}\AgdaSymbol{\}}\AgdaSpace{}%
\AgdaBound{β}\AgdaSpace{}%
\AgdaSymbol{=}\AgdaSpace{}%
\AgdaFunction{J}\AgdaSpace{}%
\AgdaFunction{D}\AgdaSpace{}%
\AgdaFunction{d}\AgdaSpace{}%
\AgdaBound{a}\AgdaSpace{}%
\AgdaBound{b}\AgdaSpace{}%
\AgdaBound{q}\AgdaSpace{}%
\AgdaBound{c}\AgdaSpace{}%
\AgdaBound{β}\<%
\\
\>[2][@{}l@{\AgdaIndent{0}}]%
\>[4]\AgdaKeyword{where}\<%
\\
\>[4][@{}l@{\AgdaIndent{0}}]%
\>[6]\AgdaFunction{D}\AgdaSpace{}%
\AgdaSymbol{:}\AgdaSpace{}%
\AgdaSymbol{(}\AgdaBound{a}\AgdaSpace{}%
\AgdaBound{b}\AgdaSpace{}%
\AgdaSymbol{:}\AgdaSpace{}%
\AgdaBound{A}\AgdaSymbol{)}\AgdaSpace{}%
\AgdaSymbol{→}\AgdaSpace{}%
\AgdaBound{a}\AgdaSpace{}%
\AgdaOperator{\AgdaDatatype{≡}}\AgdaSpace{}%
\AgdaBound{b}\AgdaSpace{}%
\AgdaSymbol{→}\AgdaSpace{}%
\AgdaPrimitive{Set}\<%
\\
%
\>[6]\AgdaFunction{D}\AgdaSpace{}%
\AgdaBound{a}\AgdaSpace{}%
\AgdaBound{b}\AgdaSpace{}%
\AgdaBound{q}\AgdaSpace{}%
\AgdaSymbol{=}\AgdaSpace{}%
\AgdaSymbol{(}\AgdaBound{c}\AgdaSpace{}%
\AgdaSymbol{:}\AgdaSpace{}%
\AgdaBound{A}\AgdaSymbol{)}\AgdaSpace{}%
\AgdaSymbol{\{}\AgdaBound{r'}\AgdaSpace{}%
\AgdaBound{s}\AgdaSpace{}%
\AgdaSymbol{:}\AgdaSpace{}%
\AgdaBound{b}\AgdaSpace{}%
\AgdaOperator{\AgdaDatatype{≡}}\AgdaSpace{}%
\AgdaBound{c}\AgdaSymbol{\}}\AgdaSpace{}%
\AgdaSymbol{(}\AgdaBound{β}\AgdaSpace{}%
\AgdaSymbol{:}\AgdaSpace{}%
\AgdaBound{r'}\AgdaSpace{}%
\AgdaOperator{\AgdaDatatype{≡}}\AgdaSpace{}%
\AgdaBound{s}\AgdaSymbol{)}\AgdaSpace{}%
\AgdaSymbol{→}\AgdaSpace{}%
\AgdaBound{q}\AgdaSpace{}%
\AgdaOperator{\AgdaFunction{∙}}\AgdaSpace{}%
\AgdaBound{r'}\AgdaSpace{}%
\AgdaOperator{\AgdaDatatype{≡}}\AgdaSpace{}%
\AgdaBound{q}\AgdaSpace{}%
\AgdaOperator{\AgdaFunction{∙}}\AgdaSpace{}%
\AgdaBound{s}\<%
\\
%
\>[6]\AgdaFunction{d}\AgdaSpace{}%
\AgdaSymbol{:}\AgdaSpace{}%
\AgdaSymbol{(}\AgdaBound{a}\AgdaSpace{}%
\AgdaSymbol{:}\AgdaSpace{}%
\AgdaBound{A}\AgdaSymbol{)}\AgdaSpace{}%
\AgdaSymbol{→}\AgdaSpace{}%
\AgdaFunction{D}\AgdaSpace{}%
\AgdaBound{a}\AgdaSpace{}%
\AgdaBound{a}\AgdaSpace{}%
\AgdaInductiveConstructor{r}\<%
\\
%
\>[6]\AgdaFunction{d}\AgdaSpace{}%
\AgdaBound{a}\AgdaSpace{}%
\AgdaBound{a'}\AgdaSpace{}%
\AgdaSymbol{\{}\AgdaBound{r'}\AgdaSymbol{\}}\AgdaSpace{}%
\AgdaSymbol{\{}\AgdaBound{s}\AgdaSymbol{\}}\AgdaSpace{}%
\AgdaBound{β}\AgdaSpace{}%
\AgdaSymbol{=}\AgdaSpace{}%
\AgdaFunction{iₗ}\AgdaSpace{}%
\AgdaBound{r'}\AgdaSpace{}%
\AgdaOperator{\AgdaFunction{⁻¹}}\AgdaSpace{}%
\AgdaOperator{\AgdaFunction{∙}}\AgdaSpace{}%
\AgdaBound{β}\AgdaSpace{}%
\AgdaOperator{\AgdaFunction{∙}}\AgdaSpace{}%
\AgdaFunction{iₗ}\AgdaSpace{}%
\AgdaBound{s}\<%
\\
%
\\[\AgdaEmptyExtraSkip]%
%
\>[2]\AgdaOperator{\AgdaFunction{\AgdaUnderscore{}⋆\AgdaUnderscore{}}}\AgdaSpace{}%
\AgdaSymbol{:}\AgdaSpace{}%
\AgdaSymbol{\{}\AgdaBound{A}\AgdaSpace{}%
\AgdaSymbol{:}\AgdaSpace{}%
\AgdaPrimitive{Set}\AgdaSymbol{\}}\AgdaSpace{}%
\AgdaSymbol{→}\AgdaSpace{}%
\AgdaSymbol{\{}\AgdaBound{a}\AgdaSpace{}%
\AgdaBound{b}\AgdaSpace{}%
\AgdaBound{c}\AgdaSpace{}%
\AgdaSymbol{:}\AgdaSpace{}%
\AgdaBound{A}\AgdaSymbol{\}}\AgdaSpace{}%
\AgdaSymbol{\{}\AgdaBound{p}\AgdaSpace{}%
\AgdaBound{q}\AgdaSpace{}%
\AgdaSymbol{:}\AgdaSpace{}%
\AgdaBound{a}\AgdaSpace{}%
\AgdaOperator{\AgdaDatatype{≡}}\AgdaSpace{}%
\AgdaBound{b}\AgdaSymbol{\}}\AgdaSpace{}%
\AgdaSymbol{\{}\AgdaBound{r'}\AgdaSpace{}%
\AgdaBound{s}\AgdaSpace{}%
\AgdaSymbol{:}\AgdaSpace{}%
\AgdaBound{b}\AgdaSpace{}%
\AgdaOperator{\AgdaDatatype{≡}}\AgdaSpace{}%
\AgdaBound{c}\AgdaSymbol{\}}\AgdaSpace{}%
\AgdaSymbol{(}\AgdaBound{α}\AgdaSpace{}%
\AgdaSymbol{:}\AgdaSpace{}%
\AgdaBound{p}\AgdaSpace{}%
\AgdaOperator{\AgdaDatatype{≡}}\AgdaSpace{}%
\AgdaBound{q}\AgdaSymbol{)}\AgdaSpace{}%
\AgdaSymbol{(}\AgdaBound{β}\AgdaSpace{}%
\AgdaSymbol{:}\AgdaSpace{}%
\AgdaBound{r'}\AgdaSpace{}%
\AgdaOperator{\AgdaDatatype{≡}}\AgdaSpace{}%
\AgdaBound{s}\AgdaSymbol{)}\AgdaSpace{}%
\AgdaSymbol{→}\AgdaSpace{}%
\AgdaBound{p}\AgdaSpace{}%
\AgdaOperator{\AgdaFunction{∙}}\AgdaSpace{}%
\AgdaBound{r'}\AgdaSpace{}%
\AgdaOperator{\AgdaDatatype{≡}}\AgdaSpace{}%
\AgdaBound{q}\AgdaSpace{}%
\AgdaOperator{\AgdaFunction{∙}}\AgdaSpace{}%
\AgdaBound{s}\<%
\\
%
\>[2]\AgdaOperator{\AgdaFunction{\AgdaUnderscore{}⋆\AgdaUnderscore{}}}\AgdaSpace{}%
\AgdaSymbol{\{}\AgdaBound{A}\AgdaSymbol{\}}\AgdaSpace{}%
\AgdaSymbol{\{}\AgdaArgument{q}\AgdaSpace{}%
\AgdaSymbol{=}\AgdaSpace{}%
\AgdaBound{q}\AgdaSymbol{\}}\AgdaSpace{}%
\AgdaSymbol{\{}\AgdaArgument{r'}\AgdaSpace{}%
\AgdaSymbol{=}\AgdaSpace{}%
\AgdaBound{r'}\AgdaSymbol{\}}\AgdaSpace{}%
\AgdaBound{α}\AgdaSpace{}%
\AgdaBound{β}\AgdaSpace{}%
\AgdaSymbol{=}\AgdaSpace{}%
\AgdaSymbol{(}\AgdaBound{α}\AgdaSpace{}%
\AgdaOperator{\AgdaFunction{∙ᵣ}}\AgdaSpace{}%
\AgdaBound{r'}\AgdaSymbol{)}\AgdaSpace{}%
\AgdaOperator{\AgdaFunction{∙}}\AgdaSpace{}%
\AgdaSymbol{(}\AgdaBound{q}\AgdaSpace{}%
\AgdaOperator{\AgdaFunction{∙ₗ}}\AgdaSpace{}%
\AgdaBound{β}\AgdaSymbol{)}\<%
\\
%
\\[\AgdaEmptyExtraSkip]%
%
\>[2]\AgdaOperator{\AgdaFunction{\AgdaUnderscore{}⋆'\AgdaUnderscore{}}}\AgdaSpace{}%
\AgdaSymbol{:}\AgdaSpace{}%
\AgdaSymbol{\{}\AgdaBound{A}\AgdaSpace{}%
\AgdaSymbol{:}\AgdaSpace{}%
\AgdaPrimitive{Set}\AgdaSymbol{\}}\AgdaSpace{}%
\AgdaSymbol{→}\AgdaSpace{}%
\AgdaSymbol{\{}\AgdaBound{a}\AgdaSpace{}%
\AgdaBound{b}\AgdaSpace{}%
\AgdaBound{c}\AgdaSpace{}%
\AgdaSymbol{:}\AgdaSpace{}%
\AgdaBound{A}\AgdaSymbol{\}}\AgdaSpace{}%
\AgdaSymbol{\{}\AgdaBound{p}\AgdaSpace{}%
\AgdaBound{q}\AgdaSpace{}%
\AgdaSymbol{:}\AgdaSpace{}%
\AgdaBound{a}\AgdaSpace{}%
\AgdaOperator{\AgdaDatatype{≡}}\AgdaSpace{}%
\AgdaBound{b}\AgdaSymbol{\}}\AgdaSpace{}%
\AgdaSymbol{\{}\AgdaBound{r'}\AgdaSpace{}%
\AgdaBound{s}\AgdaSpace{}%
\AgdaSymbol{:}\AgdaSpace{}%
\AgdaBound{b}\AgdaSpace{}%
\AgdaOperator{\AgdaDatatype{≡}}\AgdaSpace{}%
\AgdaBound{c}\AgdaSymbol{\}}\AgdaSpace{}%
\AgdaSymbol{(}\AgdaBound{α}\AgdaSpace{}%
\AgdaSymbol{:}\AgdaSpace{}%
\AgdaBound{p}\AgdaSpace{}%
\AgdaOperator{\AgdaDatatype{≡}}\AgdaSpace{}%
\AgdaBound{q}\AgdaSymbol{)}\AgdaSpace{}%
\AgdaSymbol{(}\AgdaBound{β}\AgdaSpace{}%
\AgdaSymbol{:}\AgdaSpace{}%
\AgdaBound{r'}\AgdaSpace{}%
\AgdaOperator{\AgdaDatatype{≡}}\AgdaSpace{}%
\AgdaBound{s}\AgdaSymbol{)}\AgdaSpace{}%
\AgdaSymbol{→}\AgdaSpace{}%
\AgdaBound{p}\AgdaSpace{}%
\AgdaOperator{\AgdaFunction{∙}}\AgdaSpace{}%
\AgdaBound{r'}\AgdaSpace{}%
\AgdaOperator{\AgdaDatatype{≡}}\AgdaSpace{}%
\AgdaBound{q}\AgdaSpace{}%
\AgdaOperator{\AgdaFunction{∙}}\AgdaSpace{}%
\AgdaBound{s}\<%
\\
%
\>[2]\AgdaOperator{\AgdaFunction{\AgdaUnderscore{}⋆'\AgdaUnderscore{}}}\AgdaSpace{}%
\AgdaSymbol{\{}\AgdaBound{A}\AgdaSymbol{\}}\AgdaSpace{}%
\AgdaSymbol{\{}\AgdaArgument{p}\AgdaSpace{}%
\AgdaSymbol{=}\AgdaSpace{}%
\AgdaBound{p}\AgdaSymbol{\}}\AgdaSpace{}%
\AgdaSymbol{\{}\AgdaArgument{s}\AgdaSpace{}%
\AgdaSymbol{=}\AgdaSpace{}%
\AgdaBound{s}\AgdaSymbol{\}}\AgdaSpace{}%
\AgdaBound{α}\AgdaSpace{}%
\AgdaBound{β}\AgdaSpace{}%
\AgdaSymbol{=}%
\>[34]\AgdaSymbol{(}\AgdaBound{p}\AgdaSpace{}%
\AgdaOperator{\AgdaFunction{∙ₗ}}\AgdaSpace{}%
\AgdaBound{β}\AgdaSymbol{)}\AgdaSpace{}%
\AgdaOperator{\AgdaFunction{∙}}\AgdaSpace{}%
\AgdaSymbol{(}\AgdaBound{α}\AgdaSpace{}%
\AgdaOperator{\AgdaFunction{∙ᵣ}}\AgdaSpace{}%
\AgdaBound{s}\AgdaSymbol{)}\<%
\\
%
\\[\AgdaEmptyExtraSkip]%
%
\>[2]\AgdaFunction{Ω}\AgdaSpace{}%
\AgdaSymbol{:}\AgdaSpace{}%
\AgdaSymbol{\{}\AgdaBound{A}\AgdaSpace{}%
\AgdaSymbol{:}\AgdaSpace{}%
\AgdaPrimitive{Set}\AgdaSymbol{\}}\AgdaSpace{}%
\AgdaSymbol{(}\AgdaBound{a}\AgdaSpace{}%
\AgdaSymbol{:}\AgdaSpace{}%
\AgdaBound{A}\AgdaSymbol{)}\AgdaSpace{}%
\AgdaSymbol{→}\AgdaSpace{}%
\AgdaPrimitive{Set}\<%
\\
%
\>[2]\AgdaFunction{Ω}\AgdaSpace{}%
\AgdaSymbol{\{}\AgdaBound{A}\AgdaSymbol{\}}\AgdaSpace{}%
\AgdaBound{a}\AgdaSpace{}%
\AgdaSymbol{=}\AgdaSpace{}%
\AgdaBound{a}\AgdaSpace{}%
\AgdaOperator{\AgdaDatatype{≡}}\AgdaSpace{}%
\AgdaBound{a}\<%
\\
%
\\[\AgdaEmptyExtraSkip]%
%
\>[2]\AgdaFunction{Ω²}\AgdaSpace{}%
\AgdaSymbol{:}\AgdaSpace{}%
\AgdaSymbol{\{}\AgdaBound{A}\AgdaSpace{}%
\AgdaSymbol{:}\AgdaSpace{}%
\AgdaPrimitive{Set}\AgdaSymbol{\}}\AgdaSpace{}%
\AgdaSymbol{(}\AgdaBound{a}\AgdaSpace{}%
\AgdaSymbol{:}\AgdaSpace{}%
\AgdaBound{A}\AgdaSymbol{)}\AgdaSpace{}%
\AgdaSymbol{→}\AgdaSpace{}%
\AgdaPrimitive{Set}\<%
\\
%
\>[2]\AgdaFunction{Ω²}\AgdaSpace{}%
\AgdaSymbol{\{}\AgdaBound{A}\AgdaSymbol{\}}\AgdaSpace{}%
\AgdaBound{a}\AgdaSpace{}%
\AgdaSymbol{=}\AgdaSpace{}%
\AgdaOperator{\AgdaDatatype{\AgdaUnderscore{}≡\AgdaUnderscore{}}}\AgdaSpace{}%
\AgdaSymbol{\{}\AgdaBound{a}\AgdaSpace{}%
\AgdaOperator{\AgdaDatatype{≡}}\AgdaSpace{}%
\AgdaBound{a}\AgdaSymbol{\}}\AgdaSpace{}%
\AgdaInductiveConstructor{r}\AgdaSpace{}%
\AgdaInductiveConstructor{r}\<%
\\
%
\\[\AgdaEmptyExtraSkip]%
%
\>[2]\AgdaFunction{lem1}\AgdaSpace{}%
\AgdaSymbol{:}\AgdaSpace{}%
\AgdaSymbol{\{}\AgdaBound{A}\AgdaSpace{}%
\AgdaSymbol{:}\AgdaSpace{}%
\AgdaPrimitive{Set}\AgdaSymbol{\}}\AgdaSpace{}%
\AgdaSymbol{→}\AgdaSpace{}%
\AgdaSymbol{(}\AgdaBound{a}\AgdaSpace{}%
\AgdaSymbol{:}\AgdaSpace{}%
\AgdaBound{A}\AgdaSymbol{)}\AgdaSpace{}%
\AgdaSymbol{→}\AgdaSpace{}%
\AgdaSymbol{(}\AgdaBound{α}\AgdaSpace{}%
\AgdaBound{β}\AgdaSpace{}%
\AgdaSymbol{:}\AgdaSpace{}%
\AgdaFunction{Ω²}\AgdaSpace{}%
\AgdaSymbol{\{}\AgdaBound{A}\AgdaSymbol{\}}\AgdaSpace{}%
\AgdaBound{a}\AgdaSymbol{)}\AgdaSpace{}%
\AgdaSymbol{→}\AgdaSpace{}%
\AgdaSymbol{(}\AgdaBound{α}\AgdaSpace{}%
\AgdaOperator{\AgdaFunction{⋆}}\AgdaSpace{}%
\AgdaBound{β}\AgdaSymbol{)}\AgdaSpace{}%
\AgdaOperator{\AgdaDatatype{≡}}\AgdaSpace{}%
\AgdaSymbol{(}\AgdaFunction{iᵣ}\AgdaSpace{}%
\AgdaInductiveConstructor{r}\AgdaSpace{}%
\AgdaOperator{\AgdaFunction{⁻¹}}\AgdaSpace{}%
\AgdaOperator{\AgdaFunction{∙}}\AgdaSpace{}%
\AgdaBound{α}\AgdaSpace{}%
\AgdaOperator{\AgdaFunction{∙}}\AgdaSpace{}%
\AgdaFunction{iᵣ}\AgdaSpace{}%
\AgdaInductiveConstructor{r}\AgdaSymbol{)}\AgdaSpace{}%
\AgdaOperator{\AgdaFunction{∙}}\AgdaSpace{}%
\AgdaSymbol{(}\AgdaFunction{iₗ}\AgdaSpace{}%
\AgdaInductiveConstructor{r}\AgdaSpace{}%
\AgdaOperator{\AgdaFunction{⁻¹}}\AgdaSpace{}%
\AgdaOperator{\AgdaFunction{∙}}\AgdaSpace{}%
\AgdaBound{β}\AgdaSpace{}%
\AgdaOperator{\AgdaFunction{∙}}\AgdaSpace{}%
\AgdaFunction{iₗ}\AgdaSpace{}%
\AgdaInductiveConstructor{r}\AgdaSymbol{)}\<%
\\
%
\>[2]\AgdaFunction{lem1}\AgdaSpace{}%
\AgdaBound{a}\AgdaSpace{}%
\AgdaBound{α}\AgdaSpace{}%
\AgdaBound{β}\AgdaSpace{}%
\AgdaSymbol{=}\AgdaSpace{}%
\AgdaInductiveConstructor{r}\<%
\\
%
\\[\AgdaEmptyExtraSkip]%
%
\>[2]\AgdaFunction{lem1'}\AgdaSpace{}%
\AgdaSymbol{:}\AgdaSpace{}%
\AgdaSymbol{\{}\AgdaBound{A}\AgdaSpace{}%
\AgdaSymbol{:}\AgdaSpace{}%
\AgdaPrimitive{Set}\AgdaSymbol{\}}\AgdaSpace{}%
\AgdaSymbol{→}\AgdaSpace{}%
\AgdaSymbol{(}\AgdaBound{a}\AgdaSpace{}%
\AgdaSymbol{:}\AgdaSpace{}%
\AgdaBound{A}\AgdaSymbol{)}\AgdaSpace{}%
\AgdaSymbol{→}\AgdaSpace{}%
\AgdaSymbol{(}\AgdaBound{α}\AgdaSpace{}%
\AgdaBound{β}\AgdaSpace{}%
\AgdaSymbol{:}\AgdaSpace{}%
\AgdaFunction{Ω²}\AgdaSpace{}%
\AgdaSymbol{\{}\AgdaBound{A}\AgdaSymbol{\}}\AgdaSpace{}%
\AgdaBound{a}\AgdaSymbol{)}\AgdaSpace{}%
\AgdaSymbol{→}\AgdaSpace{}%
\AgdaSymbol{(}\AgdaBound{α}\AgdaSpace{}%
\AgdaOperator{\AgdaFunction{⋆'}}\AgdaSpace{}%
\AgdaBound{β}\AgdaSymbol{)}\AgdaSpace{}%
\AgdaOperator{\AgdaDatatype{≡}}%
\>[63]\AgdaSymbol{(}\AgdaFunction{iₗ}\AgdaSpace{}%
\AgdaInductiveConstructor{r}\AgdaSpace{}%
\AgdaOperator{\AgdaFunction{⁻¹}}\AgdaSpace{}%
\AgdaOperator{\AgdaFunction{∙}}\AgdaSpace{}%
\AgdaBound{β}\AgdaSpace{}%
\AgdaOperator{\AgdaFunction{∙}}\AgdaSpace{}%
\AgdaFunction{iₗ}\AgdaSpace{}%
\AgdaInductiveConstructor{r}\AgdaSymbol{)}\AgdaSpace{}%
\AgdaOperator{\AgdaFunction{∙}}\AgdaSpace{}%
\AgdaSymbol{(}\AgdaFunction{iᵣ}\AgdaSpace{}%
\AgdaInductiveConstructor{r}\AgdaSpace{}%
\AgdaOperator{\AgdaFunction{⁻¹}}\AgdaSpace{}%
\AgdaOperator{\AgdaFunction{∙}}\AgdaSpace{}%
\AgdaBound{α}\AgdaSpace{}%
\AgdaOperator{\AgdaFunction{∙}}\AgdaSpace{}%
\AgdaFunction{iᵣ}\AgdaSpace{}%
\AgdaInductiveConstructor{r}\AgdaSymbol{)}\<%
\\
%
\>[2]\AgdaFunction{lem1'}\AgdaSpace{}%
\AgdaBound{a}\AgdaSpace{}%
\AgdaBound{α}\AgdaSpace{}%
\AgdaBound{β}\AgdaSpace{}%
\AgdaSymbol{=}\AgdaSpace{}%
\AgdaInductiveConstructor{r}\<%
\\
%
\\[\AgdaEmptyExtraSkip]%
%
\>[2]\AgdaComment{-- ap\textbackslash{}\AgdaUnderscore{}}\<%
\\
%
\>[2]\AgdaFunction{apf}\AgdaSpace{}%
\AgdaSymbol{:}\AgdaSpace{}%
\AgdaSymbol{\{}\AgdaBound{A}\AgdaSpace{}%
\AgdaBound{B}\AgdaSpace{}%
\AgdaSymbol{:}\AgdaSpace{}%
\AgdaPrimitive{Set}\AgdaSymbol{\}}\AgdaSpace{}%
\AgdaSymbol{→}\AgdaSpace{}%
\AgdaSymbol{\{}\AgdaBound{x}\AgdaSpace{}%
\AgdaBound{y}\AgdaSpace{}%
\AgdaSymbol{:}\AgdaSpace{}%
\AgdaBound{A}\AgdaSymbol{\}}\AgdaSpace{}%
\AgdaSymbol{→}\AgdaSpace{}%
\AgdaSymbol{(}\AgdaBound{f}\AgdaSpace{}%
\AgdaSymbol{:}\AgdaSpace{}%
\AgdaBound{A}\AgdaSpace{}%
\AgdaSymbol{→}\AgdaSpace{}%
\AgdaBound{B}\AgdaSymbol{)}\AgdaSpace{}%
\AgdaSymbol{→}\AgdaSpace{}%
\AgdaSymbol{(}\AgdaBound{x}\AgdaSpace{}%
\AgdaOperator{\AgdaDatatype{≡}}\AgdaSpace{}%
\AgdaBound{y}\AgdaSymbol{)}\AgdaSpace{}%
\AgdaSymbol{→}\AgdaSpace{}%
\AgdaBound{f}\AgdaSpace{}%
\AgdaBound{x}\AgdaSpace{}%
\AgdaOperator{\AgdaDatatype{≡}}\AgdaSpace{}%
\AgdaBound{f}\AgdaSpace{}%
\AgdaBound{y}\<%
\\
%
\>[2]\AgdaFunction{apf}\AgdaSpace{}%
\AgdaSymbol{\{}\AgdaBound{A}\AgdaSymbol{\}}\AgdaSpace{}%
\AgdaSymbol{\{}\AgdaBound{B}\AgdaSymbol{\}}\AgdaSpace{}%
\AgdaSymbol{\{}\AgdaBound{x}\AgdaSymbol{\}}\AgdaSpace{}%
\AgdaSymbol{\{}\AgdaBound{y}\AgdaSymbol{\}}\AgdaSpace{}%
\AgdaBound{f}\AgdaSpace{}%
\AgdaBound{p}\AgdaSpace{}%
\AgdaSymbol{=}\AgdaSpace{}%
\AgdaFunction{J}\AgdaSpace{}%
\AgdaFunction{D}\AgdaSpace{}%
\AgdaFunction{d}\AgdaSpace{}%
\AgdaBound{x}\AgdaSpace{}%
\AgdaBound{y}\AgdaSpace{}%
\AgdaBound{p}\<%
\\
\>[2][@{}l@{\AgdaIndent{0}}]%
\>[4]\AgdaKeyword{where}\<%
\\
\>[4][@{}l@{\AgdaIndent{0}}]%
\>[6]\AgdaFunction{D}\AgdaSpace{}%
\AgdaSymbol{:}\AgdaSpace{}%
\AgdaSymbol{(}\AgdaBound{x}\AgdaSpace{}%
\AgdaBound{y}\AgdaSpace{}%
\AgdaSymbol{:}\AgdaSpace{}%
\AgdaBound{A}\AgdaSymbol{)}\AgdaSpace{}%
\AgdaSymbol{→}\AgdaSpace{}%
\AgdaBound{x}\AgdaSpace{}%
\AgdaOperator{\AgdaDatatype{≡}}\AgdaSpace{}%
\AgdaBound{y}\AgdaSpace{}%
\AgdaSymbol{→}\AgdaSpace{}%
\AgdaPrimitive{Set}\<%
\\
%
\>[6]\AgdaFunction{D}\AgdaSpace{}%
\AgdaBound{x}\AgdaSpace{}%
\AgdaBound{y}\AgdaSpace{}%
\AgdaBound{p}\AgdaSpace{}%
\AgdaSymbol{=}\AgdaSpace{}%
\AgdaSymbol{\{}\AgdaBound{f}\AgdaSpace{}%
\AgdaSymbol{:}\AgdaSpace{}%
\AgdaBound{A}\AgdaSpace{}%
\AgdaSymbol{→}\AgdaSpace{}%
\AgdaBound{B}\AgdaSymbol{\}}\AgdaSpace{}%
\AgdaSymbol{→}\AgdaSpace{}%
\AgdaBound{f}\AgdaSpace{}%
\AgdaBound{x}\AgdaSpace{}%
\AgdaOperator{\AgdaDatatype{≡}}\AgdaSpace{}%
\AgdaBound{f}\AgdaSpace{}%
\AgdaBound{y}\<%
\\
%
\>[6]\AgdaFunction{d}\AgdaSpace{}%
\AgdaSymbol{:}\AgdaSpace{}%
\AgdaSymbol{(}\AgdaBound{x}\AgdaSpace{}%
\AgdaSymbol{:}\AgdaSpace{}%
\AgdaBound{A}\AgdaSymbol{)}\AgdaSpace{}%
\AgdaSymbol{→}\AgdaSpace{}%
\AgdaFunction{D}\AgdaSpace{}%
\AgdaBound{x}\AgdaSpace{}%
\AgdaBound{x}\AgdaSpace{}%
\AgdaInductiveConstructor{r}\<%
\\
%
\>[6]\AgdaFunction{d}\AgdaSpace{}%
\AgdaSymbol{=}\AgdaSpace{}%
\AgdaSymbol{λ}\AgdaSpace{}%
\AgdaBound{x}\AgdaSpace{}%
\AgdaSymbol{→}\AgdaSpace{}%
\AgdaInductiveConstructor{r}\<%
\\
%
\\[\AgdaEmptyExtraSkip]%
%
\>[2]\AgdaFunction{ap}\AgdaSpace{}%
\AgdaSymbol{:}\AgdaSpace{}%
\AgdaSymbol{\{}\AgdaBound{A}\AgdaSpace{}%
\AgdaBound{B}\AgdaSpace{}%
\AgdaSymbol{:}\AgdaSpace{}%
\AgdaPrimitive{Set}\AgdaSymbol{\}}\AgdaSpace{}%
\AgdaSymbol{→}\AgdaSpace{}%
\AgdaSymbol{\{}\AgdaBound{x}\AgdaSpace{}%
\AgdaBound{y}\AgdaSpace{}%
\AgdaSymbol{:}\AgdaSpace{}%
\AgdaBound{A}\AgdaSymbol{\}}\AgdaSpace{}%
\AgdaSymbol{→}\AgdaSpace{}%
\AgdaSymbol{(}\AgdaBound{f}\AgdaSpace{}%
\AgdaSymbol{:}\AgdaSpace{}%
\AgdaBound{A}\AgdaSpace{}%
\AgdaSymbol{→}\AgdaSpace{}%
\AgdaBound{B}\AgdaSymbol{)}\AgdaSpace{}%
\AgdaSymbol{→}\AgdaSpace{}%
\AgdaSymbol{(}\AgdaBound{x}\AgdaSpace{}%
\AgdaOperator{\AgdaDatatype{≡}}\AgdaSpace{}%
\AgdaBound{y}\AgdaSymbol{)}\AgdaSpace{}%
\AgdaSymbol{→}\AgdaSpace{}%
\AgdaBound{f}\AgdaSpace{}%
\AgdaBound{x}\AgdaSpace{}%
\AgdaOperator{\AgdaDatatype{≡}}\AgdaSpace{}%
\AgdaBound{f}\AgdaSpace{}%
\AgdaBound{y}\<%
\\
%
\>[2]\AgdaFunction{ap}\AgdaSpace{}%
\AgdaBound{f}\AgdaSpace{}%
\AgdaInductiveConstructor{r}\AgdaSpace{}%
\AgdaSymbol{=}\AgdaSpace{}%
\AgdaInductiveConstructor{r}\<%
\\
%
\\[\AgdaEmptyExtraSkip]%
%
\>[2]\AgdaFunction{lem20}\AgdaSpace{}%
\AgdaSymbol{:}\AgdaSpace{}%
\AgdaSymbol{\{}\AgdaBound{A}\AgdaSpace{}%
\AgdaSymbol{:}\AgdaSpace{}%
\AgdaPrimitive{Set}\AgdaSymbol{\}}\AgdaSpace{}%
\AgdaSymbol{→}\AgdaSpace{}%
\AgdaSymbol{\{}\AgdaBound{a}\AgdaSpace{}%
\AgdaSymbol{:}\AgdaSpace{}%
\AgdaBound{A}\AgdaSymbol{\}}\AgdaSpace{}%
\AgdaSymbol{→}\AgdaSpace{}%
\AgdaSymbol{(}\AgdaBound{α}\AgdaSpace{}%
\AgdaSymbol{:}\AgdaSpace{}%
\AgdaFunction{Ω²}\AgdaSpace{}%
\AgdaSymbol{\{}\AgdaBound{A}\AgdaSymbol{\}}\AgdaSpace{}%
\AgdaBound{a}\AgdaSymbol{)}\AgdaSpace{}%
\AgdaSymbol{→}\AgdaSpace{}%
\AgdaSymbol{(}\AgdaFunction{iᵣ}\AgdaSpace{}%
\AgdaInductiveConstructor{r}\AgdaSpace{}%
\AgdaOperator{\AgdaFunction{⁻¹}}\AgdaSpace{}%
\AgdaOperator{\AgdaFunction{∙}}\AgdaSpace{}%
\AgdaBound{α}\AgdaSpace{}%
\AgdaOperator{\AgdaFunction{∙}}\AgdaSpace{}%
\AgdaFunction{iᵣ}\AgdaSpace{}%
\AgdaInductiveConstructor{r}\AgdaSymbol{)}\AgdaSpace{}%
\AgdaOperator{\AgdaDatatype{≡}}\AgdaSpace{}%
\AgdaBound{α}\<%
\\
%
\>[2]\AgdaFunction{lem20}\AgdaSpace{}%
\AgdaBound{α}\AgdaSpace{}%
\AgdaSymbol{=}\AgdaSpace{}%
\AgdaFunction{iᵣ}\AgdaSpace{}%
\AgdaSymbol{(}\AgdaBound{α}\AgdaSymbol{)}\AgdaSpace{}%
\AgdaOperator{\AgdaFunction{⁻¹}}\<%
\\
%
\\[\AgdaEmptyExtraSkip]%
%
\>[2]\AgdaFunction{lem21}\AgdaSpace{}%
\AgdaSymbol{:}\AgdaSpace{}%
\AgdaSymbol{\{}\AgdaBound{A}\AgdaSpace{}%
\AgdaSymbol{:}\AgdaSpace{}%
\AgdaPrimitive{Set}\AgdaSymbol{\}}\AgdaSpace{}%
\AgdaSymbol{→}\AgdaSpace{}%
\AgdaSymbol{\{}\AgdaBound{a}\AgdaSpace{}%
\AgdaSymbol{:}\AgdaSpace{}%
\AgdaBound{A}\AgdaSymbol{\}}\AgdaSpace{}%
\AgdaSymbol{→}\AgdaSpace{}%
\AgdaSymbol{(}\AgdaBound{β}\AgdaSpace{}%
\AgdaSymbol{:}\AgdaSpace{}%
\AgdaFunction{Ω²}\AgdaSpace{}%
\AgdaSymbol{\{}\AgdaBound{A}\AgdaSymbol{\}}\AgdaSpace{}%
\AgdaBound{a}\AgdaSymbol{)}\AgdaSpace{}%
\AgdaSymbol{→}\AgdaSpace{}%
\AgdaSymbol{(}\AgdaFunction{iₗ}\AgdaSpace{}%
\AgdaInductiveConstructor{r}\AgdaSpace{}%
\AgdaOperator{\AgdaFunction{⁻¹}}\AgdaSpace{}%
\AgdaOperator{\AgdaFunction{∙}}\AgdaSpace{}%
\AgdaBound{β}\AgdaSpace{}%
\AgdaOperator{\AgdaFunction{∙}}\AgdaSpace{}%
\AgdaFunction{iₗ}\AgdaSpace{}%
\AgdaInductiveConstructor{r}\AgdaSymbol{)}\AgdaSpace{}%
\AgdaOperator{\AgdaDatatype{≡}}\AgdaSpace{}%
\AgdaBound{β}\<%
\\
%
\>[2]\AgdaFunction{lem21}\AgdaSpace{}%
\AgdaBound{β}\AgdaSpace{}%
\AgdaSymbol{=}\AgdaSpace{}%
\AgdaFunction{iᵣ}\AgdaSpace{}%
\AgdaSymbol{(}\AgdaBound{β}\AgdaSymbol{)}\AgdaSpace{}%
\AgdaOperator{\AgdaFunction{⁻¹}}\<%
\\
%
\\[\AgdaEmptyExtraSkip]%
%
\>[2]\AgdaFunction{lem2}\AgdaSpace{}%
\AgdaSymbol{:}\AgdaSpace{}%
\AgdaSymbol{\{}\AgdaBound{A}\AgdaSpace{}%
\AgdaSymbol{:}\AgdaSpace{}%
\AgdaPrimitive{Set}\AgdaSymbol{\}}\AgdaSpace{}%
\AgdaSymbol{→}\AgdaSpace{}%
\AgdaSymbol{(}\AgdaBound{a}\AgdaSpace{}%
\AgdaSymbol{:}\AgdaSpace{}%
\AgdaBound{A}\AgdaSymbol{)}\AgdaSpace{}%
\AgdaSymbol{→}\AgdaSpace{}%
\AgdaSymbol{(}\AgdaBound{α}\AgdaSpace{}%
\AgdaBound{β}\AgdaSpace{}%
\AgdaSymbol{:}\AgdaSpace{}%
\AgdaFunction{Ω²}\AgdaSpace{}%
\AgdaSymbol{\{}\AgdaBound{A}\AgdaSymbol{\}}\AgdaSpace{}%
\AgdaBound{a}\AgdaSymbol{)}\AgdaSpace{}%
\AgdaSymbol{→}\AgdaSpace{}%
\AgdaSymbol{(}\AgdaFunction{iᵣ}\AgdaSpace{}%
\AgdaInductiveConstructor{r}\AgdaSpace{}%
\AgdaOperator{\AgdaFunction{⁻¹}}\AgdaSpace{}%
\AgdaOperator{\AgdaFunction{∙}}\AgdaSpace{}%
\AgdaBound{α}\AgdaSpace{}%
\AgdaOperator{\AgdaFunction{∙}}\AgdaSpace{}%
\AgdaFunction{iᵣ}\AgdaSpace{}%
\AgdaInductiveConstructor{r}\AgdaSymbol{)}\AgdaSpace{}%
\AgdaOperator{\AgdaFunction{∙}}\AgdaSpace{}%
\AgdaSymbol{(}\AgdaFunction{iₗ}\AgdaSpace{}%
\AgdaInductiveConstructor{r}\AgdaSpace{}%
\AgdaOperator{\AgdaFunction{⁻¹}}\AgdaSpace{}%
\AgdaOperator{\AgdaFunction{∙}}\AgdaSpace{}%
\AgdaBound{β}\AgdaSpace{}%
\AgdaOperator{\AgdaFunction{∙}}\AgdaSpace{}%
\AgdaFunction{iₗ}\AgdaSpace{}%
\AgdaInductiveConstructor{r}\AgdaSymbol{)}\AgdaSpace{}%
\AgdaOperator{\AgdaDatatype{≡}}\AgdaSpace{}%
\AgdaSymbol{(}\AgdaBound{α}\AgdaSpace{}%
\AgdaOperator{\AgdaFunction{∙}}\AgdaSpace{}%
\AgdaBound{β}\AgdaSymbol{)}\<%
\\
%
\>[2]\AgdaFunction{lem2}\AgdaSpace{}%
\AgdaSymbol{\{}\AgdaBound{A}\AgdaSymbol{\}}\AgdaSpace{}%
\AgdaBound{a}\AgdaSpace{}%
\AgdaBound{α}\AgdaSpace{}%
\AgdaBound{β}\AgdaSpace{}%
\AgdaSymbol{=}\AgdaSpace{}%
\AgdaFunction{apf}\AgdaSpace{}%
\AgdaSymbol{(λ}\AgdaSpace{}%
\AgdaBound{-}\AgdaSpace{}%
\AgdaSymbol{→}\AgdaSpace{}%
\AgdaBound{-}\AgdaSpace{}%
\AgdaOperator{\AgdaFunction{∙}}\AgdaSpace{}%
\AgdaSymbol{(}\AgdaFunction{iₗ}\AgdaSpace{}%
\AgdaInductiveConstructor{r}\AgdaSpace{}%
\AgdaOperator{\AgdaFunction{⁻¹}}\AgdaSpace{}%
\AgdaOperator{\AgdaFunction{∙}}\AgdaSpace{}%
\AgdaBound{β}\AgdaSpace{}%
\AgdaOperator{\AgdaFunction{∙}}\AgdaSpace{}%
\AgdaFunction{iₗ}\AgdaSpace{}%
\AgdaInductiveConstructor{r}\AgdaSymbol{)}\AgdaSpace{}%
\AgdaSymbol{)}\AgdaSpace{}%
\AgdaSymbol{(}\AgdaFunction{lem20}\AgdaSpace{}%
\AgdaBound{α}\AgdaSymbol{)}\AgdaSpace{}%
\AgdaOperator{\AgdaFunction{∙}}\AgdaSpace{}%
\AgdaFunction{apf}\AgdaSpace{}%
\AgdaSymbol{(λ}\AgdaSpace{}%
\AgdaBound{-}\AgdaSpace{}%
\AgdaSymbol{→}\AgdaSpace{}%
\AgdaBound{α}\AgdaSpace{}%
\AgdaOperator{\AgdaFunction{∙}}\AgdaSpace{}%
\AgdaBound{-}\AgdaSymbol{)}\AgdaSpace{}%
\AgdaSymbol{(}\AgdaFunction{lem21}\AgdaSpace{}%
\AgdaBound{β}\AgdaSymbol{)}\<%
\\
%
\\[\AgdaEmptyExtraSkip]%
%
\>[2]\AgdaFunction{lem2'}\AgdaSpace{}%
\AgdaSymbol{:}\AgdaSpace{}%
\AgdaSymbol{\{}\AgdaBound{A}\AgdaSpace{}%
\AgdaSymbol{:}\AgdaSpace{}%
\AgdaPrimitive{Set}\AgdaSymbol{\}}\AgdaSpace{}%
\AgdaSymbol{→}\AgdaSpace{}%
\AgdaSymbol{(}\AgdaBound{a}\AgdaSpace{}%
\AgdaSymbol{:}\AgdaSpace{}%
\AgdaBound{A}\AgdaSymbol{)}\AgdaSpace{}%
\AgdaSymbol{→}\AgdaSpace{}%
\AgdaSymbol{(}\AgdaBound{α}\AgdaSpace{}%
\AgdaBound{β}\AgdaSpace{}%
\AgdaSymbol{:}\AgdaSpace{}%
\AgdaFunction{Ω²}\AgdaSpace{}%
\AgdaSymbol{\{}\AgdaBound{A}\AgdaSymbol{\}}\AgdaSpace{}%
\AgdaBound{a}\AgdaSymbol{)}\AgdaSpace{}%
\AgdaSymbol{→}\AgdaSpace{}%
\AgdaSymbol{(}\AgdaFunction{iₗ}\AgdaSpace{}%
\AgdaInductiveConstructor{r}\AgdaSpace{}%
\AgdaOperator{\AgdaFunction{⁻¹}}\AgdaSpace{}%
\AgdaOperator{\AgdaFunction{∙}}\AgdaSpace{}%
\AgdaBound{β}\AgdaSpace{}%
\AgdaOperator{\AgdaFunction{∙}}\AgdaSpace{}%
\AgdaFunction{iₗ}\AgdaSpace{}%
\AgdaInductiveConstructor{r}\AgdaSymbol{)}\AgdaSpace{}%
\AgdaOperator{\AgdaFunction{∙}}\AgdaSpace{}%
\AgdaSymbol{(}\AgdaFunction{iᵣ}\AgdaSpace{}%
\AgdaInductiveConstructor{r}\AgdaSpace{}%
\AgdaOperator{\AgdaFunction{⁻¹}}\AgdaSpace{}%
\AgdaOperator{\AgdaFunction{∙}}\AgdaSpace{}%
\AgdaBound{α}\AgdaSpace{}%
\AgdaOperator{\AgdaFunction{∙}}\AgdaSpace{}%
\AgdaFunction{iᵣ}\AgdaSpace{}%
\AgdaInductiveConstructor{r}\AgdaSymbol{)}\AgdaSpace{}%
\AgdaOperator{\AgdaDatatype{≡}}\AgdaSpace{}%
\AgdaSymbol{(}\AgdaBound{β}\AgdaSpace{}%
\AgdaOperator{\AgdaFunction{∙}}\AgdaSpace{}%
\AgdaBound{α}\AgdaSpace{}%
\AgdaSymbol{)}\<%
\\
%
\>[2]\AgdaFunction{lem2'}\AgdaSpace{}%
\AgdaSymbol{\{}\AgdaBound{A}\AgdaSymbol{\}}\AgdaSpace{}%
\AgdaBound{a}\AgdaSpace{}%
\AgdaBound{α}\AgdaSpace{}%
\AgdaBound{β}\AgdaSpace{}%
\AgdaSymbol{=}%
\>[21]\AgdaFunction{apf}%
\>[26]\AgdaSymbol{(λ}\AgdaSpace{}%
\AgdaBound{-}\AgdaSpace{}%
\AgdaSymbol{→}\AgdaSpace{}%
\AgdaBound{-}\AgdaSpace{}%
\AgdaOperator{\AgdaFunction{∙}}\AgdaSpace{}%
\AgdaSymbol{(}\AgdaFunction{iᵣ}\AgdaSpace{}%
\AgdaInductiveConstructor{r}\AgdaSpace{}%
\AgdaOperator{\AgdaFunction{⁻¹}}\AgdaSpace{}%
\AgdaOperator{\AgdaFunction{∙}}\AgdaSpace{}%
\AgdaBound{α}\AgdaSpace{}%
\AgdaOperator{\AgdaFunction{∙}}\AgdaSpace{}%
\AgdaFunction{iᵣ}\AgdaSpace{}%
\AgdaInductiveConstructor{r}\AgdaSymbol{))}\AgdaSpace{}%
\AgdaSymbol{(}\AgdaFunction{lem21}\AgdaSpace{}%
\AgdaBound{β}\AgdaSymbol{)}\AgdaSpace{}%
\AgdaOperator{\AgdaFunction{∙}}\AgdaSpace{}%
\AgdaFunction{apf}\AgdaSpace{}%
\AgdaSymbol{(λ}\AgdaSpace{}%
\AgdaBound{-}\AgdaSpace{}%
\AgdaSymbol{→}\AgdaSpace{}%
\AgdaBound{β}\AgdaSpace{}%
\AgdaOperator{\AgdaFunction{∙}}\AgdaSpace{}%
\AgdaBound{-}\AgdaSymbol{)}\AgdaSpace{}%
\AgdaSymbol{(}\AgdaFunction{lem20}\AgdaSpace{}%
\AgdaBound{α}\AgdaSymbol{)}\<%
\\
%
\>[2]\AgdaComment{-- apf (λ - → - ∙ (iₗ r ⁻¹ ∙ β ∙ iₗ r) ) (lem20 α) ∙ apf (λ - → α ∙ -) (lem21 β)}\<%
\\
%
\\[\AgdaEmptyExtraSkip]%
%
\>[2]\AgdaFunction{⋆≡∙}\AgdaSpace{}%
\AgdaSymbol{:}\AgdaSpace{}%
\AgdaSymbol{\{}\AgdaBound{A}\AgdaSpace{}%
\AgdaSymbol{:}\AgdaSpace{}%
\AgdaPrimitive{Set}\AgdaSymbol{\}}\AgdaSpace{}%
\AgdaSymbol{→}\AgdaSpace{}%
\AgdaSymbol{(}\AgdaBound{a}\AgdaSpace{}%
\AgdaSymbol{:}\AgdaSpace{}%
\AgdaBound{A}\AgdaSymbol{)}\AgdaSpace{}%
\AgdaSymbol{→}\AgdaSpace{}%
\AgdaSymbol{(}\AgdaBound{α}\AgdaSpace{}%
\AgdaBound{β}\AgdaSpace{}%
\AgdaSymbol{:}\AgdaSpace{}%
\AgdaFunction{Ω²}\AgdaSpace{}%
\AgdaSymbol{\{}\AgdaBound{A}\AgdaSymbol{\}}\AgdaSpace{}%
\AgdaBound{a}\AgdaSymbol{)}\AgdaSpace{}%
\AgdaSymbol{→}\AgdaSpace{}%
\AgdaSymbol{(}\AgdaBound{α}\AgdaSpace{}%
\AgdaOperator{\AgdaFunction{⋆}}\AgdaSpace{}%
\AgdaBound{β}\AgdaSymbol{)}\AgdaSpace{}%
\AgdaOperator{\AgdaDatatype{≡}}\AgdaSpace{}%
\AgdaSymbol{(}\AgdaBound{α}\AgdaSpace{}%
\AgdaOperator{\AgdaFunction{∙}}\AgdaSpace{}%
\AgdaBound{β}\AgdaSymbol{)}\<%
\\
%
\>[2]\AgdaFunction{⋆≡∙}\AgdaSpace{}%
\AgdaBound{a}\AgdaSpace{}%
\AgdaBound{α}\AgdaSpace{}%
\AgdaBound{β}\AgdaSpace{}%
\AgdaSymbol{=}\AgdaSpace{}%
\AgdaFunction{lem1}\AgdaSpace{}%
\AgdaBound{a}\AgdaSpace{}%
\AgdaBound{α}\AgdaSpace{}%
\AgdaBound{β}\AgdaSpace{}%
\AgdaOperator{\AgdaFunction{∙}}\AgdaSpace{}%
\AgdaFunction{lem2}\AgdaSpace{}%
\AgdaBound{a}\AgdaSpace{}%
\AgdaBound{α}\AgdaSpace{}%
\AgdaBound{β}\<%
\\
%
\\[\AgdaEmptyExtraSkip]%
%
\>[2]\AgdaComment{-- proven similairly to above }\<%
\\
%
\>[2]\AgdaFunction{⋆'≡∙}\AgdaSpace{}%
\AgdaSymbol{:}\AgdaSpace{}%
\AgdaSymbol{\{}\AgdaBound{A}\AgdaSpace{}%
\AgdaSymbol{:}\AgdaSpace{}%
\AgdaPrimitive{Set}\AgdaSymbol{\}}\AgdaSpace{}%
\AgdaSymbol{→}\AgdaSpace{}%
\AgdaSymbol{(}\AgdaBound{a}\AgdaSpace{}%
\AgdaSymbol{:}\AgdaSpace{}%
\AgdaBound{A}\AgdaSymbol{)}\AgdaSpace{}%
\AgdaSymbol{→}\AgdaSpace{}%
\AgdaSymbol{(}\AgdaBound{α}\AgdaSpace{}%
\AgdaBound{β}\AgdaSpace{}%
\AgdaSymbol{:}\AgdaSpace{}%
\AgdaFunction{Ω²}\AgdaSpace{}%
\AgdaSymbol{\{}\AgdaBound{A}\AgdaSymbol{\}}\AgdaSpace{}%
\AgdaBound{a}\AgdaSymbol{)}\AgdaSpace{}%
\AgdaSymbol{→}\AgdaSpace{}%
\AgdaSymbol{(}\AgdaBound{α}\AgdaSpace{}%
\AgdaOperator{\AgdaFunction{⋆'}}\AgdaSpace{}%
\AgdaBound{β}\AgdaSymbol{)}\AgdaSpace{}%
\AgdaOperator{\AgdaDatatype{≡}}\AgdaSpace{}%
\AgdaSymbol{(}\AgdaBound{β}\AgdaSpace{}%
\AgdaOperator{\AgdaFunction{∙}}\AgdaSpace{}%
\AgdaBound{α}\AgdaSymbol{)}\<%
\\
%
\>[2]\AgdaFunction{⋆'≡∙}\AgdaSpace{}%
\AgdaBound{a}\AgdaSpace{}%
\AgdaBound{α}\AgdaSpace{}%
\AgdaBound{β}\AgdaSpace{}%
\AgdaSymbol{=}\AgdaSpace{}%
\AgdaFunction{lem1'}\AgdaSpace{}%
\AgdaBound{a}\AgdaSpace{}%
\AgdaBound{α}\AgdaSpace{}%
\AgdaBound{β}\AgdaSpace{}%
\AgdaOperator{\AgdaFunction{∙}}\AgdaSpace{}%
\AgdaFunction{lem2'}\AgdaSpace{}%
\AgdaBound{a}\AgdaSpace{}%
\AgdaBound{α}\AgdaSpace{}%
\AgdaBound{β}\<%
\\
%
\\[\AgdaEmptyExtraSkip]%
%
\\[\AgdaEmptyExtraSkip]%
%
\>[2]\AgdaComment{--eckmannHilton : \{A : Set\} → (a : A) → (α β : Ω² \{A\} a) → α ∙ β ≡ β ∙ α }\<%
\\
%
\>[2]\AgdaComment{--eckmannHilton a r r = r}\<%
\\
\>[0]\<%
\end{code}

[TODO, fix without k errors]


% include most the code in the appendix

\section{Goals and Challenges}

The parser is still quite primitive, and needs to be extended extensively to
support natural language ambiguity in mathematics as well as other linguistic
nuance that GF captures well, like tense and aspect. This can follow a method
expored in Aarne's paper : "Translating between Language and Logic: What Is
Easy and What Is Difficult" where one develops a denotational semantics for
translating between natural language expressions with the desired AST. The bulk
of this work will be writing a Haskell back-end implementing this AST
transformation. The extended syntax, designed for linguistic nuance, will be
filtered into the core syntax, which is essentially what I have done.

The Resource Grammar Library (RGL) is designed for out-of-the box grammar
writing, and therefore much of the linearization nuance can be outsourced to
this robust and well-studied library. Nonetheless, each application grammar
brings its own unique challenges, and the RGL will only get one so far. My
linearization may require extensive tweaking.

Thus far, our parser is only able to parse non-cubical fragments of the
cubicalTT standard library. Dealing with Agda pattern matching, it was
realized, is outside the theoretical boundaries of GF (at least, if one were to
do it in a non ad-hoc way) due to its inability to pass arbitrary strings down
the syntax tree nodes during linearization. Pattern matching therefore needs to be dealt
with via pre and post processing.  Additionally, cubicaltt is weaker at
dealing with telescopes than Agda, and so a full generalization to Agda is not
yet possible. Universes are another feature future iterations of this Grammar
would need to deal with, but as they aren't present in most mathematician's
vernacular, it is not seen as relevant for the state of this project.

Records should also be added, but because this grammar supports sigma types,
there is no rush. The Identity type is so far deeply embedded in our grammar,
so the first code fragment may just be for explanatory purposes.  The degree to
which the library is extended to cover domain specific information is up to
debate, but for now the grammar is meant to be kept as minimal as possible.

One interesting extension, time dependnet, would be to allow for a bidrectional
feedback between GF and Agda : thereby allowing ad hoc extensions to GF's ASTs
to allow for newly defined Agda functions to be treated with more care, i.e.
have an arguement structure rather than just treating everything as variables.
This may be too ambitious for the time being.

% Random ideas

Category theory in agda paper, differences in formalization

* my agda hott library
* escardo's hott library 
  - if successful on mine, with universe support
  - mix of latex, agda code , and natural language 
* dummy example for non-hott math (spivak et al, type-in-type)
* alternatively, trying digging in the mountain at the other end, and try extedning ad-hoc grammar with various syntactic nuance
* Latex & Unicode support  - 
* Degenerate cases
  - find examples which are unable to be supported by this grammar, explain why and offer future possible patches

Talk about all the things that need to be done

Pattern Matching, additional parser vs internal to GF

How to decide an optimal phrase (this seems like itd be some rule based) from agda program

* Support for cs math - e.g. specifications of algorithms and their actual implementations
* Alternative syntaxes - graphical languages like grasshopper
* user interface
  - QA
  - Hoogle for proofs
* NL semantics (the semantic content is precisely the formal statements)
* Comparison / integration with ML approaches
* studies in concerete syntax -Harper psychology {\intersect} programming

Testing, with particular reference to the pgf grammar I developed





\newpage

\addcontentsline{toc}{section}{References}
\bibliographystyle{plain}
\bibliography{example_bibliography}

\newpage

\section{Appendix} \label{appendix}


\subsection{Martin-Löf Type Theory} \label{judge}

\subsubsection{Judgments}

\begin{displayquote}
With Kant, something important happened, namely, that the term judgement, Ger.
Urteil, came to be used instead of proposition. \emph{Per Martin-Löf} \cite{mlMeanings}.
\end{displayquote}

A central contribution of Per Martin-Löf in the development of type theory was
the recognition of the centrality of judgments in logic. Many mathematicians
aren't familiar with the spectrum of judgments available, and merely believe
they are concerned with \emph{the} notion of truth, namely \emph{the truth} of a
mathematical proposition or theorem. There are many judgments one can make which
most mathematicians aren't aware of or at least never mention. Examples of both familiar
and unfamiliar judgments include,

\begin{itemize}

\item $A$ is true
\item $A$ is a proposition
\item $A$ is possible
\item $A$ is necessarily true
\item $A$ is true at time $t$

\end{itemize}

These judgments are understood not in the object language in which we state our
propositions, possibilities, or probabilities, but as assertions in the
metalanguage which require evidence for us to know and believe them. Most
mathematicians may reach for their wallets if I come in and give a talk saying
it is possible that the Riemann Hypothesis is true, partially because they
already know that, and partially because it doesn't seem particularly
interesting to say that something is possible, in the same way that a physicist
may flinch if you say alchemy is possible. Most mathematicians, however, would
agree that $P = NP$ is a proposition, and it is also possible, but isn't true.

For the logician these judgments may well be interesting because their may be
logics in which the discussion of possibility or necessity is even more
interesting than the discussion of truth. And for the type theorist interested
in designing and building programming languages over many various logics, these
judgments become a prime focus. The role of the type-checker in a programming
language is to present evidence for, or decide the validity of the judgments.
The four main judgments of type theory are given in natural language on the left
and symbolically on the right :

\begin{multicols}{2}
\begin{itemize}
\item $T$ is a type
\item $T$ and $T'$ are equal types
\item $t$ is a term of type $T$
\item $t$ and $t'$ are equal terms of type $T$
\item $\vdash T \; {\rm type}$
\item $\vdash T = T'$
\item $\vdash t:T$
\item $\vdash t = t':T$
\end{itemize}
\end{multicols}

Frege's turnstile, $\vdash$, denotes a judgment. These judgments become much more interesting when we add the ability for them to
be interpreted in a some context with judgment hypotheses. Given a series of
judgments $J_1,...,J_n$, denoted $\Gamma$, where $J_i$ can depend on previously
listed $J's$, we can make judgment $J$ under the hypotheses, e.g. $J_1,...,J_n
\vdash J$. Often these hypotheses $J_i$, alternatively called \emph{antecedents},
denote variables which may occur freely in the *consequent* judgment $J$. For
instance, the antecedent, $x : \mathbb{R}$ occurs freely in the syntactic
expression $\sin x$, a which is given meaning in the judgment $\vdash \sin x { :
} \mathbb{R}$. We write our hypothetical judgement as follows :

$$x : \mathbb{R} \vdash \sin x : \mathbb{R}$$



\subsubsection{Rules}

Martin-Löf systematically used the four fundamental judgments in the proof
theoretic style of Pragwitz and Gentzen. To this end, the intuitionistic formulation of the
logical connectives just gives rules which admit an immediate computational
interpretation. The main types of rules are type formation, introduction,
elimination, and computation rules. The introduction rules for a type admit an
induction principle derivable from that type's signature. Additionally, the
$\beta$ and $\eta$ computation rules are derivable via the composition of
introduction and elimination rules, which, if correctly formulated, should
satisfy a relation known as harmony.

The fundamental notion of the lambda calculus, the function, is 
abstracted over a variable and returns a term of some type when applied to an
argument which is subsequently reduced via the computational rules.
Dependent Type Theory (DTT) generalizes this to allow the return type be
parameterized by the variable being abstracted over. The dependent function
forms the basis of the LF which underlies Agda and GF. Here is the formation
rule : 

\[
  \begin{prooftree}
    \hypo{̌\Gamma  \vdash A \; {\rm type}}
    \hypo{̌\Gamma, x : A \vdash B \; {\rm type}}
    \infer2[]{\Gamma \vdash \Pi x {:} A. B}
  \end{prooftree}
\]

One reason why hypothetical judgments are so interesting is we can devise rules
which allow us to translate from the metalanguage to the object language using
lambda expressions. These play the role of a function in mathematics and
implication in logic. More generally, this is a dependent type, representing the
$\forall$ quantifier. Assuming from now on $\Gamma \vdash A \; {\rm type}$ and
$\Gamma, x : A \vdash B \; {\rm type}$, we present here the introduction rule for
the most fundamental type in Agda, denoted \term{(x : A) -> B}.

\[
  \begin{prooftree}
    \hypo{̌\Gamma, x {:} A \vdash B \; {\rm type}}
    \infer2[]{\Gamma \vdash \lambda x. b {:} \Pi x {:} A. B}
  \end{prooftree}
\]

Observe that the hypothetical judgment with $x : A$ in the hypothesis has been
reduced to the same hypothesis set below the line, with the lambda term and Pi
type now accounting for the variable.

\[
  \begin{prooftree}
    \hypo{\Gamma \vdash f {:} \Pi x {:} A. B}
    \hypo{\Gamma \vdash a {:} A}
    \infer2[]{\Gamma \vdash f\, a {:} B[x := a]}
  \end{prooftree}
\]

We briefly give the elimination rule for
Pi, application, as well as the classic $\beta$ and $\eta$ computational equality
judgments (which are actually rules, but it is standard to forego the premises): 
$$\Gamma \vdash (\lambda x.b)\, a \equiv b[x := a] {:} B[x := a]$$
$$\Gamma \vdash (\lambda x.f)\, x \equiv f {:} \Pi x{:}A. B}$$
Using this rule, we now see a typical judgment without hypothesis in a real
analysis, $\vdash \lambda x. \sin x : \mathbb{R} \rightarrow \mathbb{R}$.  This is normally
expressed as follows (knowing full well that $\sin$ actually has to be
approximated when saying what the computable function in the codomain is): 
\begin{align*}
  \sin {:} \mathbb{R} &\rightarrow \mathbb{R}\\ x &\mapsto \sin ( x )
\end{align*}
Evaluating this function on 0, we see
\begin{align*}
(\lambda x. \sin x)\, 0 &\equiv \sin 0   \\ &\equiv 0
\end{align*}

While most mathematicians take this for granted, we hope this gives some insight
into how computer scientists present functions. We recommend reading
Martin-Löf's original papers \cite{ml1984} \cite{ml79} to see all the rules
elaborated in full detail, as well as his philosophical papers
\cite{mlMeanings} \cite{mlTruth} to understand type theory as it was conceived
both practically and philosophically.

% TODO : ADD stuff about substution, variable binding

\subsection{What is a Homomorphism?}

To get a feel for the syntactic paradigm we explore in this thesis, let us look at a basic mathematical
example: that of a group homomorphism as expressed in by a variety of somewhat
randomly sampled authors.  

% Wikipedia Defn:

\begin{definition}
In mathematics, given two groups, $(G, \ast)$ and $(H, \cdot)$, a group homomorphism from $(G, \ast)$ to $(H, \cdot)$ is a function $h : G \to H$ such that for all $u$ and $v$ in $G$ it holds that
  $$h(u \ast v) = h ( u ) \cdot h ( v )$$ 
\end{definition}

% http://math.mit.edu/~jwellens/Group%20Theory%20Forum.pdf

\begin{definition}
Let $G = (G,\cdot)$ and $G' = (G',\ast)$ be groups, and let $\phi : G \to G'$ be a map between them. We call $\phi$ a \textbf{homomorphism} if for every pair of elements $g, h \in G$, we have 
% \begin{center}
  $$\phi(g \ast h) = \phi ( g ) \cdot \phi ( h )$$ 
% \end{center}
\end{definition}

% http://www.maths.gla.ac.uk/~mwemyss/teaching/3alg1-7.pdf

\begin{definition}\label{def:def3}
Let $G$, $H$, be groups.  A map $\phi : G \to H$ is called a \emph{group homomorphism} if
  $$\phi(xy) = \phi ( x ) \phi ( y )\ for\ all\ x, y \in G$$ 
(Note that $xy$ on the left is formed using the group operation in $G$, whilst the product $\phi ( x ) \phi ( y )$ is formed using the group operation $H$.)
\end{definition}

% NLab:

\begin{definition}\label{def:def4}
Classically, a group is a monoid in which every element has an inverse (necessarily unique).
\end{definition}

We inquire the reader to pay attention to nuance and difference in presentation
that is normally ignored or taken for granted by the fluent mathematician, ask
which definitions feel better, and how the reader herself might present the
definition differently.

If one wants to distill the meaning of each of these presentations, there is a
significant amount of subliminal interpretation happening very much analogous to
our innate linguistic usage. The inverse and identity are discarded, even though
they are necessary data when defining a group. The order of presenting the
information is inconsistent, as well as the choice to use symbolic or natural
language information. In Definition~\ref{def:def3}, the group operation is used
implicitly, and its clarification a side remark. Details aside, these all mean
the same thing - don't they?


We now show yet another definition of a group homomorphism formalized in the
Agda programming language:

\begin{code}[hide]%
\>[0]\AgdaComment{--\{-\# OPTIONS --cubical \#-\}}\<%
\\
\>[0]\AgdaSymbol{\{-\#}\AgdaSpace{}%
\AgdaKeyword{OPTIONS}\AgdaSpace{}%
\AgdaPragma{--cubical}\AgdaSpace{}%
\AgdaPragma{--no-import-sorts}\AgdaSpace{}%
\AgdaPragma{--safe}\AgdaSpace{}%
\AgdaSymbol{\#-\}}\<%
\\
%
\\[\AgdaEmptyExtraSkip]%
\>[0]\AgdaKeyword{module}\AgdaSpace{}%
\AgdaModule{monoid}\AgdaSpace{}%
\AgdaKeyword{where}\<%
\\
%
\\[\AgdaEmptyExtraSkip]%
\>[0]\AgdaKeyword{module}\AgdaSpace{}%
\AgdaModule{Namespace1}\AgdaSpace{}%
\AgdaKeyword{where}\<%
\\
%
\\[\AgdaEmptyExtraSkip]%
\>[0][@{}l@{\AgdaIndent{0}}]%
\>[2]\AgdaKeyword{open}\AgdaSpace{}%
\AgdaKeyword{import}\AgdaSpace{}%
\AgdaModule{Cubical.Foundations.Prelude}\<%
\\
%
\>[2]\AgdaKeyword{open}\AgdaSpace{}%
\AgdaKeyword{import}\AgdaSpace{}%
\AgdaModule{Cubical.Foundations.Equiv}\<%
\\
%
\>[2]\AgdaKeyword{open}\AgdaSpace{}%
\AgdaKeyword{import}\AgdaSpace{}%
\AgdaModule{Cubical.Foundations.Structure}\<%
\\
%
\>[2]\AgdaKeyword{open}\AgdaSpace{}%
\AgdaKeyword{import}\AgdaSpace{}%
\AgdaModule{Cubical.Algebra.Group.Base}\<%
\\
%
\>[2]\AgdaKeyword{open}\AgdaSpace{}%
\AgdaKeyword{import}\AgdaSpace{}%
\AgdaModule{Cubical.Data.Sigma}\<%
\\
%
\\[\AgdaEmptyExtraSkip]%
%
\>[2]\AgdaKeyword{private}\<%
\\
\>[2][@{}l@{\AgdaIndent{0}}]%
\>[4]\AgdaKeyword{variable}\<%
\\
\>[4][@{}l@{\AgdaIndent{0}}]%
\>[6]\AgdaGeneralizable{ℓ}\AgdaSpace{}%
\AgdaGeneralizable{ℓ'}\AgdaSpace{}%
\AgdaGeneralizable{ℓ''}\AgdaSpace{}%
\AgdaGeneralizable{ℓ'''}\AgdaSpace{}%
\AgdaSymbol{:}\AgdaSpace{}%
\AgdaPostulate{Level}\<%
\end{code}
\begin{code}%
%
\>[2]\AgdaFunction{isGroupHom}\AgdaSpace{}%
\AgdaSymbol{:}\AgdaSpace{}%
\AgdaSymbol{(}\AgdaBound{G}\AgdaSpace{}%
\AgdaSymbol{:}\AgdaSpace{}%
\AgdaFunction{Group}\AgdaSpace{}%
\AgdaSymbol{\{}\AgdaGeneralizable{ℓ}\AgdaSymbol{\})}\AgdaSpace{}%
\AgdaSymbol{(}\AgdaBound{H}\AgdaSpace{}%
\AgdaSymbol{:}\AgdaSpace{}%
\AgdaFunction{Group}\AgdaSpace{}%
\AgdaSymbol{\{}\AgdaGeneralizable{ℓ'}\AgdaSymbol{\})}\AgdaSpace{}%
\AgdaSymbol{(}\AgdaBound{f}\AgdaSpace{}%
\AgdaSymbol{:}\AgdaSpace{}%
\AgdaOperator{\AgdaFunction{⟨}}\AgdaSpace{}%
\AgdaBound{G}\AgdaSpace{}%
\AgdaOperator{\AgdaFunction{⟩}}\AgdaSpace{}%
\AgdaSymbol{→}\AgdaSpace{}%
\AgdaOperator{\AgdaFunction{⟨}}\AgdaSpace{}%
\AgdaBound{H}\AgdaSpace{}%
\AgdaOperator{\AgdaFunction{⟩}}\AgdaSymbol{)}\AgdaSpace{}%
\AgdaSymbol{→}\AgdaSpace{}%
\AgdaPrimitive{Type}\AgdaSpace{}%
\AgdaSymbol{\AgdaUnderscore{}}\<%
\\
%
\>[2]\AgdaFunction{isGroupHom}\AgdaSpace{}%
\AgdaBound{G}\AgdaSpace{}%
\AgdaBound{H}\AgdaSpace{}%
\AgdaBound{f}\AgdaSpace{}%
\AgdaSymbol{=}\AgdaSpace{}%
\AgdaSymbol{(}\AgdaBound{x}\AgdaSpace{}%
\AgdaBound{y}\AgdaSpace{}%
\AgdaSymbol{:}\AgdaSpace{}%
\AgdaOperator{\AgdaFunction{⟨}}\AgdaSpace{}%
\AgdaBound{G}\AgdaSpace{}%
\AgdaOperator{\AgdaFunction{⟩}}\AgdaSymbol{)}\AgdaSpace{}%
\AgdaSymbol{→}\AgdaSpace{}%
\AgdaBound{f}\AgdaSpace{}%
\AgdaSymbol{(}\AgdaBound{x}\AgdaSpace{}%
\AgdaOperator{\AgdaFunction{G.+}}\AgdaSpace{}%
\AgdaBound{y}\AgdaSymbol{)}\AgdaSpace{}%
\AgdaOperator{\AgdaFunction{≡}}\AgdaSpace{}%
\AgdaSymbol{(}\AgdaBound{f}\AgdaSpace{}%
\AgdaBound{x}\AgdaSpace{}%
\AgdaOperator{\AgdaFunction{H.+}}\AgdaSpace{}%
\AgdaBound{f}\AgdaSpace{}%
\AgdaBound{y}\AgdaSymbol{)}\AgdaSpace{}%
\AgdaKeyword{where}\<%
\\
\>[2][@{}l@{\AgdaIndent{0}}]%
\>[4]\AgdaKeyword{module}\AgdaSpace{}%
\AgdaModule{G}\AgdaSpace{}%
\AgdaSymbol{=}\AgdaSpace{}%
\AgdaModule{GroupStr}\AgdaSpace{}%
\AgdaSymbol{(}\AgdaField{snd}\AgdaSpace{}%
\AgdaBound{G}\AgdaSymbol{)}\<%
\\
%
\>[4]\AgdaKeyword{module}\AgdaSpace{}%
\AgdaModule{H}\AgdaSpace{}%
\AgdaSymbol{=}\AgdaSpace{}%
\AgdaModule{GroupStr}\AgdaSpace{}%
\AgdaSymbol{(}\AgdaField{snd}\AgdaSpace{}%
\AgdaBound{H}\AgdaSymbol{)}\<%
\\
%
\\[\AgdaEmptyExtraSkip]%
%
\>[2]\AgdaKeyword{record}\AgdaSpace{}%
\AgdaRecord{GroupHom}\AgdaSpace{}%
\AgdaSymbol{(}\AgdaBound{G}\AgdaSpace{}%
\AgdaSymbol{:}\AgdaSpace{}%
\AgdaFunction{Group}\AgdaSpace{}%
\AgdaSymbol{\{}\AgdaGeneralizable{ℓ}\AgdaSymbol{\})}\AgdaSpace{}%
\AgdaSymbol{(}\AgdaBound{H}\AgdaSpace{}%
\AgdaSymbol{:}\AgdaSpace{}%
\AgdaFunction{Group}\AgdaSpace{}%
\AgdaSymbol{\{}\AgdaGeneralizable{ℓ'}\AgdaSymbol{\})}\AgdaSpace{}%
\AgdaSymbol{:}\AgdaSpace{}%
\AgdaPrimitive{Type}\AgdaSpace{}%
\AgdaSymbol{(}\AgdaPrimitive{ℓ-max}\AgdaSpace{}%
\AgdaBound{ℓ}\AgdaSpace{}%
\AgdaBound{ℓ'}\AgdaSymbol{)}\AgdaSpace{}%
\AgdaKeyword{where}\<%
\\
\>[2][@{}l@{\AgdaIndent{0}}]%
\>[4]\AgdaKeyword{constructor}\AgdaSpace{}%
\AgdaInductiveConstructor{grouphom}\<%
\\
%
\\[\AgdaEmptyExtraSkip]%
%
\>[4]\AgdaKeyword{field}\<%
\\
\>[4][@{}l@{\AgdaIndent{0}}]%
\>[6]\AgdaField{fun}\AgdaSpace{}%
\AgdaSymbol{:}\AgdaSpace{}%
\AgdaOperator{\AgdaFunction{⟨}}\AgdaSpace{}%
\AgdaBound{G}\AgdaSpace{}%
\AgdaOperator{\AgdaFunction{⟩}}\AgdaSpace{}%
\AgdaSymbol{→}\AgdaSpace{}%
\AgdaOperator{\AgdaFunction{⟨}}\AgdaSpace{}%
\AgdaBound{H}\AgdaSpace{}%
\AgdaOperator{\AgdaFunction{⟩}}\<%
\\
%
\>[6]\AgdaField{isHom}\AgdaSpace{}%
\AgdaSymbol{:}\AgdaSpace{}%
\AgdaFunction{isGroupHom}\AgdaSpace{}%
\AgdaBound{G}\AgdaSpace{}%
\AgdaBound{H}\AgdaSpace{}%
\AgdaField{fun}\<%
\end{code}
This actually \emph{was} the Cubical Agda implementation of a group homomorphism
sometime around the end of 2020. We see that, while a mathematician might be
able to infer the meaning of some of the syntax, the use of levels,
distinguising between isGroupHom and GroupHom itself, and many other details
might obscure what's going on.

We finally provide the current (May 2021) definition via Cubical Agda. One may
witness a significant number of differences from the previous version : concrete
syntax differences via changes in camel case, new uses of Group vs GroupStr, as
well as, most significantly, the identity and inverse preservation data not
appearing as corollaries, but part of the definition. Additionally, we had to
refactor the commented lines to those shown below to be compatible with our
outdated version of cubical. These changes reflect interesting syntactic
changes.

\begin{code}%
%
\>[2]\AgdaKeyword{record}\AgdaSpace{}%
\AgdaRecord{IsGroupHom}\AgdaSpace{}%
\AgdaSymbol{\{}\AgdaBound{A}\AgdaSpace{}%
\AgdaSymbol{:}\AgdaSpace{}%
\AgdaPrimitive{Type}\AgdaSpace{}%
\AgdaGeneralizable{ℓ}\AgdaSymbol{\}}\AgdaSpace{}%
\AgdaSymbol{\{}\AgdaBound{B}\AgdaSpace{}%
\AgdaSymbol{:}\AgdaSpace{}%
\AgdaPrimitive{Type}\AgdaSpace{}%
\AgdaGeneralizable{ℓ'}\AgdaSymbol{\}}\<%
\\
\>[2][@{}l@{\AgdaIndent{0}}]%
\>[4]\AgdaSymbol{(}\AgdaBound{M}\AgdaSpace{}%
\AgdaSymbol{:}\AgdaSpace{}%
\AgdaRecord{GroupStr}\AgdaSpace{}%
\AgdaBound{A}\AgdaSymbol{)}\AgdaSpace{}%
\AgdaSymbol{(}\AgdaBound{f}\AgdaSpace{}%
\AgdaSymbol{:}\AgdaSpace{}%
\AgdaBound{A}\AgdaSpace{}%
\AgdaSymbol{→}\AgdaSpace{}%
\AgdaBound{B}\AgdaSymbol{)}\AgdaSpace{}%
\AgdaSymbol{(}\AgdaBound{N}\AgdaSpace{}%
\AgdaSymbol{:}\AgdaSpace{}%
\AgdaRecord{GroupStr}\AgdaSpace{}%
\AgdaBound{B}\AgdaSymbol{)}\<%
\\
%
\>[4]\AgdaSymbol{:}\AgdaSpace{}%
\AgdaPrimitive{Type}\AgdaSpace{}%
\AgdaSymbol{(}\AgdaPrimitive{ℓ-max}\AgdaSpace{}%
\AgdaBound{ℓ}\AgdaSpace{}%
\AgdaBound{ℓ'}\AgdaSymbol{)}\<%
\\
%
\>[4]\AgdaKeyword{where}\<%
\\
%
\\[\AgdaEmptyExtraSkip]%
%
\>[4]\AgdaComment{-- Shorter qualified names}\<%
\\
%
\>[4]\AgdaKeyword{private}\<%
\\
\>[4][@{}l@{\AgdaIndent{0}}]%
\>[6]\AgdaKeyword{module}\AgdaSpace{}%
\AgdaModule{M}\AgdaSpace{}%
\AgdaSymbol{=}\AgdaSpace{}%
\AgdaModule{GroupStr}\AgdaSpace{}%
\AgdaBound{M}\<%
\\
%
\>[6]\AgdaKeyword{module}\AgdaSpace{}%
\AgdaModule{N}\AgdaSpace{}%
\AgdaSymbol{=}\AgdaSpace{}%
\AgdaModule{GroupStr}\AgdaSpace{}%
\AgdaBound{N}\<%
\\
%
\\[\AgdaEmptyExtraSkip]%
%
\>[4]\AgdaKeyword{field}\<%
\\
\>[4][@{}l@{\AgdaIndent{0}}]%
\>[6]\AgdaField{pres·}\AgdaSpace{}%
\AgdaSymbol{:}\AgdaSpace{}%
\AgdaSymbol{(}\AgdaBound{x}\AgdaSpace{}%
\AgdaBound{y}\AgdaSpace{}%
\AgdaSymbol{:}\AgdaSpace{}%
\AgdaBound{A}\AgdaSymbol{)}\AgdaSpace{}%
\AgdaSymbol{→}\AgdaSpace{}%
\AgdaBound{f}\AgdaSpace{}%
\AgdaSymbol{(}\AgdaOperator{\AgdaFunction{M.\AgdaUnderscore{}+\AgdaUnderscore{}}}\AgdaSpace{}%
\AgdaBound{x}\AgdaSpace{}%
\AgdaBound{y}\AgdaSymbol{)}\AgdaSpace{}%
\AgdaOperator{\AgdaFunction{≡}}\AgdaSpace{}%
\AgdaSymbol{(}\AgdaOperator{\AgdaFunction{N.\AgdaUnderscore{}+\AgdaUnderscore{}}}\AgdaSpace{}%
\AgdaSymbol{(}\AgdaBound{f}\AgdaSpace{}%
\AgdaBound{x}\AgdaSymbol{)}\AgdaSpace{}%
\AgdaSymbol{(}\AgdaBound{f}\AgdaSpace{}%
\AgdaBound{y}\AgdaSymbol{))}\<%
\\
%
\>[6]\AgdaField{pres1}\AgdaSpace{}%
\AgdaSymbol{:}\AgdaSpace{}%
\AgdaBound{f}\AgdaSpace{}%
\AgdaFunction{M.0g}\AgdaSpace{}%
\AgdaOperator{\AgdaFunction{≡}}\AgdaSpace{}%
\AgdaFunction{N.0g}\<%
\\
%
\>[6]\AgdaField{presinv}\AgdaSpace{}%
\AgdaSymbol{:}\AgdaSpace{}%
\AgdaSymbol{(}\AgdaBound{x}\AgdaSpace{}%
\AgdaSymbol{:}\AgdaSpace{}%
\AgdaBound{A}\AgdaSymbol{)}\AgdaSpace{}%
\AgdaSymbol{→}\AgdaSpace{}%
\AgdaBound{f}\AgdaSpace{}%
\AgdaSymbol{(}\AgdaOperator{\AgdaFunction{M.-\AgdaUnderscore{}}}\AgdaSpace{}%
\AgdaBound{x}\AgdaSymbol{)}\AgdaSpace{}%
\AgdaOperator{\AgdaFunction{≡}}\AgdaSpace{}%
\AgdaOperator{\AgdaFunction{N.-\AgdaUnderscore{}}}\AgdaSpace{}%
\AgdaSymbol{(}\AgdaBound{f}\AgdaSpace{}%
\AgdaBound{x}\AgdaSymbol{)}\<%
\\
%
\>[6]\AgdaComment{-- pres· : (x y : A) → f (x M.· y) ≡ f x N.· f y}\<%
\\
%
\>[6]\AgdaComment{-- pres1 : f M.1g ≡ N.1g}\<%
\\
%
\>[6]\AgdaComment{-- presinv : (x : A) → f (M.inv x) ≡ N.inv (f x)}\<%
\\
%
\\[\AgdaEmptyExtraSkip]%
%
\>[2]\AgdaFunction{GroupHom'}\AgdaSpace{}%
\AgdaSymbol{:}\AgdaSpace{}%
\AgdaSymbol{(}\AgdaBound{G}\AgdaSpace{}%
\AgdaSymbol{:}\AgdaSpace{}%
\AgdaFunction{Group}\AgdaSpace{}%
\AgdaSymbol{\{}\AgdaGeneralizable{ℓ}\AgdaSymbol{\})}\AgdaSpace{}%
\AgdaSymbol{(}\AgdaBound{H}\AgdaSpace{}%
\AgdaSymbol{:}\AgdaSpace{}%
\AgdaFunction{Group}\AgdaSpace{}%
\AgdaSymbol{\{}\AgdaGeneralizable{ℓ'}\AgdaSymbol{\})}\AgdaSpace{}%
\AgdaSymbol{→}\AgdaSpace{}%
\AgdaPrimitive{Type}\AgdaSpace{}%
\AgdaSymbol{(}\AgdaPrimitive{ℓ-max}\AgdaSpace{}%
\AgdaGeneralizable{ℓ}\AgdaSpace{}%
\AgdaGeneralizable{ℓ'}\AgdaSymbol{)}\<%
\\
%
\>[2]\AgdaComment{-- GroupHom' : (G : Group ℓ) (H : Group ℓ') → Type (ℓ-max ℓ ℓ')}\<%
\\
%
\>[2]\AgdaFunction{GroupHom'}\AgdaSpace{}%
\AgdaBound{G}\AgdaSpace{}%
\AgdaBound{H}\AgdaSpace{}%
\AgdaSymbol{=}\AgdaSpace{}%
\AgdaFunction{Σ[}\AgdaSpace{}%
\AgdaBound{f}\AgdaSpace{}%
\AgdaFunction{∈}\AgdaSpace{}%
\AgdaSymbol{(}\AgdaBound{G}\AgdaSpace{}%
\AgdaSymbol{.}\AgdaField{fst}\AgdaSpace{}%
\AgdaSymbol{→}\AgdaSpace{}%
\AgdaBound{H}\AgdaSpace{}%
\AgdaSymbol{.}\AgdaField{fst}\AgdaSymbol{)}\AgdaSpace{}%
\AgdaFunction{]}\AgdaSpace{}%
\AgdaRecord{IsGroupHom}\AgdaSpace{}%
\AgdaSymbol{(}\AgdaBound{G}\AgdaSpace{}%
\AgdaSymbol{.}\AgdaField{snd}\AgdaSymbol{)}\AgdaSpace{}%
\AgdaBound{f}\AgdaSpace{}%
\AgdaSymbol{(}\AgdaBound{H}\AgdaSpace{}%
\AgdaSymbol{.}\AgdaField{snd}\AgdaSymbol{)}\<%
\end{code}


While the last two definitions be somewhat compressible to a programmer
or mathematician not exposed to Agda, it is certainly comprehensible to a
computer : that is, the colors indicate it type-checks on a computer where Cubical Agda is installed.
While GF is designed for multilingual syntactic transformations and is targeted
for natural language translation, its underlying theory is largely based on
ideas from the compiler communities. A cousin of the BNF Converter (BNFC), GF is
fully capable of parsing programming languages like Agda! While the Agda
definitions are present another concrete presentation of a group
homomorphism, they are distinct in that they have inherent semantic content.

While this example may not exemplify the power of Agda's type-checker, it is of
considerable interest to many. The type-checker has merely assured us that
\term{GroupHom(')} are well-formed types - not that we have a canonical representation
of a group homomorphism.
% Aarne says cut

We note that the natural
language definitions of monoid differ in form, but also in pragmatic content.
How one expresses formalities in natural language is incredibly diverse, and
Definition~\ref{def:def4} as compared with the prior homomorphism definitions is
particularly poignant in demonstrating this. These differ very much in nature to
the Agda definitions - especially pragmatically.
The differences between the Cubical
Agda definitions may be loosely called pragmatic, in the sense that the choice
of definitions may have downstream effects on readability, maintainability, modularity, and other
considerations when trying to write good code, in a burgeoning area known as proof engineering.

% TODO section name


\subsection{Twin Primes Conjecture Revisited} \label{twin}
\begin{code}[hide]%
\>[0]\AgdaKeyword{module}\AgdaSpace{}%
\AgdaModule{twinsigma}\AgdaSpace{}%
\AgdaKeyword{where}\<%
\\
%
\\[\AgdaEmptyExtraSkip]%
\>[0]\AgdaKeyword{open}\AgdaSpace{}%
\AgdaKeyword{import}\AgdaSpace{}%
\AgdaModule{Data.Nat}\AgdaSpace{}%
\AgdaKeyword{renaming}\AgdaSpace{}%
\AgdaSymbol{(}\AgdaOperator{\AgdaPrimitive{\AgdaUnderscore{}+\AgdaUnderscore{}}}\AgdaSpace{}%
\AgdaSymbol{to}\AgdaSpace{}%
\AgdaOperator{\AgdaPrimitive{\AgdaUnderscore{}∔\AgdaUnderscore{}}}\AgdaSymbol{)}\<%
\\
\>[0]\AgdaKeyword{open}\AgdaSpace{}%
\AgdaKeyword{import}\AgdaSpace{}%
\AgdaModule{Data.Product}\AgdaSpace{}%
\AgdaKeyword{using}\AgdaSpace{}%
\AgdaSymbol{(}\AgdaRecord{Σ}\AgdaSymbol{;}\AgdaSpace{}%
\AgdaOperator{\AgdaFunction{\AgdaUnderscore{}×\AgdaUnderscore{}}}\AgdaSymbol{;}\AgdaSpace{}%
\AgdaOperator{\AgdaInductiveConstructor{\AgdaUnderscore{},\AgdaUnderscore{}}}\AgdaSymbol{;}\AgdaSpace{}%
\AgdaField{proj₁}\AgdaSymbol{;}\AgdaSpace{}%
\AgdaField{proj₂}\AgdaSymbol{;}\AgdaSpace{}%
\AgdaFunction{∃}\AgdaSymbol{;}\AgdaSpace{}%
\AgdaFunction{Σ-syntax}\AgdaSymbol{;}\AgdaSpace{}%
\AgdaFunction{∃-syntax}\AgdaSymbol{)}\<%
\\
\>[0]\AgdaKeyword{open}\AgdaSpace{}%
\AgdaKeyword{import}\AgdaSpace{}%
\AgdaModule{Data.Sum}\AgdaSpace{}%
\AgdaKeyword{renaming}\AgdaSpace{}%
\AgdaSymbol{(}\AgdaOperator{\AgdaDatatype{\AgdaUnderscore{}⊎\AgdaUnderscore{}}}\AgdaSpace{}%
\AgdaSymbol{to}\AgdaSpace{}%
\AgdaOperator{\AgdaDatatype{\AgdaUnderscore{}+\AgdaUnderscore{}}}\AgdaSymbol{)}\<%
\\
\>[0]\AgdaKeyword{import}\AgdaSpace{}%
\AgdaModule{Relation.Binary.PropositionalEquality}\AgdaSpace{}%
\AgdaSymbol{as}\AgdaSpace{}%
\AgdaModule{Eq}\<%
\\
\>[0]\AgdaKeyword{open}\AgdaSpace{}%
\AgdaModule{Eq}\AgdaSpace{}%
\AgdaKeyword{using}\AgdaSpace{}%
\AgdaSymbol{(}\AgdaOperator{\AgdaDatatype{\AgdaUnderscore{}≡\AgdaUnderscore{}}}\AgdaSymbol{;}\AgdaSpace{}%
\AgdaInductiveConstructor{refl}\AgdaSymbol{;}\AgdaSpace{}%
\AgdaFunction{trans}\AgdaSymbol{;}\AgdaSpace{}%
\AgdaFunction{sym}\AgdaSymbol{;}\AgdaSpace{}%
\AgdaFunction{cong}\AgdaSymbol{;}\AgdaSpace{}%
\AgdaFunction{cong-app}\AgdaSymbol{;}\AgdaSpace{}%
\AgdaFunction{subst}\AgdaSymbol{)}\<%
\\
\>[0]\AgdaKeyword{open}\AgdaSpace{}%
\AgdaModule{Eq.≡-Reasoning}\AgdaSpace{}%
\AgdaKeyword{using}\AgdaSpace{}%
\AgdaSymbol{(}\AgdaOperator{\AgdaFunction{begin\AgdaUnderscore{}}}\AgdaSymbol{;}\AgdaSpace{}%
\AgdaOperator{\AgdaFunction{\AgdaUnderscore{}≡⟨⟩\AgdaUnderscore{}}}\AgdaSymbol{;}\AgdaSpace{}%
\AgdaFunction{step-≡}\AgdaSymbol{;}\AgdaSpace{}%
\AgdaOperator{\AgdaFunction{\AgdaUnderscore{}∎}}\AgdaSymbol{)}\<%
\\
%
\\[\AgdaEmptyExtraSkip]%
\>[0]\AgdaOperator{\AgdaFunction{\AgdaUnderscore{}-\AgdaUnderscore{}}}\AgdaSpace{}%
\AgdaSymbol{:}\AgdaSpace{}%
\AgdaDatatype{ℕ}\AgdaSpace{}%
\AgdaSymbol{→}\AgdaSpace{}%
\AgdaDatatype{ℕ}\AgdaSpace{}%
\AgdaSymbol{→}\AgdaSpace{}%
\AgdaDatatype{ℕ}\<%
\\
\>[0]\AgdaBound{n}%
\>[6]\AgdaOperator{\AgdaFunction{-}}\AgdaSpace{}%
\AgdaInductiveConstructor{zero}\AgdaSpace{}%
\AgdaSymbol{=}\AgdaSpace{}%
\AgdaBound{n}\<%
\\
\>[0]\AgdaInductiveConstructor{zero}%
\>[6]\AgdaOperator{\AgdaFunction{-}}\AgdaSpace{}%
\AgdaInductiveConstructor{suc}\AgdaSpace{}%
\AgdaBound{m}\AgdaSpace{}%
\AgdaSymbol{=}\AgdaSpace{}%
\AgdaInductiveConstructor{zero}\<%
\\
\>[0]\AgdaInductiveConstructor{suc}\AgdaSpace{}%
\AgdaBound{n}\AgdaSpace{}%
\AgdaOperator{\AgdaFunction{-}}\AgdaSpace{}%
\AgdaInductiveConstructor{suc}\AgdaSpace{}%
\AgdaBound{m}\AgdaSpace{}%
\AgdaSymbol{=}\AgdaSpace{}%
\AgdaBound{n}\AgdaSpace{}%
\AgdaOperator{\AgdaFunction{-}}\AgdaSpace{}%
\AgdaBound{m}\<%
\\
%
\\[\AgdaEmptyExtraSkip]%
\>[0]\AgdaFunction{isPrime}\AgdaSpace{}%
\AgdaSymbol{:}\AgdaSpace{}%
\AgdaDatatype{ℕ}\AgdaSpace{}%
\AgdaSymbol{→}\AgdaSpace{}%
\AgdaPrimitive{Set}\<%
\\
\>[0]\AgdaFunction{isPrime}\AgdaSpace{}%
\AgdaBound{n}\AgdaSpace{}%
\AgdaSymbol{=}\<%
\\
\>[0][@{}l@{\AgdaIndent{0}}]%
\>[2]\AgdaSymbol{(}\AgdaBound{n}\AgdaSpace{}%
\AgdaOperator{\AgdaFunction{≥}}\AgdaSpace{}%
\AgdaNumber{2}\AgdaSymbol{)}\AgdaSpace{}%
\AgdaOperator{\AgdaFunction{×}}\<%
\\
%
\>[2]\AgdaSymbol{((}\AgdaBound{x}\AgdaSpace{}%
\AgdaBound{y}\AgdaSpace{}%
\AgdaSymbol{:}\AgdaSpace{}%
\AgdaDatatype{ℕ}\AgdaSymbol{)}\AgdaSpace{}%
\AgdaSymbol{→}\AgdaSpace{}%
\AgdaBound{x}\AgdaSpace{}%
\AgdaOperator{\AgdaPrimitive{*}}\AgdaSpace{}%
\AgdaBound{y}\AgdaSpace{}%
\AgdaOperator{\AgdaDatatype{≡}}\AgdaSpace{}%
\AgdaBound{n}\AgdaSpace{}%
\AgdaSymbol{→}\AgdaSpace{}%
\AgdaSymbol{(}\AgdaBound{x}\AgdaSpace{}%
\AgdaOperator{\AgdaDatatype{≡}}\AgdaSpace{}%
\AgdaNumber{1}\AgdaSymbol{)}\AgdaSpace{}%
\AgdaOperator{\AgdaDatatype{+}}\AgdaSpace{}%
\AgdaSymbol{(}\AgdaBound{x}\AgdaSpace{}%
\AgdaOperator{\AgdaDatatype{≡}}\AgdaSpace{}%
\AgdaBound{n}\AgdaSymbol{))}\<%
\end{code}
We now give the dependent uncurring from the functions from \ref{twinprime} We
note that this perhaps is a bit more linguistically natural, because we can
refer to definitions of a prime number, sucessive prime numbers, etc. We leave
it to the reader to decide which presentation would be better suited for
translation.
\begin{code}%
\>[0]\AgdaFunction{prime}\AgdaSpace{}%
\AgdaSymbol{=}\AgdaSpace{}%
\AgdaFunction{Σ[}\AgdaSpace{}%
\AgdaBound{p}\AgdaSpace{}%
\AgdaFunction{∈}\AgdaSpace{}%
\AgdaDatatype{ℕ}\AgdaSpace{}%
\AgdaFunction{]}\AgdaSpace{}%
\AgdaFunction{isPrime}\AgdaSpace{}%
\AgdaBound{p}\<%
\\
%
\\[\AgdaEmptyExtraSkip]%
\>[0]\AgdaFunction{isSuccessivePrime}\AgdaSpace{}%
\AgdaSymbol{:}\AgdaSpace{}%
\AgdaFunction{prime}\AgdaSpace{}%
\AgdaSymbol{→}\AgdaSpace{}%
\AgdaFunction{prime}\AgdaSpace{}%
\AgdaSymbol{→}\AgdaSpace{}%
\AgdaPrimitive{Set}\<%
\\
\>[0]\AgdaFunction{isSuccessivePrime}\AgdaSpace{}%
\AgdaSymbol{(}\AgdaBound{p}\AgdaSpace{}%
\AgdaOperator{\AgdaInductiveConstructor{,}}\AgdaSpace{}%
\AgdaBound{pIsPrime}\AgdaSymbol{)}\AgdaSpace{}%
\AgdaSymbol{(}\AgdaBound{p'}\AgdaSpace{}%
\AgdaOperator{\AgdaInductiveConstructor{,}}\AgdaSpace{}%
\AgdaBound{p'IsPrime}\AgdaSymbol{)}\AgdaSpace{}%
\AgdaSymbol{=}\<%
\\
\>[0][@{}l@{\AgdaIndent{0}}]%
\>[2]\AgdaSymbol{(}\AgdaBound{(}\AgdaBound{p''}\AgdaSpace{}%
\AgdaOperator{\AgdaInductiveConstructor{,}}\AgdaSpace{}%
\AgdaBound{p''IsPrime}\AgdaBound{)}\AgdaSpace{}%
\AgdaSymbol{:}\AgdaSpace{}%
\AgdaFunction{prime}\AgdaSymbol{)}\AgdaSpace{}%
\AgdaSymbol{→}\<%
\\
%
\>[2]\AgdaBound{p}\AgdaSpace{}%
\AgdaOperator{\AgdaDatatype{≤}}\AgdaSpace{}%
\AgdaBound{p'}\AgdaSpace{}%
\AgdaSymbol{→}\AgdaSpace{}%
\AgdaBound{p}\AgdaSpace{}%
\AgdaOperator{\AgdaDatatype{≤}}\AgdaSpace{}%
\AgdaBound{p''}\AgdaSpace{}%
\AgdaSymbol{→}\AgdaSpace{}%
\AgdaBound{p'}\AgdaSpace{}%
\AgdaOperator{\AgdaDatatype{≤}}\AgdaSpace{}%
\AgdaBound{p''}\<%
\\
%
\\[\AgdaEmptyExtraSkip]%
\>[0]\AgdaFunction{successivePrimes}\AgdaSpace{}%
\AgdaSymbol{=}\<%
\\
\>[0][@{}l@{\AgdaIndent{0}}]%
\>[2]\AgdaFunction{Σ[}\AgdaSpace{}%
\AgdaBound{p}\AgdaSpace{}%
\AgdaFunction{∈}\AgdaSpace{}%
\AgdaFunction{prime}\AgdaSpace{}%
\AgdaFunction{]}\AgdaSpace{}%
\AgdaFunction{Σ[}\AgdaSpace{}%
\AgdaBound{p'}\AgdaSpace{}%
\AgdaFunction{∈}\AgdaSpace{}%
\AgdaFunction{prime}\AgdaSpace{}%
\AgdaFunction{]}\AgdaSpace{}%
\AgdaFunction{isSuccessivePrime}\AgdaSpace{}%
\AgdaBound{p}\AgdaSpace{}%
\AgdaBound{p'}\<%
\\
%
\\[\AgdaEmptyExtraSkip]%
\>[0]\AgdaFunction{primeGap}\AgdaSpace{}%
\AgdaSymbol{:}\AgdaSpace{}%
\AgdaFunction{successivePrimes}\AgdaSpace{}%
\AgdaSymbol{→}\AgdaSpace{}%
\AgdaDatatype{ℕ}\<%
\\
\>[0]\AgdaFunction{primeGap}\AgdaSpace{}%
\AgdaSymbol{((}\AgdaBound{p}\AgdaSpace{}%
\AgdaOperator{\AgdaInductiveConstructor{,}}\AgdaSpace{}%
\AgdaBound{pIsPrime}\AgdaSymbol{)}\AgdaSpace{}%
\AgdaOperator{\AgdaInductiveConstructor{,}}\AgdaSpace{}%
\AgdaSymbol{(}\AgdaBound{p'}\AgdaSpace{}%
\AgdaOperator{\AgdaInductiveConstructor{,}}\AgdaSpace{}%
\AgdaBound{p'IsPrime}\AgdaSymbol{)}\AgdaSpace{}%
\AgdaOperator{\AgdaInductiveConstructor{,}}\AgdaSpace{}%
\AgdaBound{p'-is-after-px}\AgdaSymbol{)}\AgdaSpace{}%
\AgdaSymbol{=}\AgdaSpace{}%
\AgdaBound{p}\AgdaSpace{}%
\AgdaOperator{\AgdaFunction{-}}\AgdaSpace{}%
\AgdaBound{p'}\<%
\\
%
\\[\AgdaEmptyExtraSkip]%
\>[0]\AgdaFunction{nth-pletPrimes}\AgdaSpace{}%
\AgdaSymbol{:}\AgdaSpace{}%
\AgdaFunction{successivePrimes}\AgdaSpace{}%
\AgdaSymbol{→}\AgdaSpace{}%
\AgdaDatatype{ℕ}\AgdaSpace{}%
\AgdaSymbol{→}\AgdaSpace{}%
\AgdaPrimitive{Set}\<%
\\
\>[0]\AgdaFunction{nth-pletPrimes}\AgdaSpace{}%
\AgdaSymbol{(}\AgdaBound{p}\AgdaSpace{}%
\AgdaOperator{\AgdaInductiveConstructor{,}}\AgdaSpace{}%
\AgdaBound{p'}\AgdaSpace{}%
\AgdaOperator{\AgdaInductiveConstructor{,}}\AgdaSpace{}%
\AgdaBound{p'-is-after-p}\AgdaSymbol{)}\AgdaSpace{}%
\AgdaBound{n}\AgdaSpace{}%
\AgdaSymbol{=}\<%
\\
\>[0][@{}l@{\AgdaIndent{0}}]%
\>[2]\AgdaFunction{primeGap}\AgdaSpace{}%
\AgdaSymbol{(}\AgdaBound{p}\AgdaSpace{}%
\AgdaOperator{\AgdaInductiveConstructor{,}}\AgdaSpace{}%
\AgdaBound{p'}\AgdaSpace{}%
\AgdaOperator{\AgdaInductiveConstructor{,}}\AgdaSpace{}%
\AgdaBound{p'-is-after-p}\AgdaSymbol{)}\AgdaSpace{}%
\AgdaOperator{\AgdaDatatype{≡}}\AgdaSpace{}%
\AgdaBound{n}\<%
\\
%
\\[\AgdaEmptyExtraSkip]%
\>[0]\AgdaFunction{twinPrimes}\AgdaSpace{}%
\AgdaSymbol{:}\AgdaSpace{}%
\AgdaFunction{successivePrimes}\AgdaSpace{}%
\AgdaSymbol{→}%
\>[33]\AgdaPrimitive{Set}\<%
\\
\>[0]\AgdaFunction{twinPrimes}\AgdaSpace{}%
\AgdaBound{sucPrimes}\AgdaSpace{}%
\AgdaSymbol{=}\AgdaSpace{}%
\AgdaFunction{nth-pletPrimes}\AgdaSpace{}%
\AgdaBound{sucPrimes}\AgdaSpace{}%
\AgdaNumber{2}\<%
\\
%
\\[\AgdaEmptyExtraSkip]%
\>[0]\AgdaFunction{twinPrimeConjecture}\AgdaSpace{}%
\AgdaSymbol{:}\AgdaSpace{}%
\AgdaPrimitive{Set}\<%
\\
\>[0]\AgdaFunction{twinPrimeConjecture}\AgdaSpace{}%
\AgdaSymbol{=}\AgdaSpace{}%
\AgdaSymbol{(}\AgdaBound{n}\AgdaSpace{}%
\AgdaSymbol{:}\AgdaSpace{}%
\AgdaDatatype{ℕ}\AgdaSymbol{)}\AgdaSpace{}%
\AgdaSymbol{→}\<%
\\
\>[0][@{}l@{\AgdaIndent{0}}]%
\>[2]\AgdaFunction{Σ[}\AgdaSpace{}%
\AgdaBound{sprs}\AgdaSymbol{@((}\AgdaBound{p}\AgdaSpace{}%
\AgdaOperator{\AgdaInductiveConstructor{,}}\AgdaSpace{}%
\AgdaBound{p'}\AgdaSymbol{)}\AgdaOperator{\AgdaInductiveConstructor{,}}\AgdaSpace{}%
\AgdaBound{p'-after-p}\AgdaSymbol{)}\AgdaSpace{}%
\AgdaFunction{∈}\AgdaSpace{}%
\AgdaFunction{successivePrimes}\AgdaSpace{}%
\AgdaFunction{]}\<%
\\
\>[2][@{}l@{\AgdaIndent{0}}]%
\>[4]\AgdaSymbol{(}\AgdaBound{p}\AgdaSpace{}%
\AgdaOperator{\AgdaFunction{≥}}\AgdaSpace{}%
\AgdaBound{n}\AgdaSymbol{)}\<%
\\
%
\>[2]\AgdaOperator{\AgdaFunction{×}}\AgdaSpace{}%
\AgdaFunction{twinPrimes}\AgdaSpace{}%
\AgdaBound{sprs}\<%
\end{code}


\subsection{cubicalTT} \label{cubicaltt}
\begin{verbatim}
abstract Exp = {

flags startcat = Decl ;
      -- note, cubical tt doesn't support inductive families, and therefore the datatype (& labels) need to be modified

cat
  Comment ;
  Module  ;
  AIdent ;
  Imp ; --imports, add later
  Decl ;
  Exp ;
  ExpWhere ;
  Tele ;
  Branch ;
  PTele ;
  Label ;
  [AIdent]{0} ; -- "x y z"
  [Decl]{1} ;
  [Tele]{0} ;
  [Branch]{1} ;
  [Label]{1} ;
  [PTele]{1} ;
  -- [Exp]{1};

fun

  DeclDef : AIdent -> [Tele] -> Exp -> ExpWhere -> Decl ;
  -- foo ( b : bool ) : bool = b
  DeclData : AIdent -> [Tele] -> [Label] -> Decl ;
  -- data nat : Set where zero | suc ( n : nat )
  DeclSplit : AIdent -> [Tele] -> Exp -> [Branch] -> Decl ;
  -- caseBool ( x : Set ) ( y z : x ) : bool -> Set = split false -> y || true -> z
  DeclUndef : AIdent -> [Tele] -> Exp -> Decl ;
  -- funExt ( a : Set ) ( b : a -> Set ) ( f g : ( x : a ) -> b x ) ( p : ( x : a ) -> ( b x ) ( f x ) == ( g x ) ) : ( ( y : a ) -> b y ) f == g = undefined

  Where : Exp -> [Decl] -> ExpWhere ;
  -- foo ( b : bool ) : bool =
  -- f b where f : bool -> bool = negb
  NoWhere : Exp -> ExpWhere ;
  -- foo ( b : bool ) : bool =
  -- b

  Split : Exp -> [Branch] -> Exp ;
  --split@ ( nat -> bool ) with zero  -> true || suc n -> false
  Let : [Decl] -> Exp -> Exp ;
  -- foo ( b : bool ) : bool =
  -- let f : bool -> bool = negb in f b
  Lam : [PTele] -> Exp -> Exp ;
  -- \\ ( x : bool ) -> negb x
  -- todo, allow implicit typing
  Fun : Exp -> Exp -> Exp ;
  -- Set -> Set
  -- Set -> ( b : bool ) -> ( x : Set ) -> ( f x )
  Pi    : [PTele] -> Exp -> Exp ;
  --( f : bool -> Set ) -> ( b : bool ) -> ( x : Set ) -> ( f x )
  -- ( f : bool -> Set ) ( b : bool ) ( x : Set ) -> ( f x )
  Sigma : [PTele] -> Exp -> Exp ;
  -- ( f : bool -> Set ) ( b : bool ) ( x : Set ) * ( f x )
  App : Exp -> Exp -> Exp ;
  -- proj1 ( x , y )
  Id : Exp -> Exp -> Exp -> Exp ;
  -- Set bool == nat
  IdJ : Exp -> Exp -> Exp -> Exp -> Exp -> Exp ;
  -- J e d x y p
  Fst : Exp -> Exp ; -- "proj1 x"
  Snd : Exp -> Exp ; -- "proj2 x"
  -- Pair : Exp -> [Exp] -> Exp ;
  Pair : Exp -> Exp -> Exp ;
  -- ( x , y )
  Var : AIdent -> Exp ;
  -- x
  Univ : Exp ;
  -- Set
  Refl : AIdent ; -- Exp ;
  -- refl
  --Hole : HoleIdent -> Exp ; -- need to add holes

  OBranch :  AIdent -> [AIdent] -> ExpWhere -> Branch ;
  -- suc m -> split@ ( nat -> bool ) with zero -> false || suc n -> equalNat m n
  -- for splits

  OLabel : AIdent -> [Tele] -> Label ;
  -- suc ( n : nat )
  -- fora data types

  -- construct telescope
  TeleC : AIdent -> [AIdent] -> Exp -> Tele ;
  -- "( f g : ( x : a ) -> b x )"
  -- ( a : Set ) ( b : ( a ) -> ( Set ) ) ( f g : ( x : a ) -> ( ( b ) ( x ) ) ) ( p : ( x : a ) -> ( ( ( b ) ( x ) ) ( ( f ) ( x ) ) == ( ( g ) ( x ) ) ) )

  -- why does gr with this fail so epically?
  PTeleC : Exp -> Exp -> PTele ; 
  -- ( x : Set ) -- ( y : x -> Set )" -- ( x : f y z )"

  --everything below this is just strings

  Foo : AIdent ;
  A , B , C , D , E , F , G , H , I , J , K , L , M , N , O , P , Q , R , S , T , U , V , W , X , Y , Z : AIdent ;
  True , False , Bool : AIdent ;
  NegB : AIdent ;
  CaseBool : AIdent ;
  IndBool : AIdent ;
  FunExt : AIdent ;
  Nat : AIdent ;
  Zero : AIdent ;
  Suc : AIdent ;
  EqualNat : AIdent ;
  Unit : AIdent ;
  Top : AIdent ;
  Contr : AIdent ;
  Fiber : AIdent ;
  IsEquiv : AIdent ;
  IdIsEquiv : AIdent ;
  IdFun : AIdent ;
  ContrSingl : AIdent ;
  Equiv : AIdent ;
  EqToIso : AIdent ;
  Ybar : AIdent ;
  IdFib : AIdent ;
  Identity : AIdent ;
  Lemma0 : AIdent ;
}
\end{verbatim}
\begin{verbatim}
concrete ExpCubicalTT of Exp = open Prelude, FormalTwo in {

lincat 
  Comment,
  Module ,
  AIdent,
  Imp,
  Decl ,
  ExpWhere,
  Tele,
  Branch ,
  PTele,
  Label,
    -- = Str ;
  [AIdent],
  [Decl] ,
  -- [Exp],
  [Tele],
  [Branch] ,
  [PTele],
  [Label]
    -- = {hd,tl : Str} ;
    = Str ;
  Exp = TermPrec ;

lin

  DeclDef a lt e ew = a ++ lt ++ ":" ++ usePrec 0 e ++ "=" ++ ew ;
  DeclData a t d = "data" ++ a ++ t ++ ": Set where" ++ d ;
  DeclSplit ai lt e lb = ai ++ lt ++ ":" ++ usePrec 0 e ++ "= split" ++ lb ;
  DeclUndef a lt e = a ++ lt ++ ":" ++ usePrec 0 e ++ "= undefined" ; -- postulate in agda

  Where e ld = usePrec 0 e ++ "where" ++ ld ;
  NoWhere e = usePrec 0 e ;

  Let ld e = mkPrec 0 ("let" ++ ld ++ "in" ++ (usePrec 0 e)) ;
  Split e lb = mkPrec 0 ("split@" ++ usePrec 0 e ++ "with" ++ lb) ;
  Lam pt e = mkPrec 0 ("\\" ++ pt ++ "->" ++ usePrec 0 e) ;
  Fun = infixr 1 "->" ; -- A -> Set
  Pi pt e = mkPrec 1 (pt ++ "->" ++ usePrec 1 e) ;
  Sigma pt e = mkPrec 1 (pt ++ "*" ++ usePrec 1 e) ;
  App = infixl 2 "" ;
  Id e1 e2 e3 = mkPrec 3 (usePrec 4 e1 ++ usePrec 4 e2 ++ "==" ++ usePrec 3 e3) ;
-- for an explicit vs implicit use of parameters, may have to use expressions as records, with a parameter is_implicit
  IdJ e1 e2 e3 e4 e5 = mkPrec 3 ("J" ++ usePrec 4 e1 ++ usePrec 4 e2 ++ usePrec 4 e3 ++ usePrec 4 e4 ++ usePrec 4 e5) ;
  Fst e = mkPrec 4 ("fst" ++ usePrec 4 e) ;
  Snd e = mkPrec 4 ("snd" ++ usePrec 4 e) ;
  Pair e1 e2 = mkPrec 5 ("(" ++ usePrec 0 e1 ++ "," ++ usePrec 0 e2 ++ ")") ;
  Var a = constant a ;
  Univ = constant "Set" ;
  Refl = "refl" ; -- constant "refl" ;

  BaseAIdent = "" ;
  ConsAIdent x xs = x ++ xs ;

  -- [Decl] only used in ExpWhere
  BaseDecl x = x ;
  ConsDecl x xs = x ++ "^" ++ xs ;

  -- maybe accomodate so split on empty type just gives () 
  -- BaseBranch = "" ;
  BaseBranch x = x ;
  -- ConsBranch x xs = x ++ "\n" ++ xs ;
  ConsBranch x xs = x ++ "||" ++ xs ;

  -- for data constructors
  BaseLabel x = x ;
  ConsLabel x xs = x ++ "|" ++ xs ; 

  BasePTele x = x ;
  ConsPTele x xs = x ++ xs ;

  BaseTele = "" ;
  ConsTele x xs = x ++ xs ;

  OBranch a la ew = a ++ la ++ "->" ++ ew ;
  TeleC a la e = "(" ++ a ++ la ++ ":" ++ usePrec 0 e ++ ")" ;
  PTeleC e1 e2 = "(" ++ top e1 ++ ":" ++ top e2 ++ ")" ;

  OLabel a lt = a ++ lt ;

  --object language syntax, all variables for now
  Bool = "bool" ;
  True = "true" ;
  False = "false" ;
  CaseBool = "caseBool" ;
  IndBool = "indBool" ;
  FunExt = "funExt" ;
  Nat = "nat" ;
  Zero = "zero" ;
  Suc = "suc" ;
  EqualNat = "equalNat" ;
  Unit = "unit" ;
  Top = "top" ;
  Foo = "foo" ; 
  A = "a" ;
  B = "b" ;
  C = "c" ;
  D = "d" ;
  E = "e" ;
  F = "f" ;
  G = "g" ;
  H = "h" ;
  I = "i" ;
  J = "j" ;
  K = "k" ;
  L = "l" ;
  M = "m" ;
  N = "n" ;
  O = "o" ;
  P = "p" ;
  Q = "q" ;
  R = "r" ;
  S = "s" ;
  T = "t" ;
  U = "u" ;
  V = "v" ;
  W = "w" ;
  X = "x" ;
  Y = "y" ;
  Z = "z" ;
  NegB = "negb" ;
  -- everything below is for contractible proofs
  Contr = "isContr" ;
  Fiber = "fiber" ;
  IsEquiv = "isEquiv" ;
  IdIsEquiv = "idIsEquiv" ;
  IdFun = "idfun" ;
  ContrSingl = "contrSingl" ;
  Equiv = "equiv" ;
  EqToIso = "eqToIso" ;
  Identity = "id" ;
  Ybar = "ybar"  ;
  IdFib = "idFib"  ;
  Lemma0 = "lemma0" ;
}
\end{verbatim}
The resource FormalTwo.gf merely substitutes more precedences than Formal.gf
from the RGL, in the ideal case that we could scale the grammar to include
larger and more complicated fixity information.
\begin{verbatim}
resource FormalTwo = open Prelude in {

----Everything the same up until the definition of Prec in Formal.gf


    Prec : PType = Predef.Ints 9 ;

    highest = 9 ;

    lessPrec : Prec -> Prec -> Bool = \p,q ->
      case <<p,q> : Prec * Prec> of {
        <3,9> | <2,9> | <4,9> | <5,9> | <6,9> | <7,9> | <8,9> => True ;
        <3,8> | <2,8> | <4,8> | <5,8> | <6,8> | <7,8> => True ;
        <3,7> | <2,7> | <4,7> | <5,7> | <6,7> => True ;
        <3,6> | <2,6> | <4,6> | <5,6> => True ;
        <3,5> | <2,5> | <4,5> => True ;
        <3,4> | <2,3> | <2,4> => True ;
        <1,1> | <1,0> | <0,0> => False ;
        <1,_> | <0,_>         => True ;
        _ => False
      } ;

    nextPrec : Prec -> Prec = \p -> case <p : Prec> of {
      9 => 9 ;
      n => Predef.plus n 1
      } ;
\end{verbatim}


\subsection{Hott and cubicalTT Grammars} \label{comparison}

\begin{code}[hide]%
\>[0]\AgdaSymbol{\{-\#}\AgdaSpace{}%
\AgdaKeyword{OPTIONS}\AgdaSpace{}%
\AgdaPragma{--omega-in-omega}\AgdaSpace{}%
\AgdaPragma{--type-in-type}\AgdaSpace{}%
\AgdaSymbol{\#-\}}\<%
\\
%
\\[\AgdaEmptyExtraSkip]%
\>[0]\AgdaKeyword{module}\AgdaSpace{}%
\AgdaModule{compare}\AgdaSpace{}%
\AgdaKeyword{where}\<%
\\
%
\\[\AgdaEmptyExtraSkip]%
\>[0]\AgdaKeyword{open}\AgdaSpace{}%
\AgdaKeyword{import}\AgdaSpace{}%
\AgdaModule{Agda.Builtin.Sigma}\AgdaSpace{}%
\AgdaKeyword{public}\<%
\\
%
\\[\AgdaEmptyExtraSkip]%
\>[0]\AgdaKeyword{variable}\<%
\\
\>[0][@{}l@{\AgdaIndent{0}}]%
\>[2]\AgdaGeneralizable{A}\AgdaSpace{}%
\AgdaGeneralizable{B}\AgdaSpace{}%
\AgdaSymbol{:}\AgdaSpace{}%
\AgdaPrimitive{Set}\<%
\\
%
\\[\AgdaEmptyExtraSkip]%
\>[0]\AgdaKeyword{data}\AgdaSpace{}%
\AgdaOperator{\AgdaDatatype{\AgdaUnderscore{}≡\AgdaUnderscore{}}}\AgdaSpace{}%
\AgdaSymbol{\{}\AgdaBound{A}\AgdaSpace{}%
\AgdaSymbol{:}\AgdaSpace{}%
\AgdaPrimitive{Set}\AgdaSymbol{\}}\AgdaSpace{}%
\AgdaSymbol{(}\AgdaBound{a}\AgdaSpace{}%
\AgdaSymbol{:}\AgdaSpace{}%
\AgdaBound{A}\AgdaSymbol{)}\AgdaSpace{}%
\AgdaSymbol{:}\AgdaSpace{}%
\AgdaBound{A}\AgdaSpace{}%
\AgdaSymbol{→}\AgdaSpace{}%
\AgdaPrimitive{Set}\AgdaSpace{}%
\AgdaKeyword{where}\<%
\\
\>[0][@{}l@{\AgdaIndent{0}}]%
\>[2]\AgdaInductiveConstructor{r}\AgdaSpace{}%
\AgdaSymbol{:}\AgdaSpace{}%
\AgdaBound{a}\AgdaSpace{}%
\AgdaOperator{\AgdaDatatype{≡}}\AgdaSpace{}%
\AgdaBound{a}\<%
\\
%
\\[\AgdaEmptyExtraSkip]%
\>[0]\AgdaKeyword{infix}\AgdaSpace{}%
\AgdaNumber{20}\AgdaSpace{}%
\AgdaOperator{\AgdaDatatype{\AgdaUnderscore{}≡\AgdaUnderscore{}}}\<%
\end{code}
\begin{code}%
\>[0]\AgdaFunction{id}\AgdaSpace{}%
\AgdaSymbol{:}\AgdaSpace{}%
\AgdaGeneralizable{A}\AgdaSpace{}%
\AgdaSymbol{→}\AgdaSpace{}%
\AgdaGeneralizable{A}\<%
\\
\>[0]\AgdaFunction{id}\AgdaSpace{}%
\AgdaSymbol{=}\AgdaSpace{}%
\AgdaSymbol{λ}\AgdaSpace{}%
\AgdaBound{z}\AgdaSpace{}%
\AgdaSymbol{→}\AgdaSpace{}%
\AgdaBound{z}\<%
\\
%
\\[\AgdaEmptyExtraSkip]%
\>[0]\AgdaFunction{iscontr}\AgdaSpace{}%
\AgdaSymbol{:}\AgdaSpace{}%
\AgdaSymbol{(}\AgdaBound{A}\AgdaSpace{}%
\AgdaSymbol{:}\AgdaSpace{}%
\AgdaPrimitive{Set}\AgdaSymbol{)}\AgdaSpace{}%
\AgdaSymbol{→}\AgdaSpace{}%
\AgdaPrimitive{Set}\<%
\\
\>[0]\AgdaFunction{iscontr}\AgdaSpace{}%
\AgdaBound{A}\AgdaSpace{}%
\AgdaSymbol{=}%
\>[13]\AgdaRecord{Σ}\AgdaSpace{}%
\AgdaBound{A}\AgdaSpace{}%
\AgdaSymbol{λ}\AgdaSpace{}%
\AgdaBound{a}\AgdaSpace{}%
\AgdaSymbol{→}\AgdaSpace{}%
\AgdaSymbol{(}\AgdaBound{x}\AgdaSpace{}%
\AgdaSymbol{:}\AgdaSpace{}%
\AgdaBound{A}\AgdaSymbol{)}\AgdaSpace{}%
\AgdaSymbol{→}\AgdaSpace{}%
\AgdaSymbol{(}\AgdaBound{a}\AgdaSpace{}%
\AgdaOperator{\AgdaDatatype{≡}}\AgdaSpace{}%
\AgdaBound{x}\AgdaSymbol{)}\<%
\\
%
\\[\AgdaEmptyExtraSkip]%
\>[0]\AgdaFunction{fiber}\AgdaSpace{}%
\AgdaSymbol{:}\AgdaSpace{}%
\AgdaSymbol{(}\AgdaBound{A}\AgdaSpace{}%
\AgdaBound{B}\AgdaSpace{}%
\AgdaSymbol{:}\AgdaSpace{}%
\AgdaPrimitive{Set}\AgdaSymbol{)}\AgdaSpace{}%
\AgdaSymbol{(}\AgdaBound{f}\AgdaSpace{}%
\AgdaSymbol{:}\AgdaSpace{}%
\AgdaBound{A}\AgdaSpace{}%
\AgdaSymbol{->}\AgdaSpace{}%
\AgdaBound{B}\AgdaSymbol{)}\AgdaSpace{}%
\AgdaSymbol{(}\AgdaBound{y}\AgdaSpace{}%
\AgdaSymbol{:}\AgdaSpace{}%
\AgdaBound{B}\AgdaSymbol{)}\AgdaSpace{}%
\AgdaSymbol{→}\AgdaSpace{}%
\AgdaPrimitive{Set}\<%
\\
\>[0]\AgdaFunction{fiber}\AgdaSpace{}%
\AgdaBound{A}\AgdaSpace{}%
\AgdaBound{B}\AgdaSpace{}%
\AgdaBound{f}\AgdaSpace{}%
\AgdaBound{y}\AgdaSpace{}%
\AgdaSymbol{=}\AgdaSpace{}%
\AgdaRecord{Σ}\AgdaSpace{}%
\AgdaBound{A}\AgdaSpace{}%
\AgdaSymbol{(λ}\AgdaSpace{}%
\AgdaBound{x}\AgdaSpace{}%
\AgdaSymbol{→}\AgdaSpace{}%
\AgdaBound{y}\AgdaSpace{}%
\AgdaOperator{\AgdaDatatype{≡}}\AgdaSpace{}%
\AgdaBound{f}\AgdaSpace{}%
\AgdaBound{x}\AgdaSymbol{)}\<%
\\
%
\\[\AgdaEmptyExtraSkip]%
\>[0]\AgdaFunction{isEquiv}\AgdaSpace{}%
\AgdaSymbol{:}\AgdaSpace{}%
\AgdaSymbol{(}\AgdaBound{A}\AgdaSpace{}%
\AgdaBound{B}\AgdaSpace{}%
\AgdaSymbol{:}\AgdaSpace{}%
\AgdaPrimitive{Set}\AgdaSymbol{)}\AgdaSpace{}%
\AgdaSymbol{→}\AgdaSpace{}%
\AgdaSymbol{(}\AgdaBound{f}\AgdaSpace{}%
\AgdaSymbol{:}\AgdaSpace{}%
\AgdaBound{A}\AgdaSpace{}%
\AgdaSymbol{→}\AgdaSpace{}%
\AgdaBound{B}\AgdaSymbol{)}\AgdaSpace{}%
\AgdaSymbol{→}\AgdaSpace{}%
\AgdaPrimitive{Set}\<%
\\
\>[0]\AgdaFunction{isEquiv}\AgdaSpace{}%
\AgdaBound{A}\AgdaSpace{}%
\AgdaBound{B}\AgdaSpace{}%
\AgdaBound{f}\AgdaSpace{}%
\AgdaSymbol{=}\AgdaSpace{}%
\AgdaSymbol{(}\AgdaBound{y}\AgdaSpace{}%
\AgdaSymbol{:}\AgdaSpace{}%
\AgdaBound{B}\AgdaSymbol{)}\AgdaSpace{}%
\AgdaSymbol{→}\AgdaSpace{}%
\AgdaFunction{iscontr}\AgdaSpace{}%
\AgdaSymbol{(}\AgdaFunction{fiber}\AgdaSpace{}%
\AgdaBound{A}\AgdaSpace{}%
\AgdaBound{B}\AgdaSpace{}%
\AgdaBound{f}\AgdaSpace{}%
\AgdaBound{y}\AgdaSymbol{)}\<%
\\
%
\\[\AgdaEmptyExtraSkip]%
\>[0]\AgdaFunction{isEquiv'}\AgdaSpace{}%
\AgdaSymbol{:}\AgdaSpace{}%
\AgdaSymbol{(}\AgdaBound{A}\AgdaSpace{}%
\AgdaBound{B}\AgdaSpace{}%
\AgdaSymbol{:}\AgdaSpace{}%
\AgdaPrimitive{Set}\AgdaSymbol{)}\AgdaSpace{}%
\AgdaSymbol{→}\AgdaSpace{}%
\AgdaSymbol{(}\AgdaBound{f}\AgdaSpace{}%
\AgdaSymbol{:}\AgdaSpace{}%
\AgdaBound{A}\AgdaSpace{}%
\AgdaSymbol{→}\AgdaSpace{}%
\AgdaBound{B}\AgdaSymbol{)}\AgdaSpace{}%
\AgdaSymbol{→}\AgdaSpace{}%
\AgdaPrimitive{Set}\<%
\\
\>[0]\AgdaFunction{isEquiv'}\AgdaSpace{}%
\AgdaBound{A}\AgdaSpace{}%
\AgdaBound{B}\AgdaSpace{}%
\AgdaBound{f}\AgdaSpace{}%
\AgdaSymbol{=}\AgdaSpace{}%
\AgdaSymbol{∀}\AgdaSpace{}%
\AgdaSymbol{(}\AgdaBound{y}\AgdaSpace{}%
\AgdaSymbol{:}\AgdaSpace{}%
\AgdaBound{B}\AgdaSymbol{)}\AgdaSpace{}%
\AgdaSymbol{→}\AgdaSpace{}%
\AgdaFunction{iscontr}\AgdaSpace{}%
\AgdaSymbol{(}\AgdaFunction{fiber'}\AgdaSpace{}%
\AgdaBound{y}\AgdaSymbol{)}\<%
\\
\>[0][@{}l@{\AgdaIndent{0}}]%
\>[2]\AgdaKeyword{where}\<%
\\
\>[2][@{}l@{\AgdaIndent{0}}]%
\>[4]\AgdaFunction{fiber'}\AgdaSpace{}%
\AgdaSymbol{:}\AgdaSpace{}%
\AgdaSymbol{(}\AgdaBound{y}\AgdaSpace{}%
\AgdaSymbol{:}\AgdaSpace{}%
\AgdaBound{B}\AgdaSymbol{)}\AgdaSpace{}%
\AgdaSymbol{→}\AgdaSpace{}%
\AgdaPrimitive{Set}\<%
\\
%
\>[4]\AgdaFunction{fiber'}\AgdaSpace{}%
\AgdaBound{y}\AgdaSpace{}%
\AgdaSymbol{=}\AgdaSpace{}%
\AgdaRecord{Σ}\AgdaSpace{}%
\AgdaBound{A}\AgdaSpace{}%
\AgdaSymbol{(λ}\AgdaSpace{}%
\AgdaBound{x}\AgdaSpace{}%
\AgdaSymbol{→}\AgdaSpace{}%
\AgdaBound{y}\AgdaSpace{}%
\AgdaOperator{\AgdaDatatype{≡}}\AgdaSpace{}%
\AgdaBound{f}\AgdaSpace{}%
\AgdaBound{x}\AgdaSymbol{)}\<%
\\
%
\\[\AgdaEmptyExtraSkip]%
\>[0]\AgdaComment{-- proof from Aarne}\<%
\\
\>[0]\AgdaFunction{idIsEquiv'}\AgdaSpace{}%
\AgdaSymbol{:}\AgdaSpace{}%
\AgdaSymbol{(}\AgdaBound{A}\AgdaSpace{}%
\AgdaSymbol{:}\AgdaSpace{}%
\AgdaPrimitive{Set}\AgdaSymbol{)}\AgdaSpace{}%
\AgdaSymbol{→}\AgdaSpace{}%
\AgdaFunction{isEquiv}\AgdaSpace{}%
\AgdaBound{A}\AgdaSpace{}%
\AgdaBound{A}\AgdaSpace{}%
\AgdaSymbol{(}\AgdaFunction{id}\AgdaSpace{}%
\AgdaSymbol{\{}\AgdaBound{A}\AgdaSymbol{\})}\<%
\\
\>[0]\AgdaFunction{idIsEquiv'}\AgdaSpace{}%
\AgdaBound{A}\AgdaSpace{}%
\AgdaBound{y}\AgdaSpace{}%
\AgdaSymbol{=}\AgdaSpace{}%
\AgdaFunction{ybar}\AgdaSpace{}%
\AgdaOperator{\AgdaInductiveConstructor{,}}\AgdaSpace{}%
\AgdaFunction{help}\<%
\\
\>[0][@{}l@{\AgdaIndent{0}}]%
\>[2]\AgdaKeyword{where}\<%
\\
\>[2][@{}l@{\AgdaIndent{0}}]%
\>[4]\AgdaFunction{fib'}\AgdaSpace{}%
\AgdaSymbol{:}\AgdaSpace{}%
\AgdaPrimitive{Set}\AgdaSpace{}%
\AgdaComment{-- \{y : A\}}\<%
\\
%
\>[4]\AgdaFunction{fib'}\AgdaSpace{}%
\AgdaSymbol{=}\AgdaSpace{}%
\AgdaFunction{fiber}\AgdaSpace{}%
\AgdaBound{A}\AgdaSpace{}%
\AgdaBound{A}\AgdaSpace{}%
\AgdaFunction{id}\AgdaSpace{}%
\AgdaBound{y}\<%
\\
%
\>[4]\AgdaFunction{ybar}\AgdaSpace{}%
\AgdaSymbol{:}\AgdaSpace{}%
\AgdaFunction{fib'}\<%
\\
%
\>[4]\AgdaFunction{ybar}\AgdaSpace{}%
\AgdaSymbol{=}\AgdaSpace{}%
\AgdaBound{y}\AgdaSpace{}%
\AgdaOperator{\AgdaInductiveConstructor{,}}\AgdaSpace{}%
\AgdaInductiveConstructor{r}\<%
\\
%
\>[4]\AgdaFunction{help}\AgdaSpace{}%
\AgdaSymbol{:}\AgdaSpace{}%
\AgdaSymbol{(}\AgdaBound{x}\AgdaSpace{}%
\AgdaSymbol{:}\AgdaSpace{}%
\AgdaFunction{fib'}\AgdaSymbol{)}\AgdaSpace{}%
\AgdaSymbol{→}\AgdaSpace{}%
\AgdaOperator{\AgdaDatatype{\AgdaUnderscore{}≡\AgdaUnderscore{}}}\AgdaSpace{}%
\AgdaSymbol{\{}\AgdaRecord{Σ}\AgdaSpace{}%
\AgdaBound{A}\AgdaSpace{}%
\AgdaSymbol{(}\AgdaOperator{\AgdaDatatype{\AgdaUnderscore{}≡\AgdaUnderscore{}}}\AgdaSpace{}%
\AgdaBound{y}\AgdaSymbol{)\}}\AgdaSpace{}%
\AgdaFunction{ybar}\AgdaSpace{}%
\AgdaBound{x}\<%
\\
%
\>[4]\AgdaFunction{help}\AgdaSpace{}%
\AgdaSymbol{=}\AgdaSpace{}%
\AgdaSymbol{λ}\AgdaSpace{}%
\AgdaSymbol{\{(}\AgdaBound{a}\AgdaSpace{}%
\AgdaOperator{\AgdaInductiveConstructor{,}}\AgdaSpace{}%
\AgdaInductiveConstructor{r}\AgdaSymbol{)}\AgdaSpace{}%
\AgdaSymbol{→}\AgdaSpace{}%
\AgdaInductiveConstructor{r}\AgdaSymbol{\}}\<%
\\
%
\\[\AgdaEmptyExtraSkip]%
\>[0]\AgdaFunction{equiv}\AgdaSpace{}%
\AgdaSymbol{:}\AgdaSpace{}%
\AgdaSymbol{(}\AgdaSpace{}%
\AgdaBound{a}\AgdaSpace{}%
\AgdaBound{b}\AgdaSpace{}%
\AgdaSymbol{:}\AgdaSpace{}%
\AgdaPrimitive{Set}\AgdaSpace{}%
\AgdaSymbol{)}\AgdaSpace{}%
\AgdaSymbol{→}\AgdaSpace{}%
\AgdaPrimitive{Set}\<%
\\
\>[0]\AgdaFunction{equiv}\AgdaSpace{}%
\AgdaBound{a}\AgdaSpace{}%
\AgdaBound{b}\AgdaSpace{}%
\AgdaSymbol{=}\AgdaSpace{}%
\AgdaRecord{Σ}\AgdaSpace{}%
\AgdaSymbol{(}\AgdaBound{a}\AgdaSpace{}%
\AgdaSymbol{→}\AgdaSpace{}%
\AgdaBound{b}\AgdaSymbol{)}\AgdaSpace{}%
\AgdaSymbol{λ}\AgdaSpace{}%
\AgdaBound{f}\AgdaSpace{}%
\AgdaSymbol{→}\AgdaSpace{}%
\AgdaFunction{isEquiv}\AgdaSpace{}%
\AgdaBound{a}\AgdaSpace{}%
\AgdaBound{b}\AgdaSpace{}%
\AgdaBound{f}\<%
\\
%
\\[\AgdaEmptyExtraSkip]%
\>[0]\AgdaFunction{equivId}\AgdaSpace{}%
\AgdaSymbol{:}\AgdaSpace{}%
\AgdaSymbol{(}\AgdaBound{x}\AgdaSpace{}%
\AgdaSymbol{:}\AgdaSpace{}%
\AgdaPrimitive{Set}\AgdaSymbol{)}\AgdaSpace{}%
\AgdaSymbol{→}\AgdaSpace{}%
\AgdaFunction{equiv}\AgdaSpace{}%
\AgdaBound{x}\AgdaSpace{}%
\AgdaBound{x}\<%
\\
\>[0]\AgdaFunction{equivId}\AgdaSpace{}%
\AgdaBound{x}\AgdaSpace{}%
\AgdaSymbol{=}\AgdaSpace{}%
\AgdaFunction{id}\AgdaSpace{}%
\AgdaOperator{\AgdaInductiveConstructor{,}}\AgdaSpace{}%
\AgdaSymbol{(}\AgdaFunction{idIsEquiv'}\AgdaSpace{}%
\AgdaBound{x}\AgdaSymbol{)}\<%
\\
%
\\[\AgdaEmptyExtraSkip]%
\>[0]\AgdaFunction{eqToIso}\AgdaSpace{}%
\AgdaSymbol{:}\AgdaSpace{}%
\AgdaSymbol{(}\AgdaSpace{}%
\AgdaBound{a}\AgdaSpace{}%
\AgdaBound{b}\AgdaSpace{}%
\AgdaSymbol{:}\AgdaSpace{}%
\AgdaPrimitive{Set}\AgdaSpace{}%
\AgdaSymbol{)}\AgdaSpace{}%
\AgdaSymbol{→}\AgdaSpace{}%
\AgdaOperator{\AgdaDatatype{\AgdaUnderscore{}≡\AgdaUnderscore{}}}\AgdaSpace{}%
\AgdaSymbol{\{}\AgdaPrimitive{Set}\AgdaSymbol{\}}\AgdaSpace{}%
\AgdaBound{a}\AgdaSpace{}%
\AgdaBound{b}\AgdaSpace{}%
\AgdaSymbol{→}\AgdaSpace{}%
\AgdaFunction{equiv}\AgdaSpace{}%
\AgdaBound{a}\AgdaSpace{}%
\AgdaBound{b}\<%
\\
\>[0]\AgdaFunction{eqToIso}\AgdaSpace{}%
\AgdaBound{a}\AgdaSpace{}%
\AgdaDottedPattern{\AgdaSymbol{.}}\AgdaDottedPattern{a}\AgdaSpace{}%
\AgdaInductiveConstructor{r}\AgdaSpace{}%
\AgdaSymbol{=}\AgdaSpace{}%
\AgdaFunction{equivId}\AgdaSpace{}%
\AgdaBound{a}\<%
\end{code}

Compared with the latex code

\begin{figure}[H]
 \textbf{Definition}:
 A type $A$ is contractible, if there is $a : A$, called the center of contraction, such that for all $x : A$, $\equalH {a}{x}$.

 \textbf{Definition}:
 A map $f : \arrowH {A}{B}$ is an equivalence, if for all $y : B$, its fiber, $\comprehensionH {x}{A}{\equalH {\appH {f}{x}}{y}}$, is contractible.
 We write $\equivalenceH {A}{B}$, if there is an equivalence $\arrowH {A}{B}$.

 \textbf{Lemma}:
 For each type $A$, the identity map, $\defineH {\idMapH {A}}{\typingH {\lambdaH {x}{A}{x}}{\arrowH {A}{A}}}$, is an equivalence.

 \textbf{Proof}:
 For each $y : A$, let $\defineH {\fiberH {y}{A}}{\comprehensionH {x}{A}{\equalH {x}{y}}}$ be its fiber with respect to $\idMapH {A}$ and let $\defineH {\barH {y}}{\typingH {\pairH {y}{\reflexivityH {A}{y}}}{\fiberH {y}{A}}}$.
 As for all $y : A$, $\equalH {\pairH {y}{\reflexivityH {A}{y}}}{y}$, we may apply Id-induction on $y$, $\typingH {x}{A}$ and $\typingH {z}{\idPropH {x}{y}}$ to get that \[\equalH {\pairH {x}{z}}{y}\].
 Hence, for $y : A$, we may apply $\Sigma$ -elimination on $\typingH {u}{\fiberH {y}{A}}$ to get that $\equalH {u}{y}$, so that $\fiberH {y}{A}$ is contractible.
 Thus, $\typingH {\idMapH {A}}{\arrowH {A}{A}}$ is an equivalence.
  $\Box$

 \textbf{Corollary}:
 If $U$ is a type universe, then, for $X , Y : U$, \[(*)\arrowH {\equalH {X}{Y}}{\equivalenceH {X}{Y}}\].

 \textbf{Proof}:
 We may apply the lemma to get that for $X : U$, $\equivalenceH {X}{X}$.
 Hence, we may apply Id-induction on $\typingH {X , Y}{U}$ to get that $(*)$.
  $\Box$


 \textbf{Definition}:
 A type universe $U$ is univalent, if for $X , Y : U$, the map $\equivalenceMapH {X}{Y}: \arrowH {\equalH {X}{Y}}{\equivalenceH {X}{Y}}$ in $(*)$ is an equivalence.
\end{figure}

cubicalTT parses the following.  We note an idealization : that agda supports ananymous pattern matching, so 
\codeword{\\ ( ( b , refl )}  would not typecheck in the original cubicalTT. Additionally, the reflexivity constructor is only present when the identity is inductively defined, as it is a primitive in cubical type theories.

\begin{figure}[H]
\begin{verbatim}
id ( a : Set ) : a -> a = \\ ( b : a ) -> b
isContr ( a : Set ) : Set = ( b : a ) * ( x : a ) -> a b == x
fiber ( a b  : Set ) ( f : a -> b ) ( y : b )  : Set 
  = ( x : a ) * ( x : a ) -> b y == ( f x )
isEquiv ( a b  : Set ) ( f : a -> b )   : Set 
  = ( y : b ) -> isContr ( fiber a b f y )
  where fiber ( a b  : Set ) ( f : a -> b ) ( y : b )  : Set 
    = ( x : a ) * ( x : a ) -> b y == ( f x )
equiv ( a b : Set ) : Set = ( f : a -> b ) * isEquiv a b f

idIsEquiv ( a : Set ) : isEquiv a a ( id a ) =  ( ybar , lemma0 )
  where
    idFib : Set = fiber a a id y
    ^ ybar : idFib = ( y , refl )
    ^ lemma0 ( x : idFib ) : ( ( p ) ybar == x ) 
      = \\ ( ( b , refl ) : ( c : a ) * ( a c == c ) ) -> refl

idIsEquiv ( x : Set ) : equiv x x = ( id , idIsEquiv x )
eqToIso ( a b : Set ) : ( Set a == b ) -> equiv a b 
  = split refl -> idIsEquiv a
\end{verbatim}
\end{figure}

We compare two abstract syntax trees side by side to show that they have quite different structures,

\begin{figure}
\centering
\begin{minipage}[t]{.5\textwidth}
\begin{verbatim}
Exp> 
* DeclDef
    * Contr
      ConsTele
        * TeleC
            * A
              BaseAIdent
              Univ
          BaseTele
      Univ
      NoWhere
        * Sigma
            * BasePTele
                * PTeleC
                    * Var
                        * B
                      Var
                        * A
              Pi
                * BasePTele
                    * PTeleC
                        * Var
                            * X
                          Var
                            * A
                  Id
                    * Var
                        * A
                      Var
                        * B
                      Var
                        * X
\end{verbatim}
\end{minipage}%
\begin{minipage}[t]{.55\textwidth}
\begin{verbatim}
* PredDefinition
    * type_Sort
      A_Var
      contractible_Pred
      ExistCalledProp
        * a_Var
          ExpSort
            * VarExp
                * A_Var
          FunInd
            * centre_of_contraction_Fun
          ForAllProp
            * allUnivPhrase
                * BaseVar
                    * x_Var
                  ExpSort
                    * VarExp
                        * A_Var
              ExpProp
                * DollarMathEnv
                  equalExp
                    * VarExp
                        * a_Var
                      VarExp
                        * x_Var
\end{verbatim}
\end{minipage}
\caption{Mathematical Assertions and Agda Judgements} \label{fig:I2}
\end{figure}

What we notice : 


\begin{figure}
\centering
\begin{minipage}[t]{.5\textwidth}
\begin{verbatim}
* DeclSplit
    * EqToIso
      ConsTele
        * TeleC
            * A
              ConsAIdent
                * B
                  BaseAIdent
              Univ
          BaseTele
      Fun
        * Id
            * Univ
              Var
                * A
              Var
                * B
          App
            * App
                * Var
                    * Equiv
                  Var
                    * A
              Var
                * B
      BaseBranch
        * OBranch
            * Refl
              BaseAIdent
              NoWhere
                * App
                    * Var
                        * IdIsEquiv
                      Var
                        * A
\end{verbatim}
\end{minipage}%
\begin{minipage}[t]{.55\textwidth}
\begin{verbatim}
3 PropParagraph
    * NoConclusionPhrase
      ForAllProp
        * if_thenUnivPhrase
            * BaseVar
                * U_Var
              type_universe_Sort
          ForAllProp
            * plainUnivPhrase
                * ConsVar
                    * X_Var
                      BaseVar
                        * Y_Var
                  ExpSort
                    * VarExp
                        * U_Var
              LabelledExpProp
                * DisplayMathEnv
                  StarLabel
                  mapExp
                    * equalExp
                        * VarExp
                            * X_Var
                          VarExp
                            * Y_Var
                      equivalenceExp
                        * VarExp
                            * X_Var
                          VarExp
                            * Y_Var
4 ConclusionParagraph
    1 NoConclusionPhrase
      ApplyLabelConclusion
        * the_lemma_Label
          BaseInd
          ForAllProp
            * plainUnivPhrase
                * BaseVar
                    * X_Var
                  ExpSort
                    * VarExp
                        * U_Var
              ExpProp
                * DollarMathEnv
                  equivalenceExp
                    * VarExp
                        * X_Var
                      VarExp
                        * X_Var
    2 henceConclusionPhrase
      ApplyLabelConclusion
        * id_induction_Label
          ConsInd
            * FunInd
                * ExpFun
                    * TypedExp
                        * ConsExp
                            * VarExp
                                * X_Var
                              BaseExp
                                * VarExp
                                    * Y_Var
                          VarExp
                            * U_Var
              BaseInd
          LabelProp
            * StarLabel
\end{verbatim}
\end{minipage}
\caption{Mathematical Assertions and Agda Judgements} \label{fig:I3}
\end{figure}

todo : refactor to have the final sections side-by-side, do a more "thorough analysis of the text fragment above"
namely - look at the redundancy, the intro of identity local to a definition (often having more than one proposition in a proposition) 
the failure in some instances to provide relevant info, etc.

also, refactor to have the sigma proof here

Exp> * DeclDef
    * IdIsEquiv
      ConsTele
        * TeleC
            * X
              BaseAIdent
              Univ
          BaseTele
      App
        * App
            * Var
                * Equiv
              Var
                * X
          Var
            * X
      NoWhere
        * Pair
            * Var
                * Identity
              App
                * Var
                    * IdIsEquiv
                  Var
                    * X





\subsection{HoTT Agda Corpus} \label{hottproofs}

\end{document}


