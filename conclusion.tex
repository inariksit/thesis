\section{Goals and Challenges}

The parser is still quite primitive, and needs to be extended extensively to
support natural language ambiguity in mathematics as well as other linguistic
nuance that GF captures well, like tense and aspect. This can follow a method
expored in Aarne's paper : "Translating between Language and Logic: What Is
Easy and What Is Difficult" where one develops a denotational semantics for
translating between natural language expressions with the desired AST. The bulk
of this work will be writing a Haskell back-end implementing this AST
transformation. The extended syntax, designed for linguistic nuance, will be
filtered into the core syntax, which is essentially what I have done.

The Resource Grammar Library (RGL) is designed for out-of-the box grammar
writing, and therefore much of the linearization nuance can be outsourced to
this robust and well-studied library. Nonetheless, each application grammar
brings its own unique challenges, and the RGL will only get one so far. My
linearization may require extensive tweaking.

Thus far, our parser is only able to parse non-cubical fragments of the
cubicalTT standard library. Dealing with Agda pattern matching, it was
realized, is outside the theoretical boundaries of GF (at least, if one were to
do it in a non ad-hoc way) due to its inability to pass arbitrary strings down
the syntax tree nodes during linearization. Pattern matching therefore needs to be dealt
with via pre and post processing.  Additionally, cubicaltt is weaker at
dealing with telescopes than Agda, and so a full generalization to Agda is not
yet possible. Universes are another feature future iterations of this Grammar
would need to deal with, but as they aren't present in most mathematician's
vernacular, it is not seen as relevant for the state of this project.

Records should also be added, but because this grammar supports sigma types,
there is no rush. The Identity type is so far deeply embedded in our grammar,
so the first code fragment may just be for explanatory purposes.  The degree to
which the library is extended to cover domain specific information is up to
debate, but for now the grammar is meant to be kept as minimal as possible.

One interesting extension, time dependnet, would be to allow for a bidrectional
feedback between GF and Agda : thereby allowing ad hoc extensions to GF's ASTs
to allow for newly defined Agda functions to be treated with more care, i.e.
have an arguement structure rather than just treating everything as variables.
This may be too ambitious for the time being.

% Random ideas

Category theory in agda paper, differences in formalization

* my agda hott library
* escardo's hott library 
  - if successful on mine, with universe support
  - mix of latex, agda code , and natural language 
* dummy example for non-hott math (spivak et al, type-in-type)
* alternatively, trying digging in the mountain at the other end, and try extedning ad-hoc grammar with various syntactic nuance
* Latex & Unicode support  - 
* Degenerate cases
  - find examples which are unable to be supported by this grammar, explain why and offer future possible patches

Talk about all the things that need to be done

Pattern Matching, additional parser vs internal to GF

How to decide an optimal phrase (this seems like itd be some rule based) from agda program

* Support for cs math - e.g. specifications of algorithms and their actual implementations
* Alternative syntaxes - graphical languages like grasshopper
* user interface
  - QA
  - Hoogle for proofs
* NL semantics (the semantic content is precisely the formal statements)
* Comparison / integration with ML approaches
* studies in concerete syntax -Harper psychology {\intersect} programming

Testing, with particular reference to the pgf grammar I developed


It is also worth noting that with respect to our earlier comparative analysis
of PLs and NLs [cite earlier section], there has been work comparing things like
numeracy, natural language acquisition skills, and programming language skills
\cite{prat2020relating}.  This account offers evidence that PL and NL acquisition
in humans who have no experience coding 
and it is claimed that the study ``are consistent with previous research
reporting higher or unique predictive utility of verbal aptitude tests when
compared to mathematical one" with respect to learning Python.

However, as Python is unityped, and similar experiments with a typed programming
language would perhaps be more relevant - especially for studying mathematical
abilities more consistent with the mathematicians notion of mathematics, rather
than just numeracy. Additionally, it would be interesting to explore the role of
PL syntax in such studies - and if what kind of variation could be linked to the
concrete PL syntax.

Concrete syntax is in some sense where programming language theory meets
psychology, (Bob Harper Oplss 2017)
