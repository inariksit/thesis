\section{Conclusion}

\begin{displayquote}
Concrete syntax is in some sense where programming language theory meets
psychology. \emph{Robert Harper} 
\end{displayquote}

% move to conclusion
% However, as a counterexample to the Agda developer's mentality, the so-called
% \emph{Brunerie Number}, a computation about the homotopy groups of spheres,
% doesn't terminate in Agda but can be computed with pen and paper, offers a
% counterexample even the computer scientist has to come to terms with
% \cite{brunerie2016homotopy}. It is sufficiently precise, but nonetheless poses a
% practical issue to formal systems where, until it terminates, offers presumably
% correct code which doesn't match the mathematicians linguistic proof. And while
% this one example which may see resolution, many further counterexample may
% abound. It is a speculative matter to guess which mathematics is formalizable.

There are two major problems in the reformulation of mathematics via
typed languages which underlay interactive proofs assistants that fall under the
scope of this thesis tried to grapple with. 

The first is how to make a dependently typed programming language capable of
formulating proposition and proof more amenable to mathematicians with the goal
of improving semantic adequacy. The second seeks to ask how to facilitate the
formalization of mathematics, whether that be translation of theorem statements,
proofs, or giving the mathematicians a template for using CNLs that are more
suited to their tastes and also capable of providing tractable data in various
applications.

Progress in either of these directions will only be realized and through
significant time, labor, expertise, and most vitally, \emph{original thinking}
through collaborative efforts. It is uncertain what the role of parsers,
abstract syntax trees, linearization schemes, and other components of the GF
ecosystem will have on these efforts, but we hope that our efforts have shown
that its very much an open plane of exploration. 

Our work has perhaps only made a small contribution to these incredibly
difficult problems, yet, we hope that compiling various ideas across many
different fields have at least given some philosophical clarification as to why
the problems are so difficult. Additionally, we hope our proposal to the GF
means of approaching these problems through the analysis and comparison
different grammars gives legitimation and evidence that there's a feasibility of
actually applying these ideas to solve real problems, or at the very least,
asking important questions that may influence other methods. For there is no
doubt a role to play for statistical methods in tackling these problems as well,
How these data-based methods along with the rule-based techniques is
also incredibly speculative and merits careful consideration and research.

Our contributions, partially original and partially extrapolated from others,
are the following :

\begin{itemize}
\item Introduce notions of \emph{syntactic completeness} and \emph{semantic adequacy}, so-as to
allowing understanding a piece of mathematics based off its degree of formality
as well as clarity of presentation
\item Recognition that the GF approach is limited, especially as regards pragmatic
concerns and other philosophical stances about mathematics, like the role of
visualization in mathematical thinking and presentation
\item Offer explicit comparisons, through examples of mathematics in a textual form and a type
  theoretic presentation
\item The developments of new GF grammars for analyzing this problem
\item The first comparison of all known GF grammars in this domain with respect
  to \emph{syntactic completeness} and \emph{semantic adequacy}
\item The development of an Agda library which mirrors the HoTT book so that
  future work can seek a possible ``large-scale" translation case study
\end{itemize}


 It has been remarked that the bigger grammars gets, the more it begins to
resemble a domain specific RGL \cite{angelovSS}. We advocate to actually produce
a ``formal language RGL", whereby many of the ideas advocated and observed in
this work, like document structure, latex (and symbolic support generally),
custom lexical classes (like in BNFC), and many other features not witnessed in
the small case studies undertaken here should be accounted for. Therefore, the
grammar writer's time could be better spent either focusing on the scaling of
programming language features or the actual linguistic analysis of
mathematics text - thereby making a more natural CNLs.

Despite the promise of various topics discussed here like Cubical Agda, the
Formal Abstracts Project , and the use of Lean in mathematics education
\cite{buzzard2020will}, we don't foresee a convergence of type theorists and
mathematicians, even though belief the holy trinity would compel us to think so.
GF as a PL paradigm, applied to this problem, gives us a stark contrast of how
different these two approaches to mathematical language. For the grammars of
proof are insufficient to capture the complexity and nuance about the language
of proof, so much of which has yet to captured in an existing linguistic
framework. Grammars for propositions and definitions offer a much more limited
and seemingly feasible problem, especially with GF largely because
mathematicians write them with the explicit intention of being comprehensible
and unambiguous. We hope this works serves the mathematical community in
achieving some of these goals.

\subsection{The Mathematical Library of Babel}

\emph{The Library of Babel} \cite{borges} was a profound mirror held up to the human species
as regards our comprehension of the world through language. It reflected our
inability to grasp and reconcile our own finitude. The infinite stack of
shelves, containing every book with every permutation of letters from the Hebrew
alphabet, leaves the humans who inhabit the closed space in a state of discord
as regards their failures to navigate and interpret the myriad texts.

\emph{The Library} most certainly contains all mathematical statements, with all
possible foundations of mathematics, theorems, proofs of those theorems, in all
the possible syntactic presentations. In addition, it contains a catalogue
documenting the mathematical constructions, and how these constructions can be
encoded in the multiplicity of foundational systems. If there is a master GF
grammar for translating all of the mathematics in \emph{The Library}, it certainly
contains the source code for that as well.

Unfortunately, the library also contains all the erroneous proofs, whether they
be lexical errors or a reference to flawed lemma somewhere much deeper in the
library. There are certainly proofs of the Riemann conjecture, its negation, and
its undecidability. When perceives mathematics through the lens of human
language generally, we must acknowledge that mathematical content,
constructions, and discoveries, are not developments that come by chance,
through sifting through bags of words until some gemstone gleams through the
noise. Humans have to produce mathematical constructions through hard labor,
sweat, and tears. More importantly we create mathematics through dialogue,
laughter, and occasionally even dreams.

To imbue the sentences of mathematics which we see on paper or in the terminal
with meaning we have some kind of internal mental mechanism that is at play with
our other mental faculties : our motor system and sensory capabilities generally.
We don't merely derive formulas by computing, but we distill ideas in our
linguistic capacity to some kind of unambiguous kernel. The view of mathematics,
that is just some subset of \emph{The Library}, waiting to be discovered or verified by a
machine, is an incredibly misinformed and myopic view of the subject. That
mathematics is a human endeavor, complete with all our lust, flaws, and
ingenuity should be more clear after contemplating how difficult it is to
construct a grammar of proof.


% which section, if worth keeping?
% It is also worth noting that with respect to our earlier comparative analysis
% of PLs and NLs [cite earlier section], there has been work comparing things like
% numeracy, natural language acquisition skills, and programming language skills
% \cite{prat2020relating}.  This account offers evidence that PL and NL acquisition
% in humans who have no experience coding 
% and it is claimed that the study ``are consistent with previous research
% reporting higher or unique predictive utility of verbal aptitude tests when
% compared to mathematical one" with respect to learning Python.

% However, as Python is unityped, and similar experiments with a typed programming
% language would perhaps be more relevant - especially for studying mathematical
% abilities more consistent with the mathematicians notion of mathematics, rather
% than just numeracy. Additionally, it would be interesting to explore the role of
% PL syntax in such studies - and if what kind of variation could be linked to the

 
% Andreasp comment about the proof state/terms being desugared in every known PL -
% % ask a question of how one can make the interactivity more amenable to a kind of
% mathematical oracle, and therefore give semantic, not just syntactic goal states
% (i.e help allow the programmer to reason semantically)


