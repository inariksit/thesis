\section{Grammatical Framework}

\subsection{Thinking about GF}

A grammar specification in GF is actually just an abstract syntax. With an abstract
syntax specified, one can then define various linearization rules which
compositionally evaluate to strings. An Abstract Syntax Tree (AST) may then be
linearized to various strings admitted by different concrete syntaxes.
Conversely, given a string admitted by the language being defined, GF's powerful
parser will generate all the ASTs which linearize to that tree.

When defining a GF pipeline, one has to merely to construct an abstract syntax
file and a concrete syntax file such that they are coherent. In the abstract,
one specifies the \emph{semantics} of the domain one wants to translate over,
which is ironic, because we normally associate abstract syntax with \emph{just
syntax}. However, because GF was intended for implementing the natural language
phenomena, the types of semantic categories (or sorts) can grow much bigger than
is desirable in a programming language, where minimalism is generally favored.
The \emph{foods grammar} is the \emph{hello world} of GF, and should be referred
to for those interested in example of how the abstract syntax serves as a
semantic space in non-formal NL applications \cite{ranta2011grammatical}.

Let us revisit the ``tetrahedral doctrine", now restricting our attention to the
subset of linguistics which GF occupies. We first examine how GF fits into the
trinity, as seen in  \autoref{fig:G1}. Immediately, GF abstract syntax with
dependent types can just be seen as an
implementation of MLTT with the added bonus of a parser.  Additionally, GF is a
relatively tame Type Theory, and therefore it would be easy to construct a model
in a general purpose programming language, like Agda.  Embeddings of GF already
exist in Coq [cite FraCoq], Haskell [cite pgf], and MMT [cite mmt],  These applications allow one to use GF's
parser so that a GF AST may be transformed into some kind of
inductively defined tree these languages all support. Future work could involve
modeling GF in Agda would allow one to prove things about GF
meta-theorems about soundness and termination, or perhaps statements about
specific grammars, such as one being unambiguous.

From the logical side, we note that GF's parser specification was done using
inference rules [cite krasimir]. Given the coupling of Context-Free Grammars
(CFGs) and operads (also known as multicategories) [cite lambek, etc], one could use much more
advanced mathematical machinery to articulate and understand GF.  We sketch this
briefly below [refer].

\begin{figure}[H]
\centering
\begin{tikzcd}
     &  &  & Logic                                                                                                                                             &  &  &            \\
     &  &  &                                                                                                                                                   &  &  &            \\
     &  &  & GF \arrow[uu, "GF\ Parser\ Specification"'] \arrow[llldd, "Theory\ of\ Operads"']
     \arrow[rrrdd, "Implementation\ of", bend left] \arrow[rrrdd, "Agda\ Embedding", bend right] &  &  &            \\
     &  &  &                                                                                                                                                   &  &  &            \\
Math &  &  &                                                                                                                                                   &  &  & CS\ (MLTT)
\end{tikzcd}
\caption{Models of GF} \label{fig:G1}
\end{figure}

One can additionally model these domains in GF, which is obviously the main
focus of this work. In \autoref{fig:G2}, we see that there are 3 grammars which
give allow one to translate in these domains. Ranta's grammar from CADE 2011,
built a propositional framework with a core grammar extended with other
categories to capture syntactic nuance. Ranta's grammar from the Stockholm University
mathematics seminar in 2014 took verbatim text from a publication of Peter Aczel
and sought to show that all the syntactic nuance by constructing a grammar
capable of NL translation. Finally, our work takes a BNFC grammar for a real
programming language cubicaltt [cite], GFifies it, producing an unambiguous
grammar.

\begin{figure}[H]
\centering
\begin{tikzcd}
                                              &  &  & Logic \arrow[dd, "Ranta\
                                              Logic\ (CADE\ 11)"] &  &  &                                       \\
                                              &  &  &                                          &  &  &                                       \\
                                              &  &  & GF                                       &  &  &                                       \\
                                              &  &  &                                          &  &  &                                       \\
Math \arrow[rrruu, "Ranta\ (HoTT\ 14)"] &  &  &                                          &  &  & CS\ (MLTT) \arrow[llluu, "cubicalTT"]
\end{tikzcd}
\caption{Trinitarian Grammars} \label{fig:G2}
\end{figure}

While these three grammars offer the most poignant points of comparison between
the computational, logical, and mathematical phenomena they attempt to capture,
we also note that there were many other smaller grammars developed during the
course of this work to supplement and experiment with various ideas presented.
Importantly, the ``Trinitarian Grammars" do not only model these different
domains, but they each do so in a unique way, making compromises and capturing
various linguistic and formal phenomena. The phenomena should be seen on a
spectrum of \emph{semantic adequacy} and \emph{syntactic completeness}, as in
autoref{fig:G3} .  


\begin{figure}[H]
\centering
\begin{tikzcd}
Lexicon\ Size                                                                                                                                          &  &  & Syntactic\ Completeness \\
                                                                                                                                                       &  &  & {}                      \\
                                                                                                                                                       &  &  &                         \\
Spectrum\ of\ GF \arrow[uuu, "Statistical\ Methods?"] \arrow[rrr, "Ranta\ HoTT\ '14"'] \arrow[rrruuu, "cubicalTT"] \arrow[rrruu, "Ranta\ Logic\ '14"'] &  &  & Semantic\ Adequacy
\end{tikzcd}
\caption{The Grammatical Dimension} \label{fig:G3}
\end{figure}

The cubicalTT grammar, seeking syntacitic completeness, only has a pidgin
English syntax, and therefore is only capable of parsing a programming language.
Ranta's HoTT grammar on the other hand, while capable of presenting a
quasi-logical form, would require extensive refactoring in order to transform
the ASTs to something that resembles the ASTs of a programming language. The
Logic grammar, which produces logically coherent and linguistically nuanced
expressions, does not yet cover proofs, and therefore would require additional extensions
to actually express anything a computer might understand, or, alternatively,
theorems capable of impressing a mathematician. Finally, we note that large-scale
coverage of linguistic phenomena for any of these grammars will additionally
need to incorporate statistical methods in some way. 

Before providing perspectives on the grammar design process, it is alo 
When designing grammars, the foremost question one should ask
A few remarks on designing GF grammars should be noted as well. 

The PMCFG class of languages is still quite tame when compared with, for
instance, Turing complete languages. Thus, the `abstract` and `concrete`
coupling tight, the evaluation is quite simple, and the programs tend to write
themselves once the correct types are chosen. This is not to say GF programming
is easier than in other languages, because often there are unforeseen
constraints that the programmer must get used to, limiting the palette available
when writing code. These constraints allow for fast parsing, but greatly limit
the types of programs one often thinks of writing.

\subsection{A Brief Introduction to GF}

GF is a very powerful yet simple system.  While learning the basics may not be
to difficult for the experienced programmer, GF requires the programmer to work
with, in some sense, an incredibly stiff set of constraints compared to general
purpose languages, and therefore its lack of expressiveness requires a different
way of thinking about programming.

The two functions displayed in \autoref{fig:N2}, $Parse : \{Strings\}
\rightarrow \{\{ASTs\}\}$ and $Linearize : \{ASTs\} \rightarrow \{Strings\}$, obey
the important property that :

 $$\forall s \in \{Strings\} \forall x \in (Parse(s)), Linearize(x) \equiv s$$

This seems somewhat natural from the programmers perspective. The limitation on
ASTs to linearize uniquely is actually a benefit, because it saves the user
having to make a choice about a translation (although, again, a statistical
mechanism could alleviate this constraint). We also want our translations to be
well-behaved mathematically, i.e. composing $Linearize$ and $Parse$ ad
infinitum should presumably not diverge.

GF captures languages more expressive than Chomsky's original 
CFG [cite] but is still remains decidable, with parsing in polynomial
time. Which polynomial depends on the grammar [cite krasimir]. 
It comes equipped with 6 basic judgments:

\begin{itemize}[noitemsep]
  \item Abstract : `cat, fun`
  \item Concrete : `lincat, lin, param`
  \item Auxiliary : `oper`
\end{itemize}

There are two judgments in an abstract file, for categories and named functions
defined over those categories, namely \term{cat} and \term{fun}. The categories
are just (succinct) names, and while GF allows dependent types, e.g. categories
which are parameterized over other categories and thereby allow for more
fine-grained semantic distinctions. We will leave these details aside, but do
note that GF's dependent types can be used to implement a programming language
which only parses well-typed terms (and can actually compute with them using
auxiliary declarations).

In a simply typed programming language we can choose categories, for
variables, types and expressions, or what might \term{Var}, \term{Typ}, and
\term{Exp} respectively. One can then define the functions for the simply typed
lambda calculus extended with natural numbers, known as Gödel's T.

\begin{verbatim} 
cat
  Typ ; Exp ; Var ;
fun
  Tarr : Typ -> Typ -> Typ ;
  Tnat : Typ ;

  Evar : Var -> Exp ;
  Elam : Var -> Typ -> Exp -> Exp ;
  Eapp : Exp -> Exp -> Exp ;

  Ezer : Exp ;
  Esuc : Exp -> Exp ;
  Enatrec : Exp -> Exp -> Exp ->  Exp ;

  X : Var ;
  Y : Var ;
  F : Var ;
  IntV : Int -> Var ;
\end{verbatim}

So far we have specified how to form expressions : types built out of
possibly higher order functions
between natural numbers, and expressions built out of lambda and
natural number terms. The variables are kept as a separate syntactic category,
and integers, \term{Int}, are predefined via GF's internals and simply allow one
to parse numeric expressions. One may then define a functional which takes a
function over the natural numbers and returns that function applied to $1$ - the
AST for this expression is :

\begin{verbatim} 
Elam
    F
    Tarr
        Tnat Tnat
      Eapp
        Evar
            F
        Evar
            IntV
                1
\end{verbatim} 

Dual to the abstract syntax there are parallel judgments when defining a concrete
syntax in GF, \term{lincat} and \term{lin} corresponding to \term{cat} and
\term{fun}, respectively. Wher the AST is the specification, the concrete
form is its implementation in a given lanaguage. The \term{lincat} serves to
give \emph{linearization types} which are quite simply either strings, records (or products
which support sub-typing and named fields), or tables (or coproducts) which can
make choices when computing with arbitrarily named parameters, which are
naturally isomorphic to the sets of some finite cardinality. The tables are
actually derivable from the records and their projections, which is how PGF is
defined internally, but they are so fundamental to GF programming and
expressiveness that they merit syntactic distincion.  The \term{lin}
is a term which matches the type signature of the \term{fun} with which it
shares a name. The \term{lincat} constrains the concrete types of the arguments,
and therefore subjects the GF user to how they are used. 

If we assume we are just working with strings, then we can simply define the
functions as recursively concatenating \term{++} strings. The lambda function
for pidgin English then has, as its linearization form as follows :

\begin{verbatim}
lin 
  Elam v t e = "function taking" ++ v ++ "in" ++ t ++ "to" ++ e ;
\end{verbatim}

Once all the relevant functions are giving correct linearizations, one can now
parse and linearize to the abstract syntax tree above the to string ``function
taking f in the natural numbers to the natural numbers to apply f to 1". This is
clearly unnatural for a variety of reasons, but it's an approximation of what
a computer scientist might say. Suppose instead, we choose to linearize this same
expression to a pidgin expression modeled off Haskell's syntax, ``\\ ( f
: nat -> nat ) -> f 1". We should notice the absence of parentheses for
application suggest something more subtle is happening with the linearization
process, for normally programming languages use fixity declarations to avoid
lispy looking code. Here are the linearization functions which allow for
linearization from the above AST :

\begin{verbatim}
lincat
  Typ = TermPrec ;
  Exp = TermPrec ;
lin
  Elam v t e = 
    mkPrec 0 ("\\" ++ parenth (v ++ ":" ++ usePrec 0 t) ++ "->" ++ usePrec 0 e) ;
  Eapp = infixl 2 "" ;
\end{verbatim}

Where did \term{TermPrec}, \term{infixl}, \term{parenth}, \term{mkPrec}, and
\term{usePrec} come from? These are all functions defined in the RGL. We show a
few of them below, thereby introducing the final, main GF judgments \term{param}
and \term{oper} for parameters and operators.

\begin{verbatim}
param 
  Bool = True | False ;
oper
  TermPrec : Type = {s : Str ; p : Prec} ;
  usePrec : Prec -> TermPrec -> Str = \p,x ->
    case lessPrec x.p p of {
      True => parenth x.s ;
      False => parenthOpt x.s
    } ;
  parenth : Str -> Str = \s -> "(" ++ s ++ ")" ;
  parenthOpt : Str -> Str = \s -> variants {s ; "(" ++ s ++ ")"} ;
\end{verbatim}

Parameters in GF, to a first approximation, are simply data types of unary
constructors with finite cardinality. Operators, on the other hand, encode the
logic of GF linearization rules. They are an unnecessary part of the language
because they don't introduce new logical content, but they do allow one to
abstract the function bodies of \term{lin}'s so that one may keep the actual
linearization rules looking clean. Since GF also support \term{oper}
overloading, one can often get away with often deceptively sleek looking
linearizations, and this is a key feature of the RGL. The variants is one of the
ways to encode multiple linearizations forms for a given tree, so here, for
example, we're breaking the nice property from above.

This more or less resembles a typical programming language, with very little
deviation from what when would expect specifying something in twelf.
Nonetheless, because this is both meant to somehow capture the logical form in
addition to the surface appearance of a language, the separation of concerns
leaves the user with an important decision to make regarding how one couples the
linear and abstract syntaxes. There are in some sense two extremes one can take
to get a well performing GF grammar.

Suppose you have a page of text from some random source of length $l$, and you
take it as an exercise to build a GF grammar which translates it. The first
extreme approach you could take would be to give each word in the text to a
unique category, a unique function for each category bearing the word's name,
along with a single really function with $l$ arguments for the whole sequence of
words in the text. One could then verbatim copy the words as just strings with
their corresponding names in the concrete syntax. This overfitted grammar would
fail : it wouldn't scale to other languages, wouldn't cover any texts other than
the one given to it, and wouldn't be at all informative. Alternatively, one
could create a grammar of a two categories $c$ and $s$ with two functions, $f_0
: c$ and $f_1 : c \rightarrow s$, whereby c would be given $n$ fields, each
strings, with the string given at position $i$ in $f_0$ matching $word_i$ from
the text. $f_1$ would merely concatenate it all. This grammar would be similarly
degenerate, despite also parsing the page of text.

This seemingly silly example highlights the most blatant tension the GF grammar
writer will face : how to balance syntactic and semantic content of the grammar
in between the concrete and the abstract syntax. It is also highly
relevant as concerns the domain of translation, for a programming language
with minimal syntax and the mathematicians language in expressing her ideas are
on vastly different sides of this issue.

We claim that syntactically complete grammars are much more easily dealt with
simple abstract syntax. However, to take allow a syntactically complete grammar
to capture semantic nuance and neutrality then humans requires immensely more
work on the concrete side. Semantically adequate grammars on the other hand,
require significantly more attention on the abstract side, because semantically
meaningful expressions often don't generalize - each part of an expressions
exhibits unique behaviors which can't be abstracted to apply to other parts of
the expression. Therefore, producing a syntactically complete expressions which
doesn't overgenerate parses also requires a lot work from the grammar writer.

We hope the subsequent examples will illuminate this tension. The problem with
treating a syntactically oriented domain like type theory with and a semantically
oriented one like mathematics with the same abstract syntax poses very serious
problems, but also highlights the power of other features of GF, like the RGL [cite]
and Haskell embedding PGF [cite].

The GF RGL is a very robust library for parsing grammatically coherent language.
It exists for many different natural languages with a core abstract syntax
shared by all of them. The API allows one to easily construct, sentence level
phrases once the lexicon has been defined, which are also greatly facilitated by
the API. 

PGF, is an embedding of a GF abstract syntax into Haskell, where the categories
are given ``shadow types", so that one can build turn an abstract syntax into (a
possibly massive) Generalized Algebraic Data Type (GADT) \term{Tree} with kind
\codeword{* -> *} where all the functions serve as constructors. If function
\codeword{h} returns category \codeword{c}, the Haskell constructor
\codeword{Gh} returns \codeword{Tree c}.

The PGF API also allows for the Haskell user to call the parse and linearization
functions, so that once the grammar is built, one can use Haskell as an
interface with the outside world. While GF originally was conceived as allowing
computation with ASTs, using a semantic computation judgment \term{def}, this
has approach has largely been overshadowed by Haskell. Once a grammar is
embedded in Haskell, one can use general recursion, monads, and all other types
of bells and whistles produced by the functional programming community to
compute with the embedded ASTs.

We note that this further muddies the water of
what syntax and semantics refer to in the GF lexicon. Although a GF
abstract syntax somehow represents the programmers idealized semantic domain,
once embedded in PGF the trees now may represent syntactic objects to be
evaluated or transformed to some other semantic domain which may or may not
eventually be linked back to a GF linearization.

These are all the main ingredients a GF user will hopefully need to understand
the grammars hereby elaborated, and hopefully these examples will showcase the
full potential of GF for the problem of mathematical translations.

% \subsection{Mathematical Model of GF}
% Note on the construction of free monoids

% Consider a language $L$ we want to represent, and we come up with a model that we
% build as a set of categories and functions over those categories.  Let $Cat(L)$,
% denote the categories.  Also suppose we define functions such that, given an
% ordered list $x_1,...,x_n;y \in Cat(L)$ we define a set of functions,
% $Fun_L(x_1,...,x_n;y)$ defined over the categories. In gf, a function can be
% denoted something like $\phi : x_1 \rightarrow ... \rightarrow x_n$. We may compose these based
% off their arities. So, given a function $\psi \in Fun_L(y_1,,...,y_n;z)$,
% functions $\phi_1,...\phi_n$ such that $\phi_i \in Fun_L(x_{i,1},...,x_{i,m};y_i)}$ 
%  we can plug these functions in together, or nest them such that
% $$\psi \circ (\phi_1,...,\phi_n) : \rightarrow (x_{i,j}) \rightarrow (y_{i})
% \rightarrow Z$$ 

% This is how abstract syntax trees are formed. It is also worth noting that they
% obey an associativity property, namely that 

% \begin{align*}
% &\theta \circ (\psi_1 \circ (\phi_{1,1},...,\phi_{1,k_1}),...,\psi_n \circ
% (\phi_{n,1},...,\phi_{n,k_n}))\\ = &(\theta \circ \psi_1,...\psi_n) \circ (\phi_{1,1},...,\phi_{1,k_1},...,\phi_{n,1},...,\phi_{n,k_n})
% \end{align*}

% This means that trees in GF are invariant as to how they are built - we
% can build a tree from the leaves to the root or vice versa.

% Example : consider the arithmetic grammar of exponentiation, multiplication, and
% addition defined over a single category of natural number expressions, whereby
% the function symbol is to be interpreted as a string and the tensor product,
% $\otimes$ as the concatenation during evaluation. 

% $$\_\^{}\_ : \mathds{N} \to \mathds{N} \to \mathds{N}$$
% $$\_*\_ : \mathds{N} \to \mathds{N} \to \mathds{N}$$
% $$\_+\_ : \mathds{N} \to \mathds{N} \to \mathds{N}$$

% We can think of constructing the trees by partial application, i.e., 

% $(\lambda x.\: 2 \otimes \^{} \otimes x) : \mathds{N} -> \mathds{N}$

% Lets try see the constructions yielding the string $(1 + 2) \^{} (3 * 4)$.

% We can either (i) construct this as the exponent of two fully formed expressions,
% namely a sum and a product applied to some numbers, or we can first apply the
% exponent to the two binary functions, yielding a quaternary function .

% $x ++ y$
% $x \doubleplus y$
% $``x \doubleplus y"$

% \begin{align*}
% &(\lambda x,y.\: x \otimes \^{} \otimes y)\\
% &\hspace{1cm} ((\lambda x,y.\:x \otimes + \otimes y)\; 1\; 2)\\
% &\hspace{1cm} ((\lambda x,y.\: x \otimes * \otimes y)\; 3\; 4) \\
% \mapsto\; &(\lambda x,y.\: x \otimes \^{} \otimes y)\\
% &\hspace{1cm} (1 + 2)\\
% &\hspace{1cm} (3 * 4))\\
% \mapsto\; &((1 + 2) \^{} (3 * 4))\\
% \end{align}

% \begin{align*}
% &((\lambda x,y.\: x \otimes \^{} \otimes y)\\
% &\hspace{1cm} (\lambda x,y.\:x \otimes + \otimes y)\\
% &\hspace{1cm} (\lambda x,y.\: x \otimes * \otimes y)) \\
% &\hspace{1cm} 1\; 2; 3; 4; \\
% \end{align}

% (1 + 2) \^{} (3 * 4)
  

% ((\lambda x,y. x \^{} y)
%   (\lambda x,y. x + y) 
%   (\lambda x,y. x * y))
%     1 2 3 4

% ((\lambda x,y. x + y) \^{} (\lambda x,y. x * y)) 1 2 3 4
% ((\lambda x,y. x + y) \^{} (\lambda x,y. x * y)) 1 2 3 4

% (1 + 2) \^{} (3 * 4)

% and then say
% (\lambda x. 2 \^{} x) (1 + 3) * (4 + 5)
% = 
% (\lambda x. 2 \^{} x) (1 + 3) * (4 + 5)

% $(\lambda x. 2 \wedge x) : \mathds{N} -> \mathds{N}$

% and then apply it to a complex arguement, say 
% (1 + 3) * (4 + 5)
% (\lambda x. 2 ^ x) : Nat -> Nat

% where 


% \lambda y : Pow y 1 : Nat -> Nat

% (times (plus 2 3) (plus 4 5))
% (Pow \circ (1,times)) : Nat -> Nat -> Nat

% (plus 2 3) (plus 4 5)

% can either be 

% 2^(1+3)*(4+5)


%   % \sin {:} \mathbb{R} &\rightarrow \mathbb{R}\\ x &\mapsto \sin ( x )
% % \circ (\phi_1,...,\phi_n) : \rightarrow (x_{i,j}) \rightarrow (y_{i})
% % \rightarrow Z$$ 

% The two functions displayed in, \autoref{fig:N2}.  If we can loosely call String
% the set of strings freely generated osome acan be 

% for now given a single linear presentation $C^{AST}$ , where

% AST_L String_L0 denote the sets GF ASTs and Strings in the languages generated
% by the rules of L's abstract syntax and L0s compositional representation.

% $$Parse : String -> {AST}$$
% $$Linearize : AST -> String$$

% with the important property that given a string s,


% And given an AST a, we can Parse . Linearize a belongs to {AST}

% Now we should explore why the linearizations are interesting. In part, this is
% because they have arisen from the role of grammars have played in the
% intersection and interaction between computer science and linguistics at least
% since Chomsky in the 50s, and they have different understandings and utilities
% in the respective disciplines. These two discplines converge in GF, which allows
% us to talk about natural languages (NLs) from programming languages (PLs)
% perspective.

