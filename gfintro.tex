\subsection{Grammatical Framework}
% \subsection{Intro to GF}

\subsubsection{Overview}

I will introduce Grammatical Framework (GF) through an extended example targeted
at a general audience familiar with logic, functional programming, or
mathematics. GF is a powerful tool for specifying grammars. A grammar
specification in GF is actually an abstract syntax. With an abstract syntax
specified, one can then define various linearization rules which compositionally
evaluate to strings. An Abstract Syntax Tree (AST) may then be linearized to
various strings admitted by different concrete syntaxes. Conversely, given a
string admitted by the language being defined, GF's powerful parser will
generate all the ASTs which linearize to that tree.

\subsubsection{Overview} % ## Focus of this tutorial

We introduce this subject assuming no familiarity with GF, but a general
mathematical and programming maturity. While GF is certainly geared to
applications in domain specific machine translation, this writing will hopefully
make evident that learning about GF, even without the intention of using it for
its intended application, is a useful exercise in abstractly understanding
syntax and its importance in just about any domain.

A working high-level introduction of Grammatical Framework emphasizing both
theoretical and practical elements of GF, as well as their interplay. Specific
things covered will be

\begin{itemize}[noitemsep]

\item Historical developments leading to the creation and discovery of the GF
formalism
\item The difference between Abstract and Concrete Syntax, and the linearization
and parsing functions between these two categories
\item The basic judgments :
  \begin{itemize}[noitemsep]
    \item Abstract : `cat, fun`
    \item Concrete : `lincat, lin`
    \item Auxiliary : `oper, param`
  \end{itemize}
\item A library for building natural language applications with GF
  \begin{itemize}[noitemsep]
    \item The Resource Grammar Library (RGL)
    \item A way of interfacing GF with Haskell and transforming ASTs externally
    \item The Portable Grammar Format (PGF)
\end{itemize}
\item Module System and directory structure
\item A brief comparison with other tools like BNFC

\end{itemize}

These topics will be understood via a simple example of an expression language,
`Arith`, which will serve as a case study for understanding the many robust,
theoretical topics mentioned above - indeed, mathematics is best understood and
learned through examples. The best exercises are the reader's own ideas,
questions, and contentions which may arise while reading through this.

\subsubsection{Theoretical Overview}
% \subsubsection{Theoretical Overview}

\paragraph { What is a GF grammar ? }

[TODO : update with latex]

Consider a language L, for now given a single linear presentation C^L_0, where
AST_L String_L0 denote the sets GF ASTs and Strings in the languages generated
by the rules of L's abstract syntax and L0s compositional representation.

  $Parse : String -> {AST} Linearize : AST -> String$

with the important property that given a string s,

  forall x in (Parse s), Linearize x == s

And given an AST a, we can Parse . Linearize a belongs to {AST}

Now we should explore why the linearizations are interesting. In part, this is
because they have arisen from the role of grammars have played in the
intersection and interaction between computer science and linguistics at least
since Chomsky in the 50s, and they have different understandings and utilities
in the respective disciplines. These two discplines converge in GF, which allows
us to talk about natural languages (NLs) from programming languages (PLs)
perspective. GF captures languages more expressive than Chomsky's Context Free
Grammars (CFGs) but is still decideable with parsing in (cubic?) polynomial
time, which means it still is quite domain specific and leaves a lot to be
desired as far as turing complete languages or those capable of general
recursion are concerned.

We can should note that computa

  given a string s , perhaps a phrase map Linearize (Parse s)

is understood as the set of translations of a phrase of one language to possibly
grammatical phrases in the other languages connected by a mutual abstract
syntax. So we could denote these L^English, L^Bantu, L^Swedish for
L^Englsih_American vs L^Englsih_English or L^Bantu_i with i being any one of the
Bantu languages a hypothetical set of linguists and grammar writers may want to
implement, on a given fragment of abstract syntax, with various domain
applications, as determined by L, in mind.

  One could also further elaborate these L^English_0 L^English_1 to varying
degrees of granularity, like L^English_Chomsky L^English_Partee because Chomsky
may something like "I was stoked" and Partee may only say something analogous "I
was really excited" or whatever the individual nuances come with how speakers
say what they mean with different surface syntax, and also

  L_0 < L_1 -> L_English_0 < L_English_1

But this would be similair to a set of expressions translateable between
programming languages, like Logic^English, Logic^Latex, Logic^Agda, Logic^Coq,
etc

where one could extend the

Logic_Core^English Logic_Extended^English

whereas in the PL domain

Logic_Core^Agda Logic_Extended^Agda may collapse at certain points, but also in
some ways extend beyond our Languages capacities (we can make machine code only
machines can understand, not us)

or Mathematics = Logic + domain specific Mathematics^English Mathematics^Agda

where we could have further refinements, via, for instance, the module system,
the concrete and linear designs like

Mathematics^English iI -- Something about

  The Functor (in the module sense familiar to ML programmers)

  break down to different classifications of

-- Something about

  The Functor (in the module sense familiar to ML programmers)

  break down to different classifications of

  The indexes here, while seemingly arbitrary,

  One could also further elaborate these L^English_0 L^English_1 to varying
degrees of granularity, like L^English_Chomsky L^English_Partee

  because Chomsky may something like "I was stoked" and Partee may only say
something analogous "I was really excited" or whatever the individual nuances
come with how speakers say what they mean with different surface syntax, and
also

Given a set of categories, we define the functions \phi : (c_1, ..., c_n) -> c_t
over the categories which will serve as the constructors of our ASTs. The
categories and functions are denoted in GF with the cat and fun judgment.

\subsubsection {Some preliminary observations and considerations}

There are many ways to skin a cat. While in some sense GF offers the user a
limited palette with which to paint, she nonetheless has quite a bit of
flexibility in her decision making when designing a GF grammar for a specific
domain or application. These decisions are not binary, but rest in the spectrum
of of considerations varying between :

* immediate usability and long term sustainability * prototyping and production
readiness * dependency on external resources liable to change * research or
application oriented * sensitivity and vulnerability to errors * scalability and
maintainability

Many answers to where on the spectrum a Grammar lies will only become clear a
posteriori to code being written, which often contradicts prior decisions which
had been made and requires significant efforts to refactor. General best
practices that apply to all programming languages, like effective use of
modularity, can and should be applied by the GF programmer out of the box,
whereas the strong type system also promotes a degree of rigidity with which the
programmer is forced to stay in a certain safety boundary. Nonetheless, a
grammar is in some sense a really large program, which brings a whole series of
difficulties.

When designing a GF grammar for a given application, the most immediate question
that will come to mind is separation of concerns as regards the spectrum of

[Abstract <-> Concrete] syntax

Have your cake and eat it ?

\subsubsection{A grammar for basic arithmetic}

\paragraph {Abstract Judgments}

The core syntax of GF is quite simple. The abstract syntax specification,
denoted mathematically above as _, and in GF as `Arith.gf` is given by :


\begin{verbatim} abstract Arith = { ... }
\end{verbatim}

Please note all GF files end with the `.gf` file extension. More detailed
information about abstract, concrete, modules, etc. relevant for GF is specified
internal to a `*.gf` file

The abstract specification is simple, and reveals GF's power and elegance. The
two primary abstract judgments are :

\begin{enumerate}
\item `cat` : denoting a syntactic category
\item. `fun` : denoting a n-ary function over categories. This is essentially a
labeled context-free rewrite rule with (non-)terminal string information
suppressed
\end{enumerate}

While there are more advanced abstract judgments, for instance `def` allows one
to incorporate semantic information, discussion of these will be deferred to
other resources. These core judgments have different interpretations in the
natural and formal language settings. Let's see the spine of the `Arith`
language, where we merely want to be able to write expressions like `( 3 + 4 ) *
5` in a myriad of concrete syntaxes.

\begin{verbatim} cat Exp ;

fun Add : Exp -> Exp -> Exp ; Mul : Exp -> Exp -> Exp ; EInt : Int -> Exp ;
\end{verbatim}

To represent this abstractly, we merely have two binary operators, labeled `Add`
and `Mul`, whose intended interpretation is just the operator names, and the
`EInt` function which coerces a predefined `Int`, natively supported numerical
strings `"0","1","2",...` into arithmetic expressions. We can now generate our
first abstract syntax tree, corresponding to the above expression, `Mul (Add
(EInt 3) (EInt 4)) (EInt 5)`, more readable perhaps with the tree structure
expanded :

\begin{verbatim} Mul Add EInt 3 EInt 4 EInt 5
\end{verbatim}

The trees nodes are just the function names, and the leaves, while denoted above
as numbers, are actually function names for the built-in numeric strings which
happen to be linearized to the same piece of syntax, i.e. `linearize 3 == 3`,
where the left-hand 3 has type `Int` and the right-hand 3 has type `Str`. GF has
support for very few, but important categories. These are `Int`, `Float`, and
`String`. It is my contention and that adding user defined builtin categories
would greatly ease the burden of work for people interested in using GF to model
programming languages, because `String` makes the grammars notoriously
ambiguous.

In computer science terms, to judge `Foo` to be a given category `cat Foo;`
corresponds to the definition of a given Algebraic Datatypes (ADTs) in Haskell,
or inductive definitions in Agda, whereas the function judgments `fun`
correspond to the various constructors. These connections become explicit in the
PGF embedding of GF into Haskell, but examining the Haskell code below makes one
suspect their is some equivalence lurking in the corner:

\begin{verbatim} data Exp = Add Exp Exp | Mul Exp Exp | EInt Int
\end{verbatim}

In linguistics we can interpret the judgments via alternatively simple and
standard examples:

\begin{enumerate}
\item `cat` : these could be syntactic categories like Common Nouns `CN`, Noun
Phrases `NP` , and determiners `Det`
\item `fun` : give us ways of combining words or phrases into more complex
syntactic units
\end {enumerate}

For instance, if

\begin{verbatim} fun Car_CN : CN ; The_Det : Det ; DetCN : Det -> CN -> NP ;
\end{verbatim}

Then one can form a tree `DetCN The_Det Car_CN` which should linearize to `"the
car"` in English, `bilen` in Swedish, and `imoto` in Zulu once we have
implemented concrete syntax linearizations for these respective languages, which
we will now do.

While there was an equivalence suggested Haskell ADTs should be careful not to
treat these as the same as the GF judgments. Indeed, the linguistic
interpretation breaks this analogy, because linguistic categories aren't stable
mathematical objects in the sense that they evolved and changed during the
evolution of language, and will continue to do so. Since GF is primarily
concerned with parsing and linearization of languages, the full power of
inductive definitions in Agda, for instance, doesn't seem like a particularly
natural space to study and model natural language phenomena.


\paragraph{Arith.gf}

Below we recapitulate, for completeness, the whole `Arith.gf` file with all the
pieces from above glued together, which, the reader should start to play with.

\begin{verbatim} abstract Arith = {

flags startcat = Exp ;

-- a judgement which says "Exp is a category" cat Exp ;

fun Add : Exp -> Exp -> Exp ; -- "+" Mul : Exp -> Exp -> Exp ; -- "*" EInt : Int
-> Exp ; -- "33"

}
\end{verbatim}

The astute reader will recognize some code which has not yet been described. The
comments, delegated with `--`, can have their own lines or be given at the end
of a piece of code. It is good practice to give example linearizations as
comments in the abstract syntax file, so that it can be read in a stand-alone
way.

The `flags startcat = Exp ;` line is not a judgment, but piece of metadata for
the compiler so that it knows, when generating random ASTs, to include a
function at the root of the AST with codomain `Exp`. If I hadn't included `flags
startcat = *some cat*` , and ran `gr` in the gf shell, we would get the
following error, which can be incredibly confusing but simple bug to fix if you
know what to look for!

\begin{verbatim} Category S is not in scope CallStack (from HasCallStack):
error, called at src/compiler/GF/Command/Commands.hs:881:38 in
gf-3.10.4-BNI84g7Cbh1LvYlghrRUOG:GF.Command.Commands
\end{verbatim}


\paragraph{Concrete Judgments}

We now append our abstract syntax GF file `Arith.gf` with our first concrete GF
syntax, some pigdin English way of saying our same expression above, namely `the
product of the sum of 3 and 4 and 5`. Note that `Mul` and `Add` both being
binary operators preclude this reading : `product of (the sum of 3 and 4 and 5)`
in GF, despite the fact that it seems the more natural English interpretation
and it doesn't admit a proper semantic reading.

Reflecting the tree around the `Mul` root, `Mul (EInt 5) (Add (EInt 3) (EInt
4))`, we get a reading where the 'natural interpretation' matches the actual
syntax :`the product of 5 and the sum of 3 and 4`. Let's look at the concrete
syntax which allow us to simply specify the linearization rules corresponding to
the above `fun` function judgments.

Our concrete syntax header says that `ArithEng1` is constrained by the fact that
the concrete syntaxes must share the same prefix with the abstract syntax, and
extend it with one or more characters, i.e. `Arith+.gf`.

\begin{verbatim} concrete ArithEng1 of Arith = { ... }
\end{verbatim}

We now introduce the two concrete syntax judgments which compliment those above,
namely :

* `cat` is dual to `lincat` * `fun` is dual to `lin`

Here is the first pass at an English linearization :

\begin{verbatim} lincat Exp = Str ;

lin Add e1 e2 = "the sum of" ++ e1 ++ "and" ++ e2 ; Mul e1 e2 = "the product of"
++ e1 ++ "and" ++ e2 ; EInt i = i.s ;
\end{verbatim}

The `lincat` judgement says that `Exp` category is given a linearization type
`Str`, which means that any expression is just evaluated to a string. There are
more expressive linearization types, records and tables, or products and
coproducts in the mathematician's lingo. For instance, `EInt i = i.s` that we
project the s field from the integer i (records are indexed by numbers but
rather by names in PLs). We defer a more extended discussion of linearization
types for later examples where they are not just useful but necessary, producing
grammars more expressive than CFGs called Parallel Multiple Context Free
Grammars (PMCFGs).

The linearization of the `Add` function takes two arguments, `e1` and `e2` which
must necessarily evaluate to strings, and produces a string. Strings in GF are
denoted with double quotes `"my string"` and concatenation with `++`. This
resulting string, `"the sum of" ++ e1 ++ "and" ++ e2` is the concatenation of
`"the sum of"`, the evaluated string `e1`, `"and"`, and the string of a
linearized `e2`. The linearization of `EInt` is almost an identity function,
except that the primitive Integer's are strings embedded in a record for
scalability purposes.

Here is the relationship between `fun` and `lin` from a slightly higher vantage
point. Given a `fun` judgement

  $f {:} C_0 \rightarrow C_1 \rightarrow ... \rightarrow C_n$

in the `abstract` file, the GF user provides a corresponding `lin` judgement of
the form

  $f \: c_0 \: c_1 \: ... \: c_n \: {=} \: t_0 \: \texttt{++} \: t_1 \:
\texttt{++} \: ... \: \texttt{++} \: t_m $

in the `concrete` file. Each $c_i$ must have the linearization type given in the
`lincat` of $C_i$ , e.g. if `lincat C_i = T ;` then `c_i : T`.

We step through the example above to see how the linearization recursively
evaluates, noting that this may not be the actual reduction order GF internally
performs. The relation `->*` informally used but not defined here expresses the
step function after zero or more steps of evaluating an expression. This is the
reflexive transitive closure of the single step relation `->` familiar in
operational semantics.

\begin{verbatim} linearize (Mul (Add (EInt 3) (EInt 4)) (EInt 5)) ->* "the
product of" ++ linearize (Add (EInt 3) (EInt 4)) ++ "and" ++ linearize (EInt 5)
->* "the product of" ++ ("the sum of" ++ (EInt 3) ++ (EInt 4)) ++ "and" ++ ({ s
= "5"} . s) ->* "the product of" ++ ("the sum of" ++ ({ s = "3"} . s) ++ ({ s =
"4"} . s)) ++ "and" ++ "5" ->* "the product of" ++ ("the sum of" ++ "3" ++ "and"
++ "4") ++ "and" ++ "5" ->* "the product of" ++ ("the sum of" ++ "3" ++ "and" ++
"4") ++ "and" ++ "5" ->* "the product of the sum of 3 and 4 and 5"
\end{verbatim}

The PMCFG class of languages is still quite tame when compared with, for
instance, Turing complete languages. Thus, the `abstract` and `concrete`
coupling tight, the evaluation is quite simple, and the programs tend to write
themselves once the correct types are chosen. This is not to say GF programming
is easier than in other languages, because often there are unforeseen
constraints that the programmer must get used to, limiting the palette available
when writing code. These constraints allow for fast parsing, but greatly limit
the types of programs one often thinks of writing. We touch upon this in a
[previous section](##some-preliminary-observations-and-considerations).

Now that the basics of GF have been described, we will augment our grammar so
that it becomes slightly more interesting, introduce a second `concrete` syntax,
and show how to run these in the GF shell in order to translate between our two
languages.

\subsubsection{The GF shell}

So now that we have a GF `abstract` and `concrete` syntax pair, one needs to
test the grammars.

Once GF is
[installed](https://www.grammaticalframework.org/download/index-3.10.html), one
can open both the `abstract` and `concrete` with the `gf` shell command applied
to the `concrete` syntax, assuming the `abstract` syntax is in the same
directory :

\begin{verbatim} $ gf ArithEng1.gf
\end{verbatim}

I'll forego describing many important details and commands, please refer to the
[official shell
reference](https://www.grammaticalframework.org/doc/gf-shell-reference.html) and
Inari Listenmaa's post on [tips and
gotchas](https://inariksit.github.io/gf/2018/08/28/gf-gotchas.html) for a more
nuanced take than I give here.

The `ls` of the gf repl is `gr`. What does `gr` do? Lets try it, as well as ask
gf what it does:

\begin{verbatim} Arith> gr Add (EInt 999) (Mul (Add (EInt 999) (EInt 999)) (EInt
999))

0 msec Arith> help gr gr, generate_random generate random trees in the current
abstract syntax
\end{verbatim}

We see that the tree generated isn't random - `999` is used exclusively, and
obviously if the trees were actually random, the probability of such a small
tree might be exceedingly low. The depth of the trees is cut-off, which can be
modified with the `-depth=n` flag for some number n, and the predefined
categories, `Int` in this case, are annoyingly restricted. Nonetheless, `gr` is
a quick first pass at testing your grammar.

Listed in the repl, but not shown here, are (some of the) additional flags that
`gr` command can take. The `help` command reveals other possible commands,
included is the linearization, the `l` command. We use the pipe `|` to compose
gf functions, and the `-tr` to trace the output of `gr` prior to being piped :

\begin{verbatim} Arith> gr -tr | l Add (Mul (Mul (EInt 999) (EInt 999)) (Add
(EInt 999) (EInt 999))) (Add (Add (EInt 999) (EInt 999)) (Add (EInt 999) (EInt
999)))

the sum of the product of the product of 999 and 999 and the sum of 999 and 999
and the sum of the sum of 999 and 999 and the sum of 999 and 999
\end{verbatim}

Clearly this expression is to too complex to say out loud and retain a semblance
of meaning. Indeed, most grammatical sentences aren't meaningful. Dealing with
semantics in GF is advanced, and we won't touch that here. Nonetheless, our
grammar seems to be up and running.

Let's try the sanity check referenced at the beginning of this post, namely,
observe that $linearize \circ parse$ preserves an AST, and vice versa, $parse
\circ linearize$ preserves a string. Parse is denoted `p`.

\begin{verbatim} Arith> gr -tr | l -tr | p -tr | l Add (EInt 999) (Mul (Add
(EInt 999) (EInt 999)) (Add (EInt 999) (EInt 999)))

the sum of 999 and the product of the sum of 999 and 999 and the sum of 999 and
999

Add (EInt 999) (Mul (Add (EInt 999) (EInt 999)) (Add (EInt 999) (EInt 999)))

the sum of 999 and the product of the sum of 999 and 999 and the sum of 999 and
999
\end{verbatim}

Phew, that's a relief. Note that this is an unambiguous grammar, so when I said
'preserves', I only meant it up to some notion of relatedness. This relation is
indeed equality for unambiguous grammars. Unambiguous grammars are degenerate
cases, however, so I expect this is the last time you'll see such tame behavior
when the GF parser is involved. Now that all the main ingredients have been
introduced, the reader

\ Exercises

**Exercise 1.1 :** Extend the `Arith` grammar with variables. Specifically,
modify both `Arith.gf` and `ArithEng1.gf` arithmetic with two additional unique
variables, `x` and `y`, such that the string `product of x and y` parses
uniquely {: .notice--danger}

**Exercise 1.2 :** Extend *Exercise 1.1* with unary predicates, so that `3 is
prime` and `x is odd` parse. Then include binary predicates, so that `3 equals
3` parses. {: .notice--danger}

**Exercise 2 :** Write concrete syntax in your favorite language,
`ArithFaveLang.gf` {: .notice--danger}

**Exercise 3 :** Write second English concrete syntax, `ArithEng2.gf`, that
mimics how children learn arithmetic, i.e. "3 plus 4" and "5 times 5". Observe
the ambiguous parses in the gf shell. Substitute `plus` with `+`, `times` with
`*`, and remedy the ambiguity with parentheses {: .notice--danger}

**Thought Experiment :** Observe that parentheses are ugly and unnecessary:
sophisticated folks use fixity conventions. How would one go about remedying the
ugly parentheses, at either the abstract or concrete level? Try to do it! {:
.notice--warning}

### Solutions

*Exercise 1.1*

This warm-up exercise is to get use to GF syntax. Add a new category `Var` for
variables, two lexical variable names `VarX` and `VarY`, and a coercion function
(which won't show up on the linearization) from variables to expressions. We
then augment the `concrete` syntax which is quite trivial.

```haskell -- Add the following between `{}` in `Arith.gf` cat Var ; fun VExp :
Var -> Exp ; VarX : Var ; VarY : Var ; ``` ```haskell -- Add the following
between `{}` in `ArithEng1.gf` lincat Var = Str ; lin VExp v = v ; VarX = "x" ;
VarY = "y" ; ```

*Exercise 1.2*

This is a similar augmentation as was performed above.

```haskell -- Add the following between `{}` in `Arith.gf` flags startcat = Prop
;

cat Prop ;

fun Odd : Exp -> Prop ; Prime : Exp -> Prop ; Equal : Exp -> Exp -> Prop ; ```
```haskell -- Add the following between `{}` in `ArithEng1.gf` lincat Prop = Str
; lin Odd e = e ++ "is odd"; Prime e = e ++ "is prime"; Equal e1 e2 = e1 ++
"equals" ++ e2 ; ``` The main point is to recognize that we also need to modify
the `startcat` flag to `Prop` so that `gr` generates numeric predicates rather
than just expressions. One may also use the category flag to generate trees of
any expression `gr -cat=Exp`.

*Exercise 2*

*Exercise 3*

We simply change the strings that get linearized to what follows :

```haskell lin Add e1 e2 = e1 ++ "plus" ++ e2 ; Mul e1 e2 = e1 ++ "times" ++ e2
; ```

With these minor modifications in place, we make the follow observation in the
GF shell, noting that for a given number of binary operators in an expression,
we get the [Catalan number](https://en.wikipedia.org/wiki/Catalan_number) of
parses!

```haskell Arith> gr -tr | l -tr | p -tr | l Add (Mul (VExp VarX) (Mul (EInt
999) (VExp VarX))) (EInt 999)

x times 999 times x plus 999

Add (Mul (VExp VarX) (Mul (EInt 999) (VExp VarX))) (EInt 999) Add (Mul (Mul
(VExp VarX) (EInt 999)) (VExp VarX)) (EInt 999) Mul (VExp VarX) (Add (Mul (EInt
999) (VExp VarX)) (EInt 999)) Mul (VExp VarX) (Mul (EInt 999) (Add (VExp VarX)
(EInt 999))) Mul (Mul (VExp VarX) (EInt 999)) (Add (VExp VarX) (EInt 999))

x times 999 times x plus 999 x times 999 times x plus 999 x times 999 times x
plus 999 x times 999 times x plus 999 x times 999 times x plus 999 ```

```haskell Arith> gr -tr | l -tr | p -tr Add (EInt 999) (Mul (VExp VarY) (Mul
(EInt 999) (EInt 999)))

( 999 + ( y * ( 999 * 999 ) ) )

Add (EInt 999) (Mul (VExp VarY) (Mul (EInt 999) (EInt 999))) ```



...Blog post...



